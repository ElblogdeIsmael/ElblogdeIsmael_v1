\section{Tema 2: Concepto de Agente Inteligente}

La Teoría de Agentes(denotado a continuación como \textbf{TA}) es un sistema de ordenador, situado en un entorno, capaz de realizar acciones de manera autónoma y que es flexible en base a las situaciones que se le presentan.

\subsection{Agentes Inteligentes}

Desarrolla un enfoque basado en agentes y lleva a cabo percepciones procesada mediante algoritmos de análisis y toma de decisiones. 
La TA se usa actualmente en la mayoría de ámbitos y es lo que más se usa hoy en día.

En cuanto a los tipos de agentes encontramos:

\begin{itemize}
    \item Reactivo: reacciona en base al entorno.
    \item Pro-activo: no solo actúan en respuesta al entorno, sino que puedan analizar acciones a llevar a cabo.
    \item  Social: son además capaces de interactuar.
    \item Multiagente: esta implementado como varios agentes interactuando.
\end{itemize}

La interacción entre agentes se lleva a cabo mediante: cooperación, coordinación y la negociación.


\subsection{Arquitecturas de Agentes}

Podemos distinguir entre:

\begin{itemize}
    \item Reactivo: reacciona en base a la situación en la que se encuentra, eligiendo la más adecuada en base a lo que sabe.
    \item Deliberativo: toma decisiones basadas en razonamiento, planificación y modelos internos del mundo.
\end{itemize}

