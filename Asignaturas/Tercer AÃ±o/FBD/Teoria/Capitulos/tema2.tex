% \section{Lenguajes de una BD}

% Recomendación ANSI/SPARC. Se propone un lenguaje de base de datos. En este se debe de definir, controlar y manipular los datos. Se denomina sub-lenguaje de la base de datos al que esta implementado por el propio SGBD. Tenemos distintas partes:
% \begin{itemize}
%     \item DDL: definición de estructura de datos.
%     \item DML: modificación, borrado y consulta de datos, además, se permite consultar los esquemas definidos de la Base de Datos.
%     \item DCL: gestiona los requisitos de acceso a los datos y otro tipo de tareas de administración(creación de usuarios,...).
% \end{itemize}

% Este grupo propone que haya cada uno de los anteriores en cada uno de los niveles de la arquitectura. Esto no tiene mucho sentido. Esto se debe a que no tiene sentido tener alguno de los anteriores en determinados niveles. Si meto el DDL en el nivel interno, estoy dependiendo del nivel interno de la máquina.

% En la realidad ha surgido la idea de que todo sea un estándar, pero cada fabricant lleva sus propias estrategias. Hay diferentes versiones de SQL y los SGBD han ido proporcionando soporte.

% Acto seguido, nos centramos en el desarrollo de aplicaciones. Son de propósito general, ya que usan lenguajes como C++, Java,... Se usan diversas herramientas como es Oracle APEX, Sysbase PowerBuilder,... Con esto se proporciona un procesamiento avanzado de datos y una gestión eficaz de la interfaz del usuario.

% Debemos de establecer un mecanismo que nos permite establecer una comunicación entre el lenguaje anfitrión y el de datos. Esto se conoce como \textit{acoplamiento}. Distinguimos dos categorías.
% \begin{itemize}
%     \item Débilmente acoplados. Lenguajes de propósito general y en este caso el programador puede distinguir entre sentencias del lenguaje anfitrión y las dispuestas por la propia BD.
%     \item Fuertemente acoplados. Lenguajes y herramientas de propósito específico. Se parte del DSL como elemento central y se le van incorporando características para facilitar el desarrollo de apliaciones.
% \end{itemize}

% Hay varias alternativas para implementar el acoplamiento débil:

% \begin{itemize}
%     \item \textbf{Usar APIs de acceso a la BD}. Acceder a la BD desde el código fuente del lenguaje anfitrión.
%     \item \textbf{DSL inmerso en el código fuente del lenguaje anfitrión.} Se escribe código híbrido.
% \end{itemize}

% Pasa lo mismo con el acoplamiento fuerte:

% Las propuestas son ya propietarias, o bien la ejecución de Java sobre una máquina virtual que esta en el propio SGBD.

% También han aparecido numerosos entornos de desarrollo que son específicos para las aplicaciones de gestión.


\section{Relación de Ejercicios T2}

\begin{enumerate}
    \item Explicar la relación existente entre los niveles de una base de datos y el concepto de independencia.
    

    La relación entre los niveles de una base de datos y el concepto de independencia radica en la estructura de tres niveles propuesta por el modelo ANSI/SPARC, que busca separar la forma en que los datos se almacenan, representan y acceden. Estos niveles son:
    \begin{itemize}
        \item \textbf{Nivel interno (físico)}: Describe cómo se almacenan los datos físicamente en el sistema de almacenamiento.
        \item \textbf{Nivel conceptual}: Proporciona una vista abstracta de la base de datos, definiendo la estructura de los datos y sus relaciones sin preocuparse por los detalles de almacenamiento.
        \item \textbf{Nivel externo (vista de usuario)}: Presenta a los usuarios una versión personalizada de los datos, mostrando solo la información relevante para ellos.
    \end{itemize}

    \textbf{Independencia de los datos}

    Este concepto se divide en dos tipos:
    \begin{itemize}
        \item \textbf{Independencia lógica de los datos}: Se refiere a la capacidad de modificar el esquema conceptual sin afectar los esquemas externos. Por ejemplo, se pueden agregar nuevas entidades o atributos sin necesidad de cambiar la forma en que los usuarios ven los datos.
        \item \textbf{Independencia física de los datos}: Permite cambiar la forma en que los datos están almacenados (como cambiar de disco duro o modificar estructuras de almacenamiento) sin afectar la forma en que los datos son representados en los niveles superiores.
    \end{itemize}

    \textit{Podemos concluir con que la arquitectura de tres niveles facilita la independencia de datos, permitiendo modificar estructuras internas o lógicas sin afectar la interacción del usuario con la base de datos.}

    \item Explicar la diferencia entre esquema externo y aplicaciones de usuario.
    

    La diferencia entre el esquema externo y las aplicaciones de usuario radica en su propósito y función dentro del sistema de bases de datos.

    \subsubsection*{Esquema Externo}

    El esquema externo representa la vista de los datos desde la perspectiva de los usuarios. Cada usuario o grupo de usuarios puede tener un esquema externo diferente, dependiendo de sus necesidades y permisos. Se define a partir del esquema conceptual y no afecta la estructura interna de la base de datos.

    Por ejemplo, un \textit{cajero} en un banco solo puede ver los saldos y transacciones de clientes, mientras que un gerente puede acceder a reportes financieros más detallados.

    \subsubsection*{Aplicaciones de Usuario}

    Las aplicaciones de usuario son los programas o interfaces que interactúan con la base de datos para realizar operaciones como consultas, inserciones, modificaciones o eliminaciones de datos. Estas aplicaciones pueden utilizar SQL, interfaces gráficas o APIs para comunicarse con la base de datos y presentar los datos según el esquema externo definido para el usuario.

    \subsubsection*{Diferencia Clave}

    \begin{itemize}
        \item El esquema externo es una estructura lógica que define qué datos puede ver un usuario y cómo están organizados.
        \item Las aplicaciones de usuario son herramientas o programas que permiten interactuar con la base de datos utilizando el esquema externo para presentar la información.
    \end{itemize}

    \textit{El esquema externo define la visión restringida de los datos para cada usuario, mientras que las aplicaciones de usuario son los medios mediante los cuales se accede y manipulan esos datos.}

    \item Explica el motivo por el que, a tu juicio, no se han desarrollado DDLs a nivel interno.
    

    No se han desarrollado DDLs a nivel interno porque la gestión del almacenamiento físico \textit{es responsabilidad del SGBD}. Esto garantiza eficiencia, independencia de datos, seguridad y portabilidad, evitando que los usuarios interfieran con la optimización del sistema.

    \item Explica el motivo por el que, a tu juicio, no se han desarrollado DMLs a nivel externo.
    
    No se han desarrollado DDLs a nivel interno porque el SGBD gestiona automáticamente el almacenamiento físico para optimizar rendimiento, garantizar independencia de datos y evitar configuraciones ineficientes por parte del usuario.

    \item Buscar tres ejemplos de lenguajes de cuarta generación. Indicar sus objetivos o funciones.
    \item ¿Cuál es el enfoque actual del concepto de lenguaje anfitrión? Dar ejemplos de lenguajes anfitrión.
    
    El lenguaje anfitrión es un lenguaje de programación que integra comandos SQL para interactuar con una base de datos. Actualmente, el enfoque se centra en la integración fluida con los SGBD y la seguridad en la ejecución de consultas.

    Ejemplos: Java (JDBC), Python (SQLAlchemy), PHP (PDO), C\# (Entity Framework).

    \item ¿Qué elementos conciernen al nivel interno de una base de datos?
    
    Incluye estructuras físicas de almacenamiento, organización de archivos, índices, métodos de acceso y estrategias de optimización para el almacenamiento y recuperación de datos.

    \item ¿Qué cuestiones debe cubrir a tu juicio una buena herramienta de gestión privilegios de usuarios?
    
    \begin{enumerate}[label=\alph*)]
        \item Control de acceso (roles y permisos).
        \item Control y registro de actividad.
        \item Administración centralizada.
        \item Compatibilidad con autenticación externa.
        \item Facilidad de uso y escalabilidad.
    \end{enumerate}

    \item Explicar las ventajas de la arquitectura cliente/servidor a tres niveles.
    
    \begin{enumerate}[label=\alph*)]
        \item Escalabilidad: Mejora el rendimiento al distribuir la carga.
        \item Seguridad: Separa la lógica de negocio y el acceso a datos.
        \item Mantenimiento fácil: Cambios en una capa no afectan las demás.
        \item Optimización de recursos: Reduce la dependencia del cliente en el procesamiento.
    \end{enumerate}

\end{enumerate}