\section{Relación de Ejercicios T1}

\begin{enumerate}
    \item ¿Cuáles son las principales diferencias entre un sistema de procesamiento de archivos y un sistema de bases de datos?
    \\\\
    Por un lado, en un sistema de procesammiento de archivos los datos se almacenan en archivos y para trabajar con ellos se utilizan programas específicos. Esto hace que sean desordenados, con repeticiones causando problemas de organización. Además, cabe destacar que la seguridad es limitada. 

    Por otro lado, en un sistema de bases de datos, los datos se estructuran en tablas relacionadas entre sí. Para gestionar y manipular estos datos, se utiliza un lenguaje de consulta estructurado conocido como SQL, que facilita su manejo de manera eficiente y segura. Además, las bases de datos ofrecen mejores mecanismos de seguridad y son altamente escalables, permitiendo gestionar grandes volúmenes de datos y múltiples usuarios simultáneamente sin inconvenientes.


    \item Describe las características más importantes en un sistema de base de datos y también las propiedades más deseables. Explica a tu juicio cuál es la propiedad más importante.
    
    Características y propiedades más importantes de un SGBD:

    \begin{enumerate}[label=\alph*)]
        \item Estructura y organización: los datos estan estructurados de manera eficiente, lo que permite un acceso rápido y fácil a la información.
        \item Integridad: los datos deben ser precisos y consistentes, evitando errores y duplicaciones.
        \item Seguridad: el acceso a los datos debe estar controlado y protegido contra accesos no autorizados.
        \item Escalabilidad: el sistema debe ser capaz de manejar un aumento en la cantidad de datos y usuarios sin perder rendimiento.
        \item Confiabilidad: el sistema debe ser capaz de recuperarse de fallos y mantener la disponibilidad de los datos.
    \end{enumerate}

    Bajo mi punto de vista, la propiedad más importante es la seguridad ya que de esta manera se evitan accesos a la información no autorizados y otros problemas, como la pérdida de datos o la corrupción de la base de datos. La seguridad es fundamental para proteger la información sensible y garantizar la confianza de los usuarios en el sistema.

    \item Hemos conocido las ventajas de utilizar un sistema de bases de datos, ¿podrías comentar también algunos inconvenientes?
    
    \begin{itemize}
        \item \textbf{Costo:} La implementación y configuración de un sistema de bases de datos puede ser costosa, especialmente en proyectos grandes o en empresas con necesidades complejas. Además, el mantenimiento continuo del sistema también implica costos adicionales.
        \item \textbf{Complejidad:} Es necesario contar con personal capacitado con conocimientos técnicos especializados para gestionar y utilizar la base de datos de manera eficiente. Asimismo, la organización y estructuración de grandes volúmenes de datos o relaciones complejas entre entidades puede ser un desafío.
        \item \textbf{Seguridad:} Las bases de datos pueden ser vulnerables a ataques de hackers interesados en acceder a información sensible. Por ello, es fundamental implementar medidas de seguridad sólidas y contar con un plan de recuperación ante desastres, lo que puede aumentar la complejidad de la gestión.
    \end{itemize}

    \item Explica la diferencia entre independencia física e independencia lógica.
    
    La independencia física permite cambiar cómo se almacenan los datos sin afectar su uso lógico, es decir, que si se cambia el hardware las aplicaciones no se ven afectadas, mientras que la independencia lógica permite modificar la organización y presentación de los datos sin alterar su almacenamiento físico. En esencia, la primera se centra en el almacenamiento y la segunda en la estructura lógica de cara al usuario.

    \item Definir brevemente los siguientes conceptos:
    \begin{itemize}
        \item Base de datos: Conjunto de datos relacionados y organizados de manera estructurada, que se almacenan y gestionan mediante un sistema de gestión de bases de datos (SGBD).
        \item DBMS (DataBase Management System): Programas para describir las estructuras y gestionar la información de BD.
        \item DBA (DataBase Administrator): Persona responsable de la administración y gestión de una base de datos, asegurando su rendimiento, seguridad y disponibilidad.
        \item Acceso concurrente: Capacidad de múltiples usuarios para acceder y manipular datos en una base de datos al mismo tiempo, garantizando la integridad y consistencia de la información.
        \item Vista de usuario: Representación personalizada de los datos en una base de datos, que permite a los usuarios interactuar con la información de manera específica y adaptada a sus necesidades, sin necesidad de conocer la estructura interna de la base de datos.
    \end{itemize}
    \item Explicar brevemente los conceptos de: Integridad, fiabilidad y seguridad en una base de datos.
        \begin{itemize}
            \item Ordenarlos por importancia, explicando los criterios utilizados para elaborar el orden.
            
            \begin{enumerate}
                \item Seguridad: Es fundamental proteger la información sensible y garantizar que solo los usuarios autorizados tengan acceso a los datos. Sin seguridad, la integridad y fiabilidad de los datos pueden verse comprometidas.
                \item Integridad: La precisión y consistencia de los datos son esenciales para garantizar que la información sea útil y confiable. Si los datos no son íntegros, las decisiones basadas en ellos pueden ser erróneas.
                \item Fiabilidad: Aunque la fiabilidad es importante, si los datos son seguros e íntegros, la fiabilidad se convierte en un aspecto secundario. La fiabilidad se basa en la seguridad y la integridad de los datos, por lo que su importancia es menor en comparación con los otros dos aspectos.
            \end{enumerate}

            \item ¿En qué etapa de la vida de una base de datos se deben tener en cuenta unos y otros?
            
            \begin{itemize}
                \item Seguridad: Desde el inicio del diseño de la base de datos, es fundamental establecer políticas y medidas de seguridad para proteger la información.
                \item Integridad: Durante la fase de diseño y modelado de la base de datos, se deben definir reglas y restricciones para garantizar la integridad de los datos.
                \item Fiabilidad: A lo largo de toda la vida útil de la base de datos, es importante mantener un sistema confiable mediante copias de seguridad y recuperación ante desastres.
            \end{itemize}

            \item ¿Cómo se mantienen en una base de datos? 
            \begin{itemize}
                \item Seguridad: Se implementan controles de acceso, autenticación y autorización, así como cifrado de datos y auditorías de seguridad.
                \item Integridad: Se utilizan restricciones de integridad, como claves primarias, foráneas y reglas de validación, para asegurar la calidad de los datos.
                \item Fiabilidad: Se realizan copias de seguridad periódicas, pruebas de recuperación y mantenimiento preventivo del sistema para garantizar su disponibilidad y rendimiento.
            \end{itemize}
        \end{itemize}
\end{enumerate}
