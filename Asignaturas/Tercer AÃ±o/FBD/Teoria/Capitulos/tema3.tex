\section{Definición}

Previamente, debemos de llevar a cabo un análisis de los datos y de esta manera obtenemos el esquema conceptual y lógico de la BD. En este tema entre la siguiente fase de \textit{Diseño}, dando como resultado el modelo lógico. Este se trata de uno ya implementado, por eso se dice que es un modelo implementable. 

En cuanto al proceso de transformación, distinguimos:

\begin{enumerate}
    \item Mundo real 
    \item Datos operativos
    \item Esquema conceptual
\end{enumerate}

En esta etapa se introduce lo que conocemos como \textit{tablas}. En la última fase llevamos a cabo la implementación de esa tabla mediante un lenguaje.

Definición: Mecanismo formal para representar y manipular la información con la que voy a trabajar. Debe de constar de datos, operaciones y reglas de integridad.

La necesidad de usar el modelo de datos son:
\begin{itemize}
    \item Se usan lenguajes de definición de datos.
    \item Es de muy bajo nivel.
    \item Se necesitan niveles superior para su abstracción.
\end{itemize}

El objetivo que se tiene es establecer que representen los datos y que los describan de una forma entendible y manipulable. Según ANSI distinguimos tres niveles:
\begin{itemize}
    \item Externo
    \item Conceptual
    \item Interno
\end{itemize}

En cuanto a la clasificación:
\begin{itemize}
    \item Basados en registros.
    \item Basados en objetos.
    \item Físico.
\end{itemize}

Los dos primeros son de nivel externo y conceptual, mientras que el último es de nivel interno.

\section{Modelo Relacional}

La información se organiza en tablas: 

\begin{enumerate}
    \item La estructura de almacenamiento son las tablas.
    \item Integridad.
    \item Consulta y Manipulación.
\end{enumerate}

Esta tabla es a nivel lógico, en cuanto al nivel físico depende de varias estructuras, como son las listas enlazadas,...

Elementos a conocer:
\begin{itemize}
    \item Atributo.
    \item Dominio: rango de valores que puede tomar (enteros, letras,...)
    \item Relación: Se conoce como cualquier subconjunto del producto cartesiano $D_1 \times D_2 \times ... D_n$. En este caso a relación se refiere a la tabla. Se denota como $R(A_1..A_n)$. De esta manera conseguimos todos los valores que pueden tomar y de aquí sacamos los datos (solo a nivel teórico).
    \item Tupla: filas de la tabla.
    \item Cardinalidad de una relación: número de tuplas.
    \item Esquema de una relación R: Atributos de la relación junto a su dominio.
    \item Grado de una relación: Número de atributos. Aunque se supone que es invariable, esto depende ya que se puede modificar.
    \item Instancia de una relación: datos que tengo en un determinado momento.
    \item Esquema de la BD: colección de esquemas de relaciones junto con las restricciones de integridad.
    \item Instancia o estado de una BD: colección de instancias de relaciones que verifican las condiciones de integridad.
    \item BD relaciones: instancia de la BD junto con su esquema.
\end{itemize}
