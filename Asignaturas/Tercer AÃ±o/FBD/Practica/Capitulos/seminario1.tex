% \begin{center}
%     \begin{tikzpicture}[
%         entity/.style={rectangle, draw, minimum width=3cm, minimum height=1cm},
%         relationship/.style={diamond, draw, minimum width=2cm, minimum height=1cm},
%         attribute/.style={ellipse, draw},
%         line/.style={draw, -latex},
%         singleline/.style={draw, -},
%         doubleline/.style={draw, double distance=2pt, -}
%     ]
%         % Entidades
%         \node[entity] (libro) {Libro};
%         \node[entity, right=4cm of libro] (autor) {Autor};
%         \node[entity, below=4cm of libro] (tema) {Tema};
%         \node[entity, below=4cm of autor] (usuario) {Usuario};

%         % Relaciones
%         \node[relationship, above right=1cm and 1cm of libro] (escribe) {Escribe};
%         \node[relationship, below=2cm of libro] (trata) {Trata de};
%         \node[relationship, below=2cm of escribe] (prestamo) {Préstamo};

%         % Atributos
%         \node[attribute, above left=1cm and 1cm of libro] (atributos_libro) {ISBN, Título};
%         \node[attribute, below left=1cm and 1cm of autor] (dni_autor) {DNI};
%         \node[attribute, below right=1cm and 1cm of autor] (domicilio_autor) {Domicilio};
%         \node[attribute, below left=1cm and 1cm of usuario] (dni_usuario) {DNI};
%         \node[attribute, below right=1cm and 1cm of usuario] (domicilio_usuario) {Domicilio};
%         \node[attribute, below=1cm of tema] (nombre_tema) {Nombre};

%         % Conexiones
%         \draw[doubleline] (libro) -- (escribe);
%         \draw[line] (escribe) -- (autor);
%         \draw[line] (libro) -- (trata);
%         \draw[line] (trata) -- (tema);
%         \draw[line] (libro) -- (prestamo);
%         \draw[line] (prestamo) -- (usuario);

%         \draw[line] (libro) -- (atributos_libro);
%         \draw[line] (autor) -- (dni_autor);
%         \draw[line] (autor) -- (domicilio_autor);
%         \draw[line] (usuario) -- (dni_usuario);
%         \draw[line] (usuario) -- (domicilio_usuario);
%         \draw[line] (tema) -- (nombre_tema);
%     \end{tikzpicture}
% \end{center}

\subsection*{Ejercicio 1}
\begin{figure}[H]
    \begin{center}
        \begin{tikzpicture}[
            entity/.style={rectangle, draw, minimum width=3cm, minimum height=1cm},
            relationship/.style={diamond, draw, minimum width=2cm, minimum height=1cm},
            attribute/.style={ellipse, draw},
            line/.style={draw, -latex},
            singleline/.style={draw, -},
            doubleline/.style={draw, double distance=2pt, -},
        ]
            % Entidades
            \node[entity] (tarjeta) {Tarjeta};
            \node[entity, right=4cm of tarjeta] (cc) {C.C};
            \node[entity, below=4cm of tarjeta] (persona) {Persona};

            % Relaciones
            \node[relationship, above right=1cm and 1cm of tarjeta] (asociada) {Asociada};
            \node[relationship, below right=1cm and 1cm of cc] (vinculada) {Vinculada};
            \node[relationship, below left=1cm and 1cm of tarjeta] (titular) {Titular};

            % Atributos
            \node[attribute, above left=1cm and 1cm of tarjeta] (num_tarjeta) {\textbf{Nº}};
            \node[attribute, left=1cm of tarjeta] (tipo_tarjeta) {Tipo};
            \node[attribute, above right=1cm and 1cm of cc] (codigo_cc) {\b{Código}};
            \node[attribute, right=1cm of cc] (saldo_cc) {Saldo};
            \node[attribute, below left=1cm and 0.3cm of cc] (fecha_apertura_cc) {Fecha Apertura};
            \node[attribute, below left=1cm and 1cm of persona] (dni_persona) {\b{DNI}};
            \node[attribute, below=1cm of persona] (domicilio_persona) {Domicilio};
            \node[attribute, below right=1cm and 1cm of persona] (telefono_persona) {Teléfono};

            % Conexiones
            \draw[doubleline] (tarjeta) -- (asociada);
            \draw[line] (asociada) -- (cc);
            \draw[singleline] (cc) -- (vinculada);
            \draw[singleline] (vinculada) -- (persona);
            \draw[doubleline] (tarjeta) -- (titular);
            \draw[line] (titular) -- (persona);

            \draw[line, -*] (tarjeta) -- (num_tarjeta);  % Línea con círculo relleno al final
            \draw[line, -o] (tarjeta) -- (tipo_tarjeta); % Línea con círculo vacío al final
            \draw[line, -*] (cc) -- (codigo_cc);
            \draw[line, -o] (cc) -- (saldo_cc);
            \draw[line, -o] (cc) -- (fecha_apertura_cc);
            \draw[line, -*] (persona) -- (dni_persona);
            \draw[line, -o] (persona) -- (domicilio_persona);
            \draw[line,-o] (persona) -- (telefono_persona);
        \end{tikzpicture}
    \end{center}
    \caption{Diagrama Entidad-Relación del Ejercicio 1}
\end{figure}
\newpage
\foreach \n in {8,9,10,11,12,13,14,15,16,17,18,19,20,21,22,23,24,25,26} { 
    \begin{figure}[H]
        \centering
        \fbox{\includegraphics[width=\textwidth,height=\textheight,keepaspectratio]{Capitulos/Images_T1/EjerciciosP\n.png}}
        \caption{Relación de Ejercicios formato libreta página \the\numexpr\n-7\relax.}
        %\label{fig:ejercicio\n}
    \end{figure}
    \clearpage % Salto de página después de cada imagen
}

