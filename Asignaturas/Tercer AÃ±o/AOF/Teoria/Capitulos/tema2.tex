\section{Ejercicios}

\subsection*{Ejercicio 7 }

El Tesoro Público recibió las siguientes peticiones competitivas en una subasta de obligaciones del Estado a 10 años, además de 100 millones correspondientes a peticiones no competitivas.

\begin{table}[H]
\centering
\begin{tabular}{|c|c|}
\hline
Nominal (millones €) & Precio solicitado \\ \hline
100                  & 113,975           \\ \hline
200                  & 112,885           \\ \hline
250                  & 111,275           \\ \hline
450                  & 110,000           \\ \hline
400                  & 109,375           \\ \hline
\end{tabular}
\caption{Peticiones competitivas recibidas}
\end{table}

Estas obligaciones, que pagarán un cupón anual del 5,5\%, se emitieron el 30/03/2010 y se amortizarán el 30/03/2020.

\begin{enumerate}[label=\alph*)]
    \item Resuelve la subasta sabiendo que el Tesoro adjudicó un total de 800 millones.\\
    Solución: Precio medio 111,848; Precio marginal 110.

    % Adjudicación total: 800 millones = 100 + 200 + 250 + 450 + 400.\\
    % Tipo medio ponderado: $ 100 \times \frac{113,975}{100} + 200 \times \frac{112,885}{100} + 250 \times \frac{111,275}{100} + 450 \times \frac{110,000}{100} + 400 \times \frac{109,375}{100} = 1,1075$\\
    % Precio medio: $\frac{1000}{1}$.\\
    % Precio marginal: 110.

    \item Plantea la ecuación que verifica el tipo de interés marginal resultante de la subasta.\\
    Solución: 4,2516\%.

    \item Si un inversor participó en la subasta solicitando obligaciones a 112,885 y decide venderlas hoy, 09/04/2013, cuando cotizan al 101,245\%, plantea la ecuación que verifica la rentabilidad efectiva obtenida con ellas sabiendo que el intermediario le cobra una comisión en la operación de venta del 0,3\% sobre el nominal.\\
    Solución: 1,7529\%.

    \item Plantea la ecuación que verifica la rentabilidad que exige el mercado hoy a estas obligaciones.\\
    Solución: 5,2772\%.
\end{enumerate}

