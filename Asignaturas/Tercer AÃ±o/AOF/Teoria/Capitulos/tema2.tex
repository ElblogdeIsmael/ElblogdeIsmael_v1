\section{Ejercicios}

\subsection*{Ejercicio 1 }

\begin{enumerate}
    \item En el Boletín de Deuda Pública del día 23 de marzo de 2012, en su sección de operaciones de compraventa simple al contado sobre Deuda del Estado, podemos encontrar la siguiente información:
    
    \begin{table}[h!]
        \centering
        \begin{tabular}{|c|c|c|c|c|}
            \hline
            \textbf{EMISIÓN} & \textbf{Cupón} & \textbf{Amortización} & \textbf{Precio medio ex cupón} & \textbf{TIR} \\
            \hline
            ES00000123B9 O EST & 5.50 & 30.04.21 & 101,105 & 5,34 \\
            \hline
        \end{tabular}
    \end{table}

    \begin{tikzpicture}
        % Línea de tiempo
         \draw[-] (0,0) -- (12,0) node[right] {Tiempo};
    
        % Emisión inicial
        \draw[-{Latex}] (0,0) -- (0,-1) node[below right, yshift=-0.5cm, xshift=-1.5cm, align=center, text width=3cm] {30/04/12\\Emisión\\1000};
    
        % Pagos periódicos
        \draw[-{Latex}] (2,0) -- (2,1) node[above, yshift=0.4cm, align=center, text width=3cm] {30/04/14};
        \draw[-{Latex}] (4,0) -- (4,1) node[above, yshift=0.4cm, align=center, text width=3cm] {30/04/15};
        \draw[-{Latex}] (6,0) -- (6,1) node[above, yshift=0.4cm, align=center, text width=3cm] {...};
        \draw[-{Latex}] (12,0) -- (12,1) node[above, yshift=0.4cm, align=center, text width=3cm] {30/04/21\\};
    
        % % Emisión final
        % \draw[-{Latex}] (11.8,0) -- (11.8,-1) node[below right, yshift=-0.5cm, xshift=-1.5cm, align=center, text width=3cm] {30/04/21\\};
    
    
    
        % Conversión
        % \draw[-{Latex},red] (3,0) -- (3,-1) node[below left, yshift=-0.8cm, xshift=-0.3cm, align=center, text width=3cm] {31/3/06\\Conversión};
    
        % % Venta
        % \draw[-{Latex},blue] (3.5,0) -- (3.5,1) node[above, yshift=0.8cm, align=center, text width=3cm] {30/4/06\\Venta};
    
        % % Amortización final
        % \draw[-{Latex}] (10,0) -- (10,-1) node[below right, yshift=-0.8cm, xshift=0.3cm, align=center, text width=3cm] {31/12/09\\Amortización\\1020};
    
    \end{tikzpicture}
    
    \begin{enumerate}
        \item[a)] Si se supone que se ha comprado esta obligación por el precio medio, ¿Cuánto se ha pagado por ella? \textbf{Sol: 1.060,34€}
        
        \begin{itemize}
            \item Del 30/04/11 al 23/03/12 hay 328 días.
            \item Del 23/03/12 al 30/04/12 hay 38 días.
            \item El total es de 366 días.
        \end{itemize}

        \begin{equation*}
            P_{\text{total}} = 101,105 \times 1000 + \frac{55}{366} \times 328 = 1.060,34
        \end{equation*}
        \item[b)] Plantea la ecuación que verifica la TIR con la que se está contratando esta obligación y calcula su valor. \textbf{Sol: 5,34\%}
        \begin{equation*}
            1060,34 = \left[55 \times a_{10,TIR}+\frac{1000}{(1+TIR)^{10}}\right]\times(1+TIR)^{\frac{328}{366}}
        \end{equation*}
    \end{enumerate}
\end{enumerate}

\textit{Debemos de hacer ciertas suposiciones como es el caso de que al amortizarse en 30.04.21, y estamos a 23.03.12, el comienzo de la vida de la obligación es el 30.04.12.}
\\\\

% \begin{tikzpicture}
%     % Línea de tiempo
%      \draw[-] (0,0) -- (12,0) node[right] {Tiempo};

%     % Emisión inicial
%     \draw[-{Latex}] (0,0) -- (0,-1) node[below right, yshift=-0.5cm, xshift=-1.5cm, align=center, text width=3cm] {30/04/12\\Emisión\\1000};

%     % Pagos periódicos
%     \draw[-{Latex}] (2,0) -- (2,1) node[above, yshift=0.4cm, align=center, text width=3cm] {30/04/14};
%     \draw[-{Latex}] (4,0) -- (4,1) node[above, yshift=0.4cm, align=center, text width=3cm] {30/04/15};
%     \draw[-{Latex}] (6,0) -- (6,1) node[above, yshift=0.4cm, align=center, text width=3cm] {...};
%     \draw[-{Latex}] (12,0) -- (12,1) node[above, yshift=0.4cm, align=center, text width=3cm] {30/04/21\\};

%     % % Emisión final
%     % \draw[-{Latex}] (11.8,0) -- (11.8,-1) node[below right, yshift=-0.5cm, xshift=-1.5cm, align=center, text width=3cm] {30/04/21\\};



%     % Conversión
%     % \draw[-{Latex},red] (3,0) -- (3,-1) node[below left, yshift=-0.8cm, xshift=-0.3cm, align=center, text width=3cm] {31/3/06\\Conversión};

%     % % Venta
%     % \draw[-{Latex},blue] (3.5,0) -- (3.5,1) node[above, yshift=0.8cm, align=center, text width=3cm] {30/4/06\\Venta};

%     % % Amortización final
%     % \draw[-{Latex}] (10,0) -- (10,-1) node[below right, yshift=-0.8cm, xshift=0.3cm, align=center, text width=3cm] {31/12/09\\Amortización\\1020};

% \end{tikzpicture}

% \begin{itemize}
%     \item Del 30/04/11 al 23/03/12 hay 328 días.
%     \item Del 23/03/12 al 30/04/12 hay 38 días.
%     \item El total es de 366 días.
% \end{itemize}
% Así que calculando el precio total con la fórmula del precio excupón tenemos que:
% \begin{itemize}
%     \item [a)] \begin{equation*}
%         P_{\text{total}} = 101,105 \times 1000 + \frac{55}{366} \times 328 = 1.060,34
%     \end{equation*}


%     \item [b)] \begin{equation*}
%         1060,34 = \left[55 \times a_{10,TIR}+\frac{1000}{(1+TIR)^{10}}\right]\times(1+TIR)^{\frac{328}{366}}
%     \end{equation*}
% \end{itemize}


\subsection*{Ejercicio 2}

\begin{enumerate}
    \item La sociedad ILIGRASA emitió, el 1 de enero de 2011, obligaciones con:
    \begin{itemize}
        \item Valor nominal de 3.000€.
        \item Cupón al 5\% nominal anual pagadero por semestres (30 de junio y 31 de diciembre de cada año).
        \item Vencimiento a 10 años.
    \end{itemize}
    Los títulos se emitieron a la par sin gastos para el suscriptor. Hoy, 21 de junio de 2013, estos títulos cotizan en el mercado secundario al 108\% excupón.

    \begin{enumerate}
        \item[a)] Plantea la ecuación que verifica la rentabilidad que el mercado exige hoy a estos títulos y calcula su valor. \textbf{Sol: 3,7997\%.}
        Como primer paso debemos de calcular el $C_s$, el cual nos queda $C_s = \frac{5\%}{2} \times 3000 = 75$\footnote{Se divide entre 2 porque el pago es semestral y me dan el tipo de interés anual.}. 

        A continuación, caculamos el precio total, el cual nos queda:
        \begin{equation*}
            {P_T}_{\textit{21.06.13}} = 108\% \times 3000 + \frac{75}{181} \times 172 = 3311,27
        \end{equation*}

        Por lo que teniendo en cuenta que los días que han pasado desde el último pago son 172, podemos plantear la ecuación de la rentabilidad que el mercado exige hoy a estos títulos:

        \begin{equation*}
            3311,27 = \left[75 + a_{16,TIR_semestral} + \frac{3000}{(1+TIR)^{16}}\right] \times (1+TIR)^{\frac{172}{181}}
        \end{equation*}
        \item[b)] Si un inversor compró 15 títulos en la emisión y los vende hoy a través de un intermediario, cobrándole éste una comisión del 0,3\% sobre el valor efectivo de la venta, ¿qué rentabilidad efectiva ha obtenido con ellos? \textbf{Sol: 8,0361\%.}\\\\
        Se emitió a la par, por lo que es igual a el VN.
        El precio de venta es $ P_{venta} = 3311,27 -0,3 \times 3311,27 = 3301,34$

        La rentabilidad efectiva es:
        \begin{equation*}
            3000 = 75 \times a_{4,TIR} + \frac{3301,34}{1+TIR}^{\alpha}
        \end{equation*}

        Donde $\alpha = \frac{4s+1ts = 172\text{días}+4s}{181}$. Donde denotamos s como semestre y ts como un trozo del simestre.
    \end{enumerate}
\end{enumerate}





\subsection*{Ejercicio 7 }

El Tesoro Público recibió las siguientes peticiones competitivas en una subasta de obligaciones del Estado a 10 años, además de 100 millones correspondientes a peticiones no competitivas.

\begin{table}[H]
\centering
\begin{tabular}{|c|c|}
\hline
Nominal (millones €) & Precio solicitado \\ \hline
100                  & 113,975           \\ \hline
200                  & 112,885           \\ \hline
250                  & 111,275           \\ \hline
450                  & 110,000           \\ \hline
400                  & 109,375           \\ \hline
\end{tabular}
\caption{Peticiones competitivas recibidas}
\end{table}

Estas obligaciones, que pagarán un cupón anual del 5,5\%, se emitieron el 30/03/2010 y se amortizarán el 30/03/2020.

\begin{enumerate}[label=\alph*)]
    \item Resuelve la subasta sabiendo que el Tesoro adjudicó un total de 800 millones.\\
    Solución: Precio medio 111,848; Precio marginal 110.

    % Adjudicación total: 800 millones = 100 + 200 + 250 + 450 + 400.\\
    % Tipo medio ponderado: $ 100 \times \frac{113,975}{100} + 200 \times \frac{112,885}{100} + 250 \times \frac{111,275}{100} + 450 \times \frac{110,000}{100} + 400 \times \frac{109,375}{100} = 1,1075$\\
    % Precio medio: $\frac{1000}{1}$.\\
    % Precio marginal: 110.

    \item Plantea la ecuación que verifica el tipo de interés marginal resultante de la subasta.\\
    Solución: 4,2516\%.

    \item Si un inversor participó en la subasta solicitando obligaciones a 112,885 y decide venderlas hoy, 09/04/2013, cuando cotizan al 101,245\%, plantea la ecuación que verifica la rentabilidad efectiva obtenida con ellas sabiendo que el intermediario le cobra una comisión en la operación de venta del 0,3\% sobre el nominal.\\
    Solución: 1,7529\%.

    \item Plantea la ecuación que verifica la rentabilidad que exige el mercado hoy a estas obligaciones.\\
    Solución: 5,2772\%.
\end{enumerate}

