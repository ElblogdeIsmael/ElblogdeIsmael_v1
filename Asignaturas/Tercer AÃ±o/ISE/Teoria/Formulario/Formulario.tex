\documentclass[a4paper,12pt]{article}
\usepackage[utf8]{inputenc}
\usepackage[spanish]{babel}
\usepackage{amsmath, amssymb}
\usepackage{tcolorbox}
\usepackage{geometry}
\usepackage{fancyhdr}
\usepackage{multicol}
\usepackage{graphicx}



% Márgenes
\geometry{left=2.5cm,right=2.5cm,top=2.5cm,bottom=2.5cm}

% Cabecera y pie de página
\pagestyle{fancy}
\fancyhf{}
\rhead{Formulario de ISE}
\lhead{Asignatura: Ingeniería de Servidores}
\rfoot{\thepage}
%\lfoot{Autor: Ismael Sallami Moreno}

\title{\textbf{Formulario de ISE}} 
\author{Ismael Sallami Moreno}
\date{\small Universidad de Granada\\ Doble Grado en Ingeniería Informática y ADE}

\begin{document}

\maketitle
\thispagestyle{fancy}

\vspace{1cm}

\section*{Optimización del Rendimiento}

\textit{Todas las variables operacionales deducidas que se usan en este apartado son valores medios.}

\begin{itemize}
    \item W: waiting time, tiempo de espera en la cola.
    \item S: service time, tiempo de servicio.
    \item R: response time, tiempo de respuesta.
\end{itemize}

\begin{tcolorbox}[colback=yellow!5!white, colframe=yellow!75!black]
    \begin{align*}
        R &= W + S \\
    \end{align*}
\end{tcolorbox}

\textbf{Variables y leyes operacionales:}

\begin{itemize}
    \item $N_0$: número de trabajos en el servidor.
    \item $N_z$: número de clientes en reflexión (esperando a que los clientes vuelvan a introducirlos en el servidor).
    \item $T$: duración del periodo de media para el que se extrae el modelo.
    \item $A_i$: número de trabajos solicitados a la estación (\textbf{arrivals}).
    \item $B_i$: tiempo que el dispositivo ha estado en uso (\textbf{busy}).
    \item $C_i$: número de trabajos completados en el periodo (\textbf{completed}).
    \item $S_i$: tiempo medio de servicio (\textbf{service}). Se mide en $\frac{segundos}{trabajo}$ o bien en segundos.
    \item $W_i$: tiempo medio de espera en la cola (\textbf{waiting}). Se mide en segundos [/trabajo].
    \item $R_i$: tiempo medio de respuesta (\textbf{response}). Se mide en segundos [/trabajo].


\end{itemize}

\begin{tcolorbox}[colback=yellow!5!white, colframe=yellow!75!black]
    \begin{align*}
        S_i = \frac{B_i}{C_i} \quad \quad \quad \quad R_i = W_i + S_i
    \end{align*}
\end{tcolorbox}

%\newpage
\begin{itemize}
    \item $\lambda_i$: tasa media de llegada (\textbf{arrival rate}). Unidades $\frac{trabajos}{segundos}$.
    \item $X_i$: Productividad media (\textbf{throughput}). Unidades $\frac{trabajos}{segundos}$.
    \item $U_i$: Utilización media (\textbf{utilization}). Unidades $\%$, pero no suele tener. $\text{Valor máx} = U_{i,max} = 1 \rightarrow 100 \%$
\end{itemize}

\begin{tcolorbox}[colback=yellow!5!white, colframe=yellow!75!black]
    \begin{align*}
        U_i = \frac{B_i}{T} \quad \quad \lambda_i = \frac{A_i}{T} \quad \quad S_i = \frac{B_i}{C_i} \quad \quad X_i = \frac{C_i}{T} \\
    \end{align*}
\end{tcolorbox}

\textbf{Haciendo referencia al número de trabajos en la estación de servicio:}
\begin{itemize}
    \item $N_i$: Número de trabajos en la estación de servicio.
    \item $Q_i$: Número medio de trabajos en la cola de espera.
    \item $U_i$: Número medio de trabajos siendo servidos por el dispositivo.
\end{itemize}

\begin{tcolorbox}[colback=yellow!5!white, colframe=yellow!75!black]
    \begin{align*}
        U_i = N_i - Q_i \quad \quad \text{Coincide númericamente con la Utilización Media}\\
    \end{align*}
\end{tcolorbox}

\textbf{Variables operacionales de un servidor:}


\begin{itemize}
    \item Básicas:
    \begin{itemize}
        \item $A_0$: número de trabajos solicitados al servidor.
        \item $C_0$: número de trabajos completados en el servidor.
    \end{itemize}
    \item Deducidas:
    \begin{itemize}
        \item $\lambda_0$: tasa media de llegada al servidor.
        \item $X_0$: Productividad media del servidor.
        \item $N_0$: Número medio de trabajos en el servidor.
        \item $R_0$: Tiempo medio de respuesta del servidor. 
    \end{itemize}
\end{itemize}
\begin{tcolorbox}[colback=yellow!5!white, colframe=yellow!75!black]
    \begin{align*}
        \lambda_0 = \frac{A_0}{T} \quad \quad X_i = \frac{C_0}{T} \\
    \end{align*}
\end{tcolorbox}

\newpage
\textbf{Razón de visita y demanda de servicio:}

\begin{itemize}
    \item Razón media de visita al servidor: $V_i$ (\textbf{visit ratio}).
    \item Demanda de servicio: $D_i$ (\textbf{service demand}).
\end{itemize}


\begin{tcolorbox}[colback=yellow!5!white, colframe=yellow!75!black]
    \begin{align*}
        V_i = \frac{C_i}{C_0} \quad \quad D_i = \frac{B_i}{C_0} = V_i \times S_i \\
    \end{align*}
\end{tcolorbox}

\textbf{Ley del fujo forzado}

\begin{tcolorbox}[colback=yellow!5!white, colframe=yellow!75!black]
    \begin{align*}
        \forall i = 1, \ldots, K \quad X_i = X_0 \times V_i = \lambda_0 \times V_i = \lambda_i \quad \text{Si hay equilibrio de flujo}\\  
    \end{align*}
\end{tcolorbox}

\textbf{Relación Utilización-demanda de servicio}

\begin{tcolorbox}[colback=yellow!5!white, colframe=yellow!75!black]
    \begin{align*}
        \forall i = 1, \ldots, K \quad U_i = X_0 \times D_i = \lambda_0 \times D_i \quad \text{Si hay equilibrio de flujo} \\
    \end{align*}
\end{tcolorbox}

\textbf{Ley de Little}

\begin{itemize}
    \item Aplicada a un servidor:
    
    \begin{tcolorbox}[colback=yellow!5!white, colframe=yellow!75!black]
        \begin{align*}
            N_0 = \lambda_0 \times R_0 = X_0 \times R_0 \\
        \end{align*}
    \end{tcolorbox}
    \item Aplicada a toda una estación de servicio:
    \begin{tcolorbox}[colback=yellow!5!white, colframe=yellow!75!black]
        \begin{align*}
            N_i = \lambda_i \times R_i = X_i \times R_i \\
        \end{align*}
    \end{tcolorbox}
    \item Aplicada a una cola de una estación de servicio:
    \begin{tcolorbox}[colback=yellow!5!white, colframe=yellow!75!black]
        \begin{align*}
            Q_i = \lambda_i \times W_i = X_i \times W_i \\
        \end{align*}
    \end{tcolorbox}
    %\textbf{Nota:} En equilibrio de flujos $X_i = \lambda_i$ (lo mismo se aplica a $X_0 = \lambda_0$).
\end{itemize}

\textbf{Ley general del tiempo de respuesta}

\begin{tcolorbox}[colback=yellow!5!white, colframe=yellow!75!black]
    \begin{align*}
        R_0 = \sum_{i=1}^{K} V_i \times R_i \\ 
    \end{align*}
\end{tcolorbox}


\textbf{Ley del tiempo de respuesta interactivo}

\begin{tcolorbox}[colback=yellow!5!white, colframe=yellow!75!black]
    \begin{align*}
        R_0 = \frac{N_T}{X_0} - Z \\
    \end{align*}
\end{tcolorbox}

\textbf{Identificación del cuello de botella}

\begin{itemize}
    \item \textit{b (bottleneck): índice del dispositivo cuello de botella}
\end{itemize}

\begin{tcolorbox}[colback=yellow!5!white, colframe=yellow!75!black]
    \begin{align*}
        D_b = \max_{i=1, \ldots, K} D_i = V_b \times S_b \\ 
        U_b = \max_{i=1, \ldots, K} U_i = X_0 \times D_b \\
    \end{align*}
\end{tcolorbox}

\textbf{Saturación del servidor}

\begin{itemize}
    \item El saturación, el cuello de botella está al máximo de su productividad.
\end{itemize}

\begin{tcolorbox}[colback=yellow!5!white, colframe=yellow!75!black]
    \begin{align*}
        1 = U_b = X_b \times S_b \Rightarrow X_b = \frac{1}{S_b} \\
    \end{align*}
\end{tcolorbox}

\textbf{Límites optimistas: redes abiertas}

\begin{tcolorbox}[colback=yellow!5!white, colframe=yellow!75!black]
    \begin{align*}
        R_0 \underset{\text{optimista = min}}{\Rightarrow} R_0^{min} = \sum_{i=1}^{K} V_i \times S_i = \sum_{i=1}^{K} D_i \equiv D \\ \\
        \text{Si } U_b = 1 \Rightarrow X_0^{max} = \frac{1}{D_b} \\ \\
        \text{Cuando } \lambda_0 \leq X_0^{max} \text{ estamos en equilibrio de flujo }\\
    \end{align*}
\end{tcolorbox}

\textbf{Límites optimistas: redes cerradas}
\begin{itemize}
    \item Valores de carga altos

\begin{tcolorbox}[colback=yellow!5!white, colframe=yellow!75!black]
    \begin{align*}
        \text{Cuando esta cerca de la saturación: } \text{Si } U_b = 1 \Rightarrow X_0^{max} = \frac{1}{D_b} \\ \\
        \text{Valor optimista de respuesta medio: } 
        R_0 = \left(\frac{N_T}{X_0^{max}}\right) - Z = D_b \times N_T - Z
    \end{align*}
\end{tcolorbox}

\item Valores de carga bajos
\begin{tcolorbox}[colback=yellow!5!white, colframe=yellow!75!black]
    \begin{align*}
        \text{Valor optimista de respuesta medio: } 
        R_0^{min} = \sum_{i=1}^{K} V_i \times S_i = \sum_{i=1}^{K} D_i \equiv D
        \\
        \text{Valor optimista de productividad media: }
        X_0 = \frac{N_T}{R_0^{min} + Z} = \frac{N_T}{D + Z} 
    \end{align*}
\end{tcolorbox}



\end{itemize}

\textbf{Punto teórico de saturación}
\begin{tcolorbox}[colback=yellow!5!white, colframe=yellow!75!black]
    \begin{align*}
        D = D_b \times N_T^* - Z \Rightarrow N_T^* = \frac{D + Z}{D_b}
    \end{align*}
\end{tcolorbox}


\end{document}
