\section{Ejercicio Propuestos}
\subsection{Ejercicio Propuesto 1}


La empresa BARCA, S.A. adquiere un bono el 01/04/20X3 con la intención de negociar los flujos de caja contractuales hasta abril de 20X6, fecha de amortización del bono. Las características del mismo son las siguientes:
\begin{itemize}
    \item Valor de emisión: 98.000 €.
    \item Comisión de adquisición: 2.000 €.
    \item Valor de reembolso: 104.000 €.
    \item Tipo de interés efectivo anual de interés sobre el valor nominal (VN = 100.000 €).
\end{itemize}

Su cuadro de amortización a interés efectivo (7,006\%) es el siguiente:
\begin{table}
\centering
\begin{tabular}{|p{2cm}|p{2cm}|p{2cm}|p{2cm}|p{2cm}|p{2cm}}
    \hline
    Plazo & Intereses devengados & I. Explícitos y reembolso & I. implícitos & Coste amortizado \\
    \hline
    01/04/20X3 & & & & 100.000,00 \\
    \hline
    01/04/20X4 & 6.865,88 & 5.000 & -1.865,88 & 99.865,88 \\
    \hline
    01/04/20X5 & 6.998,65 & 5.000 & -1.998,65 & 99.865,88 \\
    \hline
    01/04/20X6 & 7.136,48 & 109.000 & -2.136,48 & 0,00 \\
    \hline
    \end{tabular}
\end{table}


SE PIDE: Realizar los asientos contables en el libro diario de la sociedad BARCA S.A. relativos a las siguientes operaciones:

\begin{enumerate}[label=\alph*)] % Configura la numeración con "a)"
    \item Contabilización, el 01/04/20X3 de la compra del bono.
    \item A 31/12/20X3, si procede, contabilice el devengo de intereses de la operación del bono.
    \item A 01/04/20X4, contabilice el devengo de intereses de la operación del bono desde enero de 20X4.
    \item A 01/04/20X4, contabilice el cobro de los intereses de la operación del bono.
    \item A 31/12/20X4, si procede, contabilice el devengo de intereses de la operación del bono.
    \item A 01/04/20X5, contabilice el devengo de intereses de la operación del bono.
    \item A 01/04/20X6, contabilice el cobro del reembolso del bono.
\end{enumerate}


\begin{table}[H]
    \centering
    \begin{tabular}{|p{3cm}|p{6cm}|p{3cm}|}
    \hline
    \rowcolor{blue!30}
    \textbf{DEBE} & \textbf{Contabilización 01/04/2023 de la compra de bono} & \textbf{HABER} \\
    \hline
    98000& \cuenta{251} & \\
    \hline
    &  \cuenta{572}& 98000\\
    \hline
    \end{tabular}
\end{table}

\begin{table}[H]
    \centering
    \begin{tabular}{|p{3cm}|p{6cm}|p{3cm}|}
    \hline
    \rowcolor{blue!30}
    \textbf{DEBE} & \textbf{ A 31/12/2023, si procede, contabilice el devengo de los intereses} & \textbf{HABER}\\
    \hline
    3727,03& \cuenta{546} & \\
    \hline
    1378,55&  \cuenta{251}& \\
    \hline
    &  \cuenta{761}& 5105,58 = $98000 \times (1,07006)^{\frac{9}{12}} - 98000$\\
    \hline
    \end{tabular}
\end{table}

\begin{table}[H]
    \centering
    \begin{tabular}{|p{3cm}|p{6cm}|p{3cm}|}
    \hline
    \rowcolor{blue!30}
    \textbf{DEBE} & \textbf{01/04/2024 devengo de los intereses desde Enero 2020} & \textbf{HABER} \\
    \hline
    1272,96 $\rightarrow I=[(100000+3727,03)\times(1,05)^{\frac{3}{12}} - (100000+3727,03)]$& \cuenta{546} & \\
    \hline
    1272,96&  \cuenta{251}& \\
    \hline
    487,47& \cuenta{761} & 1760,23 $\rightarrow I = [(98000+5105,58)\times(1,07006)^{\frac{3}{12}}-(98000+5105,58)]$\\
    \hline
    \end{tabular}
\end{table}

\begin{table}[H]
    \centering
    \begin{tabular}{|p{3cm}|p{6cm}|p{3cm}|}
    \hline
    \rowcolor{blue!30}
    \textbf{DEBE} & \textbf{01/04/2024 cobro intereses de la operación del bono} & \textbf{HABER} \\
    \hline
    5000&  \cuenta{572}& \\
    \hline
    &  \cuenta{546}& 5000\\
    \hline
    \end{tabular}
\end{table}

\begin{table}[H]
    \centering
    \begin{tabular}{|p{3cm}|p{6cm}|p{3cm}|}
    \hline
    \rowcolor{blue!30}
    \textbf{DEBE} & \textbf{31/12/2024 devengo de intereses} & \textbf{HABER} \\
    \hline
    3723,03& \cuenta{546} & \\
    \hline
    1475,76&  \cuenta{251}& \\
    \hline
    & \cuenta{761} & 5202,79 $\rightarrow I = [(99865,88 \times 1,07006)^{\frac{9}{12}} - 99865,88]$\\
    \hline
    \end{tabular}
\end{table}

\begin{table}[H]
    \centering
    \begin{tabular}{|p{3cm}|p{6cm}|p{3cm}|}
    \hline
    \rowcolor{blue!30}
    \textbf{DEBE} & \textbf{04/04/26} & \textbf{HABER} \\
    \hline
    104000&  \cuenta{572}& \\
    \hline
    & \cuenta{541} & 104000\\
    \hline
    \end{tabular}
\end{table}




\subsection{Ejercicio Propuesto 2}
En el balance de situación de la empresa “TRAG, S.A.”, a 31 de diciembre del año 2018 figura una letra del tesoro, adquirida el 1 de julio de dicho año, por la que se pagaron 130.000 euros. Su vencimiento es el 30 de junio del año 2020. El nominal de la letra es de 150.000 euros. Se espera que mantener hasta vencimiento este instrumento. El tipo de interés efectivo anual por vencido es del 7,417231\%. La tabla que recoge el coste amortizado a este tipo de interés es la siguiente:

\begin{table}[H]
\centering
\begin{tabular}{|p{2cm}|p{2cm}|p{2cm}|p{2cm}|p{2cm}|p{2cm}|}
    \hline
    Fecha & Pago & Cobros & Intereses devengados (Gastos financieros) & Saldo amortizado & Saldo pendiente de amortizar (Balance) \\
    \hline
    01/07/2018 & 130.000 & 0 & 0 & 0 & 130.000 \\
    \hline
    31/12/2018 & 0 & 0 & 4.734.97 & -4.734.97 & 134.734.97 \\
    \hline
    31/12/2019 & 0 & 0 & 9.993.60 & -9.993.60 & 144.728.57 \\
    \hline
    01/07/2020 & 150.000 & 0 & 5.271.43 & 144.728.57 & 0 \\
    \hline
    \end{tabular}
\end{table}


\textbf{SE PIDE:} Realice las anotaciones contables relacionadas con el enunciado anterior.


\begin{table}[H]
    \centering
    \begin{tabular}{|p{3cm}|p{6cm}|p{3cm}|}
    \hline
    \rowcolor{blue!30}
    \textbf{DEBE} & \textbf{Contabilización el 01/07/2018 de la compra de la letra del tesoro} & \textbf{HABER} \\
    \hline
    130000& \cuenta{251} & \\
    \hline
    &  \cuenta{572}& 130000\\
    \hline
    \end{tabular}
\end{table}

\begin{table}[H]
    \centering
    \begin{tabular}{|p{3cm}|p{6cm}|p{3cm}|}
    \hline
    \rowcolor{blue!30}
    \textbf{DEBE} & \textbf{A 31/12/2018, si procede, contabilice el devengo de intereses} & \textbf{HABER} \\
    \hline
    4734,97&  \cuenta{251}& \\
    \hline
    &  \cuenta{761}& 4734,97\\
    \hline
    \end{tabular}
\end{table}

\begin{table}[H]
    \centering
    \begin{tabular}{|p{3cm}|p{6cm}|p{3cm}|}
    \hline
    \rowcolor{blue!30}
    \textbf{DEBE} & \textbf{A 31/12/2018, si procede, contabilice el cobro de intereses} & \textbf{HABER} \\
    \hline
    &  NPC& \\
    \hline
    \end{tabular}
\end{table}

\begin{table}[H]
    \centering
    \begin{tabular}{|p{3cm}|p{6cm}|p{3cm}|}
    \hline
    \rowcolor{blue!30}
    \textbf{DEBE} & \textbf{A 31/12/2019, si procede, contabilice el devengo de los intereses} & \textbf{HABER} \\
    \hline
    9993,60&  \cuenta{251}& \\
    \hline
    &  \cuenta{761}& 9993,60\\
    \hline
    \end{tabular}
\end{table}


\subsection{Ejercicio Propuesto 3}

La sociedad BENGALS adquiere, el 02/03/2022, 5.000 acciones de la sociedad COWBOYS por importe de 25 €/acción. La operación conlleva unos gastos de gestión que ascienden al 1\% del total de la operación. La inversión se realiza con un carácter de permanencia.

A cierre del ejercicio 2022, estos títulos cotizan a 30 € cada uno, y a cierre del ejercicio 2023 su cotización es de 28 €/acción.

Por necesidades de liquidez, el 23/05/2024 se venden la totalidad de las acciones por un importe de 32 €/acción, con unos costes de transacción del 2\% sobre el precio de venta.

\textbf{SE PIDE:} Contabilizar en el libro diario de BENGALS, S.A. las siguientes operaciones, sin tener en cuenta el posible efecto impositivo:

a) Compra de los títulos a 02/03/2022.

b) Si procede, operaciones derivadas de la valoración de los títulos a 31/12/2022.

c) Si procede, operaciones derivadas de la valoración de los títulos a 31/12/2023.

d) Todas las operaciones derivadas de la venta de los títulos a 23/05/2024.


\begin{table}[H]
    \centering
    \begin{tabular}{|p{3cm}|p{6cm}|p{3cm}|}
    \hline
    \rowcolor{blue!30}
    \textbf{DEBE} & \textbf{Compra de los títulos 02/03/2022} & \textbf{HABER} \\
    \hline
    126250&\cuenta{250}  & \\
    \hline
    &  \cuenta{572}& 126250 = $(5000 \times 25)(1+0,01)$ \\
    \hline
    \end{tabular}
\end{table}

\begin{table}[H]
    \centering
    \begin{tabular}{|p{3cm}|p{6cm}|p{3cm}|}
    \hline
    \rowcolor{blue!30}
    \textbf{DEBE} & \textbf{Si procede, operaciones derivadas de la valoración de los títulos a 31/12/2022} & \textbf{HABER} \\
    \hline
    23750 = $5000 \times 30 = 150000 - 126250$&  \cuenta{250}& \\
    \hline
    &  \cuenta{900} &23750 \\
    \hline
    23750& \cuenta{900} & \\
    \hline
    & \cuenta{133} & 23750\\
    \hline
    \end{tabular}
\end{table}

\begin{table}[H]
    \centering
    \begin{tabular}{|p{3cm}|p{6cm}|p{3cm}|}
    \hline
    \rowcolor{blue!30}
    \textbf{DEBE} & \textbf{Si procede, operaciones derivadas de la valoración de los títulos a 31/12/2023} & \textbf{HABER} \\
    \hline
    10000 &  \cuenta{800}& \\
    \hline
    &  \cuenta{250}& 10000 = $5000 \times 28 = 140000-150000=10000$\\
    \hline
    10000&  \cuenta{133}& \\
    \hline
    &  \cuenta{800}& 10000\\
    \hline
    \end{tabular}
\end{table}

\begin{table}[H]
    \centering
    \begin{tabular}{|p{3cm}|p{6cm}|p{3cm}|}
    \hline
    \rowcolor{blue!30}
    \textbf{DEBE} & \textbf{Todas las operaciones derivadas de la venta de los títulos a 23/05/2024} & \textbf{HABER} \\
    \hline
    156800 = $5000 \times 32 = 160000 - 2 \% \times 160000$&  \cuenta{572}& \\
    \hline
    &  \cuenta{250}& 140000 \\
    \hline
    &  \cuenta{766}& 16800\\
    \hline
    13750 = $23750 - 10000$&  \cuenta{133}& \\
    \hline
    &  \cuenta{802}& 13750\\
    \hline
    13750 &  \cuenta{802}& \\
    \hline
    &  \cuenta{7632}& 13750\\
    \hline
    \end{tabular}
\end{table}

\subsection{Ejercicio Propuesto 4}

La empresa EL FIJITIVO, S.L. compra el 01/11/X1, 8.000 acciones del banco MALO, S.A., que cotizan en bolsa a 15 €/acción. Los gastos derivados de la adquisición son 100 €. La inversión tiene carácter especulativo. A 31/12/X1, las acciones de MALO, S.A. cotizan a 13 €/acción y los costes de transacción previstos son de 200 €. El 01/03/X2 se venden las acciones a 12 €/acción con unos gastos de 150 €.

\textbf{SE PIDE:} Contabilice exclusivamente la compra y la venta de las acciones.

\textit{Nos da la pista de que son de carácter especulativo, por lo que debemos de introducir los cambios en la cuenta de Pérdidas y Ganacias}

\begin{table}[H]
    \centering
    \begin{tabular}{|p{3cm}|p{6cm}|p{3cm}|}
    \hline
    \rowcolor{blue!30}
    \textbf{DEBE} & \textbf{Contabilización de la compra de las acciones el 01/11/X1} & \textbf{HABER} \\
    \hline
      120.000 = $8000 \times 15$ &\cuenta{540}  & \\
    \hline
      100 & \cuenta{669} & \\
    \hline
    &  \cuenta{572}& 120.100\\
    \hline
    \end{tabular}
\end{table}


\subsubsection*{Anotaciones extra:}
\begin{itemize}
    \item Los gastos son de 100 €.
    \item El VR = $8000 \times 12 = 96000$ (no se tiene en cuenta cuanto cuesta venderlos).
    \item El VC = $8000 \times 15 = 120000$
    \item El VR = $8000 \times 13 = 104000 \rightarrow$ debemos de contabilizar una pérdida de 16000 = $120000 - 104000$, para anotarlo debemos de usar la cuenta del grupo 6 (Pérdidas de la cartera de la negociación). \textit{Lo recogemos directamente, por ende no debemos de dotar de deterioro previamente.}
\end{itemize}

\begin{table}[H]
    \centering
    \begin{tabular}{|p{3cm}|p{6cm}|p{3cm}|}
    \hline
    \rowcolor{blue!30}
    \textbf{DEBE} & \textbf{Contabilización, el 01/03/X2 de la venta de las acciones} & \textbf{HABER} \\
    \hline
    95.850 = $(8000 \times 12) - 150 $&  \cuenta{572}& \\
    \hline
    8150 &  \cuenta{666}& \\
    \hline
    &  \cuenta{540}& 104.000\\
    \hline
    \end{tabular}
\end{table}

Si suponemos el caso en el que el importe de la venta es de 6000 €, debemos de realizar el cociente $\frac{104000}{8000} = 13$ \euro/acción, por lo que tendríamos $6000 \times 13 = 78000$, de manera que si el importe neto es 71.850, debemos de contabilizar una pérdida de 6150.

\newpage
\section{Otros Ejercicios}


\subsection*{\textcolor{red}{\textbf{Ejercicio 1}}}

El banco EUROPA, S.A. concede un préstamo a la empresa COPA, S.A. por un valor nominal de 50.000 € el día 1 de marzo de 2019 con un vencimiento a tres años y un valor de reembolso de 52.000 €, al tipo de interés nominal del 5\%. Los gastos de la operación (que corren a cargo de la empresa COPA) ascienden a 1.000 €. El tipo de interés efectivo de la operación es del 6,2535\%. El cuadro de amortización calculado en base al tipo de interés efectivo es el siguiente:

\begin{table}[H]
\centering
\begin{tabular}{|c|c|c|c|p{4cm}|}
    \hline
    Plazo & Intereses Devengados & Pagos & Saldo amortizado & Saldo pendiente de amortizar \\
    \hline
    01/01/2019 & & & & 50.000 \\
    \hline
    31/12/2019 & 3.126,75 & 2.500 & 626,75 & 50.626,75 \\
    \hline
    31/12/2020 & 3.165,94 & 2.500 & 665,94 & 50.292,69 \\
    \hline
    31/12/2021 & 3.207,58 & 54.500 & 51.292,69 & 0 \\
    \hline
\end{tabular}
\end{table}

\textbf{SE PIDE:} Contabilizar las siguientes operaciones:

\begin{enumerate}[label=\alph*)]
    \item Concesión del crédito a 01/01/2019.
    
    \begin{table}[H]
        \centering
        \begin{tabular}{|p{3cm}|p{6cm}|p{3cm}|}
        \hline
        \rowcolor{blue!30}
        \textbf{DEBE} & \textbf{Concesión del crédito a 01/01/2019} & \textbf{HABER} \\
        \hline
        50.000 & (252) Créditos a largo plazo & \\
        \hline
        & (572) Bancos c/c & 50.000 \\
        \hline
        \end{tabular}
        \caption{Asiento a. Ejercicio 1.}
        \label{tabla:asiento1ej1T2}
    \end{table}

    \item Contabilización de las operaciones necesarias a 31/12/2019.
    
    \begin{table}[H]
        \centering
        \begin{tabular}{|p{3cm}|p{6cm}|p{3cm}|}
        \hline
        \rowcolor{blue!30}
        \textbf{DEBE} & \textbf{Contabilización de las operaciones necesarias a 31/12/2019} & \textbf{HABER} \\
        \hline
        2.500 & (572) Bancos c/c & \\
        \hline
        626,75 & (252) Créditos a largo plazo & \\
        \hline
        & (762) Ingresos de créditos & 3.126,75 \\
        \hline
        \end{tabular}
        \caption{Asiento b. Ejercicio 1.}
        \label{tabla:asiento2ej1T2}
    \end{table}


    \item Contabilización de la reclasificación del derecho de cobro a 31/12/2020.
    \begin{table}[H]
        \centering
        \begin{tabular}{|p{3cm}|p{6cm}|p{3cm}|}
        \hline
        \rowcolor{blue!30}
        \textbf{DEBE} & \textbf{Devengo de intereses a 31/12/2020} & \textbf{HABER} \\
        \hline
        2.500 & (572) Bancos c/c & \\
        \hline
        665,94 & (252) Créditos a largo plazo & \\
        \hline
        & (762) Ingresos de créditos & 3.165,94 \\
        \hline
        \end{tabular}
        \caption{Asiento c-1. Ejercicio 1.}
        \label{tabla:asiento3ej1T2}
    \end{table}
    \begin{align*}
        \text{Saldo de la cuenta (252) Créditos a largo plazo} = \\ = 50.000 + 626,75 + 665,94 = 51.292,69 \text{ \euro}
    \end{align*}

    \begin{table}[H]
        \centering
        \begin{tabular}{|p{3cm}|p{6cm}|p{3cm}|}
        \hline
        \rowcolor{blue!30}
        \textbf{DEBE} & \textbf{Contabilización de la reclasificación del derecho de cobro a 31/12/2020} & \textbf{HABER} \\
        \hline
        51.292,69 & (542) Créditos a corto plazo & \\
        \hline
        & (252) Créditos a largo plazo & 51.292,69 \\
        \hline
        \end{tabular}
        \caption{Asiento c-2. Ejercicio 1.}
        \label{tabla:asiento4ej1T2}
    \end{table}

    \item Contabilización de la cancelación del derecho de cobro EXCLUSIVAMENTE por su valor de reembolso el 31/12/2021.
    
    \begin{table}[H]
        \centering
        \begin{tabular}{|p{3cm}|p{6cm}|p{3cm}|}
        \hline
        \rowcolor{blue!30}
        \textbf{DEBE} & \textbf{Devengo de intereses a 31/12/2021} & \textbf{HABER} \\
        \hline
        2.500 & (572) Bancos c/c & \\
        \hline
        707,58 & (542) Créditos a largo plazo & \\
        \hline
        & (762) Ingresos de créditos & 3.207,58 \\
        \hline
        \end{tabular}
        \caption{Asiento d-1. Ejercicio 1.}
        \label{tabla:asiento5ej1T2}
    \end{table}
    \begin{align*}
        \text{Saldo de la cuenta (542) Créditos a corto plazo} = \\ = 51.292,69 + 707,58 \approx \text{52.000} \text{ \euro}
    \end{align*}

    \begin{table}[H]
        \centering
        \begin{tabular}{|p{3cm}|p{6cm}|p{3cm}|}
        \hline
        \rowcolor{blue!30}
        \textbf{DEBE} & \textbf{Contabilización de la cancelación del derecho de cobro EXCLUSIVAMENTE por su valor de reembolso el 31/12/2021} & \textbf{HABER} \\
        \hline
        52.000 & (572) Bancos c/c & \\
        \hline
        & (542) Créditos a corto plazo & 52.000 \\
        \hline
        \end{tabular}
        \caption{Asiento d-2. Ejercicio 1.}
        \label{tabla:asiento6ej1T2}
    \end{table}
\end{enumerate}

\newpage 
\subsection*{\textcolor{red}{\textbf{Ejercicio 2}}}







