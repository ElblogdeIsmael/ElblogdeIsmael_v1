\section{Ejercicio Propuesto 1}

\begin{table}[H]
    \centering
    \begin{tabular}{|p{3cm}|p{6cm}|p{3cm}|}
    \hline
    \textbf{DEBE} & \textbf{Contabilización 01/04/2023 de la compra de bono} & \textbf{HABER} \\
    \hline
    98000& \cuenta{251} & \\
    \hline
    &  \cuenta{572}& 98000\\
    \hline
    \end{tabular}
\end{table}

\begin{table}[H]
    \centering
    \begin{tabular}{|p{3cm}|p{6cm}|p{3cm}|}
    \hline
    \textbf{DEBE} & \textbf{ A 31/12/2023, si procede, contabilice el devengo de los intereses} & \textbf{HABER}\\
    \hline
    3727,03& \cuenta{546} & \\
    \hline
    1378,55&  \cuenta{251}& \\
    \hline
    &  \cuenta{761}& 5105,58 = $98000 \times (1,07006)^{\frac{9}{12}} - 98000$\\
    \hline
    \end{tabular}
\end{table}

\begin{table}[H]
    \centering
    \begin{tabular}{|p{3cm}|p{6cm}|p{3cm}|}
    \hline
    \textbf{DEBE} & \textbf{01/04/2024 devengo de los intereses desde Enero 2020} & \textbf{HABER} \\
    \hline
    1272,96 $\rightarrow I=[(100000+3727,03)\times(1,05)^{\frac{3}{12}} - (100000+3727,03)]$& \cuenta{546} & \\
    \hline
    1272,96&  \cuenta{251}& \\
    \hline
    487,47& \cuenta{761} & 1760,23 $\rightarrow I = [(98000+5105,58)\times(1,07006)^{\frac{3}{12}}-(98000+5105,58)]$\\
    \hline
    \end{tabular}
\end{table}

\begin{table}[H]
    \centering
    \begin{tabular}{|p{3cm}|p{6cm}|p{3cm}|}
    \hline
    \textbf{DEBE} & \textbf{01/04/2024 cobro intereses de la operación del bono} & \textbf{HABER} \\
    \hline
    5000&  \cuenta{572}& \\
    \hline
    &  \cuenta{546}& 5000\\
    \hline
    \end{tabular}
\end{table}

\begin{table}[H]
    \centering
    \begin{tabular}{|p{3cm}|p{6cm}|p{3cm}|}
    \hline
    \textbf{DEBE} & \textbf{31/12/2024 devengo de intereses} & \textbf{HABER} \\
    \hline
    3723,03& \cuenta{546} & \\
    \hline
    1475,76&  \cuenta{251}& \\
    \hline
    & \cuenta{761} & 5202,79 $\rightarrow I = [(99865,88 \times 1,07006)^{\frac{9}{12}} - 99865,88]$\\
    \hline
    \end{tabular}
\end{table}

\begin{table}[H]
    \centering
    \begin{tabular}{|p{3cm}|p{6cm}|p{3cm}|}
    \hline
    \textbf{DEBE} & \textbf{04/04/26} & \textbf{HABER} \\
    \hline
    104000&  \cuenta{572}& \\
    \hline
    & \cuenta{541} & 104000\\
    \hline
    \end{tabular}
\end{table}




\section{Ejercicio Propuesto 2}

\begin{table}[H]
    \centering
    \begin{tabular}{|p{3cm}|p{6cm}|p{3cm}|}
    \hline
    \textbf{DEBE} & \textbf{Contabilización el 01/07/2018 de la compra de la letra del tesoro} & \textbf{HABER} \\
    \hline
    130000& \cuenta{251} & \\
    \hline
    &  \cuenta{572}& 130000\\
    \hline
    \end{tabular}
\end{table}

\begin{table}[H]
    \centering
    \begin{tabular}{|p{3cm}|p{6cm}|p{3cm}|}
    \hline
    \textbf{DEBE} & \textbf{A 31/12/2018, si procede, contabilice el devengo de intereses} & \textbf{HABER} \\
    \hline
    4734,97&  \cuenta{251}& \\
    \hline
    &  \cuenta{761}& 4734,97\\
    \hline
    \end{tabular}
\end{table}

\begin{table}[H]
    \centering
    \begin{tabular}{|p{3cm}|p{6cm}|p{3cm}|}
    \hline
    \textbf{DEBE} & \textbf{A 31/12/2018, si procede, contabilice el cobro de intereses} & \textbf{HABER} \\
    \hline
    &  NPC& \\
    \hline
    \end{tabular}
\end{table}

\begin{table}[H]
    \centering
    \begin{tabular}{|p{3cm}|p{6cm}|p{3cm}|}
    \hline
    \textbf{DEBE} & \textbf{A 31/12/2019, si procede, contabilice el devengo de los intereses} & \textbf{HABER} \\
    \hline
    9993,60&  \cuenta{251}& \\
    \hline
    &  \cuenta{761}& 9993,60\\
    \hline
    \end{tabular}
\end{table}


\section{Ejercicio Propuesto 3}

\begin{table}[H]
    \centering
    \begin{tabular}{|p{3cm}|p{6cm}|p{3cm}|}
    \hline
    \textbf{DEBE} & \textbf{Compra de los títulos 02/03/2022} & \textbf{HABER} \\
    \hline
    126250&\cuenta{250}  & \\
    \hline
    &  \cuenta{572}& 126250 = $(5000 \times 25)(1+0,01)$ \\
    \hline
    \end{tabular}
\end{table}

\begin{table}[H]
    \centering
    \begin{tabular}{|p{3cm}|p{6cm}|p{3cm}|}
    \hline
    \textbf{DEBE} & \textbf{Si procede, operaciones derivadas de la valoración de los títulos a 31/12/2022} & \textbf{HABER} \\
    \hline
    23750 = $5000 \times 30 = 150000 - 126250$&  \cuenta{250}& \\
    \hline
    &  \cuenta{900} &23750 \\
    \hline
    23750& \cuenta{900} & \\
    \hline
    & \cuenta{133} & 23750\\
    \hline
    \end{tabular}
\end{table}

\begin{table}[H]
    \centering
    \begin{tabular}{|p{3cm}|p{6cm}|p{3cm}|}
    \hline
    \textbf{DEBE} & \textbf{Si procede, operaciones derivadas de la valoración de los títulos a 31/12/2023} & \textbf{HABER} \\
    \hline
    10000 &  \cuenta{800}& \\
    \hline
    &  \cuenta{250}& 10000 = $5000 \times 28 = 140000-150000=10000$\\
    \hline
    10000&  \cuenta{133}& \\
    \hline
    &  \cuenta{800}& 10000\\
    \hline
    \end{tabular}
\end{table}

\begin{table}[H]
    \centering
    \begin{tabular}{|p{3cm}|p{6cm}|p{3cm}|}
    \hline
    \textbf{DEBE} & \textbf{Todas las operaciones derivadas de la venta de los títulos a 23/05/2024} & \textbf{HABER} \\
    \hline
    156800 = $5000 \times 32 = 160000 - 2 \% \times 160000$&  \cuenta{572}& \\
    \hline
    &  \cuenta{250}& 140000 \\
    \hline
    &  \cuenta{766}& 16800\\
    \hline
    13750 = $23750 - 10000$&  \cuenta{133}& \\
    \hline
    &  \cuenta{802}& 13750\\
    \hline
    13750 &  \cuenta{802}& \\
    \hline
    &  \cuenta{7632}& 13750\\
    \hline
    \end{tabular}
\end{table}

\section{Ejercicio Propuesto 4}

\begin{table}[H]
    \centering
    \begin{tabular}{|p{3cm}|p{6cm}|p{3cm}|}
    \hline
    \textbf{DEBE} & \textbf{Contabilización de la } & \textbf{HABER} \\
    \hline
    &  & \\
    \hline
    &  & \\
    \hline
    &  & \\
    \hline
    &  & \\
    \hline
    \end{tabular}
\end{table}

\begin{table}[H]
    \centering
    \begin{tabular}{|p{3cm}|p{6cm}|p{3cm}|}
    \hline
    \textbf{DEBE} & \textbf{} & \textbf{HABER} \\
    \hline
    &  & \\
    \hline
    &  & \\
    \hline
    &  & \\
    \hline
    &  & \\
    \hline
    \end{tabular}
\end{table}


