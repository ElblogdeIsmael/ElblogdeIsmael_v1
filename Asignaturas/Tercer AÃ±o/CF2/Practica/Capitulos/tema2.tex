\section{Ejercicio Propuesto 1}

\begin{table}[H]
    \centering
    \begin{tabular}{|p{3cm}|p{6cm}|p{3cm}|}
    \hline
    \textbf{DEBE} & \textbf{Contabilización 01/04/2023 de la compra de bono} & \textbf{HABER} \\
    \hline
    98000& \cuenta{251} & \\
    \hline
    &  \cuenta{572}& 98000\\
    \hline
    \end{tabular}
\end{table}

\begin{table}[H]
    \centering
    \begin{tabular}{|p{3cm}|p{6cm}|p{3cm}|}
    \hline
    \textbf{DEBE} & \textbf{ A 31/12/2023, si procede, contabilice el devengo de los intereses} & \textbf{HABER}\\
    \hline
    3727,03& \cuenta{546} & \\
    \hline
    1378,55&  \cuenta{251}& \\
    \hline
    &  \cuenta{761}& 5105,58 = $98000 \times (1,07006)^{\frac{9}{12}} - 98000$\\
    \hline
    \end{tabular}
\end{table}

\begin{table}[H]
    \centering
    \begin{tabular}{|p{3cm}|p{6cm}|p{3cm}|}
    \hline
    \textbf{DEBE} & \textbf{01/04/2024 devengo de los intereses desde Enero 2020} & \textbf{HABER} \\
    \hline
    1272,96 $\rightarrow I=[(100000+3727,03)\times(1,05)^{\frac{3}{12}} - (100000+3727,03)]$& \cuenta{546} & \\
    \hline
    1272,96&  \cuenta{251}& \\
    \hline
    487,47& \cuenta{761} & 1760,23 $\rightarrow I = [(98000+5105,58)\times(1,07006)^{\frac{3}{12}}-(98000+5105,58)]$\\
    \hline
    \end{tabular}
\end{table}

\begin{table}[H]
    \centering
    \begin{tabular}{|p{3cm}|p{6cm}|p{3cm}|}
    \hline
    \textbf{DEBE} & \textbf{01/04/2024 cobro intereses de la operación del bono} & \textbf{HABER} \\
    \hline
    5000&  \cuenta{572}& \\
    \hline
    &  \cuenta{546}& 5000\\
    \hline
    \end{tabular}
\end{table}

\begin{table}[H]
    \centering
    \begin{tabular}{|p{3cm}|p{6cm}|p{3cm}|}
    \hline
    \textbf{DEBE} & \textbf{31/12/2024 devengo de intereses} & \textbf{HABER} \\
    \hline
    3723,03& \cuenta{546} & \\
    \hline
    1475,76&  \cuenta{251}& \\
    \hline
    & \cuenta{761} & 5202,79 $\rightarrow I = [(99865,88 \times 1,07006)^{\frac{9}{12}} - 99865,88]$\\
    \hline
    \end{tabular}
\end{table}

\begin{table}[H]
    \centering
    \begin{tabular}{|p{3cm}|p{6cm}|p{3cm}|}
    \hline
    \textbf{DEBE} & \textbf{04/04/26} & \textbf{HABER} \\
    \hline
    104000&  \cuenta{572}& \\
    \hline
    & \cuenta{541} & 104000\\
    \hline
    \end{tabular}
\end{table}




\section{Ejercicio Propuesto 2}
En el balance de situación de la empresa “TRAG, S.A.”, a 31 de diciembre del año 2018 figura una letra del tesoro, adquirida el 1 de julio de dicho año, por la que se pagaron 130.000 euros. Su vencimiento es el 30 de junio del año 2020. El nominal de la letra es de 150.000 euros. Se espera que mantener hasta vencimiento este instrumento. El tipo de interés efectivo anual por vencido es del 7,417231\%. La tabla que recoge el coste amortizado a este tipo de interés es la siguiente:

\begin{table}[H]
\centering
\begin{tabular}{|p{2cm}|p{2cm}|p{2cm}|p{2cm}|p{2cm}|p{2cm}|}
    \hline
    Fecha & Pago & Cobros & Intereses devengados (Gastos financieros) & Saldo amortizado & Saldo pendiente de amortizar (Balance) \\
    \hline
    01/07/2018 & 130.000 & 0 & 0 & 0 & 130.000 \\
    \hline
    31/12/2018 & 0 & 0 & 4.734.97 & -4.734.97 & 134.734.97 \\
    \hline
    31/12/2019 & 0 & 0 & 9.993.60 & -9.993.60 & 144.728.57 \\
    \hline
    01/07/2020 & 150.000 & 0 & 5.271.43 & 144.728.57 & 0 \\
    \hline
    \end{tabular}
\end{table}


\textbf{SE PIDE:} Realice las anotaciones contables relacionadas con el enunciado anterior.


\begin{table}[H]
    \centering
    \begin{tabular}{|p{3cm}|p{6cm}|p{3cm}|}
    \hline
    \textbf{DEBE} & \textbf{Contabilización el 01/07/2018 de la compra de la letra del tesoro} & \textbf{HABER} \\
    \hline
    130000& \cuenta{251} & \\
    \hline
    &  \cuenta{572}& 130000\\
    \hline
    \end{tabular}
\end{table}

\begin{table}[H]
    \centering
    \begin{tabular}{|p{3cm}|p{6cm}|p{3cm}|}
    \hline
    \textbf{DEBE} & \textbf{A 31/12/2018, si procede, contabilice el devengo de intereses} & \textbf{HABER} \\
    \hline
    4734,97&  \cuenta{251}& \\
    \hline
    &  \cuenta{761}& 4734,97\\
    \hline
    \end{tabular}
\end{table}

\begin{table}[H]
    \centering
    \begin{tabular}{|p{3cm}|p{6cm}|p{3cm}|}
    \hline
    \textbf{DEBE} & \textbf{A 31/12/2018, si procede, contabilice el cobro de intereses} & \textbf{HABER} \\
    \hline
    &  NPC& \\
    \hline
    \end{tabular}
\end{table}

\begin{table}[H]
    \centering
    \begin{tabular}{|p{3cm}|p{6cm}|p{3cm}|}
    \hline
    \textbf{DEBE} & \textbf{A 31/12/2019, si procede, contabilice el devengo de los intereses} & \textbf{HABER} \\
    \hline
    9993,60&  \cuenta{251}& \\
    \hline
    &  \cuenta{761}& 9993,60\\
    \hline
    \end{tabular}
\end{table}


\section{Ejercicio Propuesto 3}

La sociedad BENGALS adquiere, el 02/03/2022, 5.000 acciones de la sociedad COWBOYS por importe de 25 €/acción. La operación conlleva unos gastos de gestión que ascienden al 1\% del total de la operación. La inversión se realiza con un carácter de permanencia.

A cierre del ejercicio 2022, estos títulos cotizan a 30 € cada uno, y a cierre del ejercicio 2023 su cotización es de 28 €/acción.

Por necesidades de liquidez, el 23/05/2024 se venden la totalidad de las acciones por un importe de 32 €/acción, con unos costes de transacción del 2\% sobre el precio de venta.

\textbf{SE PIDE:} Contabilizar en el libro diario de BENGALS, S.A. las siguientes operaciones, sin tener en cuenta el posible efecto impositivo:

a) Compra de los títulos a 02/03/2022.

b) Si procede, operaciones derivadas de la valoración de los títulos a 31/12/2022.

c) Si procede, operaciones derivadas de la valoración de los títulos a 31/12/2023.

d) Todas las operaciones derivadas de la venta de los títulos a 23/05/2024.


\begin{table}[H]
    \centering
    \begin{tabular}{|p{3cm}|p{6cm}|p{3cm}|}
    \hline
    \textbf{DEBE} & \textbf{Compra de los títulos 02/03/2022} & \textbf{HABER} \\
    \hline
    126250&\cuenta{250}  & \\
    \hline
    &  \cuenta{572}& 126250 = $(5000 \times 25)(1+0,01)$ \\
    \hline
    \end{tabular}
\end{table}

\begin{table}[H]
    \centering
    \begin{tabular}{|p{3cm}|p{6cm}|p{3cm}|}
    \hline
    \textbf{DEBE} & \textbf{Si procede, operaciones derivadas de la valoración de los títulos a 31/12/2022} & \textbf{HABER} \\
    \hline
    23750 = $5000 \times 30 = 150000 - 126250$&  \cuenta{250}& \\
    \hline
    &  \cuenta{900} &23750 \\
    \hline
    23750& \cuenta{900} & \\
    \hline
    & \cuenta{133} & 23750\\
    \hline
    \end{tabular}
\end{table}

\begin{table}[H]
    \centering
    \begin{tabular}{|p{3cm}|p{6cm}|p{3cm}|}
    \hline
    \textbf{DEBE} & \textbf{Si procede, operaciones derivadas de la valoración de los títulos a 31/12/2023} & \textbf{HABER} \\
    \hline
    10000 &  \cuenta{800}& \\
    \hline
    &  \cuenta{250}& 10000 = $5000 \times 28 = 140000-150000=10000$\\
    \hline
    10000&  \cuenta{133}& \\
    \hline
    &  \cuenta{800}& 10000\\
    \hline
    \end{tabular}
\end{table}

\begin{table}[H]
    \centering
    \begin{tabular}{|p{3cm}|p{6cm}|p{3cm}|}
    \hline
    \textbf{DEBE} & \textbf{Todas las operaciones derivadas de la venta de los títulos a 23/05/2024} & \textbf{HABER} \\
    \hline
    156800 = $5000 \times 32 = 160000 - 2 \% \times 160000$&  \cuenta{572}& \\
    \hline
    &  \cuenta{250}& 140000 \\
    \hline
    &  \cuenta{766}& 16800\\
    \hline
    13750 = $23750 - 10000$&  \cuenta{133}& \\
    \hline
    &  \cuenta{802}& 13750\\
    \hline
    13750 &  \cuenta{802}& \\
    \hline
    &  \cuenta{7632}& 13750\\
    \hline
    \end{tabular}
\end{table}

\section{Ejercicio Propuesto 4}

La empresa EL FIJITIVO, S.L. compra el 01/11/X1, 8.000 acciones del banco MALO, S.A., que cotizan en bolsa a 15 €/acción. Los gastos derivados de la adquisición son 100 €. La inversión tiene carácter especulativo. A 31/12/X1, las acciones de MALO, S.A. cotizan a 13 €/acción y los costes de transacción previstos son de 200 €. El 01/03/X2 se venden las acciones a 12 €/acción con unos gastos de 150 €.

\textbf{SE PIDE:} Contabilice exclusivamente la compra y la venta de las acciones.


\begin{table}[H]
    \centering
    \begin{tabular}{|p{3cm}|p{6cm}|p{3cm}|}
    \hline
    \textbf{DEBE} & \textbf{Contabilización de la } & \textbf{HABER} \\
    \hline
    &  & \\
    \hline
    &  & \\
    \hline
    &  & \\
    \hline
    &  & \\
    \hline
    \end{tabular}
\end{table}

\begin{table}[H]
    \centering
    \begin{tabular}{|p{3cm}|p{6cm}|p{3cm}|}
    \hline
    \textbf{DEBE} & \textbf{} & \textbf{HABER} \\
    \hline
    &  & \\
    \hline
    &  & \\
    \hline
    &  & \\
    \hline
    &  & \\
    \hline
    \end{tabular}
\end{table}


