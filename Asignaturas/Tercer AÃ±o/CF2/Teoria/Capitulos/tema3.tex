\section{Concepto, características y tipología de los pasivos financieros}

\subsection{Concepto}

Previamente debemos de saber que es un instrumento financiero, el cual es un contrato que da lugar a un activo financiero en una empresa, y simultáneamente, a un pasivo financiero o a un instrumento de patrimonio en otra empresa.

Los instrumentos financieros emitidos, incurridos o asumidos se clasificarán como pasivos financiero, en su totalidad o en una de sus partes, siempre que de acuero a su realidad económica, suponga una obligación contractual de entregar efectivo u otro activo financiero, intercambiar activos o pasivos financieros con terceros potencialmente desfavorables, también se clasificarán como un pasivo financiero, todo contrato que pueda ser o será, liquidado con los instrumentos propios de la empresa, siempre que no sean instrumentos derivados.

\subsubsection{Pasivo financiero vs instrumento de patrimonio neto}

\begin{itemize}
    \item Pasivo financiero: existe una obligación contractual que recae sobre una de las partes implicadas en el instrumento financiero. 
    \item Intrumento de PN: no existe una obligación contractual que recae sobre una de las partes implicadas en el instrumento financiero. Aunque el comprador pueda percibir dividendos, el emisor no tiene la obligación de repartir los dividendos.
\end{itemize}

\subsection{Diferentes tipologías}

\begin{itemize}
    \item Débitos por operaciones comerciales
    \item Deudas con entidades de crédito
    \item Obligaciones y otros valores negociables emitidos, tales como bonos y pagarés.
    \item Derivados con valoración desfavorable para la empresa.
    \item Deudas con características especiales, tales como deudas subordinadas.
    \item Otros pasivos financieros, tales como deudas con proveedores.
\end{itemize}

\subsection{Tipos de pasivos financieros}
\begin{itemize}
    \item Pasivo a coste amortizado: Todos en esta categoría, excepto cuando deban de valorarse a VR con cambios en la cuenta de pérdidas y ganancias. En este incluidos débito por operaciones comerciales y débito por operaciones no comerciales.
    \item Pasivo a valor razonable con cambios en la cuenta de pérdidas y ganancias: Deben de cumplir:
    \begin{itemize}
        \item Pasivos que se mantienen para negociar.
        \begin{itemize}
            \item se emite con el propósito de readquirirlo en el corto plazo.
            \item obligación de que un vendedor de un activo financiero lo readquiera.
            \item Forme parte de una cartera de activos financieros identificados que se gestionan conjuntamente.
        \end{itemize}
        \item Desde el momento de reconocimiento inicial, ha sido designado por la empresa como a valor razonable con cambios en la cuenta de pérdidas y ganancias.
    \end{itemize}
\end{itemize}

\subsection{¿Qué cuentas son con las que más vamos a trabajar?}

\begin{itemize}
    \item 15, 16, 17, 18\footnote{Para saber que cuentan son consulte el cuadro de cuentas o bien el manual, página 117-118}.
    \item 50, 51, 52, 56.
\end{itemize}

No se recogen los derivados de operaciones de tráfico, ya que se estudian en la asignatura de CF1 (operaciones con proveedores y acreedores,...).

\subsection{Valoración de los pasivos financieros}

\subsubsection{A coste amortizado}
\begin{itemize}
    \item Valoración inicial: Inicialmente a VR, es decir, el precio de la transacción. Consideramos los costes que son para emitirlos. Por ejemplo, si pido un préstamo de una cuantía X, el importe extra por gestión, se debe de incluir como un mayor importe en el valor del pasivo.
    \item Valoración posterior: Por su coste amortizado, los interéses devengados, se contabilizarán en la cuenta de pérdidas y ganancias, aplicando el método del tanto efectivo.
\end{itemize}

\textit{Nota: Debemos de tener en cuenta que la empresa que recibe el préstamo tiene un pasivo, mientras que la empresa que lo da tiene un activo.}

