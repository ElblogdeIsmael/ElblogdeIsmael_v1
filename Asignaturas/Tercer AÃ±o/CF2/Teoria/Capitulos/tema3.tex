\section{Concepto, características y tipología PF}

\subsection{Concepto}

Previamente debemos de saber que es un instrumento financiero. Este \textit{es un contrato que da lugar a un activo financiero en una empresa}, y simultáneamente, a un pasivo financiero o a un instrumento de patrimonio en otra empresa.

Los instrumentos financieros emitidos, incurridos o asumidos se clasificarán como pasivos financiero, en su totalidad o en una de sus partes, siempre que de acuerdo a su realidad económica, suponga una obligación contractual de entregar efectivo u otro activo financiero, intercambiar activos o pasivos financieros con terceros potencialmente desfavorables, también se clasificarán como un pasivo financiero, todo contrato que pueda ser o será, liquidado con los instrumentos propios de la empresa, siempre que no sean instrumentos derivados.

\subsubsection{Pasivo financiero VS instrumento de patrimonio neto}

\begin{itemize}
    \item Pasivo financiero: existe una obligación contractual que recae sobre una de las partes implicadas en el instrumento financiero. 
    \item Intrumento de PN: no existe una obligación contractual que recae sobre una de las partes implicadas en el instrumento financiero. Aunque el comprador pueda percibir dividendos, el emisor no tiene la obligación de repartir los dividendos.
\end{itemize}

\subsection{Diferentes tipologías}

\begin{itemize}
    \item Débitos por operaciones comerciales
    \item Deudas con entidades de crédito
    \item Obligaciones y otros valores negociables emitidos, tales como bonos y pagarés.
    \item Derivados con valoración desfavorable para la empresa.
    \item Deudas con características especiales, tales como deudas subordinadas.
    \item Otros pasivos financieros, tales como deudas con proveedores.
\end{itemize}

\subsection{Tipos de pasivos financieros}
\begin{itemize}
    \item \textbf{Pasivo a coste amortizado:} Todos en esta categoría, excepto cuando deban de valorarse a VR con cambios en la cuenta de pérdidas y ganancias. En este incluidos débito por operaciones comerciales y débito por operaciones no comerciales.
    \begin{itemize}
        \item Débitos por operaciones comerciales: aquellos que se originan en la compra de bienes o servicios.
        \item Débitos por operaciones no comerciales: son aquellos que no siendo instrumentos derivados, no tienen origen comercial, sino que proceden de operaciones de préstamo o crédito recibidos por la empresa.
    \end{itemize}
    \item \textbf{Pasivo a valor razonable con cambios en la cuenta de pérdidas y ganancias}: Deben de cumplir:
    \begin{itemize}
        \item \textit{Pasivos que se mantienen para negociar}.
        \begin{itemize}
            \item se emite con el propósito de readquirirlo en el corto plazo.
            \item obligación de que un vendedor de un activo financiero lo readquiera.
            \item Forme parte de una cartera de activos financieros identificados que se gestionan conjuntamente.
        \end{itemize}
        \item Desde el momento de reconocimiento inicial, ha sido designado por la empresa como a \textit{valor razonable con cambios en la cuenta de pérdidas y ganancias}.
        \begin{itemize}
            \item Se elimina o reduzca de manera significativa una incoherencia o ``asimetría contable''.
            \item Un grupo de pasivos financieros o de activos y pasivos financieros se gestione y su rendimiento se evalúe sobre la base del valor razonable de acuerdo con la estrategia de gestión de riesgos o de inversión de la empresa.
        \end{itemize}
    \end{itemize}
\end{itemize}

\subsection{¿Qué cuentas son con las que más vamos a trabajar?}

% \begin{itemize}
%     \item 15, 16, 17, 18\footnote{Para saber que cuentan son consulte el cuadro de cuentas o bien el manual, página 117-118}.
%     \item 50, 51, 52, 56.
% \end{itemize}

\begin{itemize}
    \item 15. DEUDAS A LARGO PLAZO CON CARACTERÍSTICAS ESPECIALES.
    \item 16. DEUDAS A LARGO PLAZO CON PARTES VINCULADAS.
    \item 17. DEUDAS A LARGO PLAZO POR PRÉSTAMOS RECIBIDOS, EMPRÉSTITOS Y OTROS CONCEPTOS.
    \item 18. PASIVOS POR FIANZAS, GARANTÍAS Y OTROS CONCEPTOS A LARGO PLAZO.
    \end{itemize}

    \begin{itemize}
        \item 50. EMPRÉSTITOS, DEUDAS CON CARACTERÍSTICAS ESPECIALES Y OTRAS EMISIONES ANÁLOGAS A CORTO PLAZO.
        \item 51. DEUDAS A CORTO PLAZO CON PARTES VINCULADAS.
        \item 52. DEUDAS A CORTO PLAZO POR PRÉSTAMOS RECIBIDOS Y OTROS CONCEPTOS.
        \item 56. FIANZAS Y DEPÓSITOS RECIBIDOS Y CONSTITUIDOS A CORTO PLAZO Y AJUSTES POR PERIODIFICACIÓN.
    \end{itemize}

No se recogen los derivados de operaciones de tráfico, ya que se estudian en la asignatura de CF1 (operaciones con proveedores y acreedores,...).
% \\\\
% \textit{Retomando la clasificación de los pasivos finanacieros, vamos a tratar esta más a fondo:}

% \begin{enumerate}
%     \item Pasivos financieros a coste amortizado.
%     \begin{
%     \item Pasivos financieros a valor razonable con cambios en la cuenta de pérdidas y ganancias.
% \end{enumerate}


\section{Pasivos a coste amortizado}

\begin{enumerate}
    \item Pasivos financieros a coste amortizado.
    \begin{itemize}
        \item Valoración inicial: Inicialmente a VR, salvo evidencia en contrario, será el precio de la transacción, que equivaldrá al VR de la contraprestación entregada más los costes de transacción.
        \item Valoración posterior: Por su coste amortizado, los intereses devengados se contabilizarán en la cuenta de pérdidas y ganancias, aplicando el método del tipo de interés efectivo (TIE).
    \end{itemize}
\end{enumerate}
\newpage
\subsection*{\textcolor{blue}{Ejemplo 1}}

La entidad A concede un crédito el 1-1-2018 a la entidad B con las siguientes características:
\begin{itemize}
    \item Nominal: 100.000 €.
    \item Comisión de apertura (liquidada al inicio): 1.200 €.
    \item Tipo de interés anual: 6,75\%.
    \item Plazo: 3 años.
    \item Cuota anual constante: 37.931 €.
\end{itemize}

\textbf{SE PIDE:} Contabilizar las operaciones en la empresa B.


En primer lugar, vamos a dibujar el esquema de flujos de efectivo del ejemplo 1.



\begin{figure}[H]
    \centering
    \begin{tikzpicture}
        % Eje horizontal
        \draw[-] (0,0) -- (10,0) node[right] {Tiempo};
        
        % Eje vertical
        \draw[-{Latex}] (0,0) -- (0,5) node[above left] {Flujo de efectivo};
        
        % Flechas principales
        \draw[-{Latex},red] (0,0) -- (0,4) node[above left] {100.000 \texteuro};
        \draw[-{Latex},red] (0,0) -- (0,3) node[below left] {-1.200 \texteuro};
        
        % Flechas de pagos anuales
        \draw[-{Latex},red] (3,0) -- (3,-2) node[below] {37.931 \texteuro};
        \draw[-{Latex},red] (6,0) -- (6,-2) node[below] {37.931 \texteuro};
        \draw[-{Latex},red] (9,0) -- (9,-2) node[below] {37.931 \texteuro};
        
        % Nodos de tiempo en la línea del eje
        \draw (-0.2,0)  (-0.2,0.2) node[above] {0};
        \foreach \x/\t in { 3/1, 6/2, 9/3} {
            \draw (\x,0) -- (\x,0.2) node[above] {\t};
        }
    \end{tikzpicture}
    \caption{Esquema de flujos de efectivo del ejemplo 1.}
    \label{fig:flujo-efectivo-ejemplo1}
\end{figure}


\begin{table}[H]
    \centering
    \begin{tabular}{|c|c|c|c|c|}
        \hline
        \rowcolor{blue!30}
        \textbf{Plazo} & \textbf{Cuota} & \textbf{Capital} & \textbf{Intereses} & \textbf{Capital pendiente} \\
        \hline
        01-01-2018 &   &   &   & 100.000 € \\
        \hline
        31-12-2018 & 37.931 € & 31.181 € & 6.750 € & 68.819 € \\
        \hline
        31-12-2019 & 37.931 € & 33.286 € & 4.645 € & 35.533 € \\
        \hline
        31-12-2020 & 37.931 € & 35.533 € & 2.398 € & 0 € \\
        \hline
    \end{tabular}
    \caption{Cuadro de amortización al 6,75\%.}
    \label{tabla:cuadro_amortizacion}
\end{table}

El tipo de interés que tiene la empresa financiera no tiene las mismas reglas que la que recibe el préstamo. En este caso, la empresa que recibe el préstamo tendrá el tipo de interés de la Figura \ref{fig:nuevo_tipo_interes} y el cuadro de amortización del cuadro \ref{tabla:cuadro_amortizacion_ej1}.

\begin{figure}[H]
    \begin{equation*}
        100\p000 - 1\p200 = \frac{37\p931}{(1+i)} + \frac{37\p931}{(1+i)^2} + \frac{37\p931}{(1+i)^3}
    \end{equation*}
    \begin{equation*}
        i = 7,41\%
    \end{equation*}
    \caption{Nuevo tipo de interés.}
    \label{fig:nuevo_tipo_interes}
\end{figure}

\begin{table}[H]
    \centering
    \begin{tabular}{|c|c|c|c|c|}
        \hline
        \rowcolor{blue!30}
        \textbf{Plazo} & \textbf{Cuota} & \textbf{Capital} & \textbf{Intereses} & \textbf{Coste Amortizado} \\
        \hline
        01-01-2018 & - & - & - & 98.800 € \\
        \hline
        31-12-2018 & 37.931 € & 30.609 € & 7.322 € & 68.191 € \\
        \hline
        31-12-2019 & 37.931 € & 32.877 € & 5.054 € & 35.314 € \\
        \hline
        31-12-2020 & 37.931 € & 35.314 € & 2.617 € & 0 € \\
        \hline
    \end{tabular}
    \caption{Cuadro de amortización al 7,41\%.}
    \label{tabla:cuadro_amortizacion_ej1}
\end{table}

Procedemos a la contabilización del reconocimiento inicial del préstamo.

\begin{table}[H]
    \centering
    \begin{tabular}{|c|c|p{2cm}|}
        \hline
        \rowcolor{blue!30}
        \textbf{DEBE} & \textbf{Reconocimiento inicial (01-01-2018)} & \textbf{HABER} \\
        \hline
        98.800 & (572) Bancos e instituciones de crédito c/c vista, euros & \\
        \hline
        & (170) Deudas a largo plazo con entidades de crédito & 68.191 = 32.877 + 35.314 \\
        \hline
        & (5200) Préstamos a corto plazo de entidades de crédito & 30.609 \\
        \hline
    \end{tabular}
    \caption{Reconocimiento inicial del préstamo.}
    \label{tabla:reconocimiento_inicial}
\end{table}

A continuación, al final del ejercicio debemos de hacer efectivo el pago de la cuota y el reflejo contable de los intereses.
 
\begin{table}[H]
    \centering
    \begin{tabular}{|p{2cm}|p{6cm}|p{2cm}|}
        \hline
        \rowcolor{blue!30}
        \textbf{DEBE} & \textbf{Pago de la cuota (31-12-2018)} & \textbf{HABER} \\
        \hline
        7.322 & (6623) Intereses de deudas con entidades de crédito & \\
        \hline
        30.609 & (5200) Préstamos a corto plazo de entidades de crédito & \\
        \hline
        & (572) Bancos e instituciones de crédito c/c vista, euros & 37.931 \\
        \hline
    \end{tabular}
    \caption{Pago de la cuota (31-12-2018).}
    \label{tabla:pago_cuota_2018}
\end{table}

Ahora debemos de reclasificar la deuda a largo plazo a corto plazo.

\begin{table}[H]
    \centering
    \begin{tabular}{|c|p{6cm}|c|}
        \hline
        \rowcolor{blue!30}
        \textbf{DEBE} & \textbf{Reclasificación de la deuda (31-12-2018)} & \textbf{HABER} \\
        \hline
        32.877 & (170) Deudas a largo plazo entidades crédito & \\
        \hline
        & (5200) Préstamos a corto plazo de entidades de crédito & 32.877 \\
        \hline
    \end{tabular}
    \caption{Reclasificación de la deuda (31-12-2018).}
    \label{tabla:reclasificacion_deuda}
\end{table}

Para el ejercicio de 2019 debemos de realizar las operaciones de manera análoga a las del ejercicio anterior.

\begin{table}[H]
    \centering
    \begin{tabular}{|p{2cm}|p{6cm}|p{2cm}|}
        \hline
        \rowcolor{blue!30}
        \textbf{DEBE} & \textbf{Pago de la cuota (31-12-2019)} & \textbf{HABER} \\
        \hline
        5.054 & (6623) Intereses de deudas con entidades de crédito & \\
        \hline
        32.877 & (5200) Préstamos a corto plazo de entidades de crédito & \\
        \hline
        & (572) Bancos e instituciones de crédito c/c vista, euros & 37.931 \\
        \hline
    \end{tabular}
    \caption{Pago de la cuota (31-12-2019).}
    \label{tabla:pago_cuota_2019}
\end{table}

\begin{table}[H]
    \centering
    \begin{tabular}{|c|p{6cm}|c|}
        \hline
        \rowcolor{blue!30}
        \textbf{DEBE} & \textbf{Reclasificación de la deuda (31-12-2019)} & \textbf{HABER} \\
        \hline
        35.314 & (170) Deudas a largo plazo entidades crédito & \\
        \hline
        & (5200) Préstamos a corto plazo de entidades de crédito & 35.314 \\
        \hline
    \end{tabular}
    \caption{Reclasificación de la deuda (31-12-2019).}
    \label{tabla:reclasificacion_deuda_2019}
\end{table}

\begin{table}[H]
    \centering
    \begin{tabular}{|p{2cm}|p{6cm}|p{2cm}|}
        \hline
        \rowcolor{blue!30}
        \textbf{DEBE} & \textbf{Pago de la cuota (31-12-2020)} & \textbf{HABER} \\
        \hline
        2.617 & (6623) Intereses de deudas con entidades de crédito & \\
        \hline
        35.314 & (5200) Préstamos a corto plazo de entidades de crédito & \\
        \hline
        & (572) Bancos e instituciones de crédito c/c vista, euros & 37.931 \\
        \hline
    \end{tabular}
    \caption{Pago de la cuota (31-12-2020).}
    \label{tabla:pago_cuota_2020}
\end{table}
\newpage
\subsection*{\textcolor{blue}{Ejemplo 2}}

La empresa VENKA, S.A. concierta el 1 de julio de 2020 una operación de préstamo con la entidad financiera LOPRESTOTODO, S.A. por importe de 100.000 € a 3 años con pagos semestrales con un tipo de interés nominal anual del 5\%, con una comisión de estudio del 2,5\% y comisión de apertura del 1\%. El cuadro de amortización del préstamo proporcionado por la entidad es el siguiente:

\begin{table}[h]
    \centering
    \begin{tabular}{p{2cm}p{2cm}p{2cm}p{2cm}p{2cm}p{2cm}}
        \toprule
        \textbf{Fecha} & \textbf{Cuota} & \textbf{Interés} & \textbf{Capital} & \textbf{Capital Amortizado Acumulado} & \textbf{Capital Pendiente} \\
        \midrule
        01-07-20 & & & & & 100.000,00 \\
        01-01-21 & 18.155,00 & 2.500,00 & 15.655,00 & 15.655,00 & 84.345,00 \\
        01-07-21 & 18.155,00 & 2.108,63 & 16.046,37 & 31.701,37 & 68.298,63 \\
        01-01-22 & 18.155,00 & 1.707,47 & 16.447,53 & 48.148,90 & 51.851,10 \\
        01-07-22 & 18.155,00 & 1.296,28 & 16.858,72 & 65.007,62 & 34.992,38 \\
        01-01-23 & 18.155,00 & 874,81 & 17.280,19 & 82.287,81 & 17.712,19 \\
        01-07-23 & 18.155,00 & 442,80 & 17.712,19 & 100.000,00 & 0,00 \\
        \bottomrule
    \end{tabular}
    \caption{Cuadro de amortización tipo interés nominal (5\%)}
    \label{tab:amortizacion}
\end{table}

\textbf{SE PIDE:} Contabilización de la operación de endeudamiento relativa a la valoración inicial y al pago de las dos primeras cuotas.
\\\\
\textbf{SOLUCIÓN:} \\\\

Para el cálculo del nuevo tipo de interés debemos de realizarlo de manera análoga al ejemplo anterior, cabe destacar que en el examen no se pedirá la realización del cuadro.


\begin{table}[h]
    \centering
    \begin{tabular}{p{2cm}p{2cm}p{2cm}p{2cm}p{2cm}p{2cm}}
        \toprule
        \textbf{Fecha} & \textbf{Cuota} & \textbf{Interés} & \textbf{Capital} & \textbf{Capital Amortizado Acumulado} \\
        \midrule
        01-07-20 & & & & 96.500,02 \\
        01-01-21 & 18.155,00 & 3.450,48 & 14.704,52 & 81.795,50 \\
        01-07-21 & 18.155,00 & 2.924,70 & 15.230,30 & 66.565,20 \\
        01-01-22 & 18.155,00 & 2.380,13 & 15.774,87 & 50.790,33 \\
        01-07-22 & 18.155,00 & 1.816,07 & 16.338,93 & 34.451,40 \\
        01-01-23 & 18.155,00 & 1.231,85 & 16.923,15 & 17.528,25 \\
        01-07-23 & 18.154,99 & 626,74 & 17.528,25 & 0,00 \\
        \bottomrule
    \end{tabular}
    \caption{Cuadro de amortización tipo interés efectivo (7,28\%) \\ 
    Tiempo de interés efectivo semestral (3,58\%)}
    \label{tab:amortizacion_efectivo}
\end{table}

Ahora procedemos a la contabilización de la valoración inicial del préstamo, teniendo en cuenta que la comisión de estudio y apertura\footnote{Estos ascienden a 3.500 \e.}. Debemos de distinguir entre la parte a corto y largo plazo, siendo la de corto plazo la correspondiente a la figura \ref{fig:corto_plazo_ej2} y la de largo plazo las restantes.

\begin{figure}[H]
    \begin{align*}
        \text{Corto plazo (Las dos primeras cuotas):} \Rightarrow 14.704,52 + 15.230,30 & = 29.934,82 
    \end{align*}
    \caption{Parte a corto plazo del préstamo.}
    \label{fig:corto_plazo_ej2}
\end{figure}

\begin{table}[H]
    \centering
    \begin{tabular}{|c|p{6cm}|c|}
        \hline
        \rowcolor{blue!30}
        \textbf{DEBE} & \textbf{Registro contable de la valoración inicial (01-07-2020)} & \textbf{HABER} \\
        \hline
        96.500 & (572) Bancos e instituciones de crédito c/c vista, euros & \\
        \hline
        & (520) Deudas a corto plazo con entidades de crédito & 29.934,80 \\
        \hline
        & (170) Deudas a largo plazo con entidades de crédito & 66.565,20 \\
        \hline
    \end{tabular}
    \caption{Registro contable de la valoración inicial del préstamo.}
    \label{tabla:valoracion_inicial}
\end{table}

El siguiente registro es el correspondiente al pago de la primera cuota al final de 2020.

\begin{table}[H]
    \centering
    \begin{tabular}{|p{2cm}|p{6cm}|p{2cm}|}
        \hline
        \rowcolor{blue!30}
        \textbf{DEBE} & \textbf{Pago de la primera cuota (01-01-2021)} & \textbf{HABER} \\
        \hline
        3.450,48 & (662) Intereses de deudas & \\
        \hline
        14.704,52 & (520) Deudas a corto plazo con entidades de crédito & \\
        \hline
        & (572) Bancos e instituciones de crédito c/c vista, euros & 18.155 \\
        \hline
    \end{tabular}
    \caption{Pago de la primera cuota (01-01-2021).}
    \label{tabla:pago_primera_cuota}
\end{table}

El siguiente registro es el correspondiente al pago de la segunda cuota al final de 2021.

\begin{table}[H]
    \centering
    \begin{tabular}{|p{2cm}|p{6cm}|p{2cm}|}
        \hline
        \rowcolor{blue!30}
        \textbf{DEBE} & \textbf{Pago de la segunda cuota (01-07-2021)} & \textbf{HABER} \\
        \hline
        2.924,70 & (662) Intereses de deudas & \\
        \hline
        15.230,30 & (520) Deudas a corto plazo con entidades de crédito & \\
        \hline
        & (572) Bancos e instituciones de crédito c/c vista, euros & 18.155 \\
        \hline
    \end{tabular}
    \caption{Pago de la segunda cuota (01-07-2021).}
    \label{tabla:pago_segunda_cuota}
\end{table}
\newpage
\subsection*{\textcolor{blue}{Ejemplo 3}}

La entidad financiera SED, S.A. concede un crédito el 1-9-2018 a la entidad BIT, S.A. con las siguientes características:
\begin{itemize}
    \item Nominal: 150.000 €.
    \item Comisión de apertura (liquidada al inicio): 1.200 €.
    \item Tipo de interés anual: 6,75\%.
    \item Plazo: 3 años.
    \item Cuota anual constante: 56.987 €.
\end{itemize}

\textbf{SE PIDE:} Contabilizar la operación en la empresa BIT, S.A. para los ejercicios 2018 y 2019.
\\\\
\textbf{SOLUCIÓN:} \\\\
En primer lugar, vamos a dibujar el esquema de flujos de efectivo del ejemplo 3.
\begin{figure}[H]
    \centering
    \begin{tikzpicture}
        % Eje horizontal
        \draw[-] (0,0) -- (10,0) node[right] {Tiempo};
        
        % Eje vertical
        \draw[-{Latex}] (0,0) -- (0,5) node[above left] {Flujo de efectivo};
        
        % Flechas principales
        \draw[-{Latex},red] (0,0) -- (0,4) node[above left] {150.000 \texteuro};
        \draw[-{Latex},red] (0,0) -- (0,3) node[below left] {-1.200 \texteuro};
        
        % Flechas de pagos anuales
        \draw[-{Latex},red] (3,0) -- (3,-2) node[below] {56.987 \texteuro};
        \draw[-{Latex},red] (6,0) -- (6,-2) node[below] {56.987 \texteuro};
        \draw[-{Latex},red] (9,0) -- (9,-2) node[below] {56.987 \texteuro};
        
        % Nodos de tiempo en la línea del eje
        \draw (-0.3,0)  (-0.3,0.2) node[above] {0};
        \foreach \x/\t in { 3/1, 6/2, 9/3} {
            \draw (\x,0) -- (\x,0.2) node[above] {\t};
        }
    \end{tikzpicture}
    \caption{Esquema de flujos de efectivo del ejemplo 2.}
    \label{fig:flujo-efectivo-ejemplo2}
\end{figure}


\begin{table}[h]
    \centering
    \begin{tabular}{ccccc}
        \toprule
        \textbf{Plazo} & \textbf{Cuota} & \textbf{Capital} & \textbf{Intereses explícitos} & \textbf{Capital pendiente} \\
        \midrule
        01-09-2018 & & & & 150.000,00 \\
        01-09-2019 & 56.896,86 & 46.771,86 & 10.125,00 & 103.228,14 \\
        01-09-2020 & 56.896,86 & 49.928,97 & 6.967,90 & 53.299,17 \\
        01-09-2021 & 56.896,86 & 53.299,17 & 3.597,69 & 0 \\
        \bottomrule
    \end{tabular}
    \caption{Cuadro de amortización al 6,75\%}
    \label{tab:amortizacion_675}
\end{table}

\begin{figure}[H]
    \centering
    \begin{equation*}
        150.000 - 1.200 = \frac{56.897}{(1+i)} + \frac{56.897}{(1+i)^2} + \frac{56.897}{(1+i)^3}
    \end{equation*}
    \begin{equation*}
        i = 7,189457\%
    \end{equation*}
    \caption{Cálculo del interés efectivo.}
    \label{fig:interes-efectivo}
\end{figure}

\begin{table}[h]
    \centering
    \begin{tabular}{ccccc}
        \toprule
        \textbf{Plazo} & \textbf{Cuota} & \textbf{Capital} & \textbf{Intereses devengados} & \textbf{Coste amortizado} \\
        \midrule
        01-09-2018 & & & & 148.800,00 \\
        01-09-2019 & 56.896,86 & 46.198,95 & 10.697,91 & 102.601,05 \\
        01-09-2020 & 56.896,86 & 49.520,40 & 7.376,46 & 53.080,65 \\
        01-09-2021 & 56.896,86 & 53.080,65 & 3.816,21 & 0 \\
        \bottomrule
    \end{tabular}
    \caption{Cuadro de amortización a coste efectivo 7,189457\%}
    \label{tab:amortizacion_7189}
\end{table}

Vamos a proceder a la descomposición de los intereses.

\begin{table}[H]
    \centering
    \begin{tabular}{|p{2cm}|p{2cm}|p{2cm}|p{2cm}|p{2cm}|p{2cm}|p{2cm}|}
        \hline
        \rowcolor{yellow!30}
        \textbf{Plazo} & \textbf{Cuota} & \textbf{Intereses efectivos} & \textbf{Intereses explícitos} & \textbf{Intereses implícitos} & \textbf{Capital} & \textbf{Coste amortizado}\\
        \hline
        01-09-2018 &&&&&&148.800\\
        \hline
        01-09-2019 & 56.896,86 & 10.697,91 & 10.125,00 & 572,91 & 46.198,95 & 102.601,05\\
        \hline
        01-09-2020 & 56.896,86 & 7.376,46 & 6.967,90 & 408,56 & 49.520,40 & \\
        \hline
        01-09-2021 & 56.896,86 & 3.816,21 & 3.597,69 & 218,52 & 53.080,65 & \\
        \hline
    \end{tabular}
    \caption{Descomposición de intereses del cuadro de amortización a coste efectivo 7,189457\%.}
    \label{tab:descomposicion_intereses}
\end{table}

Ahora procedemos a la contabilización de la valoración inicial del préstamo.

\begin{table}[H]
    \centering
    \begin{tabular}{|c|c|p{2cm}|}
        \hline
        \rowcolor{blue!30}
        \textbf{DEBE} & \textbf{Reconocimiento inicial (01-09-2018)} & \textbf{HABER} \\
        \hline
        148.800 & (572) Bancos e instituciones de crédito c/c vista, euros & \\
        \hline
        & (170) Deudas a largo plazo con entidades de crédito & 102.601,05 \\
        \hline
        & (5200) Préstamos a corto plazo de entidades de crédito & 46.198,95 \\
        \hline
    \end{tabular}
    \caption{Reconocimiento inicial del préstamo.}
    \label{tabla:reconocimiento_inicial}
\end{table}

En el siguiente asiento se realizará el reconocimiento contable de los intereses devengados usando el ti efectivo.

En este caso no se produce el pago de la primera cuota.

Este asiento recoge los interses devengados y \c{no vencidos}.

\begin{itemize}
    \item La cuenta 6623 refleja los intereses en su totalidad.
    \item La 527 refleja los intereses devengados y no pagados.
    \item La parte de los intereses implícitos que no se pagan suponen un incremento del coste amortizado de la deuda a corto plazo, que se refleja en la cuenta 5200.
\end{itemize}

\begin{align*}
    \text{INTERÉS DEVENGADO} = \\ = [148\p800 \times (1 + 0,07189457)^{1/12} - 148\p800 = 3\p483,75 €] \\
    \\ 
    \text{INTERÉS DEVENGADO BANCO} = \\ = [150\p000 \times (1 + 0,0675)^{1/12} - 150\p000 = 3\p301,78 €]
\end{align*}

\begin{table}[H]
    \centering
    \begin{tabular}{|p{3cm}|p{6cm}|p{2cm}|}
        \hline
        \rowcolor{blue!30}
        \textbf{DEBE} & \textbf{Devengo de intereses (31-12-2018)} & \textbf{HABER} \\
        \hline
        3.483,75 & (6623) Intereses de deudas con entidades de crédito  & \\
        \hline
        & (527) Intereses a corto plazo de deudas con Entidades de crédito   & 3.301,78 \\
        \hline
        & (5200) Préstamos a corto plazo de entidades de crédito & 181,97 \\
        \hline
    \end{tabular}
    \caption{Devengo de intereses (31-12-2018).}
    \label{tabla:devengo_intereses}
\end{table}

Ahora procedemos a elaborar el asiento contable de los intereses devengados desde el 1-1-2019 hasta el 1-9-2019 (8 meses).
\begin{align*}
    \text{INTERÉS DEVENGADO} = \\
    = [(148\p800 + 3\p483,75) \times (1 + 0,07189457)^{8/12}] - (148\p800 + 3\p483,75) = 7\p214,14 € \\
    \\
    \text{INTERÉS DEVENGADO BANCO} = \\
    = [(150\p000 + 3\p301,78) \times (1 + 0,0675)^{8/12}] - (150\p000 + 3\p301,78) = 6\p823,20 €
\end{align*}

\begin{table}[H]
    \centering
    \begin{tabular}{|c|p{6cm}|c|}
        \hline
        \rowcolor{blue!30}
        \textbf{DEBE} & \textbf{Devengo de intereses (01-09-2019)} & \textbf{HABER} \\
        \hline
        7.214,14 & (6623) Intereses de deudas con entidades de crédito & \\
        \hline
        & (527) Intereses a corto plazo de deudas con Entidades de crédito & 6.823,20 \\
        \hline
        & (5200) Préstamos a corto plazo de entidades de crédito & 390,94 \\
        \hline
    \end{tabular}
    \caption{Devengo de intereses (01-09-2019).}
    \label{tabla:devengo_intereses_2019}
\end{table}

Ahora procedemos a la contabilización del pago de la primera cuota. 

\begin{table}[H]
    \centering
    \begin{tabular}{|p{2cm}|p{6cm}|p{2cm}|}
        \hline
        \rowcolor{blue!30}
        \textbf{DEBE} & \textbf{Pago de la cuota (01-09-2019)} & \textbf{HABER} \\
        \hline
        10.125,00 & (527) Intereses a corto plazo de deudas con Entidades de crédito & \\
        \hline
        46.771,86 & (5200) Préstamos a corto plazo de entidades de crédito & \\
        \hline
        & (572) Bancos e instituciones de crédito c/c vista, euros & 56.896,86 \\
        \hline
    \end{tabular}
    \caption{Pago de la cuota (01-09-2019).}
    \label{tabla:pago_cuota_2019}
\end{table}

Ahora debemos de reclasificar la deuda a largo plazo a corto plazo para el ejercicio 2019.

\begin{table}[H]
    \centering
    \begin{tabular}{|c|p{6cm}|c|}
        \hline
        \rowcolor{blue!30}
        \textbf{DEBE} & \textbf{Reclasificación de la deuda (01-09-2019)} & \textbf{HABER} \\
        \hline
        49.520,40 & (170) Deudas a largo plazo con entidades de crédito & \\
        \hline
        & (5200) Préstamos a corto plazo de entidades de crédito & 49.520,40 \\
        \hline
    \end{tabular}
    \caption{Reclasificación de la deuda (01-09-2019).}
    \label{tabla:reclasificacion_deuda_2019}
\end{table}
\begin{tcolorbox}[colframe=yellow!30!black, colback=white, sharp corners=all, boxrule=0.5pt]
    Los siguientes asientos contables son análogos a los anteriores, pero con las cuantías que corresponden al correspondiente ejercicio. Los cálculos de los intereses devengados se realizarán de manera análoga a los anteriores, por lo que no se incluirán en el documento.
\end{tcolorbox}


\begin{table}[H]
    \centering
    \begin{tabular}{|p{2cm}|p{6cm}|p{2cm}|}
        \hline
        \rowcolor{blue!30}
        \textbf{DEBE} & \textbf{Devengo de intereses (31-12-2019)} & \textbf{HABER} \\
        \hline
        2.402,12 & (6623) Intereses de deudas con entidades de crédito & \\
        \hline
        & (527) Intereses a corto plazo de deudas con Entidades de crédito & 2.272,23 \\
        \hline
        & (5200) Préstamos a corto plazo de entidades de crédito & 129,89 \\
        \hline
    \end{tabular}
    \caption{Devengo de intereses (31-12-2019).}
    \label{tabla:devengo_intereses_2019}
\end{table}

\begin{table}[H]
    \centering
    \begin{tabular}{|p{2cm}|p{6cm}|p{2cm}|}
        \hline
        \rowcolor{blue!30}
        \textbf{DEBE} & \textbf{Devengo de intereses (01-09-2020)} & \textbf{HABER} \\
        \hline
        4.974,27 & (6623) Intereses de deudas con entidades de crédito & \\
        \hline
        & (527) Intereses a corto plazo de deudas con Entidades de crédito & 4.695,65 \\
        \hline
        & (5200) Préstamos a corto plazo de entidades de crédito & 278,62 \\
        \hline
    \end{tabular}
    \caption{Devengo de intereses (01-09-2020).}
    \label{tabla:devengo_intereses_2020}
\end{table}

\begin{table}[H]
    \centering
    \begin{tabular}{|p{2cm}|p{6cm}|p{2cm}|}
        \hline
        \rowcolor{blue!30}
        \textbf{DEBE} & \textbf{Pago de la cuota (01-09-2020)} & \textbf{HABER} \\
        \hline
        6.967,88 & (527) Intereses a corto plazo de deudas con Entidades de crédito & \\
        \hline
        49.928,91 & (5200) Préstamos a corto plazo de entidades de crédito & \\
        \hline
        & (572) Bancos e instituciones de crédito c/c vista, euros & 56.896,86 \\
        \hline
    \end{tabular}
    \caption{Pago de la cuota (01-09-2020).}
    \label{tabla:pago_cuota_2020}
\end{table}

\begin{table}[H]
    \centering
    \begin{tabular}{|p{2cm}|p{6cm}|p{2cm}|}
        \hline
        \rowcolor{blue!30}
        \textbf{DEBE} & \textbf{Reclasificación de la deuda (01-09-2020)} & \textbf{HABER} \\
        \hline
        53.080,65 & (170) Deudas a largo plazo con Entidades de crédito & \\
        \hline
        & (5200) Préstamos a corto plazo de entidades de crédito & 53.080,65 \\
        \hline
    \end{tabular}
    \caption{Reclasificación de la deuda (01-09-2020).}
    \label{tabla:reclasificacion_deuda_2020}
\end{table}

\begin{table}[H]
    \centering
    \begin{tabular}{|p{2cm}|p{6cm}|p{2cm}|}
        \hline
        \rowcolor{blue!30}
        \textbf{DEBE} & \textbf{Devengo de intereses (31-12-2020)} & \textbf{HABER} \\
        \hline
        1.242,73 & (6623) Intereses de deudas con entidades de crédito & \\
        \hline
        & (527) Intereses a corto plazo de deudas con Entidades de crédito & 1.173,20 \\
        \hline
        & (5200) Préstamos a corto plazo de entidades de crédito & 69,53 \\
        \hline
    \end{tabular}
    \caption{Devengo de intereses (31-12-2020).}
    \label{tabla:devengo_intereses_2020}
\end{table}

\begin{table}[H]
    \centering
    \begin{tabular}{|c|p{6cm}|c|}
        \hline
        \rowcolor{blue!30}
        \textbf{DEBE} & \textbf{Devengo de intereses (01-09-2021)} & \textbf{HABER} \\
        \hline
        2.573,43 & (6623) Intereses de deudas con entidades de crédito & \\
        \hline
        & (527) Intereses a corto plazo de deudas con Entidades de crédito & 2.424,47 \\
        \hline
        & (5200) Préstamos a corto plazo de entidades de crédito & 148,96 \\
        \hline
    \end{tabular}
    \caption{Devengo de intereses (01-09-2021).}
    \label{tabla:devengo_intereses_2021}
\end{table}

\begin{table}[H]
    \centering
    \begin{tabular}{|p{2cm}|p{6cm}|p{2cm}|}
        \hline
        \rowcolor{blue!30}
        \textbf{DEBE} & \textbf{Pago de la cuota (01-09-2021)} & \textbf{HABER} \\
        \hline
        3.597,69 & (527) Intereses a corto plazo de deudas con Entidades de crédito & \\
        \hline
        53.299,18 & (5200) Préstamos a corto plazo de entidades de crédito & \\
        \hline
        & (572) Bancos e instituciones de crédito c/c vista, euros & 56.896,86 \\
        \hline
    \end{tabular}
    \caption{Pago de la cuota (01-09-2021).}
    \label{tabla:pago_cuota_2021}
\end{table}
\newpage
\subsection*{\textcolor{blue}{Ejemplo 4}}

El 1 de abril de 2018 la compañía CASA, S.A. encarga a la empresa INDUSTRIAL, S.A. la fabricación de una máquina especial, acordándose las siguientes condiciones: el precio global de la máquina es de 1.000.000 € presentando la siguiente estructura de financiación:

\begin{itemize}
    \item 1 de abril de 2018: Anticipo a cuenta por 300.000 € más IVA.
    \item Pago del 30\% más todo el IVA de la operación en el momento de entrega de la máquina (1 de enero de 2019).
    \item Por el resto, aplazamiento de pago mediante cuatro letras de 107.610,82 € cada una de ellas, con vencimientos anuales, desde la fecha de entrega de la máquina, según el siguiente detalle:
\end{itemize}

\begin{table}[H]
    \centering
    \begin{tabular}{|p{3cm}|p{2cm}|p{2cm}|p{2cm}|p{2cm}|p{2cm}|}
        \hline
        \rowcolor{blue!30}
        \textbf{Fecha} & \textbf{Importe letras} & \textbf{Capital final} & \textbf{Intereses} & \textbf{Amortización del principal} & \textbf{Coste amortizado} \\
        \hline
        1 enero 2019 & & & & & 400.000,00 \\
        \hline
        1 enero 2019 & 107.610,82 & 409.251,30 & 9.251,30 & 98.359,52 & 301.640,48 \\
        \hline
        1 enero 2020 & 107.610,82 & 308.616,90 & 6.976,42 & 100.634,40 & 201.006,08 \\
        \hline
        1 enero 2021 & 107.610,82 & 210.411,44 & 9.405,36 & 98.205,46 & 102.800,62 \\
        \hline
        1 enero 2022 & 107.610,82 & 107.610,81 & 4.810,19 & 102.800,62 & 0 \\
        \hline
        &&&30.443,27&400.000&\\
        \hline
    \end{tabular}
    \caption{Cuadro de amortización.}
    \label{tabla:cuadro_amortizacion_ej4}
\end{table}

El tipo de interés efectivo de la operación es 4,6791\% anual y el IVA vigente es del 21\% (deducible), que se paga por transferencia bancaria en el momento de su devengo.

\textbf{SE PIDE:} Reflejo contable en el libro diario de la compañía CASA las operaciones hasta el 1 de enero de 2020.

\textbf{SOLUCIÓN:} \\\\

Primero, debemos de reconocer el anticipo a cuenta por la fabricación de la máquina especial.

\begin{table}[H]
    \centering
    \begin{tabular}{|c|p{8cm}|c|}
        \hline
        \rowcolor{blue!30}
        \textbf{DEBE} & \textbf{Entrega del anticipo el 1/4/2018 por transferencia bancaria} & \textbf{HABER} \\
        \hline
        300.000 & (239) Anticipos para inmovilizaciones materiales & \\
        \hline
        63.000 & (472) Hacienda Pública IVA Soportado & \\
        \hline
        & (572) Banco c/c & 363.000 \\
        \hline
    \end{tabular}
    \caption{Entrega del anticipo el 1/4/2018.}
    \label{tabla:anticipo_2018}
\end{table}

A continuación, debemos de reconocer la recepción de la máquina el 1/1/2019.

\begin{table}[H]
    \centering
    \begin{tabular}{|p{2cm}|p{8cm}|c|}
        \hline
        \rowcolor{blue!30}
        \textbf{DEBE} & \textbf{Recepción de la máquina el 1/1/2019} & \textbf{HABER} \\
        \hline
        1.000.000 & (213) Maquinaria & \\
        \hline
        147.000 = 1.000.000 $\times$ 0,21 - 63.000& (472) Hacienda Pública, IVA Soportado & \\
        \hline
        & (239) Anticipos para inmovilizaciones materiales & 300.000 \\
        \hline
        & (175) Efectos a pagar a largo plazo & 201.006,08 \\
        \hline
        & (525) Efectos a pagar a corto plazo & 198.993,92 \\
        \hline
        & (572) Banco c/c & 447.000 \\
        \hline
    \end{tabular}
    \caption{Recepción de la máquina el 1/1/2019.}
    \label{tabla:recepcion_maquina}
\end{table}

Posteriormente, debemos de contabilizar el pago de la primera letra el 1/1/2019.

\begin{table}[H]
    \centering
    \begin{tabular}{|c|p{8cm}|c|}
        \hline
        \rowcolor{blue!30}
        \textbf{DEBE} & \textbf{Pago de la letra con vencimiento 1/1/2019} & \textbf{HABER} \\
        \hline
        98.359,52 & (525) Efectos a pagar a corto plazo & \\
        \hline
        9.251,30 & (662) Intereses de deudas & \\
        \hline
        & (572) Banco c/c & 107.610,82 \\
        \hline
    \end{tabular}
    \caption{Pago de la letra con vencimiento 1/1/2019.}
    \label{tabla:pago_letra_2019}
\end{table}

Contabilizamos la segunda letra con vencimiento el 1/1/2020.

\begin{table}[H]
    \centering
    \begin{tabular}{|c|p{8cm}|c|}
        \hline
        \rowcolor{blue!30}
        \textbf{DEBE} & \textbf{Pago de la letra con vencimiento 1/1/2020} & \textbf{HABER} \\
        \hline
        100.634,40 & (525) Efectos a pagar a corto plazo & \\
        \hline
        6.976,42 & (662) Intereses de deudas & \\
        \hline
        & (572) Banco c/c & 107.610,82 \\
        \hline
    \end{tabular}
    \caption{Pago de la letra con vencimiento 1/1/2020.}
    \label{tabla:pago_letra_2020}
\end{table}

\newpage
\subsection*{\textcolor{blue}{Ejemplo 5}}

La entidad NUEVA, S.A. compra el 01/04/2019 una máquina valorada en 491.525,43 euros (más IVA del 18\%). El pago se realizará de la siguiente forma:
\begin{itemize}
    \item La mitad a través de transferencia bancaria en el momento de la compra.
    \item El resto se pagará en 2 cuotas anuales de igual cantidad a pagar dentro de uno y dos años respectivamente, de acuerdo al siguiente cuadro de amortización (interés del 6\%):
\end{itemize}

\begin{table}[H]
    \centering
    \begin{tabular}{|c|c|c|c|}
        \hline
        \rowcolor{blue!30}
        \textbf{Momento Temporal} & \textbf{Capital amortizado} & \textbf{Intereses} & \textbf{Cuota anual} \\
        \hline
        01/04/2020 & 140.776,70 & 17.000,00 & 158.176,70 \\
        \hline
        01/04/2022 & 149.223,30 & 8.953,40 & 158.176,70 \\
        \hline
        \textbf{TOTAL} & 290.000,00 & 26.353,40 & 316.353,40 \\
        \hline
    \end{tabular}
    \caption{Cuadro de amortización.}
    \label{tabla:cuadro_amortizacion_ej5}
\end{table}

Además, la máquina necesita un periodo de instalación de 18 meses, por lo que no podrá ponerse en funcionamiento hasta el 30/09/2020.

\textbf{SE PIDE:} Contabilizar para la empresa NUEVA, S.A. las siguientes operaciones:
\begin{enumerate}[label=\alph*)]
    \item Registro contable de la compra de la máquina el 01/04/2019.
    
    Sabemos que:
    \begin{itemize}
        \item Necesita un periodo de instalación de 18 meses.
        \item El pago se realizará en dos cuotas anuales de igual cantidad.
    \end{itemize}

    \begin{table}[H]
        \centering
        \begin{tabular}{|c|p{8cm}|p{2cm}|}
            \hline
            \rowcolor{blue!30}
            \textbf{DEBE} & \textbf{Registro contable de compra de la máquina (01/04/2019)} & \textbf{HABER} \\
            \hline
            491.525,43 & (233) Maquinaria en montaje & \\
            \hline
            88.474,57 & (472) HP IVA soportado & \\
            \hline
            & (572) Bancos c/c & 290.000,00 \\
            \hline
            & (523) Proveedores de Inmovilizado a corto plazo & 140.776,70 \\
            \hline
            & (173) Proveedores de Inmovilizado a largo plazo & 149.223,30 \\
            \hline
        \end{tabular}
        \caption{Registro contable de compra de la máquina (01/04/2019).}
        \label{tabla:compra_maquina}
    \end{table}



    \item Devengo de intereses al cierre del ejercicio 31/12/2019.
    
    \begin{align*}
        I_{\text{totales}} = \\
        = 290.000 \times 1,06^{\frac{9}{12}} - 290.000 = 12.954,49 €
    \end{align*}

    Cuando se pagan estos intereses, no se usa la cuenta de Bancos, si no que aumenta la deuda.

    \begin{table}[H]
        \centering
        \begin{tabular}{|c|p{6cm}|c|}
            \hline
            \rowcolor{blue!30}
            \textbf{DEBE} & \textbf{Devengo de intereses al cierre del ejercicio 31/12/2019} & \textbf{HABER} \\
            \hline
            12.954,49 & (662) Intereses de Deudas & \\
            \hline
            & (523) Proveedores de Inmovilizado a corto plazo & 12.954,49 \\
            \hline
        \end{tabular}
        \caption{Devengo de intereses al cierre del ejercicio 31/12/2019.}
        \label{tabla:devengo_intereses_2019}
    \end{table}

    \item Capitalización de intereses a 31/12/2019.

    \textit{De la asignatura de Contabilidad Financiera I sabemos que los inmovilizados que necesitan un periodo de tiempo superior a un año para estar en condiciones de funcionamiento, se incluirán en el precio de adquisición o coste de producción los gastos fianancieros que se hayan devengado abtes dea fecha de puesta en funcionamiento.}
    
    Como en el enunciado se expone que necesita un periodo de instalación de 18 meses, si cumple el requisito.

    Contabilizamos la capitalización de los intereses usando la cuenta 736.

    \begin{table}[H]
        \centering
        \begin{tabular}{|c|c|c|c|}
            \hline
            \rowcolor{blue!30}
            \textbf{DEBE} & \textbf{Capitalización de intereses a 31/12/2019} & \textbf{HABER} \\
            \hline
            12.954,49 & (233) Maquinaria en montaje & \\
            \hline
            & (736) Incorporación al activo de gastos financieros & 12.954,49 \\
            \hline
        \end{tabular}
        \caption{Capitalización de intereses a 31/12/2019.}
        \label{tabla:capitalizacion_intereses}
    \end{table}
    

    \item Devengo de intereses a 01/04/2020.
    
    \begin{align*}
        I_{\text{totales}} = \\
        = (290.000 + 12.954,49) \times 1,06^{\frac{3}{12}} - (290.000 + 12.954,49) = 4.445,51 €
    \end{align*}

    \begin{table}[H]
        \centering
        \begin{tabular}{|c|p{6cm}|c|}
            \hline
            \rowcolor{blue!30}
            \textbf{DEBE} & \textbf{Devengo de intereses a 01/04/2020} & \textbf{HABER} \\
            \hline
            4.445,51 & (662) Intereses de Deudas & \\
            \hline
            & (523) Proveedores de Inmovilizado a corto plazo & 4.445,51 \\
            \hline
        \end{tabular}
        \caption{Devengo de intereses a 01/04/2020.}
        \label{tabla:devengo_intereses_abril_2020}
    \end{table}

    \item Pago de la primera cuota el 01/04/2020.

    Ahora se debe de hacer frente a la primera cuota. La cantidad a pagar es de 158.176,70 €, donde 17.400 € son intereses y 140.776,70 € es pago de deuda.

    Repasando las operaciones que han ocurrido:
    \begin{itemize}
        \item 01.04.2019: deuda de 140.776,70 €.
        \item 31.12.2019: deuda de 12.954,49 €.
        \item 01.04.2020: deuda de 4.445,51 € de interreses.
    \end{itemize}

    La deuda total asciende a 158.176,70 €.

    \begin{table}[H]
        \centering
        \begin{tabular}{|p{2cm}|p{6cm}|p{2cm}|}
            \hline
            \rowcolor{blue!30}
            \textbf{DEBE} & \textbf{Pago de la primera cuota a 01/04/2020} & \textbf{HABER} \\
            \hline
            158.176,70 & (523) Proveedores de Inmovilizado a corto plazo & \\
            \hline
            & (572) Bancos c/c & 158.176,70 \\
            \hline
        \end{tabular}
        \caption{Pago de la primera cuota a 01/04/2020.}
        \label{tabla:pago_primera_cuota_2020}
    \end{table}

    \item Devengo de intereses a 30/09/2020.
    
    Se debe de registrar debido a que es cuando la máquina está en condiciones de funcionamiento. Como ya se ha pagado la primera cuota, la deuda asciende a 149.223,30 €.

    \begin{align*}
        I_{\text{totales}} = \\
        = 149.223,30 \times 1,06^{\frac{6}{12}} - 149.223,30 = 4.411,49 €
    \end{align*}

    \begin{table}[H]
        \centering
        \begin{tabular}{|c|p{6cm}|c|}
            \hline
            \rowcolor{blue!30}
            \textbf{DEBE} & \textbf{Devengo de intereses a 30/09/2020} & \textbf{HABER} \\
            \hline
            4.411,49 & (662) Intereses de Deudas & \\
            \hline
            & (523) Proveedores de Inmovilizado a corto plazo & 4.411,49 \\
            \hline
        \end{tabular}
        \caption{Devengo de intereses a 30/09/2020.}
        \label{tabla:devengo_intereses_septiembre_2020}
    \end{table}

    \item Capitalización de intereses al finalizar la instalación a 30/09/2020.
    
    La cantidad a capitalizar es de 4.411,49 € + 4.445,51 € = 8.857,00 €.

    \begin{table}[H]
        \centering
        \begin{tabular}{|c|p{8cm}|c|}
            \hline
            \rowcolor{blue!30}
            \textbf{DEBE} & \textbf{Capitalización de intereses al finalizar la instalación a 30/09/2020} & \textbf{HABER} \\
            \hline
            8.857,00 & (233) Maquinaria en montaje & \\
            \hline
            & (736) Incorporación al activo de gastos financieros & 8.857,00 \\
            \hline
        \end{tabular}
        \caption{Capitalización de intereses al finalizar la instalación a 30/09/2020.}
        \label{tabla:capitalizacion_intereses_septiembre_2020}
    \end{table}

    A partir de este momento, los intereses generados por la deuda con los proveedores de inmovilizado, aunque se contabilicen, no se prodrán capitalizar.

    Hay una \textbf{cuestión importante} que no piden contabilizar en este ejercicio, pero que se debe tener en cuenta. El 30/09/2020 la máquina está en condiciones de funcionamiento, por lo tanto, en esa fecha se debería reflejar contablemente, dando de baja la cuenta (233) Maquinaria en montaje y cargando la cuenta (213) Maquinaria, por el importe total de todos los cargos efectuados en la cuenta de inmovilizado en curso desde que se adquirió ese inmovilizado hasta que está terminado.
    

    \item Devengo de intereses al cierre del ejercicio 31/12/2020.
    
    \begin{align*}
        I_{\text{totales}} = \\
        = (149.223,30 + 4.411,49) \times 1,06^{\frac{3}{12}} - (149.223,30 + 4.411,49) = 2.254,41 €
    \end{align*}

    \begin{table}[H]
        \centering
        \begin{tabular}{|c|p{6cm}|c|}
            \hline
            \rowcolor{blue!30}
            \textbf{DEBE} & \textbf{Devengo de intereses al cierre del ejercicio 31/12/2020} & \textbf{HABER} \\
            \hline
            2.254,41 & (662) Intereses de Deudas & \\
            \hline
            & (523) Proveedores de Inmovilizado a corto plazo & 2.254,41 \\
            \hline
        \end{tabular}
        \caption{Devengo de intereses al cierre del ejercicio 31/12/2020.}
        \label{tabla:devengo_intereses_2020}
    \end{table}

    \item Reclasificación de la deuda a 31/12/2020.
    
    Como la empresa decide cancelar el pasivo en abril de 2021, se debe de reclasificar la deuda indicando que ahora es a corto plazo.

    \begin{table}[H]
        \centering
        \begin{tabular}{|c|p{8cm}|c|}
            \hline
            \rowcolor{blue!30}
            \textbf{DEBE} & \textbf{Reclasificación de la deuda a 31/12/2020} & \textbf{HABER} \\
            \hline
            149.223,30 & (173) Proveedores de Inmovilizado a largo plazo & \\
            \hline
            & (523) Proveedores de Inmovilizado a corto plazo & 149.223,30 \\
            \hline
        \end{tabular}
        \caption{Reclasificación de la deuda a 31/12/2020.}
        \label{tabla:reclasificacion_deuda_2020}    
    \end{table}

    La deuda total  a 31/12/2020 es de 149.223,30 € que se acaban de registrar más los 4.411,49 € de intereses (septiembre) y los 2.254,41 € de intereses (diciembre), que asciende a 155.888,20 €.



    \item Devengo de intereses a 01/04/2021.
    
    \begin{align*}
        I_{\text{totales}} = \\
        = (149.223,30 + 4.411,49 + 2.254,41) \times 1,06^{\frac{3}{12}}\\ - (149.223,30 + 4.411,49 + 2.254,41) = 2.287,50 €
    \end{align*}

    \begin{table}[H]
        \centering
        \begin{tabular}{|c|p{6cm}|c|}
            \hline
            \rowcolor{blue!30}
            \textbf{DEBE} & \textbf{Devengo de intereses a 01/04/2021} & \textbf{HABER} \\
            \hline
            2.287,50 & (662) Intereses de Deudas & \\
            \hline
            & (523) Proveedores de Inmovilizado a corto plazo & 2.287,50 \\
            \hline
        \end{tabular}
        \caption{Devengo de intereses a 01/04/2021.}
        \label{tabla:devengo
        _intereses_abril_2021}
    \end{table}

    \item Pago de la segunda cuota a 01/04/2021.
    
    La deuda total a 01/04/2021 es de 149.223,30 €(amortizacion y todo lo demás interreses) + 4.411,49 € + 2.254,41 € + 2.287,50 € = 158.176,70 €.

    \begin{table}[H]
        \centering
        \begin{tabular}{|c|p{8cm}|c|}
            \hline
            \rowcolor{blue!30}
            \textbf{DEBE} & \textbf{Pago de la segunda cuota a 01/04/2021} & \textbf{HABER} \\
            \hline
            158.176,70 & (523) Proveedores de Inmovilizado a corto plazo & \\
            \hline
            & (572) Bancos c/c & 158.176,70 \\
            \hline
        \end{tabular}
        \caption{Pago de la segunda cuota a 01/04/2021.}
        \label{tabla:pago_segunda_cuota_2021}
    \end{table}

\end{enumerate}

\newpage
\section{Pasivos financieros a valor razonble}

\begin{itemize}
    \item Valoración inicial: VR, que equivaldrá a el precio de la transacción. Los costes de transacción se contabilizarán en la cuenta de pérdidas y ganancias.
    \item Valoración posterior: A VR, los cambios/diferencias en el VR se contabilizarán en la cuenta de pérdidas y ganancias.
\end{itemize}


\subsection*{\textcolor{blue}{Ejemplo 6}}

La empresa WERT, S.A. emitió el 2 de noviembre de 2020 obligaciones que cotizan en la bolsa de Madrid con las siguientes condiciones: una prima de emisión del 5\% y la intención de recuperarlas en el corto plazo. Los gastos de emisión ascendieron a 4.000 €. El valor en libros a 31 de diciembre de 2020 de este empréstito es de 4.400.000 € (4.000 títulos y valor nominal 1.000 €) registrado en la partida del balance “Valores representativos de deuda a corto plazo” (VRD).

El 3 de febrero de 2021 la empresa decide adquirir 2.000 de estos títulos de la bolsa de Madrid que cotizan al 125\% de su valor nominal, teniendo en cuenta que los gastos de la operación ascienden a un 0,1\%.

\textbf{SE PIDE:}

\begin{enumerate}[label=\textbf{\alph*})]
    
\item Analizar la posibilidad de clasificación del pasivo financiero.

En el enunciado se nos dice que clasificó su emisión como VRD a c/p, esto podría implicar que la intención es de una emisión de readquisición a corto plazo. Esta es una de las características de los pasivos financieros valorados a VR con cambios en la cuenta de pérdidas y ganancias (la empresa debe de haberlo clasificado como pasivo con valoración posterior a VR). Además se emite en el mercado y presentan un valor razonable fiable. 



\item Reflejar contablemente, en función de la cartera de clasificación elegida en el apartado A, la emisión de los títulos.

\begin{table}[H]
    \centering
    \begin{tabular}{|p{3cm}|p{6cm}|p{3cm}|}
    \hline
    \rowcolor{blue!30}
    \textbf{DEBE} & \textbf{Emisión del empréstito 02.11.2020} & \textbf{HABER} \\
    \hline
    3.796.000&  \cuenta{572}& \\
    \hline
    4.000&  \cuenta{669}& \\
    \hline
    &  \cuenta{500}
        Valor de emisión = 4.000 títulos x 1000 =
        4.000.000 -5\% \text{de la prima de emisión} = 
        4.000.000 - 200.000 = 
        3.800.000
    & 3.800.000\\
    \hline
    \end{tabular}
    \caption{Asiento 1. Ejercicio 6.}
    \label{tabla:asiento1ej6-2}
\end{table}


\item Reflejar contablemente, en función de la cartera de clasificación elegida en el apartado A, el reembolso de los títulos.

Debemos de amortizar los títulos, en este caso, como estos títulos cotizan en el mercado secundario, a final del ejercicio se había producido un incremento del VR de los títulos. Por lo que el valor del título era de 1.100 \e y un beneficio que se imputó en la Cuenta de PyG.

En esta fecha se deben de sacar del mercado (decide amortizar) 2.000 títulos a 1.250 \e/título, lo que supone que si comparamos este valor en libros a 31 de Diciembre, una pérdida de 150 \e/título.

\begin{table}[H]
    \centering
    \begin{tabular}{|p{3cm}|p{6cm}|p{3cm}|}
    \hline
    \rowcolor{blue!30}
    \textbf{DEBE} & \textbf{} & \textbf{HABER} \\
    \hline
    2.200.000 & (500) Obligaciones y bonos a corto plazo \newline Damos de baja los títulos por su valor contable = \newline 4.400.000 €/4.000 títulos = 1.100 € título x 2.000 títulos & \\
    \hline
    2.500 & (669) Otros gastos financieros \newline Gastos = 0,001 x 2.500.000 € = 2.500 € & \\
    \hline
    300.000 & (675) Pérdidas por operaciones con obligaciones propias & \\
    \hline
    & (572) Banco X, cuenta corriente \newline (valor razonable 125\%/s/VN = 1.250 € x 2.000 títulos = \newline 2.500.000 € más 2.500 € de gastos) & 2.502.500 \\
    \hline
    \end{tabular}
    \caption{Amortización del empréstito* (03-02-2021).}
    \label{tabla:Asiento2-Ejercicio6-2}
\end{table}

\end{enumerate}



% \subsection{Valoración de los pasivos financieros}

% \subsubsection{A coste amortizado}
% \begin{itemize}
%     \item Valoración inicial: Inicialmente a VR, es decir, el precio de la transacción. Consideramos los costes que son para emitirlos. Por ejemplo, si pido un préstamo de una cuantía X, el importe extra por gestión, se debe de incluir como un mayor importe en el valor del pasivo.
%     \item Valoración posterior: Por su coste amortizado, los interéses devengados, se contabilizarán en la cuenta de pérdidas y ganancias, aplicando el método del tanto efectivo.
% \end{itemize}

% \textit{Nota: Debemos de tener en cuenta que la empresa que recibe el préstamo tiene un pasivo, mientras que la empresa que lo da tiene un activo.}


\newpage 
\section{Reclasificación y baja de PF}

Una entidad no puede reclasificar los pasivos financieros de una categoría a otra. La empresa dará de baja a un pasivo financiero, o parte del mismo, cuando la obligación se haya extinguido. No obtante, se define de la misma forma la posibilidad de que se lleve a cabo un intercambio de instrumentos de deuda entre \c{prestamista} y \c{prestatario}, o lo que se conoce como \c{refinación de la deuda}. Se genera en condiciones similares o, por si el contrario, las condiciones son sustancialmente diferentes.

\begin{enumerate}[label=$\rightarrow$]
    \item Si las condiciones son diferentes, se reconocerá la baja del pasivo financiero original y se reconocerá el nuevo pasivo que surja. Las diferencias entre ambos se contabilizarán en la cuenta de pérdidas y ganancias. La diferencia se determinará entre el valor en libros del pasivo financiero o de la parte del mismo que se da de baja y la contraprestación pagada incluidos los costes o comisiones en que se incurra y en la que se recogerá asimismo cualquier activo cedido diferente del efectivo o pasivo asumido, se reconocerá en la cuenta de pérdidas y ganancias de ejercicio que tenga lugar.
    \item Si las condiciones son similares, el pasivo no se dará de baja. Cualquier coste de transacción o comisión  incurrida ajustará el valor en libros del pasivo financiero. A partir de esa fecha, el coste amortizado del pasivo financiero se determinará aplicando el tipo de interés efectivo que iguale el valor en libros del pasivo financiero con los flujos de efectivo a pagar según las nuevas condiciones. 
\end{enumerate}

Para establecer cuando las condiciones de los dos pasivos financieros son distintos, tenemos dos tests para su determinación:

\begin{enumerate}[label=\arabic*)]
    \item \b{Test cualitativo}: consiste en determinar si el valor actual de los flujos de efectivo difiere en al menos un 10\% entre ambos pasivos financieros. Cuando el valor de flujos de efectivo del nuevo contrato, incluida cualquier comisión pagada neta de caulquier comisión recibida, difiera al menos en un 10\% del valor de los flujos de efectivo del pasivo financiero original, actualizados ambos importes al tipo de interés efectivo de este último.
    \item \textbf{Test cualitativo}: hace referencia a la existencia de ciertas modificaciones de los flujos de efectivo de los pasivos que no superan las condiciones del test cuantitativo, pero que por sus condiciones conllevan una modificación sustancial del pasivo. A modo de ejemplo, citamos los siguientes:
    \begin{enumerate}[label=$\rightarrow$]
        \item Un cambio en el tipo de interés fijo a variable en la remuneración del pasivo,
        \item La re-expresión del pasivo a una divisa distinta,
        \item Un bono a tipo de interés fijo renegociado a un bono a tipo variable,
        \item Un préstamo a tipo de interés fijo que se convierte en un préstamo participativo, entre otros casos.
    \end{enumerate}
\end{enumerate}

\newpage
\subsection*{\textcolor{blue}{Ejemplo 7}}

La empresa VENKA, S.A. concierta el 1 de julio de 2020 una operación de préstamo con la entidad financiera LOPRESTOTODO, S.A. por importe de 100.000 € a 3 años con pagos semestrales con un tipo de interés nominal anual del 5\%, con una comisión de estudio\footnote{Este gasto hace referencia al estudio de la viabilidad del préstamo en base a la evaluación de riesgos.} del 2,5\% y comisión de apertura del 1\%. Al final de los primeros pagos de las cuotas se le plantean dos opciones:

\begin{itemize}
    \item[\textbf{A)}] Refinanciar la deuda con un interés anual nominal del 4\% a 3 años de duración con cuotas anuales constantes, sin comisiones.
    \item[\textbf{B)}] Sustituir el préstamo por un pagaré con vencimiento a 5 años y un nominal de 60.000 € con un tipo de interés de mercado del 10\%.
\end{itemize}

\subsection*{SE PIDE:}

\begin{enumerate}
    \item Analizar la clasificación del pasivo financiero según el Plan General de Contabilidad (PGC).
    \item Elaborar los cuadros de amortización del préstamo para la entidad financiera y para VENKA, considerando el coste amortizado.
    \item Registrar contablemente las dos primeras cuotas del préstamo inicial.
    \item Registrar contablemente la primera cuota del nuevo préstamo en la opción A, así como las nuevas cuotas derivadas de la opción B.
\end{enumerate}

\subsubsection*{Solución Opción A}

\begin{enumerate}
    \item En este caso, se dudece de la lectura de la NV 9ª del PGC que esta operación no puede ser \c{Pasivos Financieros cambios a VR}, si no que es un \c{Pasivo Financiero a coste amortizado}.
    \item Calculamos el cuadro de amortización, y para ello debemos de tener en cuenta:
    \begin{itemize}
        \item Calcular el VR inicial, teniendo en cuenta los 100.000 \e - gastos de transacción, que en este caso es la comisión de estudio y la comisión de apertura (2.500 + 1.000 = 3.500 \e).
        \item Tipo de interés efectivo es de 7,28\%, o bien un 3,58 \% semestral\footnote{Los cuadros que se muestran nos lo proporcionan en el examen, aunque hay que saber como se elaboran.}.
    \end{itemize}
\end{enumerate}

\begin{table}[H]
    \centering
    \begin{tabular}{|p{2cm}|p{2cm}|p{2cm}|p{2cm}|p{2cm}|p{2cm}|}
    \hline
    \rowcolor{blue!30}
    \textbf{Fecha} & \textbf{Cuota} & \textbf{Interés} & \textbf{Capital} & \textbf{Capital Amortizado Acumulado} & \textbf{Capital Pendiente} \\
    \hline
    01-07-20 & 18.155,00 & 2.500,00 & 15.655,00 & 15.655,00 & 84.345,00 \\
    \hline
    01-01-21 & 18.155,00 & 2.108,63 & 16.046,37 & 31.701,37 & 68.298,63 \\
    \hline
    01-07-21 & 18.155,00 & 1.707,47 & 16.447,53 & 48.148,90 & 51.851,10 \\
    \hline
    01-01-22 & 18.155,00 & 1.296,28 & 16.858,72 & 65.007,62 & 34.992,38 \\
    \hline
    01-07-22 & 18.155,00 & 874,81 & 17.280,19 & 82.287,81 & 17.712,19 \\
    \hline
    01-01-23 & 18.154,99 & 442,80 & 17.712,19 & 100.000,00 & 0,00 \\
    \hline
    \end{tabular}
    \caption{Cuadro de amortización tipo interés nominal de la entidad financiera (5\%).}
    \label{tabla:Asiento1-Ejercicio7-tema2}
\end{table}

Ahora debemos de calcular el nuevo cuadro de en base al tipo de interés del 7,28\% anual.

\begin{table}[H]
    \centering
    \begin{tabular}{|p{2cm}|p{2cm}|p{2cm}|p{2cm}|p{2cm}|}
    \hline
    \rowcolor{blue!30}
    \textbf{Fecha} & \textbf{Cuota} & \textbf{Interés} & \textbf{Capital} & \textbf{Coste Amortizado} \\
    \hline
    01-07-20 & 18.155,00 & 3.450,48 & 14.704,52 & 96.500,00 \\
    \hline
    01-01-21 & 18.155,00 & 2.924,70 & 15.230,30 & 81.795,50 \\
    \hline
    01-07-21 & 18.155,00 & 2.380,13 & 15.774,87 & 66.565,20 \\
    \hline
    01-01-22 & 18.155,00 & 1.816,07 & 16.338,93 & 50.790,33 \\
    \hline
    01-07-22 & 18.155,00 & 1.231,85 & 16.923,15 & 34.451,40 \\
    \hline
    01-01-23 & 18.154,99 & 626,74 & 17.528,25 & 17.528,25 \\
    \hline
    \end{tabular}
    \caption{Cuadro de amortización tipo interés efectivo (7,28\%) con interés efectivo semestral (3,58\%).}
    \label{tabla:amortizacion_efectivo}
\end{table}

Una vez llegado al pago de la 2º deuda, debemos de estudiar la renegociación de la deuda. La deuda que queda pendiente es 68.298,63 \e. Teniendo en cuenta las condiciones:
\begin{itemize}
    \item Refinanciación de la deuda.
    \item 4\% de interés anual nominal.
    \item 3 años de duración.
    \item Cuotas anuales constantes.
\end{itemize}

El nuevo cuadro que nos queda es:

\begin{table}[H]
    \centering
    \begin{tabular}{|p{2cm}|p{2cm}|p{2cm}|p{2cm}|p{2cm}|p{2cm}|}
    \hline
    \rowcolor{blue!30}
    \textbf{Fecha} & \textbf{Cuota} & \textbf{Interés} & \textbf{Capital} & \textbf{Capital Amortizado Acumulado} & \textbf{Capital Pendiente} \\
    \hline
    01-07-22 & 24.611,31 & 2.731,95 & 21.879,36 & 21.879,36 & 46.419,27 \\
    \hline
    01-07-23 & 24.611,31 & 1.856,77 & 22.754,54 & 44.633,90 & 23.664,73 \\
    \hline
    01-07-24 & 24.611,32 & 946,59 & 23.664,73 & 68.298,63 & 0,00 \\
    \hline
    \end{tabular}
    \caption{Cuadro de amortización tipo interés nominal (4\%) para la nueva deuda opción A.}
    \label{tabla:Tabla3-Ejercicio7-tema2}
\end{table}

\c{Debemos de calcular el nuevo valor actual:}

\begin{align*}
    \text{Valor actual del Préstamo Opción A con condiciones iniciales } = \\
    \text{64.258,86} = \frac{\text{24.611,31}}{1,0728} + \frac{\text{24.611,31}}{1,0728^2} + \frac{\text{24.611,31}}{1,0728^3}
\end{align*}

\begin{tcolorbox}[colback=blue!5!white,colframe=blue!75!black]
    \textbf{Nota:} Dado que el importe pendiente tras el pago de las dos primeras cuotas a coste amortizado es de 66.565,20 \e y el valor actual del nuevo préstamo con las características del préstamo original es de 64.258,86 \e, \textit{la diferencia de 2.306,34 \e es menor que el 10\% sobre el importe de 66.565,20 \e,} lo que implica que \textit{no difieren sustancialmente, por lo que el pasivo original no se dará de baja del balance, Por tanto, podemos seguir contabilizando el mismo importe pendiente de la deuda original.}
    
\end{tcolorbox}

\begin{align*}
    \text{Nuevo tipo de interés efectivo}  = \\
    \text{66.565,20} = \\
    \frac{\text{24.611,31}}{1 + i} + \frac{\text{24.611,31}}{(1 + i)^2} + \frac{\text{24.611,31}}{(1 + i)^3} \rightarrow \\
    \rightarrow i = 5,3664\%
\end{align*}

\begin{table}[H]
    \centering
    \begin{tabular}{|p{2cm}|p{2cm}|p{2cm}|p{2cm}|p{2cm}|p{2cm}|}
    \hline
    \rowcolor{blue!30}
    \textbf{Fecha} & \textbf{Cuota} & \textbf{Interés} & \textbf{Capital} & \textbf{Capital Amortizado Acumulado} & \textbf{Coste Amortizado} \\
    \hline
    01-07-21 & - & - & - & - & 66.365,20 \\
    \hline
    01-07-22 & 24.611,31 & 3.572,15 & 21.039,16 & 21.039,16 & 45.526,04 \\
    \hline
    01-07-23 & 24.611,31 & 2.443,10 & 22.168,20 & 43.207,36 & 23.357,83 \\
    \hline
    01-07-24 & 24.611,31 & 1.253,47 & 23.357,83 & 66.565,20 & 0,00 \\
    \hline
    \end{tabular}
    \caption{Cuadro de amortización a coste amortizado tipo interés efectivo (5,3663932\%).}
    \label{tabla:amortizacion_coste_amortizado}
\end{table}


\begin{table}[H]
    \centering
    \begin{tabular}{|p{2cm}|p{6cm}|p{2cm}|}
    \hline
    \rowcolor{blue!30}
    \textbf{DEBE} & \textbf{} & \textbf{HABER} \\
    \hline
    21.039,16 & (170) Deudas a largo plazo con entidades financieras & \\
    \hline
    & (520) Deudas a corto plazo con entidades financieras&  21.039,16 \\
    \hline
    \end{tabular}
    \caption{Reclasificación de la deuda de la primera cuota del largo plazo al corto plazo (1-7-2021).}
    \label{tabla:reclasificacion}
\end{table}

\begin{table}[H]
    \centering
    \begin{tabular}{|p{2cm}|p{6cm}|p{2cm}|}
    \hline
    \rowcolor{blue!30}
    \textbf{DEBE} & \textbf{} & \textbf{HABER} \\
    \hline
    1.762,73 & (662) Intereses de deudas & \\
    \hline
    & INTERÉS DEVENGADO = $[\text{66.565,20} \times (1 + 0,053663932)^(1/2)] - 66.565,20 = \newline \text{1.762,73539} € $& \\
    \hline
    &(527) Intereses a corto plazo de deudas con Entidades de crédito &  1.762,73 \\
    \hline
    \end{tabular}
    \caption{Reflejo contable de los intereses devengados de la primera cuota (31-12-2021).}
    \label{tabla:intereses_2021}
\end{table}
\begin{table}[H]
    \centering
    \begin{tabular}{|p{2cm}|p{6cm}|p{2cm}|}
    \hline
    \rowcolor{blue!30}
    \textbf{DEBE} & \textbf{} & \textbf{HABER} \\
    \hline
    1.809,41 & (662) Intereses de deudas & \\
    \hline
     & INTERÉS DEVENGADO = $[(\text{66.565,20} + \text{1.762,73}) \times (1 + 0,053663932)^{(1/2)}] - \newline (\text{66.565,20} + \text{1.762,73}) = \text{1.809,41} € $& \\
    \hline
    &(527) Intereses a corto plazo de deudas con Entidades de crédito &  1.809,41 \\
    \hline
    \end{tabular}
    \caption{Reflejo contable de los intereses devengados de la primera cuota en el año 2022 (1-7-2022).}
    \label{tabla:intereses_2022}
\end{table}
\begin{table}[H]
    \centering
    \begin{tabular}{|p{2cm}|p{6cm}|p{2cm}|}
    \hline
    \rowcolor{blue!30}
    \textbf{DEBE} & \textbf{} & \textbf{HABER} \\
    \hline
    3.572,15 & (527) Intereses a corto plazo de deudas con Entidades de crédito & \\
    \hline
    21.039,16 & (520) Deudas a corto plazo con entidades financieras & \\
    \hline
    &(572) Banco X c/c &  24.611,31 \\
    \hline
    \end{tabular}
    \caption{Reflejo contable del pago de la primera cuota (1-7-2022).}
    \label{tabla:pago_cuota}
\end{table}


\subsubsection*{Opción B}

Debemos de proceder de igual manera a realizar el test cuantitativo cuyo límite es del 10\% sobre el mismo importe de 66.565,30 \e. Debemos de calcular el valor actual del nuevo pagaré.

\begin{align*}
    \text{Valor actual pagaré} = \\
    \frac{\text{60.000}}{(1,0728)^5} = 42.233,81 \e
\end{align*}

El límite entre ambas valoraciones es de 6.656,52 \e = 66.565,20 - 42.223,81 \e. Por lo que la diferencia es de 24.331,39 \e, contablemente debemos de dar de baja a la antigua operación de endeudamiento y dar de alta a la nueva, imputando la diferencia en la cuenta de pérdidas y ganancias.

\begin{align*}
    \text{Valor actual del pagaré} = \\
    \frac{\text{60.000}}{1,1^5} = 37.255,28 \e
\end{align*}

\begin{table}[H]
    \centering
    \begin{tabular}{|p{2cm}|p{2cm}|p{2cm}|}
    \hline
    \rowcolor{blue!30}
    \textbf{Vencimientos Fecha} & \textbf{Cuota de Interés} & \textbf{Capital Pendiente} \\
    \hline
    01/07/2021 & & 37.255,28 \\
    \hline
    01/07/2022 & 3.725,53 & 40.980,81 \\
    \hline
    01/07/2023 & 4.098,08 & 45.078,89 \\
    \hline
    01/07/2024 & 4.507,89 & 49.586,78 \\
    \hline
    01/07/2025 & 4.958,68 & 54.545,46 \\
    \hline
    01/07/2026 & 5.454,55 & 60.000,00 \\
    \hline
    \end{tabular}
    \caption{Amortización del pagaré.}
    \label{tabla:amortizacion_pagare}
\end{table}

\begin{table}[H]
    \centering
    \begin{tabular}{|p{2cm}|p{6cm}|p{2cm}|}
    \hline
    \rowcolor{blue!30}
    \textbf{DEBE} & \textbf{Cambio de endeudamiento (01-07-2021)} & \textbf{HABER} \\
    \hline
    66.565,52 & (170) Deudas a largo plazo con entidades de crédito & \\
    \hline
    & (170) Deudas a largo plazo con entidades de crédito (Pagaré) & 37.255,28 \\
    \hline
    & (7691) Otros ingresos financieros derivados de intercambio de deudas & 29.309,92 \\
    \hline
    \end{tabular}
    \caption{Cambio de endeudamiento.}
    \label{tabla:cambio_endeudamiento}
\end{table}


% Durante los cinco próximos años se producen los asientos contables pertenecientes al devengo de los intereses y al pago de los mismos\footnote{Ver completo en el libro.}.

A continuación, procedemos a dar de baja al pasivo antiguo y dar de alta al nuevo pasivo.

\begin{table}[H]
    \centering
    \begin{tabular}{|p{2cm}|p{6cm}|p{2cm}|}
    \hline
    \rowcolor{blue!30}
    \textbf{DEBE} & \textbf{Cambio de endeudamiento (01-07-2021)} & \textbf{HABER} \\
    \hline
    66.565,52 & (170) Deudas a largo plazo con entidades de crédito & \\
    \hline
    & (170) Deudas a largo plazo con entidades de crédito (Pagaré) & 37.255,28 \\
    \hline
    & (7691) Otros ingresos financieros derivados de intercambio de deudas & 29.309,92 \\
    \hline
    \end{tabular}
    \caption{Cambio de endeudamiento.}
    \label{tabla:cambio_endeudamiento}
\end{table}

Duante los próximos cinco años se producen los asientos contables pertenecientes al devengo de los intereses y al pago de los mismos.



\begin{table}[H]
    \centering
    \begin{tabular}{|p{2cm}|p{6cm}|p{2cm}|}
        \hline
        \rowcolor{blue!30}
        \textbf{DEBE} & \textbf{Reflejo contable de los intereses devengados de la primera cuota (31-12-2021)} & \textbf{HABER} \\
        \hline
        1\p818,38 & (662) Intereses de deudas & \\
        \hline
        & (170) Deudas a largo plazo con Entidades de crédito & 1\p818,38 \\
        \hline
    \end{tabular}
    \caption{Reflejo contable de los intereses devengados de la primera cuota (31-12-2021).}
    \label{tabla:intereses_2021}
\end{table}

\begin{align*}
    \text{Interés devengado} &= 37\p255,28 \times (1 + 0,1)^{1/12} - 37\p255,28 = 1\p818,38
\end{align*}

\begin{table}[H]
    \centering
    \begin{tabular}{|p{2cm}|p{6cm}|p{2cm}|}
        \hline
        \rowcolor{blue!30}
        \textbf{DEBE} & \textbf{Reflejo contable de los intereses devengados de la primera cuota en el año 2022 (1-7-2022)} & \textbf{HABER} \\
        \hline
        1\p907,15 & (662) Intereses de deudas & \\
        \hline
        & (170) Deudas a largo plazo con Entidades de crédito & 1\p907,15 \\
        \hline
    \end{tabular}
    \caption{Reflejo contable de los intereses devengados de la primera cuota en el año 2022 (1-7-2022)\p}
    \label{tabla:intereses_2022}
\end{table}

\begin{align*}
    \text{Interés devengado} &= 37\p255,28 \times (1 + 0,1)^{6/12} - 37\p255,28 = 1\p907,15
\end{align*}

\begin{table}[H]
    \centering
    \begin{tabular}{|p{2cm}|p{6cm}|p{2cm}|}
        \hline
        \rowcolor{blue!30}
        \textbf{DEBE} & \textbf{Reflejo contable de los intereses devengados de la segunda cuota (31-12-2022)} & \textbf{HABER} \\
        \hline
        2\p000,22 & (662) Intereses de deudas & \\
        \hline
        & (170) Deudas a largo plazo con Entidades de crédito & 2\p000,22 \\
        \hline
    \end{tabular}
    \caption{Reflejo contable de los intereses devengados de la segunda cuota (31-12-2022)\p}
    \label{tabla:intereses_2022_2}
\end{table}

\begin{align*}
    \text{Interés devengado} &= 40\p980,81 \times (1 + 0,1)^{1/12} - 40\p980,81 = 2\p000,22
\end{align*}

\begin{table}[H]
    \centering
    \begin{tabular}{|p{2cm}|p{6cm}|p{2cm}|}
        \hline
        \rowcolor{blue!30}
        \textbf{DEBE} & \textbf{Reflejo contable de los intereses devengados de la segunda cuota en el año 2023 (1-7-2023)} & \textbf{HABER} \\
        \hline
        2\p097,86 & (662) Intereses de deudas & \\
        \hline
        & (170) Deudas a largo plazo con Entidades de crédito & 2\p097,86 \\
        \hline
    \end{tabular}
    \caption{Reflejo contable de los intereses devengados de la segunda cuota en el año 2023 (1-7-2023)\p}
    \label{tabla:intereses_2023}
\end{table}

\begin{align*}
    \text{Interés devengado} &= 40\p980,81 \times (1 + 0,1)^{6/12} - 40\p980,81 = 2\p097,86
\end{align*}

\begin{table}[H]
    \centering
    \begin{tabular}{|p{2cm}|p{6cm}|p{2cm}|}
        \hline
        \rowcolor{blue!30}
        \textbf{DEBE} & \textbf{Reflejo contable de los intereses devengados de la tercera cuota (31-12-2023)} & \textbf{HABER} \\
        \hline
        2\p200,24 & (662) Intereses de deudas & \\
        \hline
        & (170) Deudas a largo plazo con Entidades de crédito & 2\p200,24 \\
        \hline
    \end{tabular}
    \caption{Reflejo contable de los intereses devengados de la tercera cuota (31-12-2023)\p}
    \label{tabla:intereses_2023_2}
\end{table}

\begin{align*}
    \text{Interés devengado} &= 45\p087,89 \times (1 + 0,1)^{1/12} - 45\p087,89 = 2\p200,24
\end{align*}

\begin{table}[H]
    \centering
    \begin{tabular}{|p{2cm}|p{6cm}|p{2cm}|}
        \hline
        \rowcolor{blue!30}
        \textbf{DEBE} & \textbf{Reflejo contable de los intereses devengados de la tercera cuota en el año 2024 (1-7-2024)} & \textbf{HABER} \\
        \hline
        2\p307,64 & (662) Intereses de deudas & \\
        \hline
        & (170) Deudas a largo plazo con Entidades de crédito & 2\p307,64 \\
        \hline
    \end{tabular}
    \caption{Reflejo contable de los intereses devengados de la tercera cuota en el año 2024 (1-7-2024)\p}
    \label{tabla:intereses_2024}
\end{table}

\begin{align*}
    \text{Interés devengado} &= 45\p087,89 \times (1 + 0,1)^{6/12} - 45\p087,89 = 2\p307,64
\end{align*}



\begin{table}[H]
    \centering
    \begin{tabular}{|p{2cm}|p{6cm}|p{2cm}|}
        \hline
        \rowcolor{blue!30}
        \textbf{DEBE} & \textbf{Reflejo contable de los intereses devengados de la cuarta cuota (31-12-2024)} & \textbf{HABER} \\
        \hline
        2\p420,27 & (662) Intereses de deudas & \\
        \hline
        & (170) Deudas a largo plazo con Entidades de crédito & 2\p420,27 \\
        \hline
    \end{tabular}
    \caption{Reflejo contable de los intereses devengados de la cuarta cuota (31-12-2024)\p}
    \label{tabla:intereses_2024}
\end{table}

\begin{align*}
    \text{Interés devengado} &= 49\p586,78 \times (1 + 0,1)^{1/12} - 49\p586,78 = 2\p420,27
\end{align*}

\begin{table}[H]
    \centering
    \begin{tabular}{|p{2cm}|p{6cm}|p{2cm}|}
        \hline
        \rowcolor{blue!30}
        \textbf{DEBE} & \textbf{Reflejo contable de los intereses devengados de la cuarta cuota en el año 2025 (1-7-2025)} & \textbf{HABER} \\
        \hline
        2\p538,40 & (662) Intereses de deudas & \\
        \hline
        & (170) Deudas a largo plazo con Entidades de crédito & 2\p538,40 \\
        \hline
    \end{tabular}
    \caption{Reflejo contable de los intereses devengados de la cuarta cuota en el año 2025 (1-7-2025)\p}
    \label{tabla:intereses_2025}
\end{table}

\begin{align*}
    \text{Interés devengado} &= \left[49\p586,78 + 2\p420,27\right] \times (1 + 0,1)^{1/2} \\ - \left[49\p586,78 + 2\p420,27\right] = 2\p538,40
\end{align*}

\begin{table}[H]
    \centering
    \begin{tabular}{|p{2cm}|p{6cm}|p{2cm}|}
        \hline
        \rowcolor{blue!30}
        \textbf{DEBE} & \textbf{Reflejo contable de los intereses devengados de la quinta cuota (31-12-2025)} & \textbf{HABER} \\
        \hline
        2\p662,30 & (662) Intereses de deudas & \\
        \hline
        & (170) Deudas a largo plazo con Entidades de crédito & 2\p662,30 \\
        \hline
    \end{tabular}
    \caption{Reflejo contable de los intereses devengados de la quinta cuota (31-12-2025)\p}
    \label{tabla:intereses_2025_2}
\end{table}

\begin{align*}
    \text{Interés devengado} &= 54\p545,46 \times (1 + 0,1)^{1/12} - 54\p545,46 = 2\p662,30
\end{align*}

\begin{table}[H]
    \centering
    \begin{tabular}{|p{2cm}|p{6cm}|p{2cm}|}
        \hline
        \rowcolor{blue!30}
        \textbf{DEBE} & \textbf{Reflejo contable de la reclasificación de la deuda a 31-12-2025} & \textbf{HABER} \\
        \hline
        57\p207,76 & (170) Deudas a largo plazo con Entidades de crédito & \\
        \hline
        & (520) Deudas a corto plazo con Entidades de crédito & 57\p207,79 \\
        \hline
    \end{tabular}
    \caption{Reflejo contable de la reclasificación de la deuda a 31-12-2025\p}
    \label{tabla:reclasificacion_2025}
\end{table}

\begin{align*}
    \text{Reclasificación de la deuda} &= 54\p545,46 + 2\p662,30 = 57\p207,76
\end{align*}

\begin{table}[H]
    \centering
    \begin{tabular}{|p{2cm}|p{6cm}|p{2cm}|}
        \hline
        \rowcolor{blue!30}
        \textbf{DEBE} & \textbf{Reflejo contable de los intereses devengados de la quinta cuota en el año 2026 (1-7-2026)} & \textbf{HABER} \\
        \hline
        2\p792,24 & (662) Intereses de deudas & \\
        \hline
        & (170) Deudas a largo plazo con Entidades de crédito & 2\p792,24 \\
        \hline
    \end{tabular}
    \caption{Reflejo contable de los intereses devengados de la quinta cuota en el año 2026 (1-7-2026)\p}
    \label{tabla:intereses_2026}
\end{table}

\begin{align*}
    \text{Interés devengado} &= \left[54\p545,46 + 2\p662,30\right] \times (1 + 0,1)^{1/2} \\ - \left[54\p545,46 + 2\p662,30\right] = 2\p792,24
\end{align*}

\begin{table}[H]
    \centering
    \begin{tabular}{|p{2cm}|p{6cm}|p{2cm}|}
        \hline
        \rowcolor{blue!30}
        \textbf{DEBE} & \textbf{Devolución del pagaré (01-07-2026)} & \textbf{HABER} \\
        \hline
        60\p000 & (520) Deudas a corto plazo con entidades de crédito & \\
        \hline
        & (572) Bancos e instituciones de crédito c/c vista, euros & 60\p000 \\
        \hline
    \end{tabular}
    \caption{Devolución del pagaré (01-07-2026)\p}
    \label{tabla:devolucion_pagare_2026}
\end{table}

\begin{table}[H]
    \centering
    \begin{tabular}{|p{2cm}|p{6cm}|p{2cm}|}
        \hline
        \rowcolor{blue!30}
        \textbf{DEBE} & \textbf{Devolución del pagaré (01-07-2026)} & \textbf{HABER} \\
        \hline
        60\p000 & (520) Deudas a corto plazo con entidades financieras & \\
        \hline
        & (572) Bancos e instituciones de crédito c/c vista, euros & 60\p000 \\
        \hline
    \end{tabular}
    \caption{Devolución del pagaré (01-07-2026)\p}
    \label{tabla:devolucion_pagare_2026_2}
\end{table}

\begin{tcolorbox}[colback=blue!5!white,colframe=blue!75!black]
    \textbf{Nota:} El ejemplo 8 del tipo de cambio no cae\p
\end{tcolorbox}
















