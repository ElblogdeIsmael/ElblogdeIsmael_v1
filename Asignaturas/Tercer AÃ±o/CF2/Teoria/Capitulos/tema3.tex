\section{Concepto, características y tipología PF}

\subsection{Concepto}

Previamente debemos de saber que es un instrumento financiero. Este \textit{es un contrato que da lugar a un activo financiero en una empresa}, y simultáneamente, a un pasivo financiero o a un instrumento de patrimonio en otra empresa.

Los instrumentos financieros emitidos, incurridos o asumidos se clasificarán como pasivos financiero, en su totalidad o en una de sus partes, siempre que de acuerdo a su realidad económica, suponga una obligación contractual de entregar efectivo u otro activo financiero, intercambiar activos o pasivos financieros con terceros potencialmente desfavorables, también se clasificarán como un pasivo financiero, todo contrato que pueda ser o será, liquidado con los instrumentos propios de la empresa, siempre que no sean instrumentos derivados.

\subsubsection{Pasivo financiero VS instrumento de patrimonio neto}

\begin{itemize}
    \item Pasivo financiero: existe una obligación contractual que recae sobre una de las partes implicadas en el instrumento financiero. 
    \item Intrumento de PN: no existe una obligación contractual que recae sobre una de las partes implicadas en el instrumento financiero. Aunque el comprador pueda percibir dividendos, el emisor no tiene la obligación de repartir los dividendos.
\end{itemize}

\subsection{Diferentes tipologías}

\begin{itemize}
    \item Débitos por operaciones comerciales
    \item Deudas con entidades de crédito
    \item Obligaciones y otros valores negociables emitidos, tales como bonos y pagarés.
    \item Derivados con valoración desfavorable para la empresa.
    \item Deudas con características especiales, tales como deudas subordinadas.
    \item Otros pasivos financieros, tales como deudas con proveedores.
\end{itemize}

\subsection{Tipos de pasivos financieros}
\begin{itemize}
    \item \textbf{Pasivo a coste amortizado:} Todos en esta categoría, excepto cuando deban de valorarse a VR con cambios en la cuenta de pérdidas y ganancias. En este incluidos débito por operaciones comerciales y débito por operaciones no comerciales.
    \begin{itemize}
        \item Débitos por operaciones comerciales: aquellos que se originan en la compra de bienes o servicios.
        \item Débitos por operaciones no comerciales: son aquellos que no siendo instrumentos derivados, no tienen origen comercial, sino que proceden de operaciones de préstamo o crédito recibidos por la empresa.
    \end{itemize}
    \item \textbf{Pasivo a valor razonable con cambios en la cuenta de pérdidas y ganancias}: Deben de cumplir:
    \begin{itemize}
        \item \textit{Pasivos que se mantienen para negociar}.
        \begin{itemize}
            \item se emite con el propósito de readquirirlo en el corto plazo.
            \item obligación de que un vendedor de un activo financiero lo readquiera.
            \item Forme parte de una cartera de activos financieros identificados que se gestionan conjuntamente.
        \end{itemize}
        \item Desde el momento de reconocimiento inicial, ha sido designado por la empresa como a \textit{valor razonable con cambios en la cuenta de pérdidas y ganancias}.
        \begin{itemize}
            \item Se elimina o reduzca de manera significativa una incoherencia o ``asimetría contable''.
            \item Un grupo de pasivos financieros o de activos y pasivos financieros se gestione y su rendimiento se evalúe sobre la base del valor razonable de acuerdo con la estrategia de gestión de riesgos o de inversión de la empresa.
        \end{itemize}
    \end{itemize}
\end{itemize}

\subsection{¿Qué cuentas son con las que más vamos a trabajar?}

% \begin{itemize}
%     \item 15, 16, 17, 18\footnote{Para saber que cuentan son consulte el cuadro de cuentas o bien el manual, página 117-118}.
%     \item 50, 51, 52, 56.
% \end{itemize}

\begin{itemize}
    \item 15. DEUDAS A LARGO PLAZO CON CARACTERÍSTICAS ESPECIALES.
    \item 16. DEUDAS A LARGO PLAZO CON PARTES VINCULADAS.
    \item 17. DEUDAS A LARGO PLAZO POR PRÉSTAMOS RECIBIDOS, EMPRÉSTITOS Y OTROS CONCEPTOS.
    \item 18. PASIVOS POR FIANZAS, GARANTÍAS Y OTROS CONCEPTOS A LARGO PLAZO.
    \end{itemize}

    \begin{itemize}
        \item 50. EMPRÉSTITOS, DEUDAS CON CARACTERÍSTICAS ESPECIALES Y OTRAS EMISIONES ANÁLOGAS A CORTO PLAZO.
        \item 51. DEUDAS A CORTO PLAZO CON PARTES VINCULADAS.
        \item 52. DEUDAS A CORTO PLAZO POR PRÉSTAMOS RECIBIDOS Y OTROS CONCEPTOS.
        \item 56. FIANZAS Y DEPÓSITOS RECIBIDOS Y CONSTITUIDOS A CORTO PLAZO Y AJUSTES POR PERIODIFICACIÓN.
    \end{itemize}

No se recogen los derivados de operaciones de tráfico, ya que se estudian en la asignatura de CF1 (operaciones con proveedores y acreedores,...).
% \\\\
% \textit{Retomando la clasificación de los pasivos finanacieros, vamos a tratar esta más a fondo:}

% \begin{enumerate}
%     \item Pasivos financieros a coste amortizado.
%     \begin{
%     \item Pasivos financieros a valor razonable con cambios en la cuenta de pérdidas y ganancias.
% \end{enumerate}


\section{Pasivos a coste amortizado}

\begin{enumerate}
    \item Pasivos financieros a coste amortizado.
    \begin{itemize}
        \item Valoración inicial: Inicialmente a VR, salvo evidencia en contrario, será el precio de la transacción, que equivaldrá al VR de la contraprestación entregada más los costes de transacción.
        \item Valoración posterior: Por su coste amortizado, los intereses devengados se contabilizarán en la cuenta de pérdidas y ganancias, aplicando el método del tipo de interés efectivo (TIE).
    \end{itemize}
\end{enumerate}

\subsection*{Ejemplo 1}

La entidad A concede un crédito el 1-1-2018 a la entidad B con las siguientes características:
\begin{itemize}
    \item Nominal: 100.000 €.
    \item Comisión de apertura (liquidada al inicio): 1.200 €.
    \item Tipo de interés anual: 6,75\%.
    \item Plazo: 3 años.
    \item Cuota anual constante: 37.931 €.
\end{itemize}

\textbf{SE PIDE:} Contabilizar las operaciones en la empresa B.


En primer lugar, vamos a dibujar el esquema de flujos de efectivo del ejemplo 1.



\begin{figure}[H]
    \centering
    \begin{tikzpicture}
        % Eje horizontal
        \draw[-] (0,0) -- (10,0) node[right] {Tiempo};
        
        % Eje vertical
        \draw[-{Latex}] (0,0) -- (0,5) node[above left] {Flujo de efectivo};
        
        % Flechas principales
        \draw[-{Latex},red] (0,0) -- (0,4) node[above left] {100.000 \texteuro};
        \draw[-{Latex},red] (0,0) -- (0,3) node[below left] {-1.200 \texteuro};
        
        % Flechas de pagos anuales
        \draw[-{Latex},red] (3,0) -- (3,-2) node[below] {37.931 \texteuro};
        \draw[-{Latex},red] (6,0) -- (6,-2) node[below] {37.931 \texteuro};
        \draw[-{Latex},red] (9,0) -- (9,-2) node[below] {37.931 \texteuro};
        
        % Nodos de tiempo en la línea del eje
        \draw (-0.2,0)  (-0.2,0.2) node[above] {0};
        \foreach \x/\t in { 3/1, 6/2, 9/3} {
            \draw (\x,0) -- (\x,0.2) node[above] {\t};
        }
    \end{tikzpicture}
    \caption{Esquema de flujos de efectivo del ejemplo 1.}
    \label{fig:flujo-efectivo-ejemplo1}
\end{figure}


\begin{table}[H]
    \centering
    \begin{tabular}{|c|c|c|c|c|}
        \hline
        \rowcolor{blue!30}
        \textbf{Plazo} & \textbf{Cuota} & \textbf{Capital} & \textbf{Intereses} & \textbf{Capital pendiente} \\
        \hline
        01-01-2018 & 37.931 € & 31.181 € & 6.750 € & 100.000 € \\
        \hline
        31-12-2018 & 37.931 € & 31.181 € & 6.750 € & 68.819 € \\
        \hline
        31-12-2019 & 37.931 € & 33.286 € & 4.645 € & 35.533 € \\
        \hline
        31-12-2020 & 37.931 € & 35.533 € & 2.398 € & 0 € \\
        \hline
    \end{tabular}
    \caption{Cuadro de amortización al 6,75\%.}
    \label{tabla:cuadro_amortizacion}
\end{table}






%%%%%%%----VOY POR AQUÍ----%%%%%%%, voy por la pagina 6 del manual de la asignatura.

\subsection{Valoración de los pasivos financieros}

\subsubsection{A coste amortizado}
\begin{itemize}
    \item Valoración inicial: Inicialmente a VR, es decir, el precio de la transacción. Consideramos los costes que son para emitirlos. Por ejemplo, si pido un préstamo de una cuantía X, el importe extra por gestión, se debe de incluir como un mayor importe en el valor del pasivo.
    \item Valoración posterior: Por su coste amortizado, los interéses devengados, se contabilizarán en la cuenta de pérdidas y ganancias, aplicando el método del tanto efectivo.
\end{itemize}

\textit{Nota: Debemos de tener en cuenta que la empresa que recibe el préstamo tiene un pasivo, mientras que la empresa que lo da tiene un activo.}


%----------------------------------------------------------------------------------

%----------------------------------------------------------------------------------




\subsection*{Ejemplo 6}

\subsection{Ejemplo 6}

La empresa WERT, S.A. emitió el 2 de noviembre de 2020 obligaciones que cotizan en la bolsa de Madrid con las siguientes condiciones: una prima de emisión del 5\% y la intención de recuperarlas en el corto plazo. Los gastos de emisión ascendieron a 4.000 €. El valor en libros a 31 de diciembre de 2020 de este empréstito es de 4.400.000 € (4.000 títulos y valor nominal 1.000 €) registrado en la partida del balance “Valores representativos de deuda a corto plazo” (VRD).

El 3 de febrero de 2021 la empresa decide adquirir 2.000 de estos títulos de la bolsa de Madrid que cotizan al 125\% de su valor nominal, teniendo en cuenta que los gastos de la operación ascienden a un 0,1\%.

\textbf{SE PIDE:}

\b{1) Analizar la posibilidad de clasificación del pasivo financiero.}

En el enunciado se nos dice que clasificó su emisión como VRD a c/p, esto podría implicar que la intención es de una emisión de readquisición a corto plazo. Esta es una de las características de los pasivos financieros valorados a VR con cambios en la cuenta de pérdidas y ganancias (la empresa debe de haberlo clasificado como pasivo con valoración posterior a VR). Además se emite en el mercado y presentan un valor razonable fiable. 



\b{2) Reflejar contablemente, en función de la cartera de clasificación elegida en el apartado A, la emisión de los títulos.}

\begin{table}[H]
    \centering
    \begin{tabular}{|p{3cm}|p{6cm}|p{3cm}|}
    \hline
    \rowcolor{blue!30}
    \textbf{DEBE} & \textbf{Emisión del empréstito 02.11.2020} & \textbf{HABER} \\
    \hline
    3.796.000&  \cuenta{572}& \\
    \hline
    4.000&  \cuenta{669}& \\
    \hline
    &  \cuenta{500}
        Valor de emisión = 4.000 títulos x 1000 =
        4.000.000 -5\% \text{de la prima de emisión} = 
        4.000.000 - 200.000 = 
        3.800.000
    & 3.800.000\\
    \hline
    \end{tabular}
    \caption{Asiento 1. Ejercicio 6.}
    \label{tabla:asiento1ej6-2}
\end{table}


\b{3) Reflejar contablemente, en función de la cartera de clasificación elegida en el apartado A, el reembolso de los títulos.}

Debemos de amortizar los títulos, en este caso, como estos títulos cotizan en el mercado secundario, a final del ejercicio se había producido un incremento del VR de los títulos. Por lo que el valor del título era de 1.100 \e y un Bf que se imputó en la Cuenta de PyG.

En esta fecha se deben de sacar del mercado (decide amortizar) 2.000 títulos a 1.250 \e/título, lo que supone que si comparamos este valor en libros a 31 de Diciembre, una pérdida de 150 \e/título.

\begin{table}[H]
    \centering
    \begin{tabular}{|p{3cm}|p{6cm}|p{3cm}|}
    \hline
    \rowcolor{blue!30}
    \textbf{DEBE} & \textbf{} & \textbf{HABER} \\
    \hline
    2.200.000 & (500) Obligaciones y bonos a corto plazo \newline Damos de baja los títulos por su valor contable = \newline 4.400.000 €/4.000 títulos = 1.100 € título x 2.000 títulos & \\
    \hline
    2.500 & (669) Otros gastos financieros \newline Gastos = 0,001 x 2.500.000 € = 2.500 € & \\
    \hline
    300.000 & (675) Pérdidas por operaciones con obligaciones propias & \\
    \hline
    & (572) Banco X, cuenta corriente \newline (valor razonable 125\%/s/VN = 1.250 € x 2.000 títulos = \newline 2.500.000 € más 2.500 € de gastos) & 2.502.500 \\
    \hline
    \end{tabular}
    \caption{Amortización del empréstito* (03-02-2021).}
    \label{tabla:Asiento2-Ejercicio6-2}
\end{table}

\newpage 
\section{Reclasificación y baja de PF}

Una entidad no puede reclasificar los pasivos financieros de una categoría a otra. La empresa dará de baja a un pasivo financiero, o parte del mismo, cuando la obligación se haya extinguido. No obtante, se define de la misma forma la posibilidad de que se lleve a cabo un intercambio de instrumentos de deuda entre \c{prestamista} y \c{prestatario}, o lo que se conoce como \c{refinación de la deuda}. Se genera en condiciones similares o, por si el contrario, las condiciones son sustancialmente diferentes.

%ACABAR



\subsection*{Ejemplo 7}

La empresa VENKA, S.A. concierta el 1 de julio de 2020 una operación de préstamo con la entidad financiera LOPRESTOTODO, S.A. por importe de 100.000 € a 3 años con pagos semestrales con un tipo de interés nominal anual del 5\%, con una comisión de estudio\footnote{Este gasto hace referencia al estudio de la viabilidad del préstamo en base a la evaluación de riesgos} del 2,5\% y comisión de apertura del 1\%. Al final de los primeros pagos de las cuotas se le plantean dos opciones:

\begin{itemize}
    \item[\textbf{A)}] Refinanciar la deuda con un interés anual nominal del 4\% a 3 años de duración con cuotas anuales constantes, sin comisiones.
    \item[\textbf{B)}] Sustituir el préstamo por un pagaré con vencimiento a 5 años y un nominal de 60.000 € con un tipo de interés de mercado del 10\%.
\end{itemize}

\subsection*{SE PIDE:}

\begin{enumerate}
    \item Analizar la clasificación del pasivo financiero según el Plan General de Contabilidad (PGC).
    \item Elaborar los cuadros de amortización del préstamo para la entidad financiera y para VENKA, considerando el coste amortizado.
    \item Registrar contablemente las dos primeras cuotas del préstamo inicial.
    \item Registrar contablemente la primera cuota del nuevo préstamo en la opción A, así como las nuevas cuotas derivadas de la opción B.
\end{enumerate}

\subsubsection*{Solución Opción A}

\begin{enumerate}
    \item En este caso, se dudece de la lectura de la NV 9ª del PGC que esta operación no puede ser \c{Pasivos Financieros cambios a VR}, si no que es un \c{Pasivo Financiero a coste amortizado}.
    \item Calculamos el cuadro de amortización, y para ello debemos de tener en cuenta:
    \begin{itemize}
        \item Calcular el VR inicial, teniendo en cuenta los 100.000 \e - gastos de transacción, que en este caso es la comisión de estudio y la comisión de apertura(2.500 + 1.000 = 3.500 \e).
        \item Tipo de interés efectivo es de 7,28\%, o bien un 3,58 \% semestral\footnote{Los cuadros que se muestran nos lo proporcionan}.
    \end{itemize}
\end{enumerate}

\begin{table}[H]
    \centering
    \begin{tabular}{|p{2cm}|p{2cm}|p{2cm}|p{2cm}|p{2cm}|p{2cm}|}
    \hline
    \rowcolor{blue!30}
    \textbf{Fecha} & \textbf{Cuota} & \textbf{Interés} & \textbf{Capital} & \textbf{Capital Amortizado Acumulado} & \textbf{Capital Pendiente} \\
    \hline
    01-07-20 & 18.155,00 & 2.500,00 & 15.655,00 & 15.655,00 & 84.345,00 \\
    \hline
    01-01-21 & 18.155,00 & 2.108,63 & 16.046,37 & 31.701,37 & 68.298,63 \\
    \hline
    01-07-21 & 18.155,00 & 1.707,47 & 16.447,53 & 48.148,90 & 51.851,10 \\
    \hline
    01-01-22 & 18.155,00 & 1.296,28 & 16.858,72 & 65.007,62 & 34.992,38 \\
    \hline
    01-07-22 & 18.155,00 & 874,81 & 17.280,19 & 82.287,81 & 17.712,19 \\
    \hline
    01-01-23 & 18.154,99 & 442,80 & 17.712,19 & 100.000,00 & 0,00 \\
    \hline
    \end{tabular}
    \caption{Cuadro de amortización tipo interés nominal de la entidad financiera (5\%).}
    \label{tabla:Asiento1-Ejercicio7-tema2}
\end{table}

Ahora debemos de calcular el nuevo cuadro de en base al tipo de interés del 7,28\% anual.

\begin{table}[H]
    \centering
    \begin{tabular}{|p{2cm}|p{2cm}|p{2cm}|p{2cm}|p{2cm}|}
    \hline
    \rowcolor{blue!30}
    \textbf{Fecha} & \textbf{Cuota} & \textbf{Interés} & \textbf{Capital} & \textbf{Coste Amortizado} \\
    \hline
    01-07-20 & 18.155,00 & 3.450,48 & 14.704,52 & 96.500,00 \\
    \hline
    01-01-21 & 18.155,00 & 2.924,70 & 15.230,30 & 81.795,50 \\
    \hline
    01-07-21 & 18.155,00 & 2.380,13 & 15.774,87 & 66.565,20 \\
    \hline
    01-01-22 & 18.155,00 & 1.816,07 & 16.338,93 & 50.790,33 \\
    \hline
    01-07-22 & 18.155,00 & 1.231,85 & 16.923,15 & 34.451,40 \\
    \hline
    01-01-23 & 18.154,99 & 626,74 & 17.528,25 & 17.528,25 \\
    \hline
    \end{tabular}
    \caption{Cuadro de amortización tipo interés efectivo (7,28\%) con interés efectivo semestral (3,58\%).}
    \label{tabla:amortizacion_efectivo}
\end{table}

Una vez llegado al pago de la 2º deuda, debemos de estudiar la renegociación de la deuda. La deuda que queda pendiente es 68.298,63 \e. Teniendo en cuenta las condiciones:
\begin{itemize}
    \item Refinanciación de la deuda.
    \item 4\% de interés anual nominal.
    \item 3 años de duración.
    \item Cuotas anuales constantes.
\end{itemize}

El nuevo cuadro que nos queda es:

\begin{table}[H]
    \centering
    \begin{tabular}{|p{2cm}|p{2cm}|p{2cm}|p{2cm}|p{2cm}|p{2cm}|}
    \hline
    \rowcolor{blue!30}
    \textbf{Fecha} & \textbf{Cuota} & \textbf{Interés} & \textbf{Capital} & \textbf{Capital Amortizado Acumulado} & \textbf{Capital Pendiente} \\
    \hline
    01-07-22 & 24.611,31 & 2.731,95 & 21.879,36 & 21.879,36 & 46.419,27 \\
    \hline
    01-07-23 & 24.611,31 & 1.856,77 & 22.754,54 & 44.633,90 & 23.664,73 \\
    \hline
    01-07-24 & 24.611,32 & 946,59 & 23.664,73 & 68.298,63 & 0,00 \\
    \hline
    \end{tabular}
    \caption{Cuadro de amortización tipo interés nominal (4\%) para la nueva deuda opción A.}
    \label{tabla:Tabla3-Ejercicio7-tema2}
\end{table}

\c{Debemos de calcular el nuevo valor actual:}

\begin{align*}
    \text{Valor actual del Préstamo Opción A con condiciones iniciales } = \\
    \text{64.258,86} = \frac{\text{24.611,31}}{1,0728} + \frac{\text{24.611,31}}{1,0728^2} + \frac{\text{24.611,31}}{1,0728^3}
\end{align*}

\begin{tcolorbox}[colback=blue!5!white,colframe=blue!75!black]
    \textbf{Nota:} Dado que el importe pendiente tras el pago de las dos primeras cuotas a coste amortizado es de 66.565,20 \e y el valor actual del nuevo préstamo con las características del préstamo original es de 64.258,86 \e, \textit{la diferencia de 2.306,34 \e es menor que el 10\% sobre el importe de 66.565,20 \e,} lo que implica que \textit{no difieren sustancialmente, por lo que el pasivo original no se dará de baja del balance, Por tanto, podemos seguir contabilizando el mismo importe pendiente de la deuda original.}
    
\end{tcolorbox}

\begin{align*}
    \text{Nuevo tipo de interés efectivo}  = \\
    \text{66.565,20} = \\
    \frac{\text{24.611,31}}{1 + i} + \frac{\text{24.611,31}}{(1 + i)^2} + \frac{\text{24.611,31}}{(1 + i)^3} \rightarrow \\
    \rightarrow i = 5,3664\%
\end{align*}

\begin{table}[H]
    \centering
    \begin{tabular}{|p{2cm}|p{2cm}|p{2cm}|p{2cm}|p{2cm}|p{2cm}|}
    \hline
    \rowcolor{blue!30}
    \textbf{Fecha} & \textbf{Cuota} & \textbf{Interés} & \textbf{Capital} & \textbf{Capital Amortizado Acumulado} & \textbf{Coste Amortizado} \\
    \hline
    01-07-21 & - & - & - & - & 66.365,20 \\
    \hline
    01-07-22 & 24.611,31 & 3.572,15 & 21.039,16 & 21.039,16 & 45.526,04 \\
    \hline
    01-07-23 & 24.611,31 & 2.443,10 & 22.168,20 & 43.207,36 & 23.357,83 \\
    \hline
    01-07-24 & 24.611,31 & 1.253,47 & 23.357,83 & 66.565,20 & 0,00 \\
    \hline
    \end{tabular}
    \caption{Cuadro de amortización a coste amortizado tipo interés efectivo (5,3663932\%).}
    \label{tabla:amortizacion_coste_amortizado}
\end{table}


\begin{table}[H]
    \centering
    \begin{tabular}{|p{2cm}|p{6cm}|p{2cm}|}
    \hline
    \rowcolor{blue!30}
    \textbf{DEBE} & \textbf{} & \textbf{HABER} \\
    \hline
    21.039,16 & (170) Deudas a largo plazo con entidades financieras & \\
    \hline
    & (520) Deudas a corto plazo con entidades financieras&  21.039,16 \\
    \hline
    \end{tabular}
    \caption{Reclasificación de la deuda de la primera cuota del largo plazo al corto plazo (1-7-2021).}
    \label{tabla:reclasificacion}
\end{table}

\begin{table}[H]
    \centering
    \begin{tabular}{|p{2cm}|p{6cm}|p{2cm}|}
    \hline
    \rowcolor{blue!30}
    \textbf{DEBE} & \textbf{} & \textbf{HABER} \\
    \hline
    1.762,73 & (662) Intereses de deudas & \\
    \hline
    & INTERÉS DEVENGADO = $[\text{66.565,20} \times (1 + 0,053663932)^(1/2)] - 66.565,20 = \newline \text{1.762,73539} € $& \\
    \hline
    &(527) Intereses a corto plazo de deudas con Entidades de crédito &  1.762,73 \\
    \hline
    \end{tabular}
    \caption{Reflejo contable de los intereses devengados de la primera cuota (31-12-2021).}
    \label{tabla:intereses_2021}
\end{table}
\begin{table}[H]
    \centering
    \begin{tabular}{|p{2cm}|p{6cm}|p{2cm}|}
    \hline
    \rowcolor{blue!30}
    \textbf{DEBE} & \textbf{} & \textbf{HABER} \\
    \hline
    1.809,41 & (662) Intereses de deudas & \\
    \hline
     & INTERÉS DEVENGADO = $[(\text{66.565,20} + \text{1.762,73}) \times (1 + 0,053663932)^{(1/2)}] - \newline (\text{66.565,20} + \text{1.762,73}) = \text{1.809,41} € $& \\
    \hline
    &(527) Intereses a corto plazo de deudas con Entidades de crédito &  1.809,41 \\
    \hline
    \end{tabular}
    \caption{Reflejo contable de los intereses devengados de la primera cuota en el año 2022 (1-7-2022).}
    \label{tabla:intereses_2022}
\end{table}
\begin{table}[H]
    \centering
    \begin{tabular}{|p{2cm}|p{6cm}|p{2cm}|}
    \hline
    \rowcolor{blue!30}
    \textbf{DEBE} & \textbf{} & \textbf{HABER} \\
    \hline
    3.572,15 & (527) Intereses a corto plazo de deudas con Entidades de crédito & \\
    \hline
    21.039,16 & (520) Deudas a corto plazo con entidades financieras & \\
    \hline
    &(572) Banco X c/c &  24.611,31 \\
    \hline
    \end{tabular}
    \caption{Reflejo contable del pago de la primera cuota (1-7-2022).}
    \label{tabla:pago_cuota}
\end{table}


\subsubsection*{Opción B}

Debemos de proceder de igual manera a realizar el test cuantitativo cuyo límite es del 10\% sobre el mismo importe de 66.565,30 \e. Debemos de calcular el valor actual del nuevo pagaré.

\begin{align*}
    \text{Valor actual pagaré} = \\
    \frac{\text{60.000}}{(1,0728)^5} = 42.233,81 \e
\end{align*}

El límite entre ambas valoraciones es de 6.656,52 \e = 66.565,20 - 42.223,81 \e. Por lo que la diferencia es de 24.331,39 \e, contablemente debemos de dar de baja a la antigua operación de endeudamiento y dar de alta a la nueva, imputando la diferencia en la cuenta de pérdidas y ganancias.

\begin{align*}
    \text{Valor actual del pagaré} = \\
    \frac{\text{60.000}}{1,1^5} = 37.255,28 \e
\end{align*}

\begin{table}[H]
    \centering
    \begin{tabular}{|p{2cm}|p{2cm}|p{2cm}|}
    \hline
    \rowcolor{blue!30}
    \textbf{Vencimientos Fecha} & \textbf{Cuota de Interés} & \textbf{Capital Pendiente} \\
    \hline
    01/07/2021 & & 37.255,28 \\
    \hline
    01/07/2022 & 3.725,53 & 40.980,81 \\
    \hline
    01/07/2023 & 4.098,08 & 45.078,89 \\
    \hline
    01/07/2024 & 4.507,89 & 49.586,78 \\
    \hline
    01/07/2025 & 4.958,68 & 54.545,46 \\
    \hline
    01/07/2026 & 5.454,55 & 60.000,00 \\
    \hline
    \end{tabular}
    \caption{Amortización del pagaré.}
    \label{tabla:amortizacion_pagare}
\end{table}

\begin{table}[H]
    \centering
    \begin{tabular}{|p{2cm}|p{6cm}|p{2cm}|}
    \hline
    \rowcolor{blue!30}
    \textbf{DEBE} & \textbf{Cambio de endeudamiento (01-07-2021)} & \textbf{HABER} \\
    \hline
    66.565,52 & (170) Deudas a largo plazo con entidades de crédito & \\
    \hline
    & (170) Deudas a largo plazo con entidades de crédito (Pagaré) & 37.255,28 \\
    \hline
    & (7691) Otros ingresos financieros derivados de intercambio de deudas & 29.309,92 \\
    \hline
    \end{tabular}
    \caption{Cambio de endeudamiento.}
    \label{tabla:cambio_endeudamiento}
\end{table}


Durante los cinco próximos años se producen los asientos contables pertenecientes al devengo de los intereses y al pago de los mismos\footnote{Ver completo en el libro.}.

\begin{tcolorbox}[colback=blue!5!white,colframe=blue!75!black]
    \textbf{Nota:} El ejemplo 8 del tipo de cambio no cae.
    
\end{tcolorbox}

%pag 154














