\section{Ejercicios}

\subsection*{Ejercicio 1}
Expresa los problemas siguientes en forma estándar y canónica:
\begin{itemize}
    \item[a)] Max. $x + y$
    \begin{align*}
        \text{s.a.} \quad & -x + y = 2 \\
        & x + 2y \leq 6 \\
        & 2x + y \geq 6 \\
        & x \geq 0, y \leq 0
    \end{align*}

    
    \item[b)] Max. $2x + 3y + z$
    \begin{align*}
        \text{s.a.} \quad & 4x + 3y + z \leq 20 \\
        & x + y \leq 20 \\
        & x \geq 0, y \leq 0
    \end{align*}
    
    \item[c)] Min. $x + y$
    \begin{align*}
        \text{s.a.} \quad & -x + y \leq 2
    \end{align*}
\end{itemize}

\subsection*{Ejercicio 2}
Resuelve mediante el método símplex el siguiente problema:
\begin{align*}
    \text{Max.} \quad & -x + y \\
    \text{s.a.} \quad & -2x + y \leq 4 \\
    & x + y \leq 1 \\
    & y \geq 0
\end{align*}

\subsection*{Ejercicio 3}
Resuelve el problema siguiente. Indica las soluciones óptimas y el valor de la función objetivo en cada una de ellas.
\begin{align*}
    \text{Opt.} \quad & x - 2y + 3z \\
    \text{s.a.} \quad & x + 2y + z \leq 4 \\
    & 2x + y - z \leq 2 \\
    & x, y, z \geq 0
\end{align*}

\subsection*{Ejercicio 4}
Resuelve:
\begin{itemize}
    \item[a)] Max. $3x + 2y + z$
    \begin{align*}
        \text{s.a.} \quad & 2x - 3y + 2z \leq 3 \\
        & -x + y + z \leq 5 \\
        & x, y, z \geq 0
    \end{align*}
    
    \item[b)] Max. $3x + y + 4z$
    \begin{align*}
        \text{s.a.} \quad & 6x + 3y + 5z \leq 25 \\
        & 3x + 4y + 5z \leq 20 \\
        & x, y, z \geq 0
    \end{align*}
\end{itemize}


\subsection*{Ejercicio 5}
Cierto fabricante produce sillas y mesas para lo que requiere la utilización de dos secciones de producción, la sección de montaje y la sección de pintura. La producción de una silla requiere una hora de trabajo en la sección de montaje y dos horas en la de pintura. Por su parte, la fabricación de una mesa precisa de tres horas en la sección de montaje y una hora en la de pintura.

La sección de montaje solo puede estar nueve horas diarias en funcionamiento, mientras que la de pintura solo ocho horas. El beneficio que se obtiene produciendo mesas es el doble que produciendo sillas.

¿Cuál debe ser la producción diaria de mesas y sillas que maximice el beneficio?

\subsection*{Ejercicio 6}
Un agricultor posee una parcela de 640 m$^2$ para dedicarla al cultivo de árboles frutales: naranjos, perales y manzanos. Se pregunta de qué forma repartirá la superficie de la parcela entre las tres variedades para conseguir el máximo beneficio sabiendo que:

\begin{itemize}
    \item Cada naranjo precisa como mínimo de 16 m$^2$, cada peral de 4 m$^2$ y cada manzano 8 m$^2$.
    \item Dispone de un total de 900 horas de trabajo por año (150 jornales), precisando cada naranjo de 30 horas por año, cada peral de 5 horas por año y cada manzano de 10 horas por año.
    \item Los beneficios unitarios son de 50, 25, 20 unidades monetarias por cada naranjo, peral y manzano respectivamente.
\end{itemize}

\subsection*{Ejercicio 7}
La empresa A se dedica al montaje de motocicletas de 500, 250, 125 y 50 c.c. Posee una planta que está estructurada en cuatro departamentos: fabricación de los chasis, pintura, montaje y el departamento de calidad.

Las horas de mano de obra que necesita cada uno de los modelos de motocicletas en los diferentes departamentos son las siguientes:

\begin{table}[H]
\centering
\begin{tabular}{|c|c|c|c|c|}
\hline
 & Chasis & Pintura & Montaje & Control calidad \\
\hline
500 c.c. & 8 & 6 & 8 & 4 \\
250 c.c. & 6 & 3 & 8 & 2 \\
125 c.c. & 4 & 2 & 6 & 2 \\
50 c.c. & 2 & 1 & 4 & 2 \\
\hline
\end{tabular}
\end{table}

La distribución de los trabajadores es la siguiente: para la fabricación de chasis se dispone de 25 trabajadores, en el departamento de pintura de 18, el de montaje de 30 y el de control de calidad de 10. Todos los trabajadores realizan una jornada laboral de 8 horas.

Si el margen de beneficio de cada uno de los modelos es de 200000, 140000, 80000 y de 40000 pesetas respectivamente, ¿cuál ha de ser la combinación óptima de motocicletas a producir para que el beneficio sea máximo?

\subsection*{Ejercicio 8}
Resolver los siguientes problemas utilizando el método de la M-grande:

\begin{itemize}
    \item[a)] Max. $3x_1 + 6x_2$
    \begin{align*}
        \text{s.a.} \quad & x_1 + 4x_2 \leq 5 \\
        & -x_1 + 3x_2 \leq -2 \\
        & x_1 - 5x_2 \leq -2 \\
        & x_1, x_2 \geq 0
    \end{align*}
    
    \item[b)] Max. $4.5x_1 + 3x_2 + 1.5x_3$
    \begin{align*}
        \text{s.a.} \quad & x_1 + 2x_2 - x_3 \leq 4 \\
        & 2x_1 - x_2 + x_3 = 8 \\
        & x_1 - x_2 \leq 6 \\
        & x_1, x_2, x_3 \geq 0
    \end{align*}
    
    \item[c)] Min. $x_1 + x_2$
    \begin{align*}
        \text{s.a.} \quad & x_1 + x_2 \geq 6 \\
        & 4x_1 + 2x_2 \geq 6 \\
        & x_1, x_2 \geq 0
    \end{align*}
\end{itemize}


\subsection*{Ejercicio 9}

Resolver utilizando el método de las dos fases:

\begin{itemize}
    \item[a)] Min. $20x_1 + 25x_2$
    \begin{align*}
        \text{s.a.} \quad & 2x_1 + 3x_2 \geq 18 \\
        & x_1 + 3x_2 \geq 12 \\
        & 4x_1 + 3x_2 \geq 24 \\
        & x_1, x_2 \geq 0
    \end{align*}

    
    \item[b)] Max. $4x_1 + 3x_2$
    \begin{align*}
        \text{s.a.} \quad & 3x_1 + 4x_2 \leq 12 \\
        & x_1 + x_2 \geq 4 \\
        & 4x_1 + 2x_2 \leq 8 \\
        & x_1, x_2 \geq 0
    \end{align*}
    
    \item[c)] Max. $x_1 - 2x_2 + 3x_3$
    \begin{align*}
        \text{s.a.} \quad & x_1 + x_2 + x_3 = 6 \\
        & x_3 \leq 2 \\
        & x_1, x_2, x_3 \geq 0
    \end{align*}

    Pasamos a forma estándar:

    \begin{align*}
        Max. \quad & x_1 - 2x_2 + 3x_3 + 0s_1 -Mt_1\\
        \text{s.a.} \quad & x_1 + x_2 + x_3 + t_1= 6 \\
        & x_3 + s_1 = 2 \\
        & x_1, x_2, x_3, s_1, t_1 \geq 0
    \end{align*}

    \subsubsection*{Fase 1}

    Paso 1. La función artificial es: 

    \begin{equation*}
        z^0=0x_1 + 0x_2 + 0x_3 + 0s_1 -t_1
    \end{equation*}

    Paso 2. Aplicar el método simplex al programa construido:

    \begin{table}[H]
    \centering
    \begin{tabular}{|c|c|c|c|c|c|c|c|}
    \hline
    &  & 0 & 0 & 0 & 0 & -1 &\\
    \hline
    & VB & $x_1$ & $x_2$ & $x_3$ & $s_1$ & $t_1$ & XB \\
    \hline
    -1 & $t_1$ & 1 & 1 & 1 & 0 & 1 & 6\\
    \hline
    0 & $s_1$ & 0 & 0 & 1 & 1& 0 &2\\
    \hline
    & $z_j - c_j$ & -1 & -1 & -1 & 0 & 0 &-6\\
    \hline
    \end{tabular}
    \end{table}

    Ahora debemos de coger la más negativa, pero al ser todos con valor -1, da igual cual cojamos, cogemos la primera, es decir, la de $x_1$.

    % \begin{tcolorbox}[colback=green!5!white,colframe=green!75!black]
        \begin{align*}
            F_p = F_1\\
            F2N = F2 (\text{Ya tiene un 0})
        \end{align*}
        
    % \end{tcolorbox}

    \begin{table}[H]
        \centering
        \begin{tabular}{|c|c|c|c|c|c|c|c|}
        \hline
        &  & 0 & 0 & 0 & 0 & -1 &\\
        \hline
        & VB & $x_1$ & $x_2$ & $x_3$ & $s_1$ & $t_1$ & XB \\
        \hline
        0 & $x_1$ & 1 & 1 & 1 & 0 & 1 & 6\\
        \hline
        0 & $s_1$ & 0 & 0 & 1 & 1& 0 &2\\
        \hline
        & $z_j - c_j$ & 0 & 0& 0 & 0 & 1 &0\\
        \hline
        \end{tabular}
    \end{table}

    En este punto no podemos continuar ya que todos los valores de la fila $z_j - c_j$ son positivos, por lo que debemos de pasar a la fase 2.

    \subsubsection*{Fase 2}

    Ahora debemos de eliminar las variables artificiales y continuar con el problema original.

    \begin{table}[H]
        \centering
        \begin{tabular}{|c|c|c|c|c|c|c|}
        \hline
        &  & 1 & -2 & 3 & 0 &\\
        \hline
        & VB & $x_1$ & $x_2$ & $x_3$ & $s_1$ & XB \\
        \hline
        1 & $x_1$ & 1 & 1 & 1 & 0 & 6\\
        \hline
        0 & $s_1$ & 0 & 0 & 1 & 1& 2\\
        \hline
        & $z_j - c_j$ & 0 & 3&-2 & 0 & 6\\
        \hline
        \end{tabular}
    \end{table}

    Cogemos la fila más negativa, en este caso la de $x_3$, con valor -2.

    \begin{align*}
        F_p = F_2\\
        F1N = F1 - F_p
    \end{align*}

    \begin{table}[H]
        \centering
        \begin{tabular}{|c|c|c|c|c|c|c|}
        \hline
        &  & 1 & -2 & 3 & 0 &\\
        \hline
        & VB & $x_1$ & $x_2$ & $x_3$ & $s_1$ & XB \\
        \hline
        1 & $x_1$ & 1 & 1 & 0 & -1 & 4\\
        \hline
        3 & $x_3$ & 0 & 0 & 1 & 1& 2\\
        \hline
        & $z_j - c_j$ & 0 & 3&0 & 2 & 10\\
        \hline
        \end{tabular}
    \end{table}

    Como todos los valores de la fila $z_j - c_j$ son positivos, hemos llegado a la solución óptima.


    
    \item[d)] Max. $x_1 + x_2 + 10x_3$
    \begin{align*}
        \text{s.a.} \quad & x_2 + 4x_3 = 2 \\
        & -2x_1 + x_2 - 6x_3 = 2 \\
        & x_1, x_2, x_3 \geq 0
    \end{align*}

    Pasamos a forma estándar:

    \begin{align*}
        Max. \quad & x_1 + x_2 + 10x_3 - t_1 -t_2\\
        \text{s.a.} \quad & x_2 + 4x_3 + t_1= 2 \\
        & -2x_1 + x_2 - 6x_3 + t_2 = 2 \\
        & x_1, x_2, x_3, s_1, t_1 \geq 0
    \end{align*}

    \subsubsection*{Fase 1}

    Paso 1. La función artificial es: 
    \begin{equation*}
        z^0=0x_1 + 0x_2 + 0x_3  -t_1 -t_2
    \end{equation*}

    Paso 2. Aplicar el método simplex al programa construido:

    \begin{table}[H]
        \centering
        \begin{tabular}{|c|c|c|c|c|c|c|c|}
        \hline
        &  & 0 & 0 & 0 & -1 & -1 &\\
        \hline
        & VB & $x_1$ & $x_2$ & $x_3$ & $t_1$ & $t_2$ & XB \\
        \hline
        -1 & $t_1$ & 0 & 1 & 4 & 1 & 0 & 2\\
        \hline
        -1 & $t_2$ & -2 & 1 & -6 & 0& 1 &2\\
        \hline
        & $z_j - c_j$ & 2 & -2& 2 & 0 & 0 &-4\\
        \hline
        \end{tabular}
    \end{table}

    Cogemos la columna más negativa, en este caso la de $x_2$.

    \begin{align*}
        F_p = F_1\\
        F2N = F_2 - F_p
    \end{align*}

    \begin{table}[H]
        \centering
        \begin{tabular}{|c|c|c|c|c|c|c|c|}
        \hline
        &  & 0 & 0 & 0 & -1 & -1 &\\
        \hline
        & VB & $x_1$ & $x_2$ & $x_3$ & $t_1$ & $t_2$ & XB \\
        \hline
        0 & $x_2$ & 0 & 1 & 4 & 1 & 0 & 2\\
        \hline
        -1 & $t_2$ & -2 & 0 & -10 & -1& 1 &0\\
        \hline
        & $z_j - c_j$ & 2 & 0& 10 & 2 & 0 &0\\
        \hline
        \end{tabular}
    \end{table}

    % Cogemos la columna más negativa, en este caso la de $x_3$.

    % \begin{align*}
    %     F_p = F_2\\
    %     F2N = F2 / 10\\
    %     F1N = F_1 - 4F_p
    % \end{align*}

    % \begin{table}[H]
    %     \centering
    %     \begin{tabular}{|c|c|c|c|c|c|c|c|}
    %     \hline
    %     &  & 0 & 0 & 0 & -1 & -1 &\\
    %     \hline
    %     & VB & $x_1$ & $x_2$ & $x_3$ & $t_1$ & $t_2$ & XB \\
    %     \hline
    %     0 & $x_2$ & 0.8 & 1 & 0 & 0.6 & 0.4 & 2\\
    %     \hline
    %     0 & $x_3$ & 0.2 & 0 & 1 & 0.1 & -0.1 & 0\\       \hline
    %     & $z_j - c_j$ & 0 & 0& 0 & 1 & 1 &0\\
    %     \hline
    %     \end{tabular}
    % \end{table}

    En este punto no podemos continuar ya que todos los valores de la fila $z_j - c_j$ son positivos, por lo que debemos de pasar a la fase 2, pero debemos de tener en cuenta que tenemos en la base la variable $t_2$ que es artificial, por lo que debemos de eliminarla. \textit{Vemos que en la 2 Fase, debemos de sacar $x_2$, pero vamos a sacar $t_2$ para que no nos de problemas. Previamente asignamos el coeficiente 0 a la variable $t_2$.}

    \subsubsection*{Fase 2}

    % Ahora debemos de eliminar las variables artificiales y continuar con el problema original.

    \begin{table}[H]
        \centering
        \begin{tabular}{|c|c|c|c|c|c|}
        \hline
        &  & 1 & 1 & 10 &\\
        \hline
        & VB & $x_1$ & $x_2$ & $x_3$ & XB \\
        \hline
        1 & $x_2$ & 0 & 1 & 4 & 2\\
        \hline
        10 & $t_2$ & -2 & 0 & -10 & 0\\
        \hline
        & $z_j - c_j$ & -1 & 0 & -6 &2\\
        \hline
        \end{tabular}
    \end{table}

    \begin{align*}
        F2/10  = F_p\\
        F1N = F1 - 4F_p
    \end{align*}

    \begin{table}[H]
        \centering
        \begin{tabular}{|c|c|c|c|c|c|}
        \hline
        &  & 1 & 1 & 10 &\\
        \hline
        & VB & $x_1$ & $x_2$ & $x_3$ & XB \\
        \hline
        1 & $x_2$ & -0.8 & 1 & 0 & 2\\
        \hline
        10 & $x_3$ & 0.2 & 0 & 1 & 0\\
        \hline
        & $z_j - c_j$ & 0.2 & 0 & 0 &2\\
        \hline
        \end{tabular}
    \end{table}

    Como todos los valores de la fila $z_j - c_j$ son positivos, hemos llegado a la solución óptima.

    
    \item[e)] Max. $x_1 + 2x_2$
    \begin{align*}
        \text{s.a.} \quad & x_1 + x_2 = 4 \\
        & 2x_1 - 3x_2 = 3 \\
        & 3x_1 - x_2 = 8 \\
        & x_1, x_2 \geq 0
    \end{align*}
\end{itemize}

\subsection*{Ejercicio 10}
Dado los siguientes problemas primales, encontrar sus problemas duales asociados:
\begin{itemize}
    \item[a)] Max. \quad $6x_1 + 4x_2$ \\
    s.a. \quad $x_1 \leq 700$ \\
    \quad \quad $3x_1 + x_2 \leq 2400$ \\
    \quad \quad $x_1 + 2x_2 \leq 1600$ \\
    \quad \quad $x_1, x_2 \geq 0$

    Debemos de calcular las siguientes matrices:
    
    $ A = 
    \begin{pmatrix}
        1 & 0 \\
        3 & 1 \\
        1 & 2
    \end{pmatrix} \quad
    A' = \begin{pmatrix}
        1 & 3 & 1 \\
        0 & 1 & 2
    \end{pmatrix}
    \quad b = \begin{pmatrix}
        700 \\
        2400 \\
        1600
    \end{pmatrix} \quad
    c = \begin{pmatrix}
        6 \\
        4
    \end{pmatrix}
    $

    De manera que nos queda:

    \begin{align*}
        Min \quad w = 700y_1 + 2400y_2 + 1600y_3 \\
        s.a. \quad y_1 + 3y_2 + y_3 \geq 6 \\
        y_2 + 2y_3 \geq 4 \\
        y_1, y_2, y_3 \geq 0
    \end{align*}

    
    
    \item[b)] Max. \quad $4.5x_1 + 3x_2 + 1.5x_3$ \\
    s.a. \quad $x_1 + 2x_2 - x_3 \leq 4$ \\
    \quad \quad $2x_1 - x_2 + x_3 = 8$ \\
    \quad \quad $x_1 - x_2 \leq 6$ \\
    \quad \quad $x_1, x_2, x_3 \geq 0$

    Debemos de calcular las siguientes matrices:

    $ A =
    \begin{pmatrix}
        1 & 2 & -1 \\
        2 & -1 & 1 \\
        1 & -1 & 0
    \end{pmatrix} \quad
    A' = \begin{pmatrix}
        1 & 2 & 1 \\
        2 & -1 & -1 \\
        -1 & 1 & 0
    \end{pmatrix}
    \quad b = \begin{pmatrix}
        4 \\
        8 \\
        6
    \end{pmatrix} \quad c = \begin{pmatrix}
        4.5 \\
        3 \\
        1.5
    \end{pmatrix}
    $

    De manera que nos queda:

    \begin{align*}
        Min \quad w = 4y_1 + 8y_2 + 6y_3 \\
        s.a. \quad y_1 + 2y_2 + y_3 \geq 4.5 \\
        2y_1 - y_2 - y_3 \geq 3 \\
        - y_1 + y_2 \geq 1.5 \\
        y_1, y_3 \geq 0 \quad y_2 \rightarrow \text{libre}
    \end{align*}
    
    \item[c)] Min. \quad $6x_1 + 4x_2$ \\
    s.a. \quad $x_1 \leq 700$ \\
    \quad \quad $3x_1 + x_2 \geq 2400$ \\
    \quad \quad $x_1 + 2x_2 \leq 1600$ \\
    \quad \quad $x_1, x_2 \geq 0$

    Debemos de calcular las siguientes matrices:

    $ A =
    \begin{pmatrix}
        1 & 0 \\
        3 & 1 \\
        1 & 2
    \end{pmatrix} \quad
    A' = \begin{pmatrix}
        1 & 3 & 1 \\
        0 & 1 & 2
    \end{pmatrix}
    \quad b = \begin{pmatrix}
        700 \\
        2400 \\
        1600
    \end{pmatrix} \quad c = \begin{pmatrix}
        6 \\
        4
    \end{pmatrix}
    $

    De manera que nos queda:

    \begin{align*}
        Max \quad w = 700y_1 + 2400y_2 + 1600y_3 \\
        s.a. \quad y_1 + 3y_2 + y_3 \leq 6 \\
        y_2 + 2y_3 \leq 4 \\
        y_2 \geq 0, y_1, y_3 \leq 0  
    \end{align*}
    
    \item[d)] Max. \quad $4.5x_1 + 3x_2 + 1.5x_3$ \\
    s.a. \quad $x_1 + 2x_2 - x_3 \leq 4$ \\
    \quad \quad $2x_1 - x_2 + x_3 \leq 8$ \\
    \quad \quad $x_1 - x_2 \leq 6$ \\
    \quad \quad $x_1, x_3 \geq 0, \quad x_2 \quad \text{sin restricciones}$

    Debemos de calcular las siguientes matrices:

    $ A =
    \begin{pmatrix}
        1 & 2 & -1 \\
        2 & -1 & 1 \\
        1 & -1 & 0
    \end{pmatrix} \quad
    A' = \begin{pmatrix}
        1 & 2 & 1 \\
        2 & -1 & -1 \\
        -1 & 1 & 0
    \end{pmatrix}
    \quad b = \begin{pmatrix}
        4 \\
        8 \\
        6
    \end{pmatrix} \quad c = \begin{pmatrix}
        4.5 \\
        3 \\
        1.5
    \end{pmatrix}
    $

    De manera que nos queda:    

    \begin{align*}
        Min \quad w = 4y_1 + 8y_2 + 6y_3 \\
        s.a. \quad y_1 + 2y_2 + y_3 \geq 4,5 \\
        2y_1 - y_2 - y_3 = 3 \\
        - y_1 + y_2 \geq 1,5 \\
        y_1, y_2, y_3 \geq 0 
    \end{align*}
    
    \item[e)] Min. \quad $6x_1 + 4x_2$ \\
    s.a. \quad $x_1 \leq 700$ \\
    \quad \quad $3x_1 + x_2 \geq 2400$ \\
    \quad \quad $x_1 + 2x_2 = 1600$ \\
    \quad \quad $x_1 \geq 0, \quad x_2 \quad \text{sin restricciones}$

    Debemos de calcular las siguientes matrices:

    $ A =
    \begin{pmatrix}
        1 & 0 \\
        3 & 1 \\
        1 & 2
    \end{pmatrix} \quad
    A' = \begin{pmatrix}
        1 & 3 & 1 \\
        0 & 1 & 2
    \end{pmatrix}
    \quad b = \begin{pmatrix}
        700 \\
        2400 \\
        1600
    \end{pmatrix} \quad c = \begin{pmatrix}
        6 \\
        4
    \end{pmatrix}
    $       

    De manera que nos queda:

    \begin{align*}
        Max \quad w = 700y_1 + 2400y_2 + 1600y_3 \\
        s.a. \quad y_1 + 3y_2 + y_3 \leq 6 \\
        y_2 + 2y_3 = 4 \\
        y_1\leq0 , y_2 \geq 0 \quad y_3 \rightarrow \text{libre}
    \end{align*}
\end{itemize}

