\documentclass[a4paper,12pt]{article}

% Paquetes básicos
\usepackage[utf8]{inputenc}
\usepackage[T1]{fontenc}
\usepackage[spanish]{babel}
\usepackage{graphicx}
\usepackage{xcolor}
\usepackage{lipsum}
\usepackage{geometry}
\geometry{top=3cm, bottom=3cm, left=2.5cm, right=2.5cm}

% Paquetes para diseño
\usepackage{titlesec}
\usepackage{fancyhdr}
\usepackage{amsmath}
\usepackage{amssymb}
\usepackage{hyperref}

% Paquetes para el entorno lstlisting
\usepackage{listings}
\usepackage{inconsolata}

%encabezado y pie de página nivel profesional
\usepackage{fancyhdr}
\pagestyle{fancy}
\fancyhf{}
\fancyhead[L]{\leftmark}
\fancyhead[R]{\rightmark}
\fancyfoot[L]{\textbf{Ismael Sallami Moreno - GIIADE}}
\fancyfoot[C]{\thepage}
\fancyfoot[R]{\textbf{(UGR)} \today}
\renewcommand{\headrulewidth}{0.4pt}
\renewcommand{\footrulewidth}{0.4pt}
\setlength{\headheight}{15pt}
\setlength{\headsep}{10pt}
\setlength{\footskip}{20pt}
\usepackage{truncate}
\fancyhead[L]{\truncate{0.5\headwidth}{\leftmark}}
\fancyhead[R]{\truncate{0.5\headwidth}{\rightmark}}
\usepackage{mathpazo}
\usepackage{tcolorbox}


% Paquete para fondo
\usepackage{background}
\usepackage{float}

% Configuración de lstlisting
\lstset{
    inputencoding=utf8,          % Permite UTF-8
    extendedchars=true,          % Reconoce caracteres extendidos
    literate=                    % Configuración manual para tildes y símbolos
        {á}{{\'a}}1
        {é}{{\'e}}1
        {í}{{\'i}}1
        {ó}{{\'o}}1
        {ú}{{\'u}}1
        {ñ}{{\~n}}1
        {Á}{{\'A}}1
        {É}{{\'E}}1
        {Í}{{\'I}}1
        {Ó}{{\'O}}1
        {Ú}{{\'U}}1
        {Ñ}{{\~N}}1
        {¿}{{\textquestiondown}}1
        {¡}{{\textexclamdown}}1,
    basicstyle=\ttfamily,        % Fuente monoespaciada
    breaklines=true,             % Habilita salto de línea automático
    frame=single,                % Marco alrededor del código
    backgroundcolor=\color{gray!10}, % Fondo gris claro
    keywordstyle=\color{blue},   % Color para palabras clave
    commentstyle=\color{green},  % Color para comentarios
    stringstyle=\color{red}      % Color para strings
}
\lstdefinestyle{customcpp}{
    language=C++,                % Lenguaje de programación
    showspaces=false,            % No mostrar espacios
    showtabs=false,              % No mostrar tabulaciones
    tabsize=4,                   % Tamaño de tabulación
    showstringspaces=false,      % No mostrar espacios en strings
    numbers=left,                % Números de línea a la izquierda
    numberstyle=\tiny\color{gray}, % Estilo de los números de línea
    numbersep=5pt,               % Separación de los números de línea
    stepnumber=1,                % Mostrar número en cada línea
    basicstyle=\ttfamily\footnotesize, % Estilo básico del código
    keywordstyle=\bfseries\color{blue}, % Estilo de las palabras clave
    commentstyle=\itshape\color{green!50!black}, % Estilo de los comentarios
    stringstyle=\color{red},     % Estilo de los strings
    identifierstyle=\color{black}, % Estilo de los identificadores
    % procnamekeys={def,class},    % Palabras clave para nombres de funciones
    morekeywords={constexpr,nullptr,size_t}, % Más palabras clave
    emph={int,char,double,float,unsigned}, % Palabras a enfatizar
    emphstyle=\color{magenta},   % Estilo de las palabras enfatizadas
    backgroundcolor=\color{gray!10}, % Color de fondo
    frame=shadowbox,             % Marco con sombra
    rulesepcolor=\color{gray},   % Color de la línea de separación
    breakatwhitespace=false,     % No cortar en espacios en blanco
    breaklines=true,             % Cortar líneas largas
    captionpos=b,                % Posición del título (abajo)
    escapeinside={(*@}{@*)},     % Delimitadores para escapar a LaTeX
    morecomment=[l][\color{magenta}]{\#}, % Comentarios de una línea
    morecomment=[s][\color{orange}]{/*}{*/}, % Comentarios multilínea
    morestring=[b]",             % Strings entre comillas dobles
    morestring=[b]'              % Strings entre comillas simples
}
%lstlistin personlizado para java y ruby
\lstdefinestyle{customjava}{
    language=Java,                % Lenguaje de programación
    showspaces=false,            % No mostrar espacios
    showtabs=false,              % No mostrar tabulaciones
    tabsize=4,                   % Tamaño de tabulación
    showstringspaces=false,      % No mostrar espacios en strings
    numbers=left,                % Números de línea a la izquierda
    numberstyle=\tiny\color{gray}, % Estilo de los números de línea
    numbersep=5pt,               % Separación de los números de línea
    stepnumber=1,                % Mostrar número en cada línea
    basicstyle=\ttfamily\footnotesize, % Estilo básico del código
    keywordstyle=\bfseries\color{blue}, % Estilo de las palabras clave
    commentstyle=\itshape\color{green!50!black}, % Estilo de los comentarios
    stringstyle=\color{red},     % Estilo de los strings
    identifierstyle=\color{black}, % Estilo de los identificadores
    % procnamekeys={def,class},    % Palabras clave para nombres de funciones
    morekeywords={constexpr,nullptr,size_t}, % Más palabras clave
    emph={int,char,double,float,unsigned}, % Palabras a enfatizar
    emphstyle=\color{magenta},   % Estilo de las palabras enfatizadas
    backgroundcolor=\color{gray!10}, % Color de fondo
    frame=shadowbox,             % Marco con sombra
    rulesepcolor=\color{gray},   % Color de la línea de separación
    breakatwhitespace=false,     % No cortar en espacios en blanco
    breaklines=true,             % Cortar líneas largas
    captionpos=b,                % Posición del título (abajo)
    escapeinside={(*@}{@*)},     % Delimitadores para escapar a LaTeX
    morecomment=[l][\color{magenta}]{\#}, % Comentarios de una línea
    morecomment=[s][\color{orange}]{/*}{*/}, % Comentarios multilínea
    morestring=[b]",             % Strings entre comillas dobles
    morestring=[b]'              % Strings entre comillas simples
}
\lstdefinestyle{customruby}{
    language=Ruby,                % Lenguaje de programación
    showspaces=false,            % No mostrar espacios
    showtabs=false,              % No mostrar tabulaciones
    tabsize=4,                   % Tamaño de tabulación
    showstringspaces=false,      % No mostrar espacios en strings
    numbers=left,                % Números de línea a la izquierda
    numberstyle=\tiny\color{gray}, % Estilo de los números de línea
    numbersep=5pt,               % Separación de los números de línea
    stepnumber=1,                % Mostrar número en cada línea
    basicstyle=\ttfamily\footnotesize, % Estilo básico del código
    keywordstyle=\bfseries\color{blue}, % Estilo de las palabras clave
    commentstyle=\itshape\color{green!50!black}, % Estilo de los comentarios
    stringstyle=\color{red},     % Estilo de los strings
    identifierstyle=\color{black}, % Estilo de los identificadores
    % procnamekeys={def,class},    % Palabras clave para nombres de funciones
    morekeywords={constexpr,nullptr,size_t}, % Más palabras clave
    emph={int,char,double,float,unsigned}, % Palabras a enfatizar
    emphstyle=\color{magenta},   % Estilo de las palabras enfatizadas
    backgroundcolor=\color{gray!10}, % Color de fondo
    frame=shadowbox,             % Marco con sombra
    rulesepcolor=\color{gray},   % Color de la línea de separación
    breakatwhitespace=false,     % No cortar en espacios en blanco
    breaklines=true,             % Cortar líneas largas
    captionpos=b,                % Posición del título (abajo)
    escapeinside={(*@}{@*)},     % Delimitadores para escapar a LaTeX
    morecomment=[l][\color{magenta}]{\#}, % Comentarios de una línea
    morecomment=[s][\color{orange}]{=begin}{=end}, % Comentarios multilínea
    morestring=[b]",             % Strings entre comillas dobles
    morestring=[b]'              % Strings entre com
}


% Configuración de título
\titleformat{\section}{\normalfont\Large\bfseries}{\thesection}{1em}{}

% Información del documento
\title{
    \vspace{-2cm}
    \includegraphics[width=0.3\textwidth]{images/etsiit.png} \\ % Cambia el logo si es necesario
    \LARGE Ingeniería Informática + ADE\\
    \large Universidad de Granada (UGR)\\[1cm]
}
\author{\textbf{Autor:} Ismael Sallami Moreno}
\date{\textbf{Asignatura:} Apuntes Visibilidad (PDOO)\\[1cm]}

% Configuración del fondo
\backgroundsetup{
    scale=1,
    color=black,
    opacity=0.2,
    angle=0,
    position=current page.south,
    vshift=0pt,
    hshift=0pt,
    contents={\includegraphics[width=\paperwidth,height=\paperheight,keepaspectratio]{images/granada.jpg}}
}

%comandos
\newcommand{\textorojo}[1]{\textbf{\textcolor{red}{#1}}}
\newcommand{\textoverde}[1]{\textbf{\textcolor{green}{#1}}}

% Inicio del documento
\begin{document}

% Portada
\maketitle
\thispagestyle{empty}

\begin{center}
    \includegraphics[width=\textwidth,height=0.4\textheight,keepaspectratio]{images/granada.jpg} \\ % Añade tu imagen de fondo
    \vfill
\end{center}

\newpage

% Índice (opcional)
\tableofcontents
\newpage

\section{Introducción}

\begin{itemize}
    \item Privado: Solo accesible desde la propia clase.
    \item Protegido: Solo accesible desde la propia clase y sus subclases.
    \item Público: Accesible desde cualquier clase.
\end{itemize}

\section{Java}

\subsection{Private}

\begin{itemize}
    \item Solo accesible desde la propia clase (Instancia y de clase).
    \item Se puede acceder a elementos privados de otra instancia distinta si es de la misma clase.
\end{itemize}

\subsection{Paquete}

\begin{itemize}
    \item Si no se pone ningún especificador de Visibilidad, por defecto es de paquete.
    \item son públicos dentro de paquete y privados fuera de él.
\end{itemize}

\subsection{Protected}
\begin{itemize}
    \item Públicos dentro del mismo paquete
    \item Accesible desde subclases de otros paquetes, es decir, dentro de una instancia se puede acceder a los elementos protegidos de sus superclases, con independencia del paquete en el que se encuentren.
    \item \textit{Si es de una instancia distinta}:
    \begin{itemize}
        \item Tienes que ser de la misma clase o una subclase.
        \item Debe de ser visible por mi.
    \end{itemize}
\end{itemize}


\section{Ruby}

\begin{itemize}
    \item Atributos siempre \textbf{privados}.
    \item Métodos siempre \textbf{públicos} (se pueden cambiar usando modificadores de acceso).
    \item Initialize es siempre \textbf{privado}.
\end{itemize}

\subsection{Private}
\begin{itemize}
    \item \textcolor{green}{Si se hereda se puede acceder a los métodos privados de la superclase (tanto en el ámbito de clase como de instancia).}
    \item \textcolor{red}{No se puede acceder a método privados de clase desde el ámbito de instancia y viceversa.}
\end{itemize}

\subsection{Protected}
\begin{itemize}
    \item \textcolor{green}{Se puede acceder a los métodos protegidos de la superclase desde la subclase.}
    \item \textcolor{red}{No existen métodos protegidos de clase.}
\end{itemize}

En ruby \textorojo{una clase y su instancia no son de la misma clase}. \\\\
Los atributos @@ \textoverde{si pueden ser accedidos desde ámbito de instancia}.

\section{Ejemplos en Java} 

\textit{Se especifican las soluciones en forma de comentarios.
}

\subsection{Ejemplo 1}

\begin{lstlisting}[style=customjava]
    package unPaquete;
    
    public class Padre {
        private int privado;        // Campo accesible solo dentro de esta clase.
        protected int protegido;    // Campo accesible dentro del paquete y subclases.
        int paquete;                // Campo accesible solo dentro del paquete.
        public int publico;         // Campo accesible desde cualquier lugar.
    
        // Método de instancia que accede a elementos de otra instancia de la misma clase.
        public void testInstanciaPadre(Padre o) {
            System.out.println(o.privado);   // Acceso permitido: "privado" es visible en la misma clase.
            System.out.println(o.protegido);// Acceso permitido: "protegido" es visible dentro de la clase.
            System.out.println(o.paquete);  // Acceso permitido: "paquete" es visible dentro del paquete.
            System.out.println(o.publico);  // Acceso permitido: "publico" es accesible desde cualquier lugar.
        }
    
        // Método estático que intenta acceder a elementos de una instancia.
        public static void testClasePadre(Padre o) {
            System.out.println(o.privado);   // Acceso permitido: "privado" es visible dentro de la misma clase.
            System.out.println(o.protegido);// Acceso permitido: "protegido" es visible dentro de la clase.
            System.out.println(o.paquete);  // Acceso permitido: "paquete" es visible dentro del paquete.
            System.out.println(o.publico);  // Acceso permitido: "publico" es accesible desde cualquier lugar.
        }
    }
    \end{lstlisting}

\subsection{Ejemplo 2}
\begin{lstlisting}[style=customjava]
    package unPaquete;
    
    public class HijaPaquete extends Padre {
    
        // Método de instancia que accede a elementos de la superclase y otra instancia.
        public void testInstanciaHijaPaquete(Padre o) {
            System.out.println(privado);      // Error: "privado" no es accesible fuera de su clase.
            System.out.println(o.privado);   // Error: "privado" no es accesible fuera de su clase.
    
            System.out.println(protegido);   // Acceso permitido: "protegido" es visible en subclases.
            System.out.println(o.protegido);// Acceso permitido: "protegido" es visible en el mismo paquete.
    
            System.out.println(o.paquete);   // Acceso permitido: "paquete" es visible dentro del paquete.
            System.out.println(o.publico);   // Acceso permitido: "publico" es accesible desde cualquier lugar.
        }
    
        // Método estático que accede a elementos de una instancia.
        public static void testClaseHijaPaquete(Padre o) {
            System.out.println(o.privado);   // Error: "privado" no es accesible fuera de su clase.
            System.out.println(o.protegido);// Acceso permitido: "protegido" es visible en el mismo paquete.
            System.out.println(o.paquete);  // Acceso permitido: "paquete" es visible dentro del paquete.
            System.out.println(o.publico);  // Acceso permitido: "publico" es accesible desde cualquier lugar.
        }
    }
    \end{lstlisting}

\subsection{Ejemplo 3}

\begin{lstlisting}[style=customjava]

package otroPaquete;

public class HijaOtroPaquete extends Padre {

    public void testInstanciaHijaOtroPaquete(Padre o) {
        // Acceso a elementos heredados
        System.out.println(privado);    // (Línea 7) Error: "privado" no es accesible fuera de la clase.
        System.out.println(paquete);   // (Línea 8) Error: "paquete" no es accesible desde otro paquete.
        System.out.println(protegido); // (Línea 9) Acceso permitido: "protegido" es visible en subclases.

        // Acceso a elementos de otra instancia
        System.out.println(o.privado);   // (Línea 12) Error: "privado" no es accesible desde otra instancia.
        System.out.println(o.protegido);// (Línea 13) Error: "protegido" no es accesible desde otra instancia en otro paquete.
        System.out.println(o.paquete);  // (Línea 14) Error: "paquete" no es accesible desde otro paquete.
        System.out.println(o.publico);  // (Línea 15) Acceso permitido: "publico" es accesible desde cualquier lugar.
    }

    public static void testClaseHijaOtroPaquete(Padre o) {
        // Acceso a elementos de otra instancia
        System.out.println(o.privado);   // (Línea 20) Error: "privado" no es accesible desde otra clase.
        System.out.println(o.protegido);// (Línea 21) Error: "protegido" no es accesible desde otra instancia en otro paquete.
        System.out.println(o.paquete);  // (Línea 22) Error: "paquete" no es accesible desde otro paquete.
        System.out.println(o.publico);  // (Línea 23) Acceso permitido: "publico" es accesible desde cualquier lugar.
    }
}
\end{lstlisting}

\subsection{Ejemplo 4}
\begin{lstlisting}[style=customjava]

package otroPaquete;

public class HijaOtroPaquete extends Padre {

    public void testInstanciaHijaOtroPaquete(HijaOtroPaquete o) {
        System.out.println(o.privado);   // (Línea 8) Error: "privado" no es accesible fuera de la clase.
        System.out.println(o.protegido);// (Línea 9) Acceso permitido: "protegido" es visible en subclases.
        System.out.println(o.paquete);  // (Línea 10) Error: "paquete" no es accesible desde otro paquete.
        System.out.println(o.publico);  // (Línea 11) Acceso permitido: "publico" es accesible desde cualquier lugar.
    }

    public static void testClaseHijaOtroPaquete(HijaOtroPaquete o) {
        System.out.println(o.privado);   // (Línea 15) Error: "privado" no es accesible desde otra clase.
        System.out.println(o.protegido);// (Línea 16) Acceso permitido: "protegido" es visible en subclases.
        System.out.println(o.paquete);  // (Línea 17) Error: "paquete" no es accesible desde otro paquete.
        System.out.println(o.publico);  // (Línea 18) Acceso permitido: "publico" es accesible desde cualquier lugar.
    }
}
\end{lstlisting}

\subsection{Ejemplo 5}
\begin{lstlisting}[style=customjava]

package otroPaquete;

public class HijaOtroPaquete extends Padre {

    public void testInstanciaHijaOtroPaquete(NietaOtroPaquete o) {
        System.out.println(o.privado);   // (Línea 8) Error: "privado" no es accesible fuera de la clase.
        System.out.println(o.protegido);// (Línea 9) Error: "protegido" no es accesible en instancias externas.
        System.out.println(o.paquete);  // (Línea 10) Error: "paquete" no es accesible desde otro paquete.
        System.out.println(o.publico);  // (Línea 11) Acceso permitido: "publico" es accesible desde cualquier lugar.
    }

    public static void testClaseHijaOtroPaquete(NietaOtroPaquete o) {
        System.out.println(o.privado);   // (Línea 15) Error: "privado" no es accesible desde otra clase.
        System.out.println(o.protegido);// (Línea 16) Error: "protegido" no es accesible en instancias externas.
        System.out.println(o.paquete);  // (Línea 17) Error: "paquete" no es accesible desde otro paquete.
        System.out.println(o.publico);  // (Línea 18) Acceso permitido: "publico" es accesible desde cualquier lugar.
    }
}
\end{lstlisting}
    



























% \subsection{Ejemplo 1}

% \begin{lstlisting}[style=customjava]
%     package unPaquete;
    
%     public class Padre {
%         private int privado;
%         protected int protegido;
%         int paquete;
%         public int publico;
    
%         public void testInstanciaPadre(Padre o) {
%             // No se puede acceder a 'privado' de otra instancia
%             // System.out.println(o.privado);
%             System.out.println(o.protegido);
%             System.out.println(o.paquete);
%             System.out.println(o.publico);
%         }
    
%         public static void testClasePadre(Padre o) {
%             // No se puede acceder a 'privado' de otra instancia
%             // System.out.println(o.privado);
%             System.out.println(o.protegido);
%             System.out.println(o.paquete);
%             System.out.println(o.publico);
%         }
%     }
% \end{lstlisting}

% \subsection{Ejemplo 2}

% \begin{lstlisting}[style=customjava]


% package unPaquete;

% public class HijaPaquete extends Padre {

%     public void testInstanciaHijaPaquete(Padre o) {
%         // No se puede acceder a 'privado' directamente ni a través de una instancia
%         // System.out.println(privado);
%         // System.out.println(o.privado);

%         System.out.println(protegido); // Se puede acceder porque es protegido
%         System.out.println(o.protegido); // Se puede acceder porque es protegido

%         System.out.println(o.paquete); // Se puede acceder porque está en el mismo paquete
%         System.out.println(o.publico); // Se puede acceder porque es público
%     }

%     public static void testClaseHijaPaquete(Padre o) {
%         // No se puede acceder a 'privado' de otra instancia
%         // System.out.println(o.privado);

%         System.out.println(o.protegido); // Se puede acceder porque es protegido
%         System.out.println(o.paquete); // Se puede acceder porque está en el mismo paquete
%         System.out.println(o.publico); // Se puede acceder porque es público
%     }
% }
% \end{lstlisting}

% \subsection{Ejemplo 3}

% \begin{lstlisting}[style=customjava]


% package otroPaquete;

% public class HijaOtroPaquete extends Padre {

%     public void testInstanciaHijaOtroPaquete(Padre o) {
%         // Acceso a elementos heredados
%         // No se puede acceder a 'privado' directamente
%         // System.out.println(privado);
%         // No se puede acceder a 'paquete' porque no está en el mismo paquete
%         // System.out.println(paquete);
%         System.out.println(protegido); // Se puede acceder porque es protegido

%         // Acceso a elementos de otra instancia
%         // No se puede acceder a 'privado' de otra instancia
%         // System.out.println(o.privado);
%         System.out.println(o.protegido); // Se puede acceder porque es protegido
%         // No se puede acceder a 'paquete' de otra instancia porque no está en el mismo paquete
%         // System.out.println(o.paquete);
%         System.out.println(o.publico); // Se puede acceder porque es público
%     }

%     public static void testClaseHijaOtroPaquete(Padre o) {
%         // Acceso a elementos de otra instancia
%         // No se puede acceder a 'privado' de otra instancia
%         // System.out.println(o.privado);
%         System.out.println(o.protegido); // Se puede acceder porque es protegido
%         // No se puede acceder a 'paquete' de otra instancia porque no está en el mismo paquete
%         // System.out.println(o.paquete);
%         System.out.println(o.publico); // Se puede acceder porque es público
%     }
% }

% \end{lstlisting}

% \subsection{Ejemplo 4}

% \begin{lstlisting}[style=customjava]

% package otroPaquete;

% public class HijaOtroPaquete extends Padre {

%     // El mismo código cambiando solo el tipo del parámetro
%     public void testInstanciaHijaOtroPaquete(HijaOtroPaquete o) {
%         // No se puede acceder a 'privado' de otra instancia
%         // System.out.println(o.privado);
%         System.out.println(o.protegido); // Se puede acceder porque es protegido
%         // No se puede acceder a 'paquete' de otra instancia porque no está en el mismo paquete
%         // System.out.println(o.paquete);
%         System.out.println(o.publico); // Se puede acceder porque es público
%     }

%     public static void testClaseHijaOtroPaquete(HijaOtroPaquete o) {
%         // No se puede acceder a 'privado' de otra instancia
%         // System.out.println(o.privado);
%         System.out.println(o.protegido); // Se puede acceder porque es protegido
%         // No se puede acceder a 'paquete' de otra instancia porque no está en el mismo paquete
%         // System.out.println(o.paquete);
%         System.out.println(o.publico); // Se puede acceder porque es público
%     }
% }

% \end{lstlisting}

% \subsection{Ejemplo 5}

% \begin{lstlisting}[style=customjava]

% package otroPaquete;

% public class HijaOtroPaquete extends Padre {

%     // El mismo código. El tipo del parámetro es subclase.
%     public void testInstanciaHijaOtroPaquete(NietaOtroPaquete o) {
%         // No se puede acceder a 'privado' de otra instancia
%         // System.out.println(o.privado);
%         System.out.println(o.protegido); // Se puede acceder porque es protegido
%         // No se puede acceder a 'paquete' de otra instancia porque no está en el mismo paquete
%         // System.out.println(o.paquete);
%         System.out.println(o.publico); // Se puede acceder porque es público
%     }

%     public static void testClaseHijaOtroPaquete(NietaOtroPaquete o) {
%         // No se puede acceder a 'privado' de otra instancia
%         // System.out.println(o.privado);
%         System.out.println(o.protegido); // Se puede acceder porque es protegido
%         // No se puede acceder a 'paquete' de otra instancia porque no está en el mismo paquete
%         // System.out.println(o.paquete);
%         System.out.println(o.publico); // Se puede acceder porque es público
%     }
% }

% // NietaOtroPaquete deriva de HijaOtroPaquete (ambas están en otroPaquete)

% \end{lstlisting}

\subsection{Ejemplo 6}
\begin{lstlisting}[style=customjava]
    package otroPaquete;
    
    public class NietaOtroPaquete extends HijaOtroPaquete {
    
        // Ahora probamos con un parámetro de la superclase
        public void testInstanciaNietaOtroPaquete(HijaOtroPaquete o) {
            // No se puede acceder a 'privado' de otra instancia
            // System.out.println(o.privado);
            // No se puede acceder a 'protegido' de otra instancia en un paquete diferente
            // System.out.println(o.protegido);
            // No se puede acceder a 'paquete' de otra instancia porque no está en el mismo paquete
            // System.out.println(o.paquete);
            System.out.println(o.publico); // Se puede acceder porque es público
        }
    
        public static void testClaseNietaOtroPaquete(HijaOtroPaquete o) {
            // No se puede acceder a 'privado' de otra instancia
            // System.out.println(o.privado);
            // No se puede acceder a 'protegido' de otra instancia en un paquete diferente
            // System.out.println(o.protegido);
            // No se puede acceder a 'paquete' de otra instancia porque no está en el mismo paquete
            // System.out.println(o.paquete);
            System.out.println(o.publico); // Se puede acceder porque es público
        }
    }
\end{lstlisting}

\subsection{Ejemplo 7}

\begin{lstlisting}[style=customjava]
    package otroPaquete;
    
    public class NietaOtroPaquete extends HijaOtroPaquete {
    
        public void testInstanciaNietaOtroPaquete(NietaOtroPaquete o) {
            // No se puede acceder a 'privado' de otra instancia
            // System.out.println(o.privado);
            System.out.println(o.protegido); // Se puede acceder porque es protegido
            // No se puede acceder a 'paquete' de otra instancia porque no está en el mismo paquete
            // System.out.println(o.paquete);
            System.out.println(o.publico); // Se puede acceder porque es público
        }
    
        public static void testClaseNietaOtroPaquete(NietaOtroPaquete o) {
            // No se puede acceder a 'privado' de otra instancia
            // System.out.println(o.privado);
            System.out.println(o.protegido); // Se puede acceder porque es protegido
            // No se puede acceder a 'paquete' de otra instancia porque no está en el mismo paquete
            // System.out.println(o.paquete);
            System.out.println(o.publico); // Se puede acceder porque es público
        }
    }
    \end{lstlisting}

\subsection{Ejemplo 8}

\begin{lstlisting}[style=customjava]
    // En el paquete base
    package base;
    
    public class A {
        protected int protegidoA = 0;
    }
    
    // En el paquete base
    public class B extends A {
        protected int protegidoB = 1;
    }
    
    // En el paquete base2
    package base2;
    import base.*;
    
    public class C extends B {
        protected int protegidoC = 2;
    
        public void test() {
            A a = new A();
            // No se puede acceder a 'protegidoA' de una instancia de A en un paquete diferente
            // a.protegidoA = 666;
    
            B b = new B();
            // No se puede acceder a 'protegidoB' de una instancia de B en un paquete diferente
            // b.protegidoB = 666;
    
            C c = new C();
            c.protegidoA = 555; // Se puede acceder porque C hereda de A
            c.protegidoB = 555; // Se puede acceder porque C hereda de B
    
            D d = new D();
            d.protegidoA = 555; // Se puede acceder porque D hereda de A
            d.protegidoB = 555; // Se puede acceder porque D hereda de B
            d.protegidoD = 555; // Se puede acceder porque es miembro de D
    
            D d2 = new D();
            d2.protegidoB = 555;
            // No se puede acceder a 'protegidoD' de otra instancia en un paquete diferente
            // d2.protegidoD = 555;
    
            this.protegidoA = 777; // Se puede acceder porque C hereda de A
            this.protegidoB = 777; // Se puede acceder porque C hereda de B
            this.protegidoC = 777; // Se puede acceder porque es miembro de C
        }
    }
    
    // En el paquete base2
    package base2;
    
    public class D extends C {
        protected int protegidoD = 3;
    }
\end{lstlisting}


\section{Ejemplos en Ruby}

\subsection{Ejemplo 1}
\begin{lstlisting}[style=customruby]
    class Padre
      private
      def privado
      end
    
      protected
      def protegido
      end
    
      public
      def publico
      end
    
      def test(p)
        privado
        self.privado # Correcto solo a partir de Ruby 2.7
        # p.privado # No se puede acceder a un método privado de otra instancia
    
        protegido
        self.protegido
        p.protegido # Se puede acceder a un método protegido de otra instancia
      end
    end
    \end{lstlisting}

\subsection{Ejemplo 2}
\begin{lstlisting}[style=customruby]
    # Fuera de cualquier clase
    
    Padre.new.test(Padre.new)
    p = Padre.new
    
    # Acceso a métodos fuera de la clase o subclases
    
    # p.privado # No se puede acceder a un método privado fuera de la clase
    # p.protegido # No se puede acceder a un método protegido fuera de la clase o subclases
    p.publico # Se puede acceder a un método público
    \end{lstlisting}

\subsection{Ejemplo 3}

\begin{lstlisting}[style=customruby]
    class Hija < Padre
      def test(p)
        privado
        self.privado # Correcto solo a partir de Ruby 2.7
        # p.privado # No se puede acceder a un método privado de otra instancia
    
        protegido
        self.protegido
        p.protegido # Se puede acceder a un método protegido de otra instancia
    
        publico
        self.publico
        p.publico
      end
    end
    
    # Fuera de cualquier clase
    
    Hija.new.test(Hija.new)
    Hija.new.test(Padre.new)
    h = Hija.new
    # h.privado # No se puede acceder a un método privado fuera de la clase
    # h.protegido # No se puede acceder a un método protegido fuera de la clase o subclases
    h.publico # Se puede acceder a un método público
    \end{lstlisting}

\subsection{Ejemplo 4}

\begin{lstlisting}[style=customruby]
    class Padre
      private
      def privado_instancia
      end
    
      def self.privado_clase
      end
    
      private_class_method :privado_clase
    
      public
      def test
        # No se puede acceder a un método privado de clase con un receptor explícito
        # self.class.privado_clase
      end
    
      def self.test(p)
        # No se puede acceder a un método privado de instancia de otra instancia
        # p.privado_instancia
      end
    end
    
    # Fuera de cualquier clase
    
    Padre.new.test
    Padre.test(Padre.new)
    \end{lstlisting}

\subsection{Ejemplo 5}

\begin{lstlisting}[style=customruby]
    class Padre
      @instanciaDeClase = 1
      @duda = 2
      @@deClase = 11
      @@duda = 22
    
      def initialize
        @deInstancia = 333
        @duda = 444
      end
    
      def self.salida
        # puts @instanciaDeClase + 1
        # No se puede acceder a la variable de instancia de la clase desde una subclase
        # puts @duda + 1 unless @duda.nil? # desde Hija?
        puts @@deClase + 1
        puts @@duda + 1
      end
    
      def salida
        #puts @deInstancia + 1
        #puts @duda + 1
        # No se puede acceder porque en Ruby todos los atributos son privados
        puts @@deClase + 1
        puts @@duda + 1
      end
    end
\end{lstlisting}

\subsection{Ejemplo 6}

\begin{lstlisting}[style=customruby]
    class Hija < Padre
      @instanciaDeClase = -1
    
      # Sobreescribimos el valor fijado anteriormente
      # Este atributo es compartido
      @@deClase = -11
    
      def modifica
        # Acceso a los atributos definidos en Padre
        # No se puede acceder a la variable de instancia de la clase desde una instancia
        # @duda += 111
      end #deberia de ser self.modifica
    end
    
    # Fuera de cualquier clase
    
    Padre.salida
    Hija.salida # Atención a lo que ocurre con la segunda línea
    
    p = Padre.new
    p.salida
    h = Hija.new
    h.salida
    
    # h.modifica debería de ser self.modifica
    h.salida
\end{lstlisting}

\subsection{Ejemplo 7}

\begin{lstlisting}[style=customruby]
    class A
      protected
      def protegidoA
      end
    end
    
    class B < A
      protected
      def protegidoB
      end
    end
    
    class C < B
      protected
      def protegidoC
      end
    
      public
      def test
        A.new.protegidoA
        B.new.protegidoA
        B.new.protegidoB
        C.new.protegidoA
        C.new.protegidoB
        C.new.protegidoC
        D.new.protegidoA
        # D.new.protegidoD
        # No puedo debido a que estoy en C y es D la que hereda de C, puedo acceder de D a C, pero no de C a D
      end
    end
    
    class D < C
      protected
      def protegidoD
      end
    end
    
    C.new.test
    # C.new.protegidoC
    # No se puede acceder a un método protegido fuera de la clase o subclases
\end{lstlisting}

\subsection{Ejemplo 8}
\begin{lstlisting}[style=customruby]
    class FalsaSeguridad
      # Un consultor puede ser muy peligroso
      attr_reader :lista
    
      def initialize
        @lista = [1, 2, 3, 4]
      end
    
      def info
        puts @lista.size
      end
    end
    
    # Fuera de cualquier clase
    f = FalsaSeguridad.new
    f.info # 4
    
    # Modificamos el estado interno
    # Simplemente usando un consultor
    # Aunque el atributo sea privado
    # Cuidado con las referencias
    f.lista.clear
    
    f.info # 0
\end{lstlisting}

\end{document}