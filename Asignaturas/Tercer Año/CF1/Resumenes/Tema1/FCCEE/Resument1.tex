\documentclass[a4paper,12pt]{article}

% Paquetes básicos
\usepackage[utf8]{inputenc}
\usepackage[T1]{fontenc}
\usepackage[spanish]{babel}
\usepackage{graphicx}
\usepackage{xcolor}
\usepackage{lipsum}
\usepackage{geometry}
\geometry{top=3cm, bottom=3cm, left=2.5cm, right=2.5cm}

% Paquetes para diseño
\usepackage{titlesec}
\usepackage{fancyhdr}
\usepackage{amsmath}
\usepackage{amssymb}
\usepackage{hyperref}
\usepackage{float}

% Paquetes para el entorno lstlisting
\usepackage{listings}
\usepackage{inconsolata}

% Paquete para diagramas
\usepackage{tikz}
\usetikzlibrary{shapes, arrows.meta, positioning}

% Paquete para fondo
\usepackage{background}

% Configuración de lstlisting
\lstset{
    language=Python,
    basicstyle=\ttfamily\small,
    keywordstyle=\color{blue}\bfseries,
    stringstyle=\color{teal},
    commentstyle=\color{gray}\itshape,
    numbers=left,
    numberstyle=\tiny\color{gray},
    backgroundcolor=\color{black!5},
    frame=single,
    rulecolor=\color{black!50},
    breaklines=true,
    captionpos=b,
    showstringspaces=false
}


%comandos
\usepackage{amsmath}
\newcommand{\flechita}{$\rightarrow$}

% Configuración de título
\titleformat{\section}{\normalfont\Large\bfseries}{\thesection}{1em}{}

% Información del documento
\title{
    \vspace{-2cm}
    \includegraphics[width=0.3\textwidth]{images/fccee.jpg} \\ % Cambia el logo si es necesario
    \LARGE Ingeniería Informática + ADE\\
    \large Universidad de Granada (UGR)\\[1cm]
}
\author{\textbf{Autor:} Ismael Sallami Moreno}
\date{\textbf{Asignatura:} Resúmenes de Contabilidad Financiera I Tema 1: Normalización Contable y Plan General de Contabilidad}

% Configuración del fondo
\backgroundsetup{
    scale=1,
    color=black,
    opacity=0.2,
    angle=0,
    position=current page.south,
    vshift=0pt,
    hshift=0pt,
    contents={\includegraphics[width=\paperwidth,height=\paperheight,keepaspectratio]{images/granada.jpg}}
}

% Configuración del encabezado y pie de página
\usepackage{times}
\pagestyle{fancy}
\fancyhf{}
\fancyhead[L]{\textbf{\textsf{\leftmark}}}
\fancyhead[R]{\textbf{\textsf{\thepage}}}
\fancyfoot[C]{\thepage}

% Inicio del documento
\begin{document}

% Portada
\maketitle
\thispagestyle{empty}

\begin{center}
    \includegraphics[width=\textwidth,height=0.4\textheight,keepaspectratio]{images/granada.jpg} \\ % Añade tu imagen de fondo
    \vfill
\end{center}

\newpage

% Índice (opcional)
\tableofcontents
\newpage


\section{Normas Internacionales de Información Financiera}

\subsection{Concepto}
Es el proceso mediante el que emite las normas contables para poder convertir los documentos de todas las organizaciones en documentos que sean comparables con el resto de organizaciones y poder presentar la Información de manera uniforme. Debido a la amplia gama de usuarios la información debe de cumplir con las caracterrísticas de: \textbf{Relevante, fiable y comparable}.\\

Dependiendo del país las normas contables puede ser emitidas por dos entidades:
\begin{itemize}
    \item Organizaciones Privadas
    \item Administraciones Públicas
\end{itemize}

Un caso que podemos tener en la práctica es, por ejemplo cuando compramos un coche y el anterior lo entregamos como parte de pago, en el caso de que la amortización del vehículo sea mayor que su valor contable, pues tenemos un sobrevalor, el cual podemos contabilizar de dos maneras distintas, bien podemos considerarlo como beneficios del ejercicio o bien como menor coste del vehículo adquirido.\\

De la misma manera pasaría si pensamos en el caso de que tenemos una denuncia, podemos considerar que esa denuncia implica una deuda, aunque no haya sentencia judicial firme o bien, considerar que la denuncia no implica deuda hasta que no haya sentencia judicial firme.\\

Y otros muchos casos que podemos considerar...

Organismos que emiten las Normas Contables Internacionales:
\begin{itemize}
    \item Europa: IASB
    \item EEUU: FASB
    \item Nivel Mundial: IFAC
\end{itemize}

Etapas de la normalización contable en Europa:
\begin{itemize}
    \item 1era \flechita 1970 al 1990: se emiten las directivas
    \item 2nd \flechita 1991 al 2000: emisión de las Comunicaciones
    \item 3era \flechita 2002 en adelante: emisión de reglamentos y de las NIIF por parte del IASB, las cuales se toman como base para la elaboración de las normas contables.
\end{itemize}


\subsection{Endorsement}
Entendemos por Endorsement como el mecanismo de aceptación de las NIIF, este proceso atraviesa dos etapas:
\begin{itemize}
    \item EFRAG: emite el informe sobre la utilidad de adoptar las NIIF
    \item Cómite de Reglamentación Contable debe de emitir informe sobre dichos aspectos.
\end{itemize}

Para que se acepten como NIIF deben de cumplir con los siguientes requisitos:
\begin{itemize}
    \item No ser contrarios a la imagen fiel
    \item Cumplir con relevancia, fiabilidad, comprensibilidad y comparabilidad
    \item Favorecer al interés público de Europa
\end{itemize}

\section{La Normativa Contable en España}

El encargado en España es el \textbf{ICAC}, cuyas funciones son: elaboración del PGC, adapatación del PGC, análisis y propuesta de la normativa, coordinación y cooperación internacional, informes sobre las disposiciones.\\

El principal marco legal de la normalización contable española es el PGC, el cual tiene 3 tipos de pronunciamientos: Resoluciones, Adaptaciones Sectoriales, Consultas.

\subsection{Partes del PGC}
\subsubsection{Primera Parte: Marco Conceptual}
\begin{enumerate}
    \item Cuentas anuales. Imagen fiel
    \item Requisitos de la información a incluir en las cuentas anuales
    \item Principios contables
    \item Elementos de las cuentas anuales
    \item Criterios de registro o reconocimiento contable de los elementos de las cuentas anuales
    \item Criterios de valoración
    \item Principios y normas de contabilidad generalmente aceptados
\end{enumerate}

\subsubsection{Segunda Parte: Normas de Registro y Valoración}

En esta parte se encuentran las normas de registro y valoración como pueden ser las del inmovilizado material, intangible, pasivos por retribuciones al personal, \dots

\subsubsection{Tercera Parte: Cuentas Anuales}

\begin{table}[H]
    \centering
    \begin{tabular}{|p{5cm}|p{5cm}|}
    \hline
    \textbf{CUENTA ANUAL} & \textbf{CONTENIDO/UTILIDAD} \\ \hline
    Balance & Refleja la imagen fiel del patrimonio, incluyendo activos, pasivos y patrimonio neto de la empresa \\ \hline
    Cuenta de Pérdidas y Ganancias & Muestra el resultado del ejercicio, formado por ingresos y gastos devengados en el mismo, salvo cuando proceda su imputación directa al patrimonio neto \\ \hline
    Estado de Cambios en el Patrimonio Neto & Recoge las variaciones en el patrimonio neto, incluyendo dos partes: estado de ingresos y gastos reconocidos; y estado total de cambios en el patrimonio neto \\ \hline
    Estado de Flujos de Efectivo & Informa sobre el origen y uso de los activos monetarios representativos de efectivo y otros medios líquidos, clasificando movimientos por actividades \\ \hline
    Memoria & Completa, amplía y comenta la información contenida en las otras cuatro cuentas anuales. \\ \hline
    \end{tabular}
\end{table}

Se pueden presentar dos modelos:
\begin{itemize}
    \item Modelo normal que incuye todas las cuentas.
    \item Modelo abreviado, que solo van a poder aquellas empresas que cumplan unos requisitos de tamaño. En este tenemos las cuentas: Balance, Cuenta de Pérdidas y Ganancias y la Memoria.
\end{itemize}
    
Los requisitos son los siguientes:

\begin{table}[H]
    \centering
    \begin{tabular}{|p{5cm}|p{5cm}|p{5cm}|}
    \hline
    \textbf{ESTADO FINANCIERO} & \textbf{MODELO NORMAL} & \textbf{MODELO ABREVIADO (*)} \\ \hline
    Balance & & Podrá aplicar modelos abreviados la sociedad que, a fecha de cierre del ejercicio, y durante dos ejercicios consecutivos, no supere dos de los tres límites siguientes:\newline - Total Activos \(\leq\) 4.000.000 euros.\newline - Cifra de Negocios \(\leq\) 8.000.000 euros.\newline - Número medio de trabajadores \(\leq\) 50. \textit{(Incluye Estado de Cambios de Patrimonio Neto y Memoria)}\\ \hline
    Estado de Cambios en el Patrimonio Neto & Cuando no se puedan utilizar los modelos abreviados & No será obligatorio si el balance, el ECPN y la memoria se pueden realizar en los modelos abreviados. \\ \hline
    Memoria & Criterios modificados por la Ley 14/2013, de 27 de septiembre de apoyo a emprendedores & \\ \hline
    Estado de Flujos de Efectivo & & \\ \hline
    Cuenta de Pérdidas y Ganancias & Cuando no se puedan utilizar los modelos abreviados & Podrá aplicar modelos abreviados la sociedad que, a fecha de cierre del ejercicio, y durante dos ejercicios consecutivos, no supere dos de los tres límites siguientes:\newline - Total Activos \(\leq\) 11.400.000 euros.\newline - Cifra de Negocios \(\leq\) 22.800.000 euros.\newline - Número medio de trabajadores \(\leq\) 250. \\ \hline
    \end{tabular}
\end{table}

\subsection{Cuarta Parte: Cuadro de Cuentas}

%cuadro
    \begin{table}[H]
        \centering
        \begin{tabular}{|p{3cm}|p{12cm}|}
        \hline
        \textbf{Grupo} & \textbf{Descripción} \\ \hline
        1 & Financiación Básica \\ \hline
        2 & Activo no corriente \\ \hline
        3 & Existencias \\ \hline
        4 & Acreedores y deudores por operaciones comerciales \\ \hline
        5 & Cuentas financieras \\ \hline
        6 & Compras y gastos \\ \hline
        7 & Ventas e ingresos \\ \hline
        8 & Gastos imputados al patrimonio neto \\ \hline
        9 & Ingresos imputados al patrimonio neto \\ \hline
        \end{tabular}
        \caption{Cuadro de cuentas}
    \end{table}

    Para más información sobre cada grupo, puedes visitar este enlace: \href{https://mundocontabilidad.es/plan-general-de-contabilidad/cuadro-de-cuentas/}{Guía Contabilidad 2024}

    \subsection{Quinta Parte: Definiciones y Relaciones Contables}

    \begin{itemize}
        \item Balance: Cuentas de los grupos del 1 al 5.
        \item Cuenta de Pérdidas y Ganancias: Cuentas de los grupos 6 y 7.
        \item Estado de Cambios en el Patrominio Neto: Cuentas de los grupos 8 y 9.
    \end{itemize}

    \subsubsection*{Ejercicio Práctica}
    \textbf{\underline{Ejercicio práctico} sobre selección de cuentas para reflejar elementos patrimoniales. Para cada punto, usando el cuadro de cuentas del PGC y las definiciones y relaciones contables, seleccionar la cuenta más apropiada, indicando nombre y número.}

 


    \begin{table}[H]
        \centering
        \begin{tabular}{|p{5cm}|p{5cm}|}
            \hline
            \textbf{ELEMENTO} & \textbf{CUENTA} \\ \hline
            1. Vehículo de reparto de las mercancías que vende la empresa en su actividad ordinaria & \textcolor{green}{221. Transportes} \\ \hline
            2. Vehículos en una empresa concesionaria de coches, destinados a la venta & \textcolor{green}{300. Mercaderías} \\ \hline
            3. Ordenadores que se usan en la empresa para el control del inventario de las mercancías que vende en su actividad ordinaria & \textcolor{green}{217. Equipos para procesos de información} \\ \hline
            4. Nave industrial donde la empresa fabrica los productos para su venta & \textcolor{green}{211. Construcciones} \\ \hline
            5. Préstamos suscritos con bancos, pendientes de pago, con vencimiento a largo plazo & \textcolor{green}{171. Deudas a largo plazo con entidades de crédito} \\ \hline
            6. Préstamos suscritos con bancos, pendientes de pago, con vencimiento a corto plazo & \textcolor{green}{520. Deudas a corto plazo con entidades de crédito} \\ \hline
            7. Acciones de otra empresa adquiridas en Bolsa como inversión a largo plazo & \textcolor{green}{250. Inversiones financieras a largo plazo en instrumentos de patrimonio} \\ \hline
            8. Acciones de otra empresa adquiridas en Bolsa con la intención de venderlas en un mes & \textcolor{green}{540. Inversiones financieras temporales en instrumentos de patrimonio} \\ \hline
            9. Deudas pendientes de pago con la Agencia Estatal de la Administración Tributaria & \textcolor{green}{475. Hacienda Pública acreedora por conceptos fiscales} \\ \hline
            10. Deudas mantenidas con suministradores de las materias primas que consume la empresa en su proceso productivo & \textcolor{green}{400. Proveedores} \\ \hline
        \end{tabular}
        \caption{Tabla 1 de elementos y cuentas}
    \end{table}
    


\section{Marco Conceptual de la Contabilidad en el PGC}

\subsection{Concepto y Estructura de Contabilidad}

Es una guía o referencia en la elaboración de las normas contables, es decir, partiendo de unos conceptos básicos se establecen las normas contables para su aplicación en la práctica.\\

Sirve de soporte teórico para desarrollar los fundamentos conceptuales en los que se basa la contabilidad.\\

Para mas información pincha \href{https://www.aeca.es/old/pub/documentos/sf1r.htm}{aquí.}

\subsection{El Marco Conceptual en el PGC}

Se encuentra recogido en la primera parte del PGC, y se compone de los siguientes elementos:
\begin{enumerate}
    \item Cuentas anuales. Imagen fiel
    \begin{itemize}
        \item Balance
        \item Cuenta de Pérdidas y Ganancias
        \item Estado de Cambios en el Patrimonio Neto
        \item Estado de Flujos de Efectivo
        \item Memoria
    \end{itemize}
    \item Requisitos de la información a incluir en las cuentas anuales
    \item Principios contables
    \item Elementos de las cuentas anuales
    \item Criterios de registro o reconocimiento contable de los elementos de las cuentas anuales
    \item Criterios de valoración
    \item Principios y normas de contabilidad generalmente aceptados
\end{enumerate}

\subsection{Características cualitativas de la información financiera}

Los requisitos que deben de cumplir la información financieras son:

\begin{enumerate}
    \item Relevancia: la información debe de ser útil para los usuarios.
    \item Fiabilidad: la información debe de ser fiable, libre de errores materiales y debe de ser neutral.
    \item Comparabilidad: La información debe de ser comparable con la de otras empresas.
    \item Claridad: Debe de ser clara para los usuarios.
\end{enumerate}


\section{Elementos de los Estados Financieros}

En esta parte se incluyen los activos que posee la empresa, los pasivos que tiene y el patrimonio neto.\\

Por lo que podemos distinguir entre:
\begin{itemize}
    \item \textbf{Balance de Situación}:
    \begin{itemize}
        \item \textbf{Activo}: bienes y derechos controlados económicamente por la empresa, resultado de sucesos pasados, de los que se espera que se consiga rendimiento económico en el futuro.
        
        \subsubsection*{Reconocimiento de los activos}
        Los activos deben de reconocerse en el Balance cuando sea probable la obtención a partir de los mismos de benedificios o rendimientos económicos para la empresa en el futuro y siempre que se pueda evaluar con fiablidad.\\
        Esto implica de manera análoga el reconocimiento de un pasico, disminución de un activo, reconocimiento de un ingreso o incrementos en el PN.

        \item \textbf{Pasivo}: obligaciones actuales de la empresa, surgidas de sucesos pasados, cuyo cumplimiento se espera que resulte en una salida de recursos que incorporan beneficios económicos. A estos efectos, se deben de incluit las provisiones.
        
        \subsubsection*{Reconocimiento de los pasivos}

        Los pasivos deben de reconocerse en el Balance cuando sea probable que se produzca una salida de recursos que incorporen beneficios económicos futuros y siempre que se pueda evaluar con fiabilidad.\\ 
        Esto implica de manera análoga el reconocimiento de un activo, disminución de un pasivo, reconocimiento de un gasto o disminución del PN.

        \item \textbf{Patrimonio Neto}: la parte residual de los activos de la empresa, una vez deducidos todos sus pasivos. Incluyen las aportaciones realizadas por los socios o propietarios, ya sea en el momento de la constitución de la empresa o en situaciones posteriores, que no tengan la consideración de pasivos, así como los resultados acumulados u otras variaciones que le afecten.
    \end{itemize}
    \item \textbf{Cuenta de Pérdidas y Ganancias y Estado de Cambios en el Patrimonio Neto}
    \begin{itemize}
        \item \textbf{Gastos}: Decrementos en el PN de la empresa durante el ejercicio, ya sea en forma de salidas o decrementos de valor de los activos, o incrementos de pasivos, que no estén relacionados con distribuciones a los socios. 
        
        \subsubsection*{Reconocimiento de los gastos}

        Los gastos tienen lugar como consecuencia de una disminución de los recursos de la empresa, y siempre que su cuantía pueda estimarse con fiabilidad.\\
        Este implica de manera análoga el recocimientos de un pasivo o el incremento del mismo, y la disminución o desaparición de un activo.

        \item \textbf{Ingresos}: Incrementos en el PN de la empresa durante el ejercicio, ya sea en forma de entradas o incrementos de valor de los activos, o disminuciones de pasivos, que no estén relacionados con aportaciones de los socios.
        
        \subsubsection*{Reconocimiento de los ingresos}

        Los ingresos tienen lugar como consecuencia de un incremento de los recursos de la empresa, y siempre que su cuantía pueda estimarse con fiabilidad.\\
        Este implica de manera análoga el recocimientos de un activo o el incremento del mismo, y la disminución o desaparición de un pasivo.

    \end{itemize}
    \item \textbf{Estados de flujos de efectivo}
    \begin{itemize}
        \item \textbf{Cobros}
        \item \textbf{Pagos}
    \end{itemize}
\end{itemize}


\section{Principios de la Contabilidad y Criterior de Valoración}

\subsection{Principio de la Empresa en Funcionamiento}

Se considerará salvo prueba en contrario que la gestión de la empresa continuará en el futuro, por lo que la aplicación de los principios contables no tienen el propósito de determinar el valor del PN a efectos de una transmisión global o parcial, ni el importe resultante en caso de liquidación.\\
En situaciones anormales, en las que se debe de considerar un horizonte temporal limitado para la entidad, se debe de indicar en el Memoria.\\

\subsection{Principio de Devengo}

LOs efectos de las transacciones o hechos económicos deben de registrarse cuando ocurran, imputándose al ejercicio las cuentas anuales que se refieran, los gastos e ingresos que afecten al mismo, con independencia de la fecha de su cobro o pago.\\

\subsection{Principio de Uniformidad}

Los criterios contables deben de ser aplicados de manera uniforme en los distintos ejercicios, para que la información sea comparable.\\

\subsection{Principio de Prudencia}

Cuando existan varias alternativas para la valoración de un elemento patrimonial, se debe de elegir la que menos beneficios genere, y la que más gastos genere. Siempre se debe de valorar la imagen fiel de la empresa.\\

\subsection{Principio de No Compensación}

No se deben de compensar los activos con los pasivos, ni los ingresos con los gastos.\\

\subsection{Principio de Importancia Relativa}

Se debe de aplicar el principio de importancia relativa, es decir, se deben de aplicar los principios contables de manera que se refleje la imagen fiel de la empresa, pero sin que se incurra en costes desproporcionados.\\

\subsection{Criterios de Valoración}

Se entiende por valoración al proceso mediante el que se determina el valor de los elementos patrimoniales, es decir, la unidad monetaria que se le atribuye para que figure en los estado financieros.\\

\subsubsection{Coste histórico}

Es el precio de adquisición de un activo o el importe de la contraprestación recibida por la venta de un pasivo.\\

\subsubsection{Valor razonable}

Es el precio que se obtendría en una transacción de compraventa entre partes interesadas, debidamente informadas y sin presión.\\

\subsubsection{Valor neto realizable}

Es el precio de venta estimado en el mercado, menos los costes estimados para la venta.\\

\subsubsection{Valor actual}

Es el valor actual de los flujos de efectivo futuros que se espera obtener de un activo o pasivo.\\

\subsubsection{Valor en uso}

Es el valor actual de los flujos de efectivo futuros que se espera obtener de un activo o pasivo.\\

\subsubsection{Coste amortizado}

Es el importe al que se valora un activo o pasivo financiero en el momento de su reconocimiento inicial, menos los pagos de principal que se hayan efectuado, más o menos la amortización acumulada de la diferencia entre el importe inicial y el importe de vencimiento.\\

\subsubsection{Coste de Venta}

Es el importe que se espera obtener de la venta de un activo en el curso normal de la actividad de la empresa.\\

\subsubsection{Coste de transacción atribuibles a un activo a un pasivo financiero}

Es el importe de los costes directamente atribuibles a la adquisición, emisión o venta de un activo o pasivo financiero.\\

\subsubsection{Valor contable o en libros}

Es el importe por el que se encuentra registrado un activo o pasivo en la contabilidad de la empresa, una vez deducida la amortización acumulada y las pérdidas por deterioro acumuladas.\\

\subsubsection{Valor residual}

Es el importe estimado que la empresa podría obtener por la venta de un activo, una vez deducidos los costes estimados para la venta, si el activo tuviera ya la edad y el estado esperado al final de su vida útil.\\










\newpage
\section{Cuestionario tipo test}
Para realizar el tipo test del libro pinche \href{https://elblogdeismael.github.io/Asignaturas/Tercer%20A%C3%B1o/CF1/Tests/testT1Libro.html}{aquí.}




\end{document}
