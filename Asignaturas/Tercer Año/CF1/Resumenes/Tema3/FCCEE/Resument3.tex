\documentclass[a4paper,12pt]{article}

% Paquetes básicos
\usepackage[utf8]{inputenc}
\usepackage[T1]{fontenc}
\usepackage[spanish]{babel}
\usepackage{graphicx}
\usepackage{xcolor}
\usepackage{lipsum}
\usepackage{geometry}
\geometry{top=3cm, bottom=3cm, left=2.5cm, right=2.5cm}

% Paquetes para diseño
\usepackage{titlesec}
\usepackage{fancyhdr}
\usepackage{amsmath}
\usepackage{amssymb}
\usepackage{hyperref}
\usepackage{float}

% Paquetes para el entorno lstlisting
\usepackage{listings}
\usepackage{inconsolata}
\usepackage{enumitem}

% Paquete para diagramas
\usepackage{tikz}
\usetikzlibrary{shapes, arrows.meta, positioning}

% Paquete para fondo
\usepackage{background}

% Configuración de lstlisting
\lstset{
    language=Python,
    basicstyle=\ttfamily\small,
    keywordstyle=\color{blue}\bfseries,
    stringstyle=\color{teal},
    commentstyle=\color{gray}\itshape,
    numbers=left,
    numberstyle=\tiny\color{gray},
    backgroundcolor=\color{black!5},
    frame=single,
    rulecolor=\color{black!50},
    breaklines=true,
    captionpos=b,
    showstringspaces=false
}


\usepackage{tcolorbox}

%comandos
\usepackage{amsmath}
\newcommand{\flechita}{$\rightarrow$}

% Configuración de título
\titleformat{\section}{\normalfont\Large\bfseries}{\thesection}{1em}{}

% Información del documento
\title{
    \vspace{-2cm}
    \includegraphics[width=0.3\textwidth]{images/fccee.jpg} \\ % Cambia el logo si es necesario
    \LARGE Ingeniería Informática + ADE\\
    \large Universidad de Granada (UGR)\\[1cm]
}
\author{\textbf{Autor:} Ismael Sallami Moreno}
\date{\textbf{Asignatura:} Resúmenes de Contabilidad Financiera I Tema 3: Deudores y Acreedores de la actividad habitual y otros débitos y créditos}

% Configuración del fondo
\backgroundsetup{
    scale=1,
    color=black,
    opacity=0.2,
    angle=0,
    position=current page.south,
    vshift=0pt,
    hshift=0pt,
    contents={\includegraphics[width=\paperwidth,height=\paperheight,keepaspectratio]{images/granada.jpg}}
}

% Configuración del encabezado y pie de página
\usepackage{times}
\pagestyle{fancy}
\fancyhf{}
\fancyhead[L]{\textbf{\textsf{\leftmark}}}
\fancyhead[R]{\textbf{\textsf{\thepage}}}
\fancyfoot[C]{\thepage}

% Inicio del documento
\begin{document}

% Portada
\maketitle
\thispagestyle{empty}

\begin{center}
    \includegraphics[width=\textwidth,height=0.4\textheight,keepaspectratio]{images/granada.jpg} \\ % Añade tu imagen de fondo
    \vfill
\end{center}

\newpage

% Índice (opcional)
\tableofcontents
\newpage

\section{Valoración y Registro de los créditos y partidas a cobrar}

Dentro de los activos financieros pueden establecerse los siguientes elementos: 
\begin{itemize}
    \item Efectivo y otros líquidos equivalentes.
    \item Créditos por operaciones comerciales: clientes y deudores varios.
    \item Créditos a terceros: préstamos y otros créditos.
    \item Valores representativos de la deuda de otras empresas, tales como obligaciones, bonos y pagarés.
    \item Instrumento de patrimonio de otras empresas adquiridos: acciones, participaciones y otros instrumentos de patrimonio.
\end{itemize}

\subsection{Tipología de créditos}

\subsubsection{Clasificación de los diferentes elementos incluidos en esta cartera atendiendo al origen de los créditos:}

\begin{itemize}
    \item Clientes
    \item Deudores
    \item Otras cuentas a cobrar
\end{itemize}

\subsubsection{En función del plazo de cobro o vencimiento}

\begin{itemize}
    \item Créditos a corto plazo
    \item Créditos a largo plazo
\end{itemize}

\subsubsection{En función del soporte documental de los créditos}

\begin{itemize}
    \item Créditos materializados en facturas o documentos análogos
    \item Créditos materializados en títulos de crédito, que en el PGC se denominan efectos comerciales a cobrar.
\end{itemize}

\subsubsection{En función de la vinculación de la empresa con las entidades deudoras}

\begin{itemize}
    \item Cuentas a cobrar de empresas del grupo
    \item Otras cuentas a cobrar
\end{itemize}

\subsection{Valoración de créditos comerciales y no comerciales}

Pueden clasificarse según:
\begin{itemize}
    \item Activos financieros a coste amortizado.
    \item Activos financieros a valor razonable con cambios en la cuenta de pérdidas y ganancias.
    \item Activos financieros a valor razonable con cambios en el patrimonio neto.
    \item Activos financieros a coste.
\end{itemize}


\subsubsection{Valoración inicial}

Inicialmente se valoran por su valor razonable, que salvo evidencia en contrario, será el precio de la transacción, que equivaldrá al valor razonable de la contraprestación entregada más los costes directamente atribuibles a la transacción.\\\\
Se considera que los créditos que cumplan con las siguientes características:
\begin{enumerate}
    \item vencimiento no superior a un año
    \item no tengan tipo de interés contractual explícito
    \item cuando el efecto de no actualizar los flujos de efectivo sea no significativo, se podrá valorar a valor nominal.
\end{enumerate}

\subsubsection{Valoración posterior}

Los créditos comerciales y no comerciales se valorarán por su coste amortizado,. Aplicando este criterio tenemos que los intereses devengados se contabilizarán en la cuenta de pérdidas y ganancias, aplicando el método del \textit{tipo de interés efectivo}.\\\\
Se define el coste amortizado como el importe por el cual fue inicialmente valorao un activo financiero o un pasivo financiero, menos los reembolsos de principal que se hubieran producido, más o menos, según proceda la parte imputada en las cuentas de pérdidas y ganacias,mediante la utilización del método del tipo de interés efectivo, de la diferencia entre el importe inicial y el importe de vencimiento.\\\\
Se entiende por coste efectivo:
\begin{enumerate}
    \item Valoración inicial del instrumento
    \item Menos los reembolsos de principal
    \item más o menos la imputación en resultados de la diferencia entre el valor inicial y el valor del reembolso al vencimiento , según el método del tipo de interés efectivo.
    \item Menos las reducciones del deterioro.
\end{enumerate}


\subsection{Operaciones de baja y cesión de activos financieros}

Las operaciones relaciondas con la compraventa de productos y prestación de servicios se recoge en los ''documentos comerciales'' .\\

En cuanto a los efectos que existen en contabilidad financiera vamos a ir detallando brevemente cada uno y exponiendo la cuenta que es en el cuadro de cuentas del PGC.\\
\begin{itemize}
    \item \textbf{4310 Efectos comerciales en cartera:} Los que la empresa tiene en su poder y que ha recibido de sus clientes como contrapartida de la venta de bienes o prestación de servicios.
    \item \textbf{4311 Efectos comerciales descontados:} Los efectos comerciales que la empresa ha descontado en una entidad financiera, generando así una deuda con el banco = entidad financiera, esto se suele hacer frente a la necesidad de dinero/liquidez por parte de la empresa.
    \item \textbf{4213 Efectos comerciales en gestión de cobro:} Los efectos comerciales que la empresa ha entregado a una entidad financiera para que esta se encargue de su cobro, en otras palabras, la empresa cede su nombre a la entidad financiera para que esta se encargue de cobrar los efectos comerciales, no se transmite el riesgo de impago.
    \item \textbf{4315 Clientes efectos comerciales impagados: } Los efectos comerciales que la empresa ha entregado a sus clientes y que estos no han pagado en la fecha de vencimiento.
\end{itemize}

A continuación vamos a verlos más detalladamente.\\

\textbf{Definición de efecto comercial:} medio de pago aplazado documentado en una letra de cambio.\\

\textbf{Descuento de efectos comerciales:} el descuento o negociación de efectos es una fórmula de financiación que permite cobrar antes de que la letra o pagaré venza, gracias a que la entidad bancaria anticipa el dinero.\\

\textbf{Gestión de cobro de efectos comerciales:} la empresa cede a la entidad financiera la gestión de cobro de los efectos comerciales, pero no el riesgo de impago y no anticipa la cantidad monetaria a la empresa.\\

\textbf{Implicados en estas operaciones:}
\begin{itemize}
    \item \textbf{Librador o cedente:} empresa que emite el efecto.
    \item \textbf{Librado o deudor:} empresa que debe pagar el efecto.
    \item \textbf{Cesionario o factor:} entidad financiera que realiza alguna actividad intermediaria, ya sea adelantar el dinero a la empresa o gestionar el cobro de los efectos.
\end{itemize}

La empresa dará de baja a los activos financieros cuando se haya transmitido la propiedad del mismo y ya no genere beneficios para la propia empresa.\\

\textbf{Riesgos:}
\begin{itemize}
    \item \textbf{Riesgo de crédito:} riesgo de que el deudor no pague debido a una situación de insolvencia o de quiebra.
    \item \textbf{Riesgo de pago atrasado o riesgo de mora:} a que el deudor se retrase en el pago de sus facturas.
    \item \textbf{Riesgo de tipo de cambio:} diferencias de cambio en cuanto a las operaciones con moneda extranjera.
    \item \textbf{Riesgo del tipo de interés:} riesgo de que el valor actual de la cuenta a cobrar varíe debido a cambios en los tipos de interés.
\end{itemize}

\subsubsection*{Anotaciones sobre la práctica}

Debemos de tener en cuenta que en cuanto a los intereses que se generen debemos de contabilizarlos, pero en el caso de que se valore la deuda a valor \textit{razonable}, y debemos de reconocerlos al final de ejercicio aunque no se haya acabado su período de cobro. Para el cobro de los intereses debemos de dar de alta la cuenta 430 Efectos comerciales en cartera y abonar la cuenta 762 Ingresos de créditos.\\

Cuando descontemos algún efecto como contrapartidad usamos la cuenta \textit{5208 Deudas por efectos descontados}.\\

\begin{tcolorbox}[colback=blue!5!white,colframe=blue!75!black,fonttitle=\bfseries,title=Tipos de valoración inicial]
    \begin{itemize}
        \item \textbf{Valor razonable: }Contabilizando los intereses devengados con el paso del tiempo.
        \item \textbf{Valor nominal: }Los interes ya están incluidos en el valor de la deuda, para ello debemos de tener en cuenta estas características para poder valorar a valor nominal:
        \begin{enumerate}
            \item vencimiento no superior a un año
            \item no tengan tipo de interés contractual explícito
            \item cuando el efecto de no actualizar los flujos de efectivo sea no significativo, se podrá valorar a valor nominal. (actualización poco significativa).
        \end{enumerate} 
    \end{itemize}
    
\end{tcolorbox}

\begin{tcolorbox}[colback=blue!5!white,colframe=blue!75!black,fonttitle=\bfseries,title=Valoración posterior]
    \begin{itemize}
        \item Implicación de los intereses.
        \item Test de Deterioro a final de ejercicio. 
    \end{itemize}
    
\end{tcolorbox}

\subsection{Factoring}

El factoring es una operación de cesión de créditos a una entidad financiera, que se encarga de la gestión de cobro de los mismos a entidades especializadas en esto.\\

\begin{itemize}
    \item Factoring con recurso: la empresa cedente retiene los beneficios y asume el riesgo de impago.
    \item Factoring sin recurso: la empresa cedente no asume el riesgo de impago y la entidad financiera se encarga de la gestión de cobro. Puede entenderse que el riesgo de impago lo asume la entidad financiera, ya que se ha transmitido la propiedad de los créditos.
\end{itemize}

\subsubsection*{Anotaciones sobre la práctica en cuanto al Factoring}

Para la generación de intereses debemos de usar la cuenta \textit{665 Intereses por descuento de efectos y operaciones de factoring} y como contrapartida usamos la cuenta \textit{5209 Deudas por operaciones de factoring}. En cuanto a la generaciónd e intereses debemos de tener en cuenta que el montente se va acumulando al ser capitalización compuesta.

En el caso de que se cedan los derechos de cobro y riesgo, solamente debemos de realizar el siguiente cuadro:

\begin{center}
    \begin{table}[H]
        \begin{tabular}{|l|p{5cm}|l|}
        \hline
        \textbf{DEBE} & \textbf{Registro contable de la operación de factoring sin recurso} & \textbf{HABER} \\ \hline
        193.252,53 & (572) Banco X c/C & \\ \hline
        5.747,47 & (665) Intereses por descuento de efectos y operaciones de factoring & \\ \hline
        1.000 & (626) Servicios bancarios y similares & \\ \hline
        & & (430) Clientes 200.000 \\ \hline
        \end{tabular}
        \caption{Registro contable de la operación de factoring sin recurso suponiendo valores arbitratios}
    \end{table}
\end{center}
    


\begin{tcolorbox}[colback=blue!5!white,colframe=green!75!black,fonttitle=\bfseries,title=Ejercicios Recomendados]
    Interesante hacer los ejercicios 6 y 7 del libro página 134.

    
\end{tcolorbox}

\subsection{Confirming}

El confirming es una operación de cesión de créditos a una entidad financiera, que se encarga de la gestión de pago de los mismos a los proveedores de la empresa cedente.\\


\subsubsection*{Anotaciones sobre la práctica}

Aparte de usar las cuentas que son típicas, debemos de tener en cuenta la casuística de cuandos se renegocian unos efectos y se aceptan. En este caso debemos de cobrarles una comisión extra usando la cuenta \textit{759 Ingresos por Servicios Diversos}.


\section{Deterioro de valor de los créditos y partidas a cobrar}

Como venimos diciendo, al cierre del ejercicio debemos de realizar las oportunas correciones valorativas para que existan las evidencias objetivas de que se trata de un activo financiero. Para ello debemos de calcular la diferencia entre el valor en libros\footnote{Es lo mismo que el valor contable} de los créditos comerciales menos el Valor Actual de los Flujos de Efectivo. SI los cŕeditos están valorados a valor razonable se considerará el tipo de interés efectivo calculado en el momento de su reconocimiento inicial.\\

\subsection{Métodos de estimación }

\subsubsection{Método individual}

Supone la determinación de forma separada de cada uno de los clientes frente a deudas o problemas de cobro. Esto implica el reconocimiento como créditos de \textit{dudoso cobro} en el momento en el que se detecte el riesgo de impago, con independencia de que se haya producido o no el vencimiento de la deuda. Además, debemos de contabilizar la correspondiente pérdida según el principio de prudencia.\\

\subsubsection{Método global}

La empresa determinará el importe del deterioro al final del ejercicio a través de la estimación global del riesgo de insolvencia correspondiente a los saldos de los clientes y de los deudores.\\

\subsection{Topología de débitos y partidas a pagar}

Podemos distinguir diversas tipologías:

\begin{itemize}
    \item Proveedores: deudas contraídas en el conjunto de suministradores por el suministro de bienes y servicios, esto se reconoce como \textit{efectos comerciales a pagar}.
    \item Acreedores: podemos tener dos posibles excepciones:
    \begin{itemize}
        \item Deudas que son originadas por otros gastos dentro de las actividades de Explotación derivadas del factor trabajo, se reconoce como \textit{otras cuentas a pagar}.
        \item Deudas contraídas con los suministradores de los que integran el inmovilizado de la empresa, se reconocen como \textit{proveedores de inmovilizado}. Suelen establecerse con un plazo superior a un año.
    \end{itemize}
\end{itemize}

\subsection{Valoración de débitos comerciales y no comerciales}

Los pasivos financieros se clasificarán en algunas de las siguientes categorías:
\begin{itemize}
    \item Pasivos financieros a coste amortizado.
    \item Pasivos financieros a valor razonable con cambios en la cuenta de pérdidas y ganancias.
\end{itemize}


Debemos de centranros en la primera para ver la inclusión de los débitos por operaciones comerciales y débitos por operaciones no comerciales:
\begin{itemize}
    \item Débitos por operaciones comerciales: son aquellos que resultan de la compra de bienes y servicio con pagos aplazados.
    \item Débitos por operaciones no comerciales: no tiene ningún origen comercial, sino que proceden de operaciones de préstamo o créditos recibidos por la empresa.
\end{itemize}

En cuanto a la valoración, se valorarán a valor razonable, salvo evidencia en contrario, será al precio de la transacción\footnote{Precio razonable más los costes atribuibles}. Cabe destacar que los costes asociados a la operación disminuyen la financiación recibida al calcular el tipo de interés efectivo. De nuevo en esta parte entra en juego lo de poder valorar a valor nominal\footnote{Explicido en las secciones anteriormente mencionadas}. Además, \textit{los intereses devengados se contabilizarán en la cuenta de pérdidas y ganancias, aplicando el método del tipo de interés efectivo}.

\subsubsection{Confirming}

Aunque se haya hablado anteriormente de esto, vamos a mencionar algunos detalles a tener en cuenta y explicarlo de manera más detallada.\\

Sabemos que se trata de un servicio en el que determminadas entidades financieras prestán a empresas para la tramitación de pago de facturas a proveedores. Básicamente, la empresa cedente se encarga de la gestión de pago de los proveedores y le adelanta el dinero a la empresa teniendo en cuenta los gastos de la operación.\\

Tipologías:
\begin{itemize}
    \item Confirming estándar: gestión de forma general, hasta vencimiento.
    \item Confirming por pronto pago financiado o confirming por descuento: la entidad financiera y el proveedor negocian un descuento por pronto pago.
    \item Confirming por financiación: el cliente paga al proveedor la deuda y la entidad financiera le financia el importe, teniendo la empresa la deuda a plazo posterior con la entidad financiera.
\end{itemize}

Otras clasicaciones pueden ser como la de en función de \textit{de quién asume el riesgo de impago}:
\begin{itemize}
    \item Confirming con recurso: la empresa cedente asume el riesgo de impago.
    \item Confirming sin recurso: la entidad financiera asume el riesgo de impago.
\end{itemize}

\begin{tcolorbox}[colback=blue!5!white,colframe=green!75!black,fonttitle=\bfseries,title=Ejercicios Recomendados]
    Interesante hacer el ejercicio 15 de la página 163 del libro.
\end{tcolorbox}

\section{Débitos y créditos derivados de la contabilización del impuesto de valor añadido (IVA)}

Tipos:
\begin{itemize}
    \item General: 21 \%.
    \item Reducido: 10 \%.
    \item Superreducido: 4 \%.
\end{itemize}

\subsection{Otros hechos imponibles del IVA}

En el caso de que se produzcan pagos o cobros adelantados, se ha establecido que se deberá de devengar el IVA soportado para el comprador y el IVA repercutido para el vendedor.\\

\subsection{El proceso de liquidación del IVA}

Surge debido a la relación del empresario con la Agencia Tributaria, se hará en distintos periodos de tiempo, dependiendo del tamaño de la empresa. Se deben de compensar ambas cuentas de IVA:
\begin{itemize}
    \item Si el IVA soportado deducible de la empresa es menor que el IVA repercutido, surge la cuenta  \textit{4750 Hacienda Pública, Acreedora por IVA}.
    \item Si el IVA soportado deducible de la empresa es mayor que el IVA repercutido, surge la cuenta \textit{4700 Hacienda Pública, Deudora por IVA}.
\end{itemize}

\begin{table}[H]
    \begin{tabular}{|c|l|c|}
    \hline
    \textbf{DEBE} & \textbf{Liquidación trimestral del IVA} & \textbf{HABER} \\ \hline
    X & (477) Hacienda Pública, IVA Repercutido (\textbf{Cuandos se liquida se carga}) &  \\ \hline
     & (472) Hacienda Pública, IVA Soportado (\textbf{Cuandos se liquida se abona})& Y \\ \hline
     & (4750) Hacienda Pública, Acreedor por IVA Si X > Y &  \\ \hline
    \end{tabular}
\end{table}

\begin{table}[H]
    \begin{tabular}{|c|l|c|}
    \hline
    \textbf{DEBE} & \textbf{Liquidación trimestral del IVA} & \textbf{HABER} \\ \hline
    X & (477) Hacienda Pública, IVA Repercutido (\textbf{Cuandos se liquida se carga}) &  \\ \hline
     & (472) Hacienda Pública, IVA Soportado (\textbf{Cuandos se liquida se abona})& Y \\ \hline
    & (4700) Hacienda Pública, Deudor por IVA Si X < Y &  \\ \hline
    \end{tabular}
\end{table}
    
    

Una vez que hayamos determinado el cálculo de la liquidación, el siguiente paso es proceder la liquidación de la Agencia Tributaria.

\begin{table}[H]
    \begin{tabular}{|c|c|c|}
    \hline
    \textbf{DEBE} & \textbf{Liquidación trimestral del IVA} & \textbf{HABER} \\ \hline
    X & (4750) Hacienda Pública, Acreedor por IVA & \\ \hline
     &  (572) Banco X c/c & X \\ \hline
    \end{tabular}
\end{table}
    


\begin{tcolorbox}[colback=blue!5!white,colframe=blue!75!black,fonttitle=\bfseries,title=Regla de la prorrata]
    Por cronología de la Asignatura, tocaría la regla de la prorrata, pero se afirmo en clase que esta no caía.
\end{tcolorbox}

\section{Información a cumplimentar en las cuentas anuales}

Para ello accede a las páginas 189--197 del Manual.


\newpage
\section{Cuestionario tipo test}
Para realizar el tipo test del libro pinche \href{https://elblogdeismael.github.io/Asignaturas/Tercer%20A%C3%B1o/CF1/Tests/testT3Libro.html}{aquí.}






\end{document}
