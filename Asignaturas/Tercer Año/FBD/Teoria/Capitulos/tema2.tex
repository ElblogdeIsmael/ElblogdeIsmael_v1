% \section{Lenguajes de una BD}

% Recomendación ANSI/SPARC. Se propone un lenguaje de base de datos. En este se debe de definir, controlar y manipular los datos. Se denomina sub-lenguaje de la base de datos al que esta implementado por el propio SGBD. Tenemos distintas partes:
% \begin{itemize}
%     \item DDL: definición de estructura de datos.
%     \item DML: modificación, borrado y consulta de datos, además, se permite consultar los esquemas definidos de la Base de Datos.
%     \item DCL: gestiona los requisitos de acceso a los datos y otro tipo de tareas de administración(creación de usuarios,...).
% \end{itemize}

% Este grupo propone que haya cada uno de los anteriores en cada uno de los niveles de la arquitectura. Esto no tiene mucho sentido. Esto se debe a que no tiene sentido tener alguno de los anteriores en determinados niveles. Si meto el DDL en el nivel interno, estoy dependiendo del nivel interno de la máquina.

% En la realidad ha surgido la idea de que todo sea un estándar, pero cada fabricant lleva sus propias estrategias. Hay diferentes versiones de SQL y los SGBD han ido proporcionando soporte.

% Acto seguido, nos centramos en el desarrollo de aplicaciones. Son de propósito general, ya que usan lenguajes como C++, Java,... Se usan diversas herramientas como es Oracle APEX, Sysbase PowerBuilder,... Con esto se proporciona un procesamiento avanzado de datos y una gestión eficaz de la interfaz del usuario.

% Debemos de establecer un mecanismo que nos permite establecer una comunicación entre el lenguaje anfitrión y el de datos. Esto se conoce como \textit{acoplamiento}. Distinguimos dos categorías.
% \begin{itemize}
%     \item Débilmente acoplados. Lenguajes de propósito general y en este caso el programador puede distinguir entre sentencias del lenguaje anfitrión y las dispuestas por la propia BD.
%     \item Fuertemente acoplados. Lenguajes y herramientas de propósito específico. Se parte del DSL como elemento central y se le van incorporando características para facilitar el desarrollo de apliaciones.
% \end{itemize}

% Hay varias alternativas para implementar el acoplamiento débil:

% \begin{itemize}
%     \item \textbf{Usar APIs de acceso a la BD}. Acceder a la BD desde el código fuente del lenguaje anfitrión.
%     \item \textbf{DSL inmerso en el código fuente del lenguaje anfitrión.} Se escribe código híbrido.
% \end{itemize}

% Pasa lo mismo con el acoplamiento fuerte:

% Las propuestas son ya propietarias, o bien la ejecución de Java sobre una máquina virtual que esta en el propio SGBD.

% También han aparecido numerosos entornos de desarrollo que son específicos para las aplicaciones de gestión.


\section{Relación de Ejercicios T2}

\begin{enumerate}
    \item Explicar la relación existente entre los niveles de una base de datos y el concepto de independencia.
    \item Explicar la diferencia entre esquema externo y aplicaciones de usuario.
    \item Explica el motivo por el que, a tu juicio, no se han desarrollado DDLs a nivel interno.
    \item Explica el motivo por el que, a tu juicio, no se han desarrollado DMLs a nivel externo.
    \item Buscar tres ejemplos de lenguajes de cuarta generación. Indicar sus objetivos o funciones.
    \item ¿Cuál es el enfoque actual del concepto de lenguaje anfitrión? Dar ejemplos de lenguajes anfitrión.
    \item ¿Qué elementos conciernen al nivel interno de una base de datos?
    \item ¿Qué cuestiones debe cubrir a tu juicio una buena herramienta de gestión privilegios de usuarios?
    \item Explicar las ventajas de la arquitectura cliente/servidor a tres niveles.
\end{enumerate}