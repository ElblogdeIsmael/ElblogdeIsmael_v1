\section{Ejercicios}

\subsection*{9. Resolver utilizando el método de las dos fases:}

\begin{itemize}
    \item[a)] Min. $20x_1 + 25x_2$
    \begin{align*}
        \text{s.a.} \quad & 2x_1 + 3x_2 \geq 18 \\
        & x_1 + 3x_2 \geq 12 \\
        & 4x_1 + 3x_2 \geq 24 \\
        & x_1, x_2 \geq 0
    \end{align*}
    
    \item[b)] Max. $4x_1 + 3x_2$
    \begin{align*}
        \text{s.a.} \quad & 3x_1 + 4x_2 \leq 12 \\
        & x_1 + x_2 \geq 4 \\
        & 4x_1 + 2x_2 \leq 8 \\
        & x_1, x_2 \geq 0
    \end{align*}
    
    \item[c)] Max. $x_1 - 2x_2 + 3x_3$
    \begin{align*}
        \text{s.a.} \quad & x_1 + x_2 + x_3 = 6 \\
        & x_3 \leq 2 \\
        & x_1, x_2, x_3 \geq 0
    \end{align*}

    Pasamos a forma estándar:

    \begin{align*}
        Max. \quad & x_1 - 2x_2 + 3x_3 + 0s_1 -Mt_1\\
        \text{s.a.} \quad & x_1 + x_2 + x_3 + t_1= 6 \\
        & x_3 + s_1 = 2 \\
        & x_1, x_2, x_3, s_1, t_1 \geq 0
    \end{align*}

    \subsubsection*{Fase 1}

    Paso 1. La función artificial es: 

    \begin{equation*}
        z^0=0x_1 + 0x_2 + 0x_3 + 0s_1 -t_1
    \end{equation*}

    Paso 2. Aplicar el método simplex al programa construido:

    \begin{table}[H]
    \centering
    \begin{tabular}{|c|c|c|c|c|c|c|c|}
    \hline
    &  & 0 & 0 & 0 & 0 & -1 &\\
    \hline
    & VB & $x_1$ & $x_2$ & $x_3$ & $s_1$ & $t_1$ & XB \\
    \hline
    -1 & $t_1$ & 1 & 1 & 1 & 0 & 1 & 6\\
    \hline
    0 & $s_1$ & 0 & 0 & 1 & 1& 0 &2\\
    \hline
    & $z_j - c_j$ & -1 & -1 & -1 & 0 & 0 &-6\\
    \hline
    \end{tabular}
    \end{table}

    Ahora debemos de coger la más negativa, pero al ser todos con valor -1, da igual cual cojamos, cogemos la primera, es decir, la de $x_1$.

    % \begin{tcolorbox}[colback=green!5!white,colframe=green!75!black]
        \begin{align*}
            F_p = F_1\\
            F2N = F2 (\text{Ya tiene un 0})
        \end{align*}
        
    % \end{tcolorbox}

    \begin{table}[H]
        \centering
        \begin{tabular}{|c|c|c|c|c|c|c|c|}
        \hline
        &  & 0 & 0 & 0 & 0 & -1 &\\
        \hline
        & VB & $x_1$ & $x_2$ & $x_3$ & $s_1$ & $t_1$ & XB \\
        \hline
        0 & $x_1$ & 1 & 1 & 1 & 0 & 1 & 6\\
        \hline
        0 & $s_1$ & 0 & 0 & 1 & 1& 0 &2\\
        \hline
        & $z_j - c_j$ & 0 & 0& 0 & 0 & 1 &0\\
        \hline
        \end{tabular}
    \end{table}

    En este punto no podemos continuar ya que todos los valores de la fila $z_j - c_j$ son positivos, por lo que debemos de pasar a la fase 2.

    \subsubsection*{Fase 2}

    Ahora debemos de eliminar las variables artificiales y continuar con el problema original.

    \begin{table}[H]
        \centering
        \begin{tabular}{|c|c|c|c|c|c|c|}
        \hline
        &  & 1 & -2 & 3 & 0 &\\
        \hline
        & VB & $x_1$ & $x_2$ & $x_3$ & $s_1$ & XB \\
        \hline
        1 & $x_1$ & 1 & 1 & 1 & 0 & 6\\
        \hline
        0 & $s_1$ & 0 & 0 & 1 & 1& 2\\
        \hline
        & $z_j - c_j$ & 0 & 3&-2 & 0 & 6\\
        \hline
        \end{tabular}
    \end{table}

    Cogemos la fila más negativa, en este caso la de $x_3$, con valor -2.

    \begin{align*}
        F_p = F_2\\
        F1N = F1 - F_p
    \end{align*}

    \begin{table}[H]
        \centering
        \begin{tabular}{|c|c|c|c|c|c|c|}
        \hline
        &  & 1 & -2 & 3 & 0 &\\
        \hline
        & VB & $x_1$ & $x_2$ & $x_3$ & $s_1$ & XB \\
        \hline
        1 & $x_1$ & 1 & 1 & 0 & -1 & 4\\
        \hline
        3 & $x_3$ & 0 & 0 & 1 & 1& 2\\
        \hline
        & $z_j - c_j$ & 0 & 3&0 & 2 & 10\\
        \hline
        \end{tabular}
    \end{table}

    Como todos los valores de la fila $z_j - c_j$ son positivos, hemos llegado a la solución óptima.


    
    \item[d)] Max. $x_1 + x_2 + 10x_3$
    \begin{align*}
        \text{s.a.} \quad & x_2 + 4x_3 = 2 \\
        & -2x_1 + x_2 - 6x_3 = 2 \\
        & x_1, x_2, x_3 \geq 0
    \end{align*}

    Pasamos a forma estándar:

    \begin{align*}
        Max. \quad & x_1 + x_2 + 10x_3 - t_1 -t_2\\
        \text{s.a.} \quad & x_2 + 4x_3 + t_1= 2 \\
        & -2x_1 + x_2 - 6x_3 + t_2 = 2 \\
        & x_1, x_2, x_3, s_1, t_1 \geq 0
    \end{align*}

    \subsubsection*{Fase 1}

    Paso 1. La función artificial es: 
    \begin{equation*}
        z^0=0x_1 + 0x_2 + 0x_3  -t_1 -t_2
    \end{equation*}

    Paso 2. Aplicar el método simplex al programa construido:

    \begin{table}[H]
        \centering
        \begin{tabular}{|c|c|c|c|c|c|c|c|}
        \hline
        &  & 0 & 0 & 0 & -1 & -1 &\\
        \hline
        & VB & $x_1$ & $x_2$ & $x_3$ & $t_1$ & $t_2$ & XB \\
        \hline
        -1 & $t_1$ & 0 & 1 & 4 & 1 & 0 & 2\\
        \hline
        -1 & $t_2$ & -2 & 1 & -6 & 0& 1 &2\\
        \hline
        & $z_j - c_j$ & 2 & -2& 2 & 0 & 0 &0\\
        \hline
        \end{tabular}
    \end{table}

    Cogemos la columna más negativa, en este caso la de $x_2$.

    \begin{align*}
        F_p = F_1\\
        F2N = F_p - F_2
    \end{align*}

    \begin{table}[H]
        \centering
        \begin{tabular}{|c|c|c|c|c|c|c|c|}
        \hline
        &  & 0 & 0 & 0 & -1 & -1 &\\
        \hline
        & VB & $x_1$ & $x_2$ & $x_3$ & $t_1$ & $t_2$ & XB \\
        \hline
        0 & $x_2$ & 0 & 1 & 4 & 1 & 0 & 2\\
        \hline
        -1 & $t_2$ & 2 & 0 & 10 & 1& -1 &0\\
        \hline
        & $z_j - c_j$ & -2 & 0& -10 & 0 & 2 &0\\
        \hline
        \end{tabular}
    \end{table}

    Cogemos la columna más negativa, en este caso la de $x_3$.

    \begin{align*}
        F_p = F_2\\
        F2N = F2 / 10\\
        F1N = F_1 - 4F_p
    \end{align*}

    \begin{table}[H]
        \centering
        \begin{tabular}{|c|c|c|c|c|c|c|c|}
        \hline
        &  & 0 & 0 & 0 & -1 & -1 &\\
        \hline
        & VB & $x_1$ & $x_2$ & $x_3$ & $t_1$ & $t_2$ & XB \\
        \hline
        0 & $x_2$ & 0.8 & 1 & 0 & 0.6 & 0.4 & 2\\
        \hline
        0 & $x_3$ & 0.2 & 0 & 1 & 0.1 & -0.1 & 0\\       \hline
        & $z_j - c_j$ & 0 & 0& 0 & 1 & 1 &0\\
        \hline
        \end{tabular}
    \end{table}

    En este punto no podemos continuar ya que todos los valores de la fila $z_j - c_j$ son positivos, por lo que debemos de pasar a la fase 2.

    \subsubsection*{Fase 2}

    Ahora debemos de eliminar las variables artificiales y continuar con el problema original.
    \begin{table}[H]
        \centering
        \begin{tabular}{|c|c|c|c|c|c|}
        \hline
        &  & 1 & 1 & 10 &\\
        \hline
        & VB & $x_1$ & $x_2$ & $x_3$ & XB \\
        \hline
        1 & $x_2$ & 0.8 & 1 & 0 & 2\\
        \hline
        10 & $x_3$ & 0.2 & 0 & 1 & 0\\
        \hline
        & $z_j - c_j$ & 1.8 & 0 & 0 &2\\
        \hline
        \end{tabular}
    \end{table}

    Como todos los valores de la fila $z_j - c_j$ son positivos, hemos llegado a la solución óptima.

    
    \item[e)] Max. $x_1 + 2x_2$
    \begin{align*}
        \text{s.a.} \quad & x_1 + x_2 = 4 \\
        & 2x_1 - 3x_2 = 3 \\
        & 3x_1 - x_2 = 8 \\
        & x_1, x_2 \geq 0
    \end{align*}
\end{itemize}
