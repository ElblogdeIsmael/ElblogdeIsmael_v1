\documentclass[11pt, a4paper]{article}
\usepackage[spanish]{babel}
\usepackage[utf8]{inputenc}
\usepackage[T1]{fontenc}
\usepackage{amsmath, amssymb, amsthm}
\usepackage{graphicx}
\usepackage{geometry}
\usepackage{fancyhdr}
\usepackage{hyperref}
\usepackage{xcolor}      % Para definir colores
\usepackage{tcolorbox}   % Para crear recuadros

% Configuración de página
\geometry{a4paper, left=3cm, right=3cm, top=2.5cm, bottom=2.5cm}
\pagestyle{fancy}

% Definición de colores
\definecolor{titleblue}{RGB}{0,76,151}

% Configuración de encabezado y pie de página
\fancyhead[L]{\small\textbf{\color{titleblue}Fórmulas Financieras}}
%\fancyhead[R]{\small\textbf{\color{titleblue}\autor}}
\fancyhead[R]{\small\textbf{\color{titleblue}Ismael Sallami Moreno}}
\fancyfoot[C]{\thepage}
\renewcommand{\headrulewidth}{0.4pt}
\renewcommand{\footrulewidth}{0.4pt}

% Información del documento
\title{\color{titleblue}\textbf{Formulario de Análisis de Operaciones Financieras}}
\author{Ismael Sallami Moreno}
\date{\today}

% Configuración hyperref
\hypersetup{
    colorlinks=true,
    linkcolor=titleblue,
    urlcolor=titleblue,
    pdftitle={Formulario Matemático},
    pdfauthor={Nombre del Autor},
}

\begin{document}

\maketitle
\thispagestyle{fancy}

\begin{center}
    \textbf{Curso:} Análisis de Operaciones Financieras \\
    \textbf{Institución:} Universidad de Granada \\
    \vspace{1cm}
    \includegraphics[width=0.3\textwidth]{images/UGR-Simbolo.png}
\end{center}

\tableofcontents
\newpage

\section{Tema 2}
\subsection{Precio Total}
\begin{tcolorbox}[colframe=blue!50, colback=blue!5] % Configura colores y estilo del recuadro
\begin{equation}
    \text{Precio}_{\text{Total}} = \text{Precio}_t^{\text{excupón}} + \text{CC}_t \text{(Cupón Corrido)}
\end{equation}
\end{tcolorbox}
\begin{itemize}
    \item Precio de cotización es lo mismo que el precio de excupón.
\end{itemize}

\subsection{Cupón Corrido}
\begin{tcolorbox}[colframe=blue!50, colback=blue!5, sharp corners] % Configura colores y estilo del recuadro
\begin{equation}
    CC_t =\frac{\text{cupón anual} = iV_n}{\text{nº de días por el que se paga el cupón}} \times \text{d}
\end{equation}
\end{tcolorbox}

\begin{itemize}
    \item Con d = nº de días pasados desde el último cupón hasta el tiempo t.
\end{itemize}

% \subsection{TIR}
% \begin{tcolorbox}[colframe=blue!50, colback=blue!5, sharp corners] % Configura colores y estilo del recuadro
% \begin{equation}
%     P_{\text{Total}} = \left(\text{Cupón anual} \times a_{n)TIR}+\frac{V_n}{1+TIR^n} \right) \times (1+ TIR^{\frac{d}{\text{dt}}})
% \end{equation}
% \end{tcolorbox}

\subsection{TIR}
\begin{tcolorbox}[colframe=blue!50, colback=blue!5, sharp corners] % Configura colores y estilo del recuadro
\begin{equation}
    P_{\text{Total}} = (1) = \frac{iV_n}{(1+R)^{p+1-t}} + \frac{iV_n}{(1+R)^{p+2-t}} + \cdots + \frac{iV_n+V_n}{(1+R)^{n-t}}
\end{equation}
\end{tcolorbox}

\begin{itemize}
    \item Con p = 
\end{itemize}

% \begin{itemize}
%     \item Con dt = número de días totales.
%     \item Con d = nº de días pasados desde el último cupón hasta el tiempo t.
%     \item $n$ es el número de días totales o de períodos.
% \end{itemize}

\subsection{Cambiar el TIR}
\begin{tcolorbox}[colframe=blue!50, colback=blue!5, sharp corners] % Configura colores y estilo del recuadro
\begin{equation}
    (1+i_s)^2=1+i_a
\end{equation}
\end{tcolorbox}

\subsection{Cáculo de la \textit{a}}
\begin{tcolorbox}[colframe=blue!50, colback=blue!5, sharp corners] % Configura colores y estilo del recuadro
\begin{equation}
    a_{ni}=\frac{1-(1+i)^{-n}}{i}
\end{equation}
\end{tcolorbox}

\begin{itemize}
    \item Donde $a_{ni}$ es el valor actual de la renta.
    \item $n$ es el número de períodos.
    \item $i$ es la tasa de interés por periodo.
\end{itemize}

\subsection{Valor de la obligación convertible}

\begin{tcolorbox}[colframe=blue!50, colback=blue!5, sharp corners] % Configura colores y estilo del recuadro
    \begin{equation}
        \text{Valor }_{\text{Obligación Convertible}} = \text{Valor }_{\text{Obligación simple}} + \text{Valor }_{\text{Opción de conversión}}
    \end{equation}
\end{tcolorbox}

\begin{tcolorbox}[colframe=blue!50, colback=blue!5, sharp corners] % Configura colores y estilo del recuadro
    \begin{equation}
        \text{Tasa de la conversión} = \frac{B}{A} = \text{nº de acciones entregadas por cada obligación}
    \end{equation}
\end{tcolorbox}

\begin{tcolorbox}[colframe=blue!50, colback=blue!5, sharp corners] % Configura colores y estilo del recuadro
    \begin{equation}
        \text{Valor de la conversión} = \frac{B}{A} \times A_c
    \end{equation}
\end{tcolorbox}

\begin{itemize}
    \item Con $A_c$ = valor de la cotización de la acción después de la conversión.
    \item Con $B$ = valor de la obligación convertible.
    \item Con $A$ = valor de la acción en el momento de la conversión.
\end{itemize}

\subsection{Obligación con posibilidad de rescate anticipado}

\begin{tcolorbox}[colframe=blue!50, colback=blue!5, sharp corners] % Configura colores y estilo del recuadro
    \begin{equation}
        \text{Valor }_{\text{Obligación rescatable}} = \text{Valor }_{\text{Obligación simple}} - \text{Valor }_{\text{Opción de compra}}
    \end{equation}
\end{tcolorbox}

\subsection{Precio de subasta o Letras del Tesoro}

\subsubsection{Capitalización simple}

\begin{tcolorbox}[colframe=blue!50, colback=blue!5, sharp corners] % Configura colores y estilo del recuadro
    \begin{equation}
        P = \frac{N}{1+\frac{n}{360}i}
    \end{equation}
\end{tcolorbox}

\subsubsection{Capitalización compuesta}

\begin{tcolorbox}[colframe=blue!50, colback=blue!5, sharp corners] % Configura colores y estilo del recuadro
    \begin{equation}
        P = \frac{N}{(1+i)^{\frac{n}{360}}}
    \end{equation}
\end{tcolorbox}

\textit{Alude a los cáculos del precio marginal y precio medio, entre otros.}









\end{document}