\section{Otros Ejercicios}

\subsection*{\textcolor{red}{Ejercicio 1}}

La empresa COPA, S.A. recibe un préstamo por un valor nominal de 50.000 € el día 1 de enero de 2019 con un vencimiento a tres años y un valor de reembolso de 52.000 €, al tipo de interés nominal del 5\%. Los gastos de la operación (que corren a cargo de la empresa COPA) ascienden a 1.000 €. El tipo de interés efectivo de la operación es del 7,006\%. El cuadro de amortización calculado en base al tipo de interés efectivo es el siguiente:

\begin{table}[h!]
\centering
\begin{tabular}{|p{2cm}|p{3cm}|p{2cm}|p{3cm}|p{3cm}|}
\hline
\textbf{Plazo} & \textbf{Intereses Devengados} & \textbf{Pagos} & \textbf{Saldo amortizado} & \textbf{Saldo pendiente de amortizar} \\ \hline
01/01/2019 & - & - & - & 50.000 \\ \hline
31/12/2019 & 3.433 & 2.500 & 933 & 49.933 \\ \hline
31/12/2020 & 3.498 & 2.500 & 998 & 50.932 \\ \hline
31/12/2021 & 3.563 & 52.500 & 50.932 & 0 \\ \hline
\end{tabular}
\caption{CUADRO DE AMORTIZACIÓN A COSTE EFECTIVO 7,006\%}
\end{table}

\textbf{SE PIDE:} contabilizar las siguientes operaciones:
\begin{enumerate}[label=\alph*)]


\item Obtención del préstamo a 01/01/2019.

\begin{table}[H]
\centering
\begin{tabular}{|p{2cm}|p{8cm}|p{2cm}|}
\hline
\rowcolor{blue!30}
\textbf{DEBE} & \textbf{Obtención del préstamo el 01/01/2019} & \textbf{HABER} \\ \hline
49.000 & (572) Bancos c/c & \\ \hline
& (170) Deudas a largo plazo con entidades de crédito & 49.000 \\ \hline
\end{tabular}
\end{table}

\item Contabilización de las operaciones necesarias a 31/12/2019.

\begin{table}[H]
\centering
\begin{tabular}{|p{2cm}|p{8cm}|p{2cm}|}
\hline
\rowcolor{blue!30}
\textbf{DEBE} & \textbf{Contabilización de las operaciones necesarias a 31/12/2019} & \textbf{HABER} \\ \hline
3.433 & (662) Intereses de deudas & \\ \hline
& (170) Deudas a l/p con entidades de crédito & 933 \\ \hline
& (572) Bancos c/c & 2.500 \\ \hline
\end{tabular}
\end{table}

\item Contabilización de la reclasificación de la deuda a 31/12/2020.

No es necesario, pero debemos de tener de asumir que antes de reclasificar la deuda a largo plazo, se han contabilizado los interses devengados y pagados.

\begin{table}[H]
\centering
\begin{tabular}{|p{2cm}|p{8cm}|p{2cm}|}
\hline
\rowcolor{blue!30}
\textbf{DEBE} & \textbf{Devengo de intereses a 31/12/2020} & \textbf{HABER} \\ \hline
3.498 & (662) Intereses de deudas & \\ \hline
& (170) Deudas a l/p con entidades de crédito & 998 \\ \hline
& (572) Bancos c/c & 2.500 \\ \hline
\end{tabular}
\end{table}

Una vez contabilizados los intereses devengados y pagados, se procede a reclasificar la deuda a largo plazo a corto plazo.

\begin{table}[H]
\centering
\begin{tabular}{|p{2cm}|p{8cm}|p{2cm}|}
\hline
\rowcolor{blue!30}
\textbf{DEBE} & \textbf{Contabilización de la reclasificación de la deuda a 31/12/2020} & \textbf{HABER} \\ \hline
50.932 & (170) Deudas largo plazo entidades crédito & \\ \hline
& (520) Deudas a corto plazo entidades crédito & 50.932 \\ \hline
\end{tabular}
\end{table}
\item Contabilización de la cancelación de la deuda EXCLUSIVAMENTE por su valor de reembolso el 31/12/2021.
\begin{table}[H]
\centering
\begin{tabular}{|p{2cm}|p{8cm}|p{2cm}|}
\hline
\rowcolor{blue!30}
\textbf{DEBE} & \textbf{Contabilización de la cancelación de la deuda EXCLUSIVAMENTE por su valor de reembolso el 31/12/2021} & \textbf{HABER} \\ \hline
52.000 & (520) Deudas a corto plazo con entidades de crédito & \\ \hline
& (572) Bancos c/c & 52.000 \\ \hline
\end{tabular}
\end{table}

\end{enumerate}


\newpage
\subsection*{\textcolor{red}{Ejercicio 2}}

La empresa TUNA, S.A. concede el 1 de enero de 2018 un préstamo de 10.000 € a la empresa VELVAT S.A, a devolver en 3 años en cuotas constantes anuales y cuyo cuadro de amortización según el tipo de interés efectivo al 7\% es el siguiente:

\begin{table}[h!]
\centering
\begin{tabular}{|c|c|c|c|c|c|}
\hline
\textbf{Fecha} & \textbf{Cobro} & \textbf{Pagos (cuota)} & \textbf{Intereses} & \textbf{Capital} & \textbf{Coste amortizado} \\ \hline
01/01/2018 & 10.000 & - & - & - & 10.000 \\ \hline
31/12/2018 & - & 3.810,52 & 700 & 3.110,52 & 6.889,48 \\ \hline
31/12/2019 & - & 3.810,52 & 482,26 & 3.328,25 & 3.561,23 \\ \hline
31/12/2020 & - & 3.810,52 & 249,29 & 3.561,23 & 0 \\ \hline
\end{tabular}
\caption{CUADRO DE AMORTIZACIÓN A COSTE EFECTIVO 7\%}
\end{table}

\textbf{SE PIDE:} contabilizar los siguientes apartados para la empresa VELVAT, S.A.:
\begin{enumerate}[label=\alph*)]
\item Asiento contable de la formalización del préstamo el 01/01/2018.

\begin{table}[H]
\centering
\begin{tabular}{|p{2cm}|p{8cm}|p{2cm}|}
\hline
\rowcolor{blue!30}
\textbf{DEBE} & \textbf{Asiento contable de la formalización del préstamo el 01/01/2018} & \textbf{HABER} \\ \hline
10.000 & (572) Bancos c/c & \\ \hline
& (520) Deudas a corto plazo con entidades de crédito & 3.110,52 \\ \hline
& (170) Deudas a largo plazo con entidades de crédito & 6.889,48 \\ \hline
\end{tabular}
\end{table}

\item Asiento contable del pago de la primera cuota el 31/12/2018.

\begin{table}[H]
\centering
\begin{tabular}{|p{2cm}|p{8cm}|p{2cm}|}
\hline
\rowcolor{blue!30}
\textbf{DEBE} & \textbf{Asiento contable del pago de la primera cuota el 31/12/2018} & \textbf{HABER} \\ \hline
3.110,52 & (520) Deudas a corto plazo con entidades de crédito & \\ \hline
700 & (662) Intereses de deudas & \\ \hline
& (572) Bancos c/c & 3.810,52 \\ \hline
\end{tabular}
\end{table}

\item Asiento contable de la reclasificación de la obligación de pago el 31/12/2019.

\begin{table}[H]
\centering
\begin{tabular}{|p{2cm}|p{8cm}|p{2cm}|}
\hline
\rowcolor{blue!30}
\textbf{DEBE} & \textbf{Asiento contable de la reclasificación de la obligación de pago el 31/12/2019} & \textbf{HABER} \\ \hline
3.561,23 & (170) Deudas a largo plazo con entidades de crédito & \\ \hline
& (520) Deudas a corto plazo con entidades de crédito & 3.561,23 \\ \hline
\end{tabular}
\end{table}

\item Asiento contable del pago de la última cuota el 31/12/2020.
\end{enumerate}

\begin{table}[H]
\centering
\begin{tabular}{|p{2cm}|p{8cm}|p{2cm}|}
\hline
\rowcolor{blue!30}
\textbf{DEBE} & \textbf{Asiento contable del pago de la última cuota el 31/12/2020} & \textbf{HABER} \\ \hline
3.561,23 & (520) Deudas a corto plazo con entidades de crédito & \\ \hline
249,29 & (662) Intereses de deudas & \\ \hline
& (572) Bancos c/c & 3.810,52 \\ \hline
\end{tabular}
\end{table}

\newpage
\subsection*{\textcolor{red}{Ejercicio 3}}

La empresa CONFINADOS, S.A. realiza una emisión de 1.000 obligaciones el 1 de julio de 2019, cobrando por ellas un total de 130.000 €. Su vencimiento, que es único, tendrá lugar el 30 de junio del año 2021, con un valor de reembolso total de 150.000 €. El nominal de las obligaciones es de 150 € cada una.

La tabla, que recoge el coste amortizado al tipo de interés efectivo del 7,417231\%, es la siguiente:

\begin{table}[h]
    \centering
    \renewcommand{\arraystretch}{1.2}
    \begin{tabular}{|c|r|r|p{2cm}|p{2cm}|p{2cm}|}
        \hline
        \textbf{Fecha} & \textbf{Cobro} & \textbf{Pagos} & \textbf{Intereses devengados} & \textbf{Saldo amortizado} & \textbf{Saldo pendiente de amortizar} \\
        \hline
        01/07/2019 & 130.000 & 0 &  &  & 130.000 \\
        31/12/2019 & 0 & 0 & 4.734,97 & 4.734,97 & 134.734,97 \\
        31/12/2020 & 0 & 0 & 9.993,60 & 9.993,60 & 144.728,57 \\
        01/07/2021 & 150.000 & 0 & 5.271,43 & 144.728,57 & 0 \\
        \hline
    \end{tabular}
    \caption{Tabla de amortización}
    \label{tab:amortizacion}
\end{table}


\textbf{SE PIDE:} Realizar los asientos contables en el libro diario de la sociedad CONFINADOS S.A. relativos a las siguientes operaciones:
\begin{enumerate}[label=\alph*)]
\item Contabilización, el 01/07/2019, de la emisión de la deuda.

\begin{table}[H]
\centering
\begin{tabular}{|p{2cm}|p{8cm}|p{2cm}|}
\hline
\rowcolor{blue!30}
\textbf{DEBE} & \textbf{Contabilización, el 01/07/2019, de la emisión de la deuda} & \textbf{HABER} \\ \hline
130.000 & (572) Bancos c/c & \\ \hline
& (177) Obligaciones y bonos & 130.000 \\ \hline
\end{tabular}
\end{table}

\item Contabilización, si procede, del devengo de intereses explícitos e implícitos a 31/12/2019.

En este caso cabe destacar que como durante ``vida'' de la deuda no se efectúa ningún pago, no van a existir intereses explícitos, por lo que los consideramos en su totalidad \c{implícitos}.

\begin{table}[H]
\centering
\begin{tabular}{|p{2cm}|p{8cm}|p{2cm}|}
\hline
\rowcolor{blue!30}
\textbf{DEBE} & \textbf{Contabilización, si procede, del devengo de intereses explícitos e implícitos a 31/12/2019} & \textbf{HABER} \\ \hline
4.734,97 & (661) Intereses de obligaciones y bonos & \\ \hline
& (177) Obligaciones y bonos & 4.734,97 \\ \hline
\end{tabular}
\end{table}

\item Contabilización, si procede, del pago de intereses explícitos a 31/12/2019.

\begin{table}[H]
\centering
\begin{tabular}{|p{2cm}|p{8cm}|p{2cm}|}
\hline
\rowcolor{blue!30}
\textbf{DEBE} & \textbf{Contabilización, si procede, del pago de intereses explícitos a 31/12/2019} & \textbf{HABER} \\ \hline
\multicolumn{3}{|c|}{NO PROCEDE ANOTACIÓN CONTABLE} \\ \hline
\end{tabular}
\end{table}

\item Asientos a realizar por la empresa CONFINADOS S.A., en su caso, a 31/12/2020.

Deberán de realizar dos anotaciones contables:
\begin{enumerate}
    \item Registrar los intereses devengados durante el año 2020.
    \item Reclasificar la deuda a largo plazo a corto plazo.
\end{enumerate}

\begin{table}[H]
\centering
\begin{tabular}{|p{2cm}|p{8cm}|p{2cm}|}
\hline
\rowcolor{blue!30}
\textbf{DEBE} & \textbf{Registrar los intereses devengados durante el año 2020} & \textbf{HABER} \\ \hline
9.993,6 & (661) Intereses de obligaciones y bonos & \\ \hline
& (177) Obligaciones y bonos & 9.993,6 \\ \hline
\end{tabular}
\end{table}

\begin{table}[H]
\centering
\begin{tabular}{|p{2cm}|p{8cm}|p{2cm}|}
\hline
\rowcolor{blue!30}
\textbf{DEBE} & \textbf{Reclasificar la deuda a largo plazo a corto plazo} & \textbf{HABER} \\ \hline
144.728,57 & (177) Obligaciones y bonos & \\ \hline
& (500) Obligaciones y bonos a corto plazo & 144.728,57 \\ \hline
\end{tabular}
\end{table}

\c{Se puede hacer en un solo asiento, pero se ha desglosado para una mejor comprensión.}

\item Contabilización, si procede, en un solo asiento, de los intereses devengados y reembolso del principal a 01/07/2021.

\begin{table}[H]
\centering
\begin{tabular}{|p{2cm}|p{8cm}|p{2cm}|}
\hline
\rowcolor{blue!30}
\textbf{DEBE} & \textbf{Contabilización, si procede, en un solo asiento, de los intereses devengados y reembolso del principal a 01/07/2021} & \textbf{HABER} \\ \hline
5.271,38 & (661) Intereses de obligaciones y bonos & \\ \hline
144.728,6 & (500) Obligaciones y bonos a corto plazo & \\ \hline
& (572) Bancos c/c & 150.000 \\ \hline
\end{tabular}
\end{table}

\end{enumerate}


\newpage
\subsection*{\textcolor{red}{Ejercicio 4}}

La empresa SABIOS, S.A. suscribe un préstamo de 50.000 € a un tipo de interés anual del 5\% a devolver en dos años con cuotas anuales constantes. La comisión de apertura ascendió a 600 € y los gastos de notaría a 500 €.

El cuadro de amortización facilitado por la entidad financiera en función del interés nominal del 5\% es el siguiente:

\begin{table}[H]
\centering
\begin{tabular}{|c|c|c|c|c|}
\hline
\textbf{Vencimiento} & \textbf{Cuota} & \textbf{Capital} & \textbf{Intereses} & \textbf{Pendiente de Amortización} \\ \hline
01/05/2019 & - & - & - & 50.000 \\ \hline
01/05/2020 & 26.890 & 24.390 & 2.500 & 25.610 \\ \hline
01/05/2021 & 26.890 & 25.610 & 1.280 & 0 \\ \hline
\end{tabular}
\caption{CUADRO DE AMORTIZACIÓN A INTERÉS NOMINAL 5\%}
\end{table}

El cuadro de amortización obtenido mediante el criterio del tipo de interés efectivo del 6,5807\% responde al siguiente detalle:

\begin{table}[h!]
\centering
\begin{tabular}{|c|c|c|c|c|}
\hline
\textbf{Vencimiento} & \textbf{Cuota} & \textbf{Capital} & \textbf{Intereses} & \textbf{Pendiente de Amortización} \\ \hline
01/05/2019 & - & - & - & 48.900 \\ \hline
01/05/2020 & 26.890 & 23.672 & 3.218 & 25.228 \\ \hline
01/05/2021 & 26.890 & 25.228 & 1.662 & 0 \\ \hline
\end{tabular}
\caption{CUADRO DE AMORTIZACIÓN A INTERÉS EFECTIVO 6,5807\%}
\end{table}

\textbf{SE PIDE:} Sabiendo que SABIOS, S.A. ha catalogado este pasivo en la categoría de Préstamos y partidas a pagar, y empleando el sistema de capitalización compuesta, reflejo contable en el libro diario de las siguientes operaciones:
\begin{enumerate}[label=\alph*)]
\item Reconocimiento inicial por la obtención y cobro del préstamo.

\begin{table}[H]
\centering
\begin{tabular}{|p{2cm}|p{8cm}|p{2cm}|}
\hline
\rowcolor{blue!30}
\textbf{DEBE} & \textbf{Reconocimiento inicial por la obtención y cobro del préstamo} & \textbf{HABER} \\ \hline
48.900 & (572) Bancos c/c & \\ \hline
& (5200) Préstamos a c/p con entidades de crédito & 23.672 \\ \hline
& (170) Deudas a l/p con entidades de crédito & 25.228 \\ \hline
\end{tabular}
\end{table}

\item Contabilización de los intereses devengados a 31/12/2019.

Por un lado, debemos de calcular los intereses totales que se han devengado durante 8 meses que podemos ver en la figura \ref{fig:intereses_devengados_ejercicio4}.

\begin{figure}[H]
    \begin{equation*}
        \text{Intereses totales}  = \left[48\p900 \times (1,0658 / 12 )\right] - 48\p900 = 2\p122,44 \text{ €}
    \end{equation*}
    \caption{Cálculo de los intereses devengados a 31/12/2019.}
    \label{fig:intereses_devengados_ejercicio4}
\end{figure} 

Ahora debemos de calcualar los intereses explícitos que podemos ver en la figura \ref{fig:intereses_explicitos_ejercicio4}, así como los intereses implícitos que podemos ver en la figura \ref{fig:intereses_implicitos_ejercicio4}.
\begin{figure}[H]
    \begin{equation*}
        \text{Intereses explícitos} = 50\p000 \times (1,058 / 12) - 50\p000 = 1\p653,07 \text{ €}
    \end{equation*}
    \caption{Cálculo de los intereses explícitos a 31/12/2019.}
    \label{fig:intereses_explicitos_ejercicio4}
\end{figure}

\begin{figure}[H]
    \begin{equation*}
        \text{Intereses implícitos} = 2\p122,44 - 1\p653,07 = 471,38 \text{ €}
    \end{equation*}
    \caption{Cálculo de los intereses implícitos a 31/12/2019.}
    \label{fig:intereses_implicitos_ejercicio4}
\end{figure}


\begin{table}[H]
\centering
\begin{tabular}{|p{2cm}|p{8cm}|p{2cm}|}
\hline
\rowcolor{blue!30}
\textbf{DEBE} & \textbf{Contabilización de los intereses devengados a 31/12/2019} & \textbf{HABER} \\ \hline
2.122,44 & (6623) Intereses de deudas con entidades de crédito & \\ \hline
& (527) Intereses a c/p deudas con entidades de crédito & 1.653,07 \\ \hline
& (5200) Préstamos a c/p con entidades de crédito & 469,37 \\ \hline
\end{tabular}
\end{table}

\item Devengo y pago, en un único asiento, de la cuota correspondiente a 01/05/2020.

\begin{align*}
    \text{Intereses totales} = \\
    \\ = \left[\left(48\p900+2\p122,44\right) \times \left(1,065807/12\right)\right] - \left(48\p900+2\p122,44\right) = 1\p095,52 \text{ €}
\end{align*}

\begin{table}[H]
\centering
\begin{tabular}{|p{2cm}|p{8cm}|p{2cm}|}
\hline
\rowcolor{blue!30}
\textbf{DEBE} & \textbf{Devengo y pago, en un único asiento, de la cuota correspondiente a 01/05/2020} & \textbf{HABER} \\ \hline
1.095,52 & (6623) Intereses de deudas a c/p con entidades de crédito & \\ \hline
1.653,07 & (527) Intereses a c/p de deudas con entidades de crédito & \\ \hline
24.141,37 & (5200) Préstamos a c/p con entidades de crédito & \\ \hline
& (572) Bancos c/c & 26.890 \\ \hline
\end{tabular}
\end{table}


\item Contabilice, exclusivamente, el pago de la cuota correspondiente a 01/05/2020.

\begin{table}[H]
\centering
\begin{tabular}{|p{2cm}|p{8cm}|p{2cm}|}
\hline
\rowcolor{blue!30}
\textbf{DEBE} & \textbf{Contabilice, exclusivamente, el pago de la cuota correspondiente a 01/05/2020} & \textbf{HABER} \\ \hline
2.500 & (527) Intereses a c/p de deudas con entidades de crédito & \\ \hline
24.390 & (5200) Préstamos a c/p con entidades de crédito & \\ \hline
& (572) Bancos c/c & 26.890 \\ \hline
\end{tabular}
\end{table}

\item Contabilice, exclusivamente, los intereses devengados a 31/12/2020.

\begin{figure}[H]
    \begin{align*}
        \text{Intereses totales} &= \left[25.228 \times \left(1,065807^{8/12}\right)\right] - 25.228 = 1.094,98 \text{ €} \\
        \text{Intereses explícitos} &= \left[25.609 \times \left(1,05^{8/12}\right)\right] - 25.609 = 846,67 \text{ €} \\
        \text{Intereses implícitos} &= 1.094,98 - 846,67 = 248,31 \text{ €}
    \end{align*}
    \caption{Cálculo de los intereses devengados a 31/12/2020.}
    \label{fig:intereses_devengados_ejercicio4_2020}
\end{figure}

\begin{table}[H]
\centering
\begin{tabular}{|p{2cm}|p{8cm}|p{2cm}|}
\hline
\rowcolor{blue!30}
\textbf{DEBE} & \textbf{Contabilice, exclusivamente, los intereses devengados a 31/12/2020} & \textbf{HABER} \\ \hline
1.094,98 & (6623) Intereses de deudas con entidades de crédito & \\ \hline
& (527) Intereses a c/p deudas con entidades de crédito & 846,67 \\ \hline
& (5200) Préstamos a c/p con entidades de crédito & 248,31 \\ \hline
\end{tabular}
\end{table}

\item Contabilice, exclusivamente, el devengo de los intereses a 01/05/2021.

\begin{figure}[H]
    \begin{align*}
        \text{Intereses totales} &= \left[\left(25.228 + 1.094,98\right) \times \left(1,065807^{4/12}\right)\right] - \left(25.228 + 1.094,98\right) = 565,19 \text{ €} \\
        \text{Intereses explícitos} &= \left[\left(25.610 + 846,67\right) \times \left(1,05^{4/12}\right)\right] - \left(25.610 + 846,67\right) = 433,80 \text{ €} \\
        \text{Intereses implícitos} &= 565,19 - 433,80 = 131,39 \text{ €}
    \end{align*}
    \caption{Cálculo de los intereses devengados a 01/05/2021.}
    \label{fig:intereses_devengados_ejercicio4_2021}
\end{figure}

\begin{table}[H]
\centering
\begin{tabular}{|p{2cm}|p{8cm}|p{2cm}|}
\hline
\rowcolor{blue!30}
\textbf{DEBE} & \textbf{Contabilice, exclusivamente, el devengo de los intereses a 01/05/2021} & \textbf{HABER} \\ \hline
565,19 & (6623) Intereses de deudas a c/p con entidades de crédito & \\ \hline
& (527) Intereses a c/p deudas con entidades de crédito & 433,80 \\ \hline
& (5200) Préstamos a c/p con entidades de crédito & 131,39 \\ \hline
\end{tabular}
\end{table}

\item Contabilice, exclusivamente, el pago de la cuota a 01/05/2021.

\begin{table}[H]
\centering
\begin{tabular}{|p{2cm}|p{8cm}|p{2cm}|}
\hline
\rowcolor{blue!30}
\textbf{DEBE} & \textbf{Contabilice, exclusivamente, el pago de la cuota a 01/05/2021} & \textbf{HABER} \\ \hline
1.280 & (527) Intereses a c/p de deudas con entidades de crédito & \\ \hline
25.610 & (5200) Préstamos a c/p con entidades de crédito & \\ \hline
& (572) Bancos c/c & 26.890 \\ \hline
\end{tabular}
\end{table}

\end{enumerate}

