\documentclass[a4paper,12pt]{book}

% Paquetes necesarios
\usepackage[utf8]{inputenc}   % Codificación de caracteres
\usepackage[spanish]{babel}   % Idioma español
\usepackage[T1]{fontenc}      % Codificación de fuentes
\usepackage{amsmath, amssymb} % Símbolos matemáticos
\usepackage{graphicx}         % Inclusión de gráficos
\usepackage{cite}             % Gestión de citas
\usepackage{hyperref}         % Enlaces y referencias
\usepackage{geometry}         % Configuración de márgenes
\usepackage{fancyhdr}         % Encabezados y pies de página
\usepackage{titlesec}         % Formato de títulos
\usepackage{booktabs}         % Tablas profesionales
\usepackage{caption}          % Personalización de leyendas
\usepackage{enumitem}         % Personalización de listas
\usepackage{float}
\usepackage{tcolorbox}
\usepackage[table]{xcolor} % Paquete para colores en tablas
\usepackage{colortbl}       % Complemento para colorear celdas específicas
\usepackage{multirow}       % Combinar celdas en tablas
\usepackage{makecell}       % Combinar celdas en tablas
\usepackage{enumitem}
\usepackage{amsmath}
\usepackage{eurosym}
\usepackage{tikz}
\usepackage{pdfpages}
\usepackage{pifont}
\usepackage{tabularx}


\newcommand{\cuenta}[1]{
    \ifnum#1=2800 2800. Amortización acumulada de investigación\fi
    \ifnum#1=251 251. Valores representativos de deuda\fi
    \ifnum#1=250 250. Inversiones financieras a l/p en instrumentos de patrimonio\fi
    \ifnum#1=133 133. Ajustes en la valoración en AF a VR[PN]\fi
    \ifnum#1=900 900. Beneficios en AF a VR[PN]\fi
    \ifnum#1=7632 7632. Beneficio de activos financieros a VR con cambios en patrimonio neto\fi
    \ifnum#1=802 802. Transferencia de beneficios de AFVR\fi
    \ifnum#1=766 766. Beneficios en participaciones y VRD\fi
    \ifnum#1=800 800. Pérdidas en AR a VR[PN]\fi
    \ifnum#1=761 761. Ingresos de valores representativos de deuda\fi
    \ifnum#1=546 546. Intereses a corto plazo de valores representativos de deuda\fi
    \ifnum#1=572 572. Bancos c/c \fi
    \ifnum#1=541 541. Valores representativos de deuda a corto plazo \fi
    \ifnum#1=669 669. Otros gastos financieros\fi
    \ifnum#1=666 666. Pérdidas en participaciones y VRD\fi
    \ifnum#1=540 540. Inversiones financieras c/p en instrumentos de patrimonio\fi
    \ifnum#1=252 252.Créditos a l/p\fi
    \ifnum#1=542 542. Créditos a c/p\fi
    \ifnum#1=76203 76203. Ingresos de créditos a largo plazo, otras empresas\fi
    \ifnum#1=547 547. Intereses a corto plazo de créditos\fi
    \ifnum#1=6968 6968. Pérdidas por deterioro de valores representativos de deuda a largo plazo, otras empresas\fi
    \ifnum#1=297 297. Deterioro de valor de valores representativos de deuda a l/p\fi
    \ifnum#1=6630 6630. Pérdidas por valoración de inversiones financieras\fi
    \ifnum#1=7630 7630. Beneficios por valoración de inversiones financiera\fi
    \ifnum#1=2404 2404. Participaciones en empresas asociadas\fi
}


\newcommand{\e}{\text{€ }}
\renewcommand{\c}[1]{\textit{#1}}

% Configuración de márgenes
\geometry{left=3cm, right=3cm, top=2.5cm, bottom=2.5cm}

% Configuración de encabezados y pies de página
% \setlength{\headheight}{14.49998pt}
\pagestyle{fancy}
\fancyhf{}
\fancyhead[L]{Universidad de Granada}
\fancyhead[L]{\nouppercase{\leftmark}}

% \fancyhead[C]{Escuela Técnica Superior de Ingenierías Informática}
%\fancyhead[L]{Universidad de Granada}
\fancyhead[R]{Contabilidad Financiera II}
\fancyfoot[L]{\rule[0pt]{\textwidth}{0.2pt}\\Ismael Sallami Moreno}
\fancyfoot[C]{\rule[0pt]{\textwidth}{0.2pt}\\\thepage}
\fancyfoot[R]{\rule[0pt]{\textwidth}{0.2pt}\\\today}

\renewcommand{\sectionmark}[1]{\markboth{#1}{}} % Configura \leftmark para que solo muestre la sección


% Formato de títulos
\titleformat{\section}{\large\bfseries}{\thesection.}{0.5em}{}
\titleformat{\subsection}{\normalsize\bfseries}{\thesubsection.}{0.5em}{}

% Datos del documento
\title{\textbf{Temario Contabilidad Financiera II}}
\author{
    Ismael Sallami Moreno \\
    \texttt{ism350zsallami@correo.ugr.es}
}
\date{
    \vspace{1cm}
    \begin{tabular}{rl}
        \textbf{Asignatura:} & Contabilidad Financiera II \\
        \textbf{Tema:} & Teoría \\
        \textbf{Fecha:} & \today
    \end{tabular}
}

% Configuración del índice
\usepackage{tocloft} % Para personalizar el índice
\usepackage{titletoc}
\renewcommand{\cftsecleader}{\cftdotfill{\cftdotsep}} % Línea de puntos espaciados
\renewcommand{\cftdot}{\normalfont .} % Puntos suspensivos espaciados
\renewcommand{\cftsecfont}{\normalfont} % Fuente normal para las secciones
\renewcommand{\cftsubsecfont}{\normalfont} % Fuente normal para las subsecciones
\renewcommand{\cftsecafterpnum}{\vspace{4pt}} % Espacio después de cada entrada
\renewcommand{\cftsubsecafterpnum}{\vspace{3pt}} % Espacio después de cada subsección
\renewcommand{\cftbeforesecskip}{5pt} % Espacio antes de cada sección
\renewcommand{\cftbeforesubsecskip}{2pt} % Espacio antes de cada subsección

% Estilo creativo para el título del índice
\renewcommand{\contentsname}{\Large\textsc{Índice}} % Título en mayúsculas y grande
\titlecontents{section}[1.5em]{\addvspace{10pt}\bfseries} % Estilo de secciones
{\contentslabel{1.5em}}{} % Formato de etiquetas
{\titlerule*[0.5pc]{.}\contentspage} % Línea de puntos y número de página


\begin{document}

% Portada
\begin{titlepage}
    \begin{center}
        % \vspace*{1cm}
        
        % \Huge
        % \textbf{Práctica Contabilidad Financiera II}
        \Huge \textbf{Temario Contabilidad Financiera II} 
        % \vspace{0.5cm}
        % \LARGE
        % \textbf{Ismael Sallami Moreno}\\
        % \LARGE
        % \texttt{ism350zsallami@correo.ugr.es}
        % \LARGE
        % \url{https://github.com/Ismael-Sallami}
        
        % \vfill
        
        % \Large
        % \textbf{Universidad de Granada}
        
        \vspace{0.8cm}
        
        \begin{tikzpicture}[remember picture, overlay]
            \node[opacity=0.2] at (current page.center) {\includegraphics[width=\paperwidth,height=\paperheight]{portada.jpg}};
            \node[align=center] at (current page.center) {
                
                \vspace{0.5cm}
                \LARGE \textbf{Ismael Sallami Moreno} \\
                \LARGE \texttt{ism350zsallami@correo.ugr.es} \\
                \LARGE \url{https://ismael-sallami.github.io/} \\
                \LARGE \url{https://elblogdeismael.github.io/} \\
                \vspace{2cm}
                \Large \textbf{Universidad de Granada} \\
                \vspace{0.8cm}
                % \Large \textbf{2025}
            };
        \end{tikzpicture}
        \vfill
        
        \Large
        \textbf{2025}
        
    \end{center}
\end{titlepage}
\newpage


\includepdf[pages=-]{../../../../licencia.pdf}

% Tabla de contenidos

\tableofcontents
\newpage

\chapter{Activos Financieros}

Uno de los ejercicios de examen será identificar los componentes de una imagen de una máquina (PC, Pórtatil).

Cada uno de los portátiles tiene un formato propio.

Se proporciona información sobre los componentes de la placa base, en especial, en los manuales de Prado.

En cuanto al montaje de los componentes de la placa base, debemos de tener cuidado:
\begin{itemize}
    \item Que esté apagado.
    \item Debemos de usar guantes que no conducen la electricidad, aunque en los actuales, suelen estar protegidos.
    \item No tocar nada metálico con la placa base.
    \item Un componente solo se instala de una manera, no debemos de forzarlo.
\end{itemize}

\subsection{Fuente de Alimentación}

Suele ser la parte que más se avería. Se puede comprobar con un polímetro. La fuente de alimentación posee numerosos puertos para cargar cada una de las partes de la máquina que requieren enegía.

Hay varios tipos de fuentes de alimentación.

Otra de las partes es el módulo regular de voltaje, esta pensado para evitar que el ordenador se pueda quemar, entre otras palabras.

En cuanto a los procesadores, hay dos tipos:
\begin{itemize}
    \item PGA.
    \item LGA.  
\end{itemize}


\begin{tcolorbox}[colback=yellow!5!white,colframe=yellow!75!black]
    \textbf{Pregunta de examen:} ¿Un conector de tipo PGA tiene pines alargados (patillas) que se enganchan al procesador? \textbf{Verdadero o Falso} Es falso.
    
\end{tcolorbox}

\begin{figure}[H]
    \centering
    \includegraphics[width=0.8\textwidth]{images/Tema2/evol.png}
    \caption{Evolución histórica de los microprocesadores.}
    \label{fig:1}
\end{figure}

\textbf{Preguntas de examen:}
\begin{itemize}
    \item ¿Que pasó en 2005 relacionado con la frecuencia de los procesadores?. Responder en base a la imagen de arriba. Que deciden poner más cores,\dots
    \item ¿Que diferencia a un procesador de sobremesa a uno de servidores? Número de cores.
    \item Diferencias enter un microprocesador de PC a los de servidores (Transparencia número 15).
    \item ¿Que es un canal de RAM? Conecta la memoria con el procesador, de manera que se puede leer más rápido.
\end{itemize}

Un servidor tiene instrucciones distintas a las de ordenadores de sobremesa, debido a que estan destinados por ejemplo a virtualización de máquinas virtuales.

Los procesadores AMD para servidores, son multichips, es decir, tienen varios procesadores en un mismo chip (no todos).

Otra pregunta:
¿La primera arquitectura de 64 bits fue en 2004? Falso, ya había en los años 90.

Otra de las partes de un ordenador que podemos destacar es el \textit{IBM POWER}, que significa Performance Optimization With Enhanced RISC.

Luego podemos encontrar el disipador de calor, que se encarga de refrigerar el procesador.

Tenemos ranuras de memoria DRAM (Dynamic RAM), que es donde se conecta los módulos de memoria principal. Dentro del procesador hay memoria RAM, pero es estática. La memoria RAM es volátil, es decir, que si se apaga el ordenador, se pierde la información.

\begin{figure}
    \centering
    \includegraphics[width=0.8\textwidth]{images/Tema2/piramide.png}
    \caption{Jerarquía de memoria.}
    \label{fig:2}
\end{figure}
Las cintas de la imagen, se usan para copias de seguridad, ya que son muy lentas.

En cuanto a la evolución histórica de la memoria RAM, podemos destacar que primero estaba la DRAM, luego surge la PM, luego la SDRAM,\dots

Al principio tenían lo que conocemos como 'patitas', luego pasan a conectores, y luego a conectores que se encuentran por delante y por detrás.

Tenemos diversos tipos de DMIM (Dual Data Rate Inline Memory Module), que son módulos de memoria RAM:
\begin{itemize}
    \item Para servidores.
    \begin{itemize}
        \item EU-DIMM. U-DDIM con corrección de errores. Por cada línea\footnote{Como línea nos referimos a las líneas de datos, de manera gráfica ver la diapositiva 23, líneas con números en ECC RAM.} de datos se pasan 8 bits de corrección de errores y otros 8 bits de datos.
        \item R-DIMM: Registered DIMM. Hay un registro que almacena
        las señales de control (operación a realizar, líneas de
        dirección…). Mayor latencia que EU-DIMM pero permiten
        módulos de mayor tamaño. Tienen ECC.
        \item LR-DIMM: Load Reduced DIMM. Hay un buffer que
        almacena tanto las señales de control como los datos a
        leer/escribir. Mayor latencia que R-DIMM, pero son las que
        permiten los módulos con mayor tamaño. Tienen ECC
    \end{itemize}
    \item Para PC y portátiles.
    \begin{itemize}
        \item DIMM: Unbuffered (ó Unregistered) DIMM.
        \item SO-DIMM. Small Outline DIMM. Tamaño más reducido para
        equipos portátiles (tienen menos contactos).
    \end{itemize}
\end{itemize}

Cuando se aumenta el número de pines es más eficiente. En cuanto al ancho de banda debemps de tener en cuenta que va aumentando también. En la diapositiva 21, no debemos de aprendernos los números, pero si debemos de saber ocmo crecen y demás.

Un procesador accede a los módulos de memoria DRAM a  través de los \textit{canales de memoria}. Debemos de tener en cuenta que cuando las encajamos no debemos de hacer mucha fuerza. Algunos módulos tienes disipadores de calor, chips,... Hay un espacio entre las memorias para que se refrigere. En cada instante de lectura no se puede leer de manera simultánea, sino que se lee de manera secuencial. Podemos distinguir entre \textit{bancos}(leemos solo de uno) y entre \textit{canales}(leemos de todos los canales a la vez), cuando son canales podemos leer al final solo de un módulo de memoria (Diapositiva 23).

El concepto de rango se refiere al hecho de tener bancos de memoria. En cuanto a la notación de $1R\times8$ se refiere a que es un módulo de memoria con un rango, y que tiene 8 chips de memoria, por lo que lee de 8 chips a la vez. (A nivel de módulo, banco es a nivel de zócalo).

Se puede dar el caso que sea Double Sided, pero 1 rank, por lo que aunque tenga rangos por delante y por detrás, solo puede leer de uno a la vez.









\chapter{Pasivos Financieros}

\section{Concepto, características y tipología de los pasivos financieros}

\subsection{Concepto}

Previamente debemos de saber que es un instrumento financiero, el cual es un contrato que da lugar a un activo financiero en una empresa, y simultáneamente, a un pasivo financiero o a un instrumento de patrimonio en otra empresa.

Los instrumentos financieros emitidos, incurridos o asumidos se clasificarán como pasivos financiero, en su totalidad o en una de sus partes, siempre que de acuero a su realidad económica, suponga una obligación contractual de entregar efectivo u otro activo financiero, intercambiar activos o pasivos financieros con terceros potencialmente desfavorables, también se clasificarán como un pasivo financiero, todo contrato que pueda ser o será, liquidado con los instrumentos propios de la empresa, siempre que no sean instrumentos derivados.

\subsubsection{Pasivo financiero vs instrumento de patrimonio neto}

\begin{itemize}
    \item Pasivo financiero: existe una obligación contractual que recae sobre una de las partes implicadas en el instrumento financiero. 
    \item Intrumento de PN: no existe una obligación contractual que recae sobre una de las partes implicadas en el instrumento financiero. Aunque el comprador pueda percibir dividendos, el emisor no tiene la obligación de repartir los dividendos.
\end{itemize}

\subsection{Diferentes tipologías}

\begin{itemize}
    \item Débitos por operaciones comerciales
    \item Deudas con entidades de crédito
    \item Obligaciones y otros valores negociables emitidos, tales como bonos y pagarés.
    \item Derivados con valoración desfavorable para la empresa.
    \item Deudas con características especiales, tales como deudas subordinadas.
    \item Otros pasivos financieros, tales como deudas con proveedores.
\end{itemize}

\subsection{Tipos de pasivos financieros}
\begin{itemize}
    \item Pasivo a coste amortizado: Todos en esta categoría, excepto cuando deban de valorarse a VR con cambios en la cuenta de pérdidas y ganancias. En este incluidos débito por operaciones comerciales y débito por operaciones no comerciales.
    \item Pasivo a valor razonable con cambios en la cuenta de pérdidas y ganancias: Deben de cumplir:
    \begin{itemize}
        \item Pasivos que se mantienen para negociar.
        \begin{itemize}
            \item se emite con el propósito de readquirirlo en el corto plazo.
            \item obligación de que un vendedor de un activo financiero lo readquiera.
            \item Forme parte de una cartera de activos financieros identificados que se gestionan conjuntamente.
        \end{itemize}
        \item Desde el momento de reconocimiento inicial, ha sido designado por la empresa como a valor razonable con cambios en la cuenta de pérdidas y ganancias.
    \end{itemize}
\end{itemize}

\subsection{¿Qué cuentas son con las que más vamos a trabajar?}

\begin{itemize}
    \item 15, 16, 17, 18\footnote{Para saber que cuentan son consulte el cuadro de cuentas o bien el manual, página 117-118}.
    \item 50, 51, 52, 56.
\end{itemize}

No se recogen los derivados de operaciones de tráfico, ya que se estudian en la asignatura de CF1 (operaciones con proveedores y acreedores,...).

\subsection{Valoración de los pasivos financieros}

\subsubsection{A coste amortizado}
\begin{itemize}
    \item Valoración inicial: Inicialmente a VR, es decir, el precio de la transacción. Consideramos los costes que son para emitirlos. Por ejemplo, si pido un préstamo de una cuantía X, el importe extra por gestión, se debe de incluir como un mayor importe en el valor del pasivo.
    \item Valoración posterior: Por su coste amortizado, los interéses devengados, se contabilizarán en la cuenta de pérdidas y ganancias, aplicando el método del tanto efectivo.
\end{itemize}

\textit{Nota: Debemos de tener en cuenta que la empresa que recibe el préstamo tiene un pasivo, mientras que la empresa que lo da tiene un activo.}




\newpage
% Referencias
\begin{thebibliography}{99}
    \bibitem{Referencia1}
    Ismael Sallami Moreno, \textbf{Estudiante del Doble Grado en Ingeniería Informática + ADE}, Universidad de Granada, 2025.
    % \bibitem{Referencia2}
    % Autor(es), \emph{Título del libro}, Editorial, año.
    
    % \bibitem{Referencia3}
    % Autor(es), \emph{Título del documento}, Nombre de la Conferencia, páginas, año.
    \end{thebibliography}

\end{document}
