\section{Concepto}

Antes debemos de plantearnos que es un instrumento financiero, pues bien, \textit{un instrumento financiero es un contrato que da ligar a un activo financiero para una empresa y, simultáneamente, a un pasivo financiero o a un instrumento de patrimonio para otra empresa.}

Como ejemplo, podemos destacar una gran cantidad, pero nos centraremos en 3 que serán los que trataremos en este capítulo:
\begin{itemize}
    \item Créditos a terceros: Préstamos y créditos financieros concedidos, incluidos los surgidos de la venta de activos no corrientes.
    \item Valores representativos de deuda de otras empresas adquiridos: obligaciones, bonos y pagarés.
    \item Instrumento de patrimonio de otras empresas adquiridos: acciones, participaciones en instituciones de inversión colectiva y otros instrumentos de patrimonio.
\end{itemize}


Dentro de esta también se regula la problemática relatica a las \textit{acciones propias}, considerando que \textit{surgirá como un activo o un pasivo financiero en operaciones que se vayan a liquidar entregando instrumentos de patrimonio propio}, es decir acciones propias de la empresa, cuando de los mismos estemos considerando que son mecanismos de pago.

La empesa reconocerá un activo financiero cuando se convierta en una parte obligada del contrato o negocio jurídico.

Si los elementos suponen un derecho de cobra, y no derivan de una relación contractual, sino que tienen su origen en un requerimiento legal o en actividades de fomento por parte de las Administraciones Públicas, \textit{no se consideran activos financieros}.

Las cuentas a tener en cuenta en este apartado, son:
\begin{itemize}
    \item 24 Inversiones financieras a largo plazo en partes vinculadas.
    \item 25 Otras inversiones financieras a largo plazo.
    \item 26 Fianzas y depósitos constituidos a largo plazo.
    \item 29 Deterioro del valor del inmovilizado.
    \item 43 Clientes.
    \item 44 Deudores.
    \item 49 Deterioro de valor de créditos comerciales y provisiones a corto plazo.
    \item 53 Inversiones financieras a corto plazo en partes vinculadas.
    \item 54 Otras inversiones financieras a corto plazo.
    \item 55 Otras cuentas no bancarias.
    \item 57 Tesorería.
    \item 59. Deterioro valor de instrumentos financieros a corto plazo y activos no corrientes.
\end{itemize}

\section{Clasificación AF}

Dependiendo dee su naturaleza, los activos financieros se clasifican en:
\begin{itemize}
    \item Activos financieros a coste amortizado.
    \item Activos financieros a valor razonable con cambios en el partrimonio neto \textit{AFVRPN}.
    \item Activos financieros a coste.
    \item Activos financieros a valor razonable con cambios en la cuenta de pérdidas y ganancias \textit{AFVRPYG}.
\end{itemize}

En términos generales, los activos financieros deben de valorarse a valor razonable con cambios en la cuenta de pérdidas y ganancia, con la excepción de:
\begin{itemize}
    \item Las invesiones valoradas a coste o inversiones en el patrimonio del grupo o asociadas, deben de ser clasificados como \textit{Activos financieros a coste}.
    \item Los activos financieros que tienen una estructura común a la del préstamo, es decir, del mismo se deriva una estructura de flujos de efectivo similar a la de amortización e intereses financieros, y además, que la empresa tenga la intención de mantenerlo para recibir esos flujos de efectivo, se clasificarán dentro de la categoría de \textit{Activos a coste amortizado}.
    \item Cuando la gestión del activo consista en recibir flujos de efectivo de un acuerdo básico de préstamos (amortización e intereses), pero también se pueda acordar su enajenación antes del vencimiento de los mismos. Estos se valoran como \textit{Activos financieros a valor razonable con cambios en el patrimonio neto [AFVRPN]}
\end{itemize}

Cuando no sea ninguna de las condiciones anteriores, el activo se valorará como \textit{Activos financieros a valor razonable con cambios en la cuenta de pérdidas y ganancias [AFVRPYG]}.

Ahora vamos a tratar los AF que pueden ser considerados en estas catergorías de forma específica.

\begin{enumerate}
    \item Activos financieros a coste amortizado.
    \begin{itemize}
        \item Créditos por operaciones comerciales: se originan por la venta de bienes y la prestación de servicios por operaciones de tráfico.
        \item Créditos por operaciones no comerciales: son aquellos AF, que no siendo isntrumentos de patrimonio ni derivados, no tienen origen comercial y cuyos cobros son de cuantía determinada o determinable, que procedes de \textit{operaciones de préstamo o crédito concedidos por la empresa}.
    \end{itemize}


    Se incluirán en esta categoría, incluso cuando esté admitido a negociación en un mercado organizado, si la empresa mantiene la inversión con el objetivo de percibir los flujos de efectivo derivados de la ejecución del contrato y las condiciones contractuales del AF dan lugar, en fechas específicas, a flujos de efectivo que son únicamente cobros de principal e intereses (SCPI).

    A continuación, definimos que entendemos por modelo de negocio y por las características del activo (SCPI):
    \begin{enumerate}
        \item Modelo de Negocio (MN): modelo que la empresa sigue en relación con la gestión del activo. Algunos aspectos a considerar son:
        \begin{itemize}
            \item Una empresa tiene varios modelos de negocio para gestionar sus carteras de activos.
            \item ¿Es posible clasificar una cartera de activos a coste amortizado aunque se vendan antes de los que se establece en el contrato? Sí es posible. \textit{La gestión de un grupo de activos financieros para obtener sus flujos contractuales no implica que la empresa haya de mantener todos los instrumentos hasta su vencimiento}.
            \item ¿Qué criterio es necesario tener en cuenta para decidir si una cartera de activos a coste amortizado cumple el modelo de negocio? Es necesario analizar \textit{frecuencia, importe y el calendario de las ventas de ejercicio anteriores, motivos de las ventas y las expectativas en relación con la actividad de ventas futuras}.
            \item Si se detectan incrementos de frecuencias en ventas, la empresa debe de explicar la razón y demostrar que no hay un cambio en el modelo de negocio. No se establece el límite cuantitativo específico para determinar los cambios en el modelo de negocio.
        \end{itemize}
        \item Características del Activo (SCPI): La condición se cumple si \textit{las condiciones contractuales del activo financiero dan lugar a flujos de efectivo que son únicamente cobros de principal e intereses sobre el importe del principal pendiente}. Cuando se cumplen las condiciones anteriores, los activos deben de clasificarse como \textit{Activos financieros a coste amortizado}. Cuando cumplan las condiciones del SCPI, pero no atienda a un modelo de negocio que viene desarrollando la empresa, se clasificarán como \textit{Activos financieros a valor razonable con cambios en el patrimonio neto[AFVRPN]}.
    \end{enumerate}

    \item Activos financieros a valor razonable con cambios en el patrimonio neto [AFVRPN].
    Se incluirán en esta categoría cuando las condiciones contractuales del activo financiero den lugar a flujos de efectivo que son únicamente cobros de principal e intereses sobre el importe del principal pendiente, \textit{y no se mantenga para negociar ni proceda clasificarlo en la categoría regulada en el apartado 2.2 de esta norma}.
    También se incluirán en esta categoría las inversiones de patrimonio para las que se haya ejercitado la opción irrevocable (no se mantengan para negociar, ni deban valorarse al coste).

    \item Activos financieros a coste.
    
    En esta categoría se incluyen:

    \begin{enumerate}
        \item Inversiones en el patrimonio de empresas del grupo y asociadas.
        \item Restantes inversiones en instrumentos de patrimonio sin valor razonable por referencia a valor cotizado.
        \item Los activos financieros híbridos cuyo VR no puede estimarse de manera fiable.
        \item Las aportanciones realizadas como consecuencia de un contrato de cuentas de participación y similares.
        \item Préstamos participativos cuyos interes tengan un carácter contingente.
        \item Cualquier otro activo financiero, cuando no sea posible obtener una estimación fiable de su valor razonable.
    \end{enumerate}

    \item Activos financieros a valor razonable con cambios en la cuenta de pérdidas y ganancias [AFVRPYG].
    
    Se debe de incluir en esta categoría salvo que proceda su clasificación en alguna de las restantes categorías.

    En cualquier caso, los activos financieros mantenidos para negociar se incluirán obligarioriamente en esta catergoría.

    %PAG 53 MANUAL

\end{enumerate}









\section{Baja de Activos financieros}

Se dará de baja a un activo financiero cuando expiren o se hayan cedido los derechos contractuales sobres los flujos de efectivo del mismo, siendo necesario que se hayan transferido de manera sustancial los riesgos y beneficios inherentes a su propiedad. 











Consideramos desinversión a la venta de algún activo. 

\section{Memoria}

Debemos de indicar los criterios que estoy aplicando en la calificación y la valoración de las diferentes catergorías de los activos financieros. 
Además, debemos de tener en cuenta las reclasificaciones de activos, compensación de AF y PF, correcciones por deterioro de valor originadas por el riesgo de crédito y por último la información de riesgos(crédito, liquidez y mercado, este último comprende el riesgo de los tipos de cambio, ti y otros tipos de riesgos relacionados con el precio).


