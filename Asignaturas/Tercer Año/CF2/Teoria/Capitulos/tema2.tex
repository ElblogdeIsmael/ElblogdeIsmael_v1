\section{Concepto}

Antes debemos de plantearnos que es un instrumento financiero, pues bien, \textit{un instrumento financiero es un contrato que da ligar a un activo financiero para una empresa y, simultáneamente, a un pasivo financiero o a un instrumento de patrimonio para otra empresa.}

Como ejemplo, podemos destacar una gran cantidad, pero nos centraremos en 3 que serán los que trataremos en este capítulo:
\begin{itemize}
    \item Créditos a terceros: Préstamos y créditos financieros concedidos, incluidos los surgidos de la venta de activos no corrientes.
    \item Valores representativos de deuda de otras empresas adquiridos: obligaciones, bonos y pagarés.
    \item Instrumento de patrimonio de otras empresas adquiridos: acciones, participaciones en instituciones de inversión colectiva y otros instrumentos de patrimonio.
\end{itemize}


Dentro de esta también se regula la problemática relatica a las \textit{acciones propias}, considerando que \textit{surgirá como un activo o un pasivo financiero en operaciones que se vayan a liquidar entregando instrumentos de patrimonio propio}, es decir acciones propias de la empresa, cuando de los mismos estemos considerando que son mecanismos de pago.

La empesa reconocerá un activo financiero cuando se convierta en una parte obligada del contrato o negocio jurídico.

Si los elementos suponen un derecho de cobra, y no derivan de una relación contractual, sino que tienen su origen en un requerimiento legal o en actividades de fomento por parte de las Administraciones Públicas, \textit{no se consideran activos financieros}.

Las cuentas a tener en cuenta en este apartado, son:
\begin{itemize}
    \item 24 Inversiones financieras a largo plazo en partes vinculadas.
    \item 25 Otras inversiones financieras a largo plazo.
    \item 26 Fianzas y depósitos constituidos a largo plazo.
    \item 29 Deterioro del valor del inmovilizado.
    \item 43 Clientes.
    \item 44 Deudores.
    \item 49 Deterioro de valor de créditos comerciales y provisiones a corto plazo.
    \item 53 Inversiones financieras a corto plazo en partes vinculadas.
    \item 54 Otras inversiones financieras a corto plazo.
    \item 55 Otras cuentas no bancarias.
    \item 57 Tesorería.
    \item 59. Deterioro valor de instrumentos financieros a corto plazo y activos no corrientes.
\end{itemize}

\section{Clasificación AF}

Dependiendo dee su naturaleza, los activos financieros se clasifican en:
\begin{itemize}
    \item Activos financieros a coste amortizado.
    \item Activos financieros a valor razonable con cambios en el partrimonio neto \textit{AFVRPN}.
    \item Activos financieros a coste.
    \item Activos financieros a valor razonable con cambios en la cuenta de pérdidas y ganancias \textit{AFVRPYG}.
\end{itemize}

En términos generales, los activos financieros deben de valorarse a valor razonable con cambios en la cuenta de pérdidas y ganancia, con la excepción de:
\begin{itemize}
    \item Las invesiones valoradas a coste o inversiones en el patrimonio del grupo o asociadas, deben de ser clasificados como \textit{Activos financieros a coste}.
    \item Los activos financieros que tienen una estructura común a la del préstamo, es decir, del mismo se deriva una estructura de flujos de efectivo similar a la de amortización e intereses financieros, y además, que la empresa tenga la intención de mantenerlo para recibir esos flujos de efectivo, se clasificarán dentro de la categoría de \textit{Activos a coste amortizado}.
    \item Cuando la gestión del activo consista en recibir flujos de efectivo de un acuerdo básico de préstamos (amortización e intereses), pero también se pueda acordar su enajenación antes del vencimiento de los mismos. Estos se valoran como \textit{Activos financieros a valor razonable con cambios en el patrimonio neto [AFVRPN]}
\end{itemize}

Cuando no sea ninguna de las condiciones anteriores, el activo se valorará como \textit{Activos financieros a valor razonable con cambios en la cuenta de pérdidas y ganancias [AFVRPYG]}.

Ahora vamos a tratar los AF que pueden ser considerados en estas catergorías de forma específica.

\begin{enumerate}
    \item Activos financieros a coste amortizado.
    \begin{itemize}
        \item Créditos por operaciones comerciales: se originan por la venta de bienes y la prestación de servicios por operaciones de tráfico.
        \item Créditos por operaciones no comerciales: son aquellos AF, que no siendo isntrumentos de patrimonio ni derivados, no tienen origen comercial y cuyos cobros son de cuantía determinada o determinable, que procedes de \textit{operaciones de préstamo o crédito concedidos por la empresa}.
    \end{itemize}


    Se incluirán en esta categoría, incluso cuando esté admitido a negociación en un mercado organizado, si la empresa mantiene la inversión con el objetivo de percibir los flujos de efectivo derivados de la ejecución del contrato y las condiciones contractuales del AF dan lugar, en fechas específicas, a flujos de efectivo que son únicamente cobros de principal e intereses (SCPI).

    A continuación, definimos que entendemos por modelo de negocio y por las características del activo (SCPI):
    \begin{enumerate}
        \item Modelo de Negocio (MN): modelo que la empresa sigue en relación con la gestión del activo. Algunos aspectos a considerar son:
        \begin{itemize}
            \item Una empresa tiene varios modelos de negocio para gestionar sus carteras de activos.
            \item ¿Es posible clasificar una cartera de activos a coste amortizado aunque se vendan antes de los que se establece en el contrato? Sí es posible. \textit{La gestión de un grupo de activos financieros para obtener sus flujos contractuales no implica que la empresa haya de mantener todos los instrumentos hasta su vencimiento}.
            \item ¿Qué criterio es necesario tener en cuenta para decidir si una cartera de activos a coste amortizado cumple el modelo de negocio? Es necesario analizar \textit{frecuencia, importe y el calendario de las ventas de ejercicio anteriores, motivos de las ventas y las expectativas en relación con la actividad de ventas futuras}.
            \item Si se detectan incrementos de frecuencias en ventas, la empresa debe de explicar la razón y demostrar que no hay un cambio en el modelo de negocio. No se establece el límite cuantitativo específico para determinar los cambios en el modelo de negocio.
        \end{itemize}
        \item Características del Activo (SCPI): La condición se cumple si \textit{las condiciones contractuales del activo financiero dan lugar a flujos de efectivo que son únicamente cobros de principal e intereses sobre el importe del principal pendiente}. Cuando se cumplen las condiciones anteriores, los activos deben de clasificarse como \textit{Activos financieros a coste amortizado}. Cuando cumplan las condiciones del SCPI, pero no atienda a un modelo de negocio que viene desarrollando la empresa, se clasificarán como \textit{Activos financieros a valor razonable con cambios en el patrimonio neto[AFVRPN]}.
    \end{enumerate}

    \item Activos financieros a valor razonable con cambios en el patrimonio neto [AFVRPN].
    Se incluirán en esta categoría cuando las condiciones contractuales del activo financiero den lugar a flujos de efectivo que son únicamente cobros de principal e intereses sobre el importe del principal pendiente, \textit{y no se mantenga para negociar ni proceda clasificarlo en la categoría regulada en el apartado 2.2 de esta norma}.
    También se incluirán en esta categoría las inversiones de patrimonio para las que se haya ejercitado la opción irrevocable (no se mantengan para negociar, ni deban valorarse al coste).

    \item Activos financieros a coste.
    
    En esta categoría se incluyen:

    \begin{enumerate}
        \item Inversiones en el patrimonio de empresas del grupo y asociadas.
        \item Restantes inversiones en instrumentos de patrimonio sin valor razonable por referencia a valor cotizado.
        \item Los activos financieros híbridos cuyo VR no puede estimarse de manera fiable.
        \item Las aportanciones realizadas como consecuencia de un contrato de cuentas de participación y similares.
        \item Préstamos participativos cuyos interes tengan un carácter contingente.
        \item Cualquier otro activo financiero, cuando no sea posible obtener una estimación fiable de su valor razonable.
    \end{enumerate}

    \item Activos financieros a valor razonable con cambios en la cuenta de pérdidas y ganancias [AFVRPYG].
    
    Se debe de incluir en esta categoría salvo que proceda su clasificación en alguna de las restantes categorías.

    En cualquier caso, los activos financieros mantenidos para negociar se incluirán obligarioriamente en esta catergoría.

\end{enumerate}

A efectos de catalogar lo que se entiende por negociación, consideramos las compras y ventas con el objetivo de generar una ganacia por las fluctuaciones a c/p. El activo \textit{adquiere esta condición} cuando:

\begin{itemize}
    \item Se origine o adquiera con el propósito de venderlo en el corto plazo.
    \item Forme parte en el momento de su reconocimiento inicial de una cartera de instrumentos financieros identificados y gestionados conjuntamente.
    \item Sea un instrumento financiero derivado
\end{itemize}

Los que se mantengan para negociar, ni se valoren al coste, la empresa puede realizar \textit{elección irrevocable}, en el momento de su reconocimiento inicial, de presentar los cambios posteriores en el VR directamente en el PN.

Por último, un activo se puede incluir en esta categoría para reducir las asimetrías contables. Esta opción hace referencia a la valoración de los activos de manera conjunta, aunque alguno de sus componentes se pueda valorar de manera separada.

La siguiente Figura \ref{fig:Acciones} muestra la clasificación de las acciones, en base a los siguientes criterios:

\begin{itemize}
    \item Exitencia de un modelo de negocio o SCPI. Si no se cumple no es posible clasificar el instrumento de patrimonio como \c{Activos financieros a coste amortizado}.
    \item Existencia o no de VR. Si no, solo puede valorarse a \c{coste}.  
    \item Intención o no de negociar. Si esta condiciones se cumple, se valorará a \c{valor razonable con cambios en la cuenta de pérdidas y ganancias}, mientras que si no se cumple debemos de valorar a \c{valor razonable con cambios en el patrimonio neto}.
\end{itemize}

\begin{figure}[H]
    \centering
    \includegraphics[width=0.8\textwidth]{images/clasacc.png}
    \caption{Clasificación de las acciones}
    \label{fig:Acciones}
\end{figure}

De la misma manera, la Figura \ref{fig:Deuda} muestra la clasificación de la deuda, en base a los mencionados anterirmente.

\begin{figure}[H]
    \centering
    \includegraphics[width=0.8\textwidth]{images/deuda.png}
    \caption{Clasificación de la deuda}
    \label{fig:Deuda}
\end{figure}

\subsection*{\textcolor{blue}{Ejemplo 2}}

La empresa CHOTIS, S.A. realiza diferentes adquisiciones de instrumentos de patrimonio y de deuda, con las siguientes características:

\begin{enumerate}
    \item Adquisición de 100 acciones cotizadas en mercado activo, donde se plantean dos opciones:
    \begin{enumerate}
        \item La empresa tiene intención de negociar.
        \item La empresa tiene intención de mantenerlas a largo plazo.
    \end{enumerate}
    \item Adquisición de 100 obligaciones cotizadas en mercado activo, donde se plantean dos opciones:
    \begin{enumerate}
        \item La empresa tiene intención de mantener el modelo de negocio, esperar a vencimiento y recibir los flujos de efectivo contractuales.
        \item La empresa tiene intención de especular o negociar con las mismas en el corto plazo y no espera recibir los flujos de efectivo contractuales.
    \end{enumerate}
    \item Adquisición de 100 acciones no cotizadas en mercado activo, donde no es posible determinar un valor razonable fiable.
    \item Adquisición de 100 obligaciones no cotizadas en mercado activo, donde se plantean dos opciones:
    \begin{enumerate}
        \item La empresa tiene intención de mantener el modelo de negocio, esperar a vencimiento y recibir los flujos de efectivo contractuales.
        \item La empresa tiene intención de especular o negociar con las mismas en el corto plazo y no espera recibir los flujos de efectivo contractuales.
    \end{enumerate}
    En ambos casos no es posible disponer de un valor razonable fiable.
\end{enumerate}

La respuesta a este ejercicio se muestra en la Cuadro \ref{tab:Chotis}.


\begin{table}
    \centering 
    % \begin{tabular}{|p{3cm}|p{3cm}|p{3cm}|p{3cm}|p{3cm}|}
    \begin{tabularx}{\textwidth}{|X|X|X|X|X|}
        \hline 
        Categorías & Compra 100 acciones, cotizadas en un mercado activo. & Compra 100 obligaciones, cotizadas en un mercado activo. & Compra 100 acciones, no cotizadas en un mercado activo & Compra 100 obligaciones, no cotizadas en un mercado activo. \\
        \hline
        Modelo de negocio / Cobros principal e intereses & No modelo de negocio y no SCPI- Existe VR fiable. Tenemos dos opciones: \begin{itemize}
            \item Intención de negociar.
            \item No negociar.
        \end{itemize}& 
        \begin{itemize}
            \item Opción A: La empresa tiene intención de percibir flujos de efectivo y MN.
            \item Opción B: Sin MN y sin intención de percibir flujos de efectivo. Intención de negociar.
        \end{itemize} & 
        No MN y no SCPI. Intención de negociar. No es posible estimar un VR fiable & 
        \begin{itemize}
            \item Opción A: No puede estimarse un VR fiable, pero si hay SCPI y MN.
            \item Opción B: Carácter inversión especulativa sin VR.
        \end{itemize} \\
        \hline
        \hline
        AVRF (PyG) & \ding{51} (A) & \ding{51} (B) & \ding{55} & \ding{55} \\
        \hline
        Activos a coste amortizado & \ding{55} & \ding{51}(A) & \ding{55} & \ding{51}(A) \\
        \hline
        Activos a coste & \ding{55} & \ding{55} & \ding{51} & \ding{51}(B) \\
        \hline
        AVRF (PN) & \ding{51}(B) & \ding{51}(A) & \ding{55} & \ding{55} \\
        \hline
    \end{tabularx}
    \caption{Clasificación de activos financieros de la Chotis S.A.}
    \label{tab:Chotis}
\end{table}

% \begin{table}
%     \centering
%     \begin{tabular}{|p{3cm}|p{3cm}|p{3cm}|p{3cm}|p{3cm}|}
%         \hline
%         AVRF (PyG) & \ding{51} (A) & \ding{51} (B) & \ding{55} & \ding{55} \\
%         \hline
%         Activos a coste amortizado & \ding{55} & \ding{51}(A) & \ding{55} & \ding{51}(A) \\
%         \hline
%         Activos a coste & \ding{55} & \ding{55} & \ding{51} & \ding{51}(B) \\
%         \hline
%         AVRF (PN) & \ding{51}(B) & \ding{51}(A) & \ding{55} & \ding{55} \\
%         \hline
%     \end{tabular}
%     \caption{Clasificación de activos financieros de la Chotis S.A.}
%     \label{tab:Chotis2}
% \end{table}

Como se ha señalado a lo largo de este capítulo, es necesario identificar, en primer lugar, si se cumple el modelo de negocio y el modelo SCPI. Además, se debe indicar si el activo tiene un valor razonable y éste es fiable, determinando si la intención es negociar con los mismos. Con estas premisas:

\begin{enumerate}
    \item Para el caso 1:
    \begin{itemize}
        \item Si no se cumplen el SCPI ni el modelo de negocio, pero existe un valor razonable fiable, la clasificación depende de la intención de la empresa.
        \begin{itemize}
            \item Sin intención de negociar: AFVRPN.
            \item Con intención de negociar: AFVRPYG.
        \end{itemize}
    \end{itemize}

    \item Para el caso 2:
    \begin{itemize}
        \item Opción a): Si se cumplen el SCPI y el modelo de negocio, el instrumento de deuda se clasifica como activo a coste amortizado.
        \item Opción b): Si no se cumple ningún criterio, los instrumentos de deuda se clasifican como AFVRPYG.
    \end{itemize}

    \item Para el caso 3: 
    \begin{itemize}
        \item Los instrumentos de patrimonio no cotizados se clasifican como activos a coste.
    \end{itemize}

    \item Para el caso 4:
    \begin{itemize}
        \item Opción a): Con modelo de negocio y SCPI, se clasifican como activos a coste amortizado.
        \item Opción b): Sin valor razonable y siendo una inversión especulativa, se clasifican como activos a coste.
    \end{itemize}
\end{enumerate}

\c{Si no existiera un valor razonable fiable para los casos 1 y 2, se podrían clasificar como activos financieros a coste, según la NV9-2.4 f).}

\subsection*{\textcolor{blue}{Ejemplo 3}}

En este veremos otros activos financieros y como es su clasificación. Podemos verlo en la Cuadro \ref{tab:otrosAF}.


% \begin{enumerate}
%     \item \textbf{Acciones ordinarias.} No cumple con la condición de SCPI, es decir, de ellos no se puede establecer que exista un acuerdo de préstamos con devolución de principal. Partiendo de esto, debemos de ver si existe in VR y si es fiable o no. Si no existiera las acciones se deben de clasificar a coste. Si existiera un VR y se hace el reconocimiento inicial, se clasifica como AFVRPN. Si se tiene la intención de negociar, o cualquier otro caso se clasificará como AFVRPYG.
%     \item \b{Bono simple a tipo de interés fijo, o bono simple, en euros o con fecha de vencimiento determinada.}: \begin{itemize}
%         \item Inversión cumple con SCPI.
%         \item Si no lo vende hasta el vencimiento se clasifica como \c{Activos financieros a coste amortizado}, si es por el contrario se vende antes, se clasifica como \c{Activos financieros a valor razonable con cambios en el patrimonio neto}.
%         \item \b{Bono con remuneración referenciada a precios de acciones}
%     \end{itemize}
% \end{enumerate}

\begin{table}[H]
\centering
\begin{tabularx}{\textwidth}{|X|X|X|}
\hline
\textbf{Activo financiero} & \textbf{¿Cumple SCPI?} & \textbf{Se clasifica como...} \\ \hline
Acciones ordinarias & 
No cumple SCPI. El rendimiento (intereses) no refleja el valor temporal del dinero por el uso de un capital principal, y depende del valor de un título en el mercado. & 
• AFVR (PyG) si hay intención de negociar y el valor razonable es fiable. 
• AFVR (PN) si el valor razonable es fiable pero no hay intención de negociar. 
• A coste, si no cotizan en mercados activos. \\ \hline
Bono simple (interés fijo) & 
Cumple SCPI, se esperan flujos de efectivo contractuales con intereses que reflejan el valor temporal del dinero. & 
• AFVRC si se piensa negociar a corto plazo. 
• A coste amortizado, si se mantienen a vencimiento.
• A coste, si no cotizan en mercados activos. \\ \hline
Bono referenciado a acciones & 
No cumple SCPI. Intereses dependen de precios de acciones, no reflejan valor temporal del dinero. & 
• AFVR (PyG) si el valor razonable es fiable y hay intención de negociar.
• AFVR (PN) si no hay intención de negociar. \\ \hline
Derivados (forwards, opciones) & 
No cumple SCPI. Los beneficios están sujetos a cambios en el mercado. & 
• AFVR (PyG) con valor razonable fiable.
• A coste, si el valor razonable no es fiable. \\ \hline
Bono a perpetuidad & 
No cumple SCPI. No refleja valor temporal del dinero. & 
• AFVR (PN), si el valor razonable es fiable.
• A coste, si no existe un valor razonable fiable. \\ \hline
\end{tabularx}
\caption{Clasificación y análisis de diferentes tipos de activos financieros.}
\label{tab:otrosAF}
\end{table}

\subsection{Valoración inicial y posterior de los activos financieros}

El siguiente paso es determinar el proceso de valoración de las mismas. (Ver Figura \ref{fig:Valoracion}).

\begin{figure}[H]
    \centering
    \includegraphics[width=0.9\textwidth]{images/cri_val.png}
    \caption{Valoración de los activos financieros}
    \label{fig:Valoracion}
\end{figure}

\section{AF con valoración posterior a coste amortizado}

\subsection{Valoración inicial}

Los activos incluidos en esta categoría se valorarán inicialmente por su VR, salvo evidencia en contrario. \c{Será el precio de la transacción, que equivladrá al VR de la contraprestación entregada más los costes de transacción.}

\subsection{Valoración posterior}

Se valorarán por su coste amortizado. Los intereses devengados se contabilizarán en la cuenta de Pérdidas y Ganancias, aplicando el tipo de interés efectivo (TIE).

Sabemos que la valoración posterior incluye alguno deterioros de valor en la mayoría de casos.

La primera parte del Marco Conceptual define el coste amortizado como \c{Importe al que inicalmente feu valorado un AF o un PF, menos los reembolsos del principal, más o menos la parte imputada en la cuenta de PyG, mediante la utilización del tipo de interés efectivo, de la diferencia entre el importe inicial y el valor de reembolso en el vencimiento.}

De esta definición cabe destacar:
\begin{enumerate}
    \item + valoración inicial del instrumento.
    \item - los reembolsos del princial realizados.
    \item $\pm$ la imputación en resultados de la diferencia entre el valor inicial y el valor de reembolso al vencimiento.
    \item - las reducciones por deterioro de valor.
\end{enumerate}

\subsection*{\textcolor{blue}{Ejemplo 4}}

La entidad A concede un crédito el 1-1-2018 a la entidad B con las siguientes características:

\begin{itemize}
    \item Nominal: 1.000.000 €.
    \item Comisión de apertura (a cargo de la empresa A): 10.000 €.
    \item Tipo de interés nominal: 6\%.
    \item Plazo: 3 años.
    \item Cuota anual constante: 374.109,81 €.
\end{itemize}

\textbf{SE PIDE:} Contabilizar la operación en la empresa A.

Debemos de identificar en  que categoría debemos de clasifiacar el activo. Para ello debemoos de ver de identificar el modelo de negocio y si sigue un modelo SCPI. En este en particular se cumplen ambos, \c{ya que la empresa esperará al vencimiento del crédito y sigue un modelo de crédito de principal más intereses.}

En el examen se nos dará el el tipo de interés efectivo, pero debemos de saber como se calcula, así que en este caso:

\begin{align*}
    \text{1.010.000} = \frac{\text{374.109,81 €}}{(1+i)} + \frac{\text{374.109,81 €}}{(1+i)^2} + \frac{\text{374.109,81 €}}{(1+i)^3} \\\\ 
    \text{i efectivo} \rightarrow 5,46\%
\end{align*}

En base a este tipo de interés calculamos la siguiente tabla:

\begin{table}[h]
    \centering
    \caption{Tabla de amortización}
    \begin{tabularx}{\textwidth}{|X|X|X|X|X|}
    \hline
    \textbf{Plazo} & \textbf{Cuota} & \textbf{Capital} & \textbf{Intereses} & \textbf{Coste amortizado} \\
    \hline
    01-01-2018 & & & & 1.010.000 \\
    \hline
    31-12-2018 & 374.110 & 318.923 & 55.187 & 691.077 \\
    \hline
    31-12-2019 & 374.110 & 336.349 & 37.761 & 354.727 \\
    \hline
    31-12-2020 & 374.110 & 354.728 & 19.382 & 0 \\
    \hline
    \end{tabularx}
    \label{tabla:amortizacion1}
\end{table}

En base a la tabla \ref{tabla:amortizacion1}, podemos ver que el coste amortizado es de 1.010.000 €, que es el importe inicial del crédito. Los interses se van calculando en base al coste amortizado, y el capital se va reduciendo en base a la cuota anual.\\
\textbf{Solución.} 
A continuación, procedemos al registro contable de las operaciones:

\begin{table}[H]
    \centering
    \begin{tabular}{|p{3cm}|p{6cm}|p{3cm}|}
    \hline
    \rowcolor{blue!30}
    \textbf{DEBE} & \textbf{Concesión del crédito el 01.01.2018} & \textbf{HABER} \\
    \hline
    691.077& \cuenta{252} & \\
    \hline
    318.923 &  \cuenta{542}& \\
    \hline
    &  \cuenta{572}& 1.010.000\\
    \hline
    \end{tabular}
    \caption{Registro contable de la concesión del crédito}
    \label{tabla:asiento1ej4}
\end{table}
\c{En el anterior asiento debemos de tener cuidado, ya que tenemos pagos que realizar a c/p y otros a l/p.}

Ahora debemos de reconocer la primera cuota, pero esta está compuesta por dos operaciones, que son la amortización del capital y los intereses devengados.

\begin{table}[H]
    \centering
    \begin{tabular}{|p{3cm}|p{6cm}|p{3cm}|}
    \hline
    \rowcolor{blue!30}
    \textbf{DEBE} & \textbf{Valoración posterior a coste amortizado el 31.12.2018} & \textbf{HABER} \\
    \hline
    374.110 & \cuenta{572} & \\
    \hline
    &  \cuenta{76203}& 55.187\\
    \hline
    &  \cuenta{542}& 318.923\\
    \hline
    \end{tabular}
    \label{tabla:asiento2ej4}
\end{table}

\c{Al final de este ejercicio debemos de reclasificar el cŕedito a largo plazo, que vence el año siguiente.}

\begin{table}[H]
    \centering
    \begin{tabular}{|p{3cm}|p{6cm}|p{3cm}|}
    \hline
    \rowcolor{blue!30}
    \textbf{DEBE} & \textbf{Reclasificaciónn el 31.12.2018} & \textbf{HABER} \\
    \hline
    336.349 &  \cuenta{542}& \\
    \hline
    &  \cuenta{252}& 336.349\\
    \hline
    \end{tabular}
    \label{tabla:asiento3ej4}
\end{table}

\c{Y así debemos de continuar con el resto de cuotas.}

\subsection*{\textcolor{blue}{Ejemplo 5}}

La entidad A concede un crédito el 1-2-2018 a la entidad B con las siguientes características:
\begin{itemize}
    \item Nominal: 1.000.000 €.
    \item Comisión de apertura (a cargo de la empresa A): 10.000 €.
    \item Tipo de interés nominal: 6\%.
    \item Plazo: 3 años.
    \item Cuota anual constante: 374.109,81 €.
\end{itemize}

\textbf{SE PIDE:} Contabilizar la operación en la empresa A.

\textbf{Solución.} En su resolución obviamos todos los comentarios que son similares a los del ejemplo 4 y nos centramos en las diferencias, principalmente, el cambio en la fecha que aparece en el cuadro de amortización.

\begin{table}[h!]
\centering
\begin{tabular}{|c|c|c|c|c|}
\hline
\textbf{Plazo} & \textbf{Cuota} & \textbf{Capital} & \textbf{Intereses} & \textbf{Coste amortizado} \\ \hline
01-02-2018 &  &  &  & 1.010.000 \\ \hline
01-02-2019 & 374.110 & 318.923 & 55.187 & 691.077 \\ \hline
01-02-2020 & 374.110 & 336.349 & 37.761 & 354.727 \\ \hline
01-02-2021 & 374.110 & 354.728 & 19.382 & 0 \\ \hline
\end{tabular}
\caption{Cuadro de amortización a coste efectivo 5,46\%}
\label{tabla:amortizacion2}
\end{table}

La contabilización de la concesión es idéntica a la figura \ref{tabla:asiento1ej4}, pero la contabilización de la primera cuota es diferente, ya que la fecha de vencimiento es diferente.

A 31.12.Año debemos de reclasificar y calcular los intereses devengados.

\begin{equation*}
    \text{Intereses devengados} = \left[\text{1.010.000} \times (1+0,055)^{\frac{11}{12}}\right] - \text{1.010.000} = \text{50.474,70 €}
\end{equation*}

\begin{table}[H]
    \centering
    \begin{tabular}{|p{3cm}|p{6cm}|p{3cm}|}
    \hline
    \rowcolor{blue!30}
    \textbf{DEBE} & \textbf{Intereses a 31.12.2018} & \textbf{HABER} \\
    \hline
    50.474,70&  \cuenta{547}& \\
    \hline
    &  \cuenta{76203}& 50.474,70\\
    \hline
    \end{tabular}
    \label{tabla:asiento1ej5}
\end{table}

\c{Ahora debemos de contabilizar el mes de enero y el cobro  de la cuenta, podemos hacerlo en uno o en dos asientos.}

\begin{align*}
    \text{Intereses devengados} = \left[\text{(1.010.000+ 50.474,70)} \times (1+0,055)^{\frac{1}{12}}\right] - \text{(1.010.000+50.474,70)} = \\ \text{50.474,70 €}
\end{align*}

\begin{table}[H]
    \centering
    \begin{tabular}{|p{3cm}|p{6cm}|p{3cm}|}
    \hline
    \rowcolor{blue!30}
    \textbf{DEBE} & \textbf{Contabilización de los intereses devengados a 01-02-2019} & \textbf{HABER} \\
    \hline
    4.711,84 & (547) Intereses a corto plazo de créditos & \\
    \hline
    & (76203) Ingresos de créditos a largo plazo, otras empresas & 4.711,84 \\
    \hline
    \end{tabular}
    \caption{Asiento 2. Ejercicio 5.}
    \label{tabla:asiento1ej1}
\end{table}

\begin{table}[H]
    \centering
    \begin{tabular}{|p{3cm}|p{6cm}|p{3cm}|}
    \hline
    \rowcolor{blue!30}
    \textbf{DEBE} & \textbf{Contabilización del cobro de la primera cuota a 01-02-2019} & \textbf{HABER} \\
    \hline
    374.110 & (572) Bancos c/c & \\
    \hline
    & (547) Intereses a corto plazo de créditos & 55.187 \\
    \hline
    & (542) Créditos a corto plazo & 318.923 \\
    \hline
    \end{tabular}
    \caption{Asiento 3. Ejercicio 5.}
    \label{tabla:asiento2ej1}
\end{table}

\c{Y así debemos de continuar con el resto de cuotas.}

Una vez visto los ejemplos anteriores, debemos de tener en cuenta que al final del ejercicio debemos de realizar un ajuste por deterioro de valor, que se calcula en base a la diferencia entre el valor en libros y el valor razonable del activo.

En el caso de la existencia de un deterioro contabilizado anteriormente, debemos de revertirlo, si se ha producido una mejora en la situación del activo, es lo que se conoce como \textit{reversión del deterioro de valor}.

\subsection*{\textcolor{blue}{Ejemplo 6}}
El 1 de octubre de 2021, "BART, S.A." vendió 50.000 € en mercancías a "VERT, S.A." que se cobrarán el 1 de octubre de 2022. En el contrato de la operación de venta se incluye un tipo de interés del 3\% anual, por lo que a la fecha de vencimiento se debe desembolsar el importe de 51.500 €.\\

A 31 de diciembre de 2021 se tiene información sobre el riesgo de insolvencia de VERT donde se considera que a la fecha de vencimiento es altamente probable que se cobre el 90\% de la valoración posterior del crédito.\\

\textbf{SE PIDE:} Realizar las anotaciones contables de las operaciones anteriores del ejercicio 2021.\\

\c{El objetivo de este ejercicio es ver como se contabiliza un deterioro de valor.}

\begin{align*}
    \text{Coste amortizado derecho de cobro a 31.12.2021} = \\
    50000(1,03)^{\frac{92}{365}} = 50.37391 €
\end{align*}

En el enunciado se nos dice que solo cobraremos el 90\% de la valoración posterior, por lo que el deterioro de valor es de 50.373,91 - 45.336,52 = 5.037,39 €.

\begin{table}[H]
    \centering
    \begin{tabular}{|p{3cm}|p{6cm}|p{3cm}|}
    \hline
    \rowcolor{blue!30}
    \textbf{DEBE} & \textbf{} & \textbf{HABER} \\
    \hline
    50.000 & (430) Clientes &  \\
    \hline
     & (700) Venta de mercaderías & 50.000 \\
    \hline
    \end{tabular}
    \caption{Asiento 1. Ejercicio 6. Registro contable de la venta a 1 de octubre de 2021.}
    \label{tabla:asiento1ej6}
\end{table}

\begin{table}[H]
    \centering
    \begin{tabular}{|p{3cm}|p{6cm}|p{3cm}|}
    \hline
    \rowcolor{blue!30}
    \textbf{DEBE} & \textbf{} & \textbf{HABER} \\
    \hline
    373,91 & (430) Clientes &  \\
    \hline
     & (76213) Ingresos de créditos a corto plazo, otras empresas & 373,91 \\
    \hline
    \end{tabular}
    \caption{Asiento 2. Ejercicio 6. Registro contable del coste amortizado a 31/12/2021.}
    \label{tabla:asiento2ej6}
\end{table}

\begin{table}[H]
    \centering
    \begin{tabular}{|p{3cm}|p{6cm}|p{3cm}|}
    \hline
    \rowcolor{blue!30}
    \textbf{DEBE} & \textbf{} & \textbf{HABER} \\
    \hline
    5.037,39 & (694) Pérdidas por deterioro de créditos por operaciones comerciales &  \\
    \hline
     & (490) Deterioro de valor de créditos comerciales & 5.037,39 \\
    \hline
    \end{tabular}
    \caption{Asiento 3. Ejercicio 6. Registro contable del deterioro a 31/12/2021.}
    \label{tabla:asiento3ej6}
\end{table}

\subsection*{\textcolor{blue}{Ejemplo 7}}

La entidad A adquiere un bono el 1-1-2018 con la intención de mantenerlo hasta vencimiento hasta dentro de 3 años. Las características son las siguientes:
\begin{itemize}
    \item Valor de emisión: 48.000 €.
    \item Comisión de adquisición: 1.000 €.
    \item Valor de reembolso: 52.000 €.
    \item Intereses: 5\% anual sobre valor nominal (VN= 50.000 €).
\end{itemize}

A 31 de diciembre de 2018 y 2019 los títulos cotizan al 95\% y al 105\%, respectivamente, valor que la empresa considera como lo suficientemente fiable como para considerarlo representativo del valor que pudiera recuperar la empresa.

\subsubsection*{Solución}

Podemos definir bono como \textit{instrumento de deuda que emite una empresa o entidad pública para financiarse.} El emisor de un bono promete devolver el dinero prestado al comprador de ese bono, normalmente más unos intereses, conocidos como cupón, por eso se conoce como instrumento de renta fija.

Como se produce la adquisición por parte de la empresa A del bono, se atiende al modelo de negocio. Y el hecho de mantenerlo hasta el vencimiento de 3 años, hace que cumpla con el SCPI. Por lo que se clasifica como activo a \c{coste amortizado}.


\c{Tenemos que calcular el VR al inicio y el valor posterior a coste amortizado del activo, siendo necesario para este último el cálculo del tipo de interés efectivo.}

\begin{align*}
    \text{49.000} \e = \\ 
    \frac{\text{2.500}}{(1+i)} + \frac{\text{2.500}}{(1+i)^2} + \frac{\text{2.500}}{(1+i)^3} \rightarrow \\
    \rightarrow \text{i efectivo} = 7,006\%
\end{align*}

A partir de este i, debemos de calcular el cuadro de amortización\footnote{Cade destacar que se proporciona en el examen, pero debemos de saber el cálculo de cada una de las celdas.}.

\begin{tcolorbox}[colback=yellow!5!white,colframe=yellow!75!black, title=Cálculo de intereses]
    \begin{itemize}
        \item Devengados: Calculados aplicando el tipo de interés efectivo.
        \item Explícitos: referencia al cupón que paga la empresa emisora.
        \item Implícitos: diferencia entre el los explícitos y los devengados.
    \end{itemize}
\end{tcolorbox}

\begin{table}[H]
    \centering
    \begin{tabular}{|p{2cm}|p{2cm}|p{2cm}|p{2cm}|p{2cm}|}
    \hline
    \rowcolor{blue!30}
    \textbf{Plazo} & \textbf{I. Devengados} & \textbf{I. Explícitos y Reembolso} & \textbf{I. Implícitos} & \textbf{Coste Amortizado} \\
    \hline
    01/01/2018 & & & & 49.000 \\
    \hline
    31/12/2018 & 3.433 & 2.500 & 933 & 49.993 \\
    \hline
    31/12/2019 & 3.498 & 2.500 & 998 & 50.932 \\
    \hline
    31/12/2020 & 3.568,29 & 54.500 & 1.068,29 & 0 \\
    \hline
    \end{tabular}
    \caption{Cuadro de amortización a coste efectivo (7,0006\%).}
    \label{tabla:ejercicio7-asiento1-tema2}
\end{table}

\begin{table}[H]
    \centering
    \begin{tabular}{|p{3cm}|p{6cm}|p{3cm}|}
    \hline
    \rowcolor{blue!30}
    \textbf{DEBE} & \textbf{Suscripción de loos títulos el 1.1.2018} & \textbf{HABER} \\
    \hline
    49.000&  \cuenta{251}& \\
    \hline
    & \cuenta{572} &49.000 \\
    \hline
    &  & \\
    \hline
    \end{tabular}
    \caption{Asiento 2. Ejercicio 7.}
    \label{tabla:asiento2Ej7T2}
\end{table}

\begin{table}[H]
    \centering
    \begin{tabular}{|p{3cm}|p{6cm}|p{3cm}|}
    \hline
    \rowcolor{blue!30}
    \textbf{DEBE} & \textbf{Valoración posterior a coste amortizado el 31.12.2018} & \textbf{HABER} \\
    \hline
    2.500&  \cuenta{572}& \\
    \hline
    & \cuenta{7613} & 2.500\\
    \hline
    \end{tabular}
    \caption{Asiento 3. Ejercicio 7.}
    \label{tabla:asiento3ej7T2}
\end{table}

\begin{table}[H]
    \centering
    \begin{tabular}{|p{3cm}|p{6cm}|p{3cm}|}
    \hline
    \rowcolor{blue!30}
    \textbf{DEBE} & \textbf{Valoración posterior a coste amortizado 31.12.2018} & \textbf{HABER} \\
    \hline
    933& \cuenta{251} & \\
    \hline
    &  \cuenta{7613}& 933\\
    \hline
    \end{tabular}
    \caption{Asiento 4. Ejercicio 7.}
    \label{tabla:asiento4ej7T2}
\end{table}

Llegado a esta fecha debemos de comprobar si hay o no deterioro, una de las formas es comparar si el valor contable coincide con los flujos de caja esperados (VAFE).

\begin{align*}
    \text{Flujos de caja esperados a recibir (bono) con ti efectio} = \\
    \text{49.933,39} \e = \\
    \frac{2500}{1,07006} + \frac{54500}{1,07006^2}
\end{align*}

Siguiendo este criterio podemos ver que no hay deterioro.

La NV9 establece que \c{como sustituto del valor actual de los flujos de efectivo futuros, se puede utilizar el valor de mercado del instrumento.}

En el enunciado se nos expone que a 31 de diciembre, los títulos cotizan al 95\% y 105\%, respectivamente.

%pag 34 del pdf














\newpage
\section{Baja de Activos financieros}

Se dará de baja a un activo financiero cuando expiren o se hayan cedido los derechos contractuales sobres los flujos de efectivo del mismo, siendo necesario que se hayan transferido de manera sustancial los riesgos y beneficios inherentes a su propiedad. 











Consideramos desinversión a la venta de algún activo. 
\newpage
\section{Memoria}

Debemos de indicar los criterios que estoy aplicando en la calificación y la valoración de las diferentes catergorías de los activos financieros. 
Además, debemos de tener en cuenta las reclasificaciones de activos, compensación de AF y PF, correcciones por deterioro de valor originadas por el riesgo de crédito y por último la información de riesgos(crédito, liquidez y mercado, este último comprende el riesgo de los tipos de cambio, ti y otros tipos de riesgos relacionados con el precio).


