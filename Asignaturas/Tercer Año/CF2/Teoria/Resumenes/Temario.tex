%%
% Copyright (c) 2017 - 2025, Pascal Wagler;
% Copyright (c) 2014 - 2025, John MacFarlane
%
% All rights reserved.
%
% Redistribution and use in source and binary forms, with or without
% modification, are permitted provided that the following conditions
% are met:
%
% - Redistributions of source code must retain the above copyright
% notice, this list of conditions and the following disclaimer.
%
% - Redistributions in binary form must reproduce the above copyright
% notice, this list of conditions and the following disclaimer in the
% documentation and/or other materials provided with the distribution.
%
% - Neither the name of John MacFarlane nor the names of other
% contributors may be used to endorse or promote products derived
% from this software without specific prior written permission.
%
% THIS SOFTWARE IS PROVIDED BY THE COPYRIGHT HOLDERS AND CONTRIBUTORS
% "AS IS" AND ANY EXPRESS OR IMPLIED WARRANTIES, INCLUDING, BUT NOT
% LIMITED TO, THE IMPLIED WARRANTIES OF MERCHANTABILITY AND FITNESS
% FOR A PARTICULAR PURPOSE ARE DISCLAIMED. IN NO EVENT SHALL THE
% COPYRIGHT OWNER OR CONTRIBUTORS BE LIABLE FOR ANY DIRECT, INDIRECT,
% INCIDENTAL, SPECIAL, EXEMPLARY, OR CONSEQUENTIAL DAMAGES (INCLUDING,
% BUT NOT LIMITED TO, PROCUREMENT OF SUBSTITUTE GOODS OR SERVICES;
% LOSS OF USE, DATA, OR PROFITS; OR BUSINESS INTERRUPTION) HOWEVER
% CAUSED AND ON ANY THEORY OF LIABILITY, WHETHER IN CONTRACT, STRICT
% LIABILITY, OR TORT (INCLUDING NEGLIGENCE OR OTHERWISE) ARISING IN
% ANY WAY OUT OF THE USE OF THIS SOFTWARE, EVEN IF ADVISED OF THE
% POSSIBILITY OF SUCH DAMAGE.
%%

%%
% This is the Eisvogel pandoc LaTeX template.
%
% For usage information and examples visit the official GitHub page:
% https://github.com/Wandmalfarbe/pandoc-latex-template
%%
% Options for packages loaded elsewhere
\PassOptionsToPackage{unicode}{hyperref}
\PassOptionsToPackage{hyphens}{url}
\PassOptionsToPackage{dvipsnames,svgnames,x11names,table}{xcolor}
\documentclass[
  paper=a4,
  ,captions=tableheading
]{scrbook}
\usepackage{xcolor}
\usepackage[margin=2.5cm]{geometry}
\usepackage{amsmath,amssymb}

\usepackage[export]{adjustbox}
\usepackage{graphicx}

% add backlinks to footnote references, cf. https://tex.stackexchange.com/questions/302266/make-footnote-clickable-both-ways
\usepackage{footnotebackref}
\setcounter{secnumdepth}{5}
\usepackage{iftex}
\ifPDFTeX
  \usepackage[T1]{fontenc}
  \usepackage[utf8]{inputenc}
  \usepackage{textcomp} % provide euro and other symbols
\else % if luatex or xetex
  \usepackage{unicode-math} % this also loads fontspec
  \defaultfontfeatures{Scale=MatchLowercase}
  \defaultfontfeatures[\rmfamily]{Ligatures=TeX,Scale=1}
\fi
\usepackage{lmodern}
\ifPDFTeX\else
  % xetex/luatex font selection
  \setmainfont[]{Latin Modern Roman}
\fi
% Use upquote if available, for straight quotes in verbatim environments
\IfFileExists{upquote.sty}{\usepackage{upquote}}{}
\IfFileExists{microtype.sty}{% use microtype if available
  \usepackage[]{microtype}
  \UseMicrotypeSet[protrusion]{basicmath} % disable protrusion for tt fonts
}{}

% Use setspace anyway because we change the default line spacing.
% The spacing is changed early to affect the titlepage and the TOC.
\usepackage{setspace}
\setstretch{1.2}
\makeatletter
\@ifundefined{KOMAClassName}{% if non-KOMA class
  \IfFileExists{parskip.sty}{%
    \usepackage{parskip}
  }{% else
    \setlength{\parindent}{0pt}
    \setlength{\parskip}{6pt plus 2pt minus 1pt}}
}{% if KOMA class
  \KOMAoptions{parskip=half}}
\makeatother
\usepackage{longtable,booktabs,array}
\usepackage{calc} % for calculating minipage widths
% Correct order of tables after \paragraph or \subparagraph
\usepackage{etoolbox}
\makeatletter
\patchcmd\longtable{\par}{\if@noskipsec\mbox{}\fi\par}{}{}
\makeatother
% Allow footnotes in longtable head/foot
\IfFileExists{footnotehyper.sty}{\usepackage{footnotehyper}}{\usepackage{footnote}}
\makesavenoteenv{longtable}
\ifLuaTeX
\usepackage[bidi=basic,shorthands=off,]{babel}
\else
\usepackage[bidi=default,shorthands=off,]{babel}
\fi
\ifLuaTeX
  \usepackage{selnolig} % disable illegal ligatures
\fi
\setlength{\emergencystretch}{3em} % prevent overfull lines
\providecommand{\tightlist}{%
  \setlength{\itemsep}{0pt}\setlength{\parskip}{0pt}}
\usepackage{bookmark}
\IfFileExists{xurl.sty}{\usepackage{xurl}}{} % add URL line breaks if available
\urlstyle{same}
\definecolor{default-linkcolor}{HTML}{A50000}
\definecolor{default-filecolor}{HTML}{A50000}
\definecolor{default-citecolor}{HTML}{4077C0}
\definecolor{default-urlcolor}{HTML}{4077C0}

\hypersetup{
  pdftitle={Contabilidad Financiera II},
  pdfauthor={Ismael Sallami Moreno},
  pdflang={esge},
  colorlinks=true,
  linkcolor={008080},
  filecolor={default-filecolor},
  citecolor={default-citecolor},
  urlcolor={default-urlcolor},
  breaklinks=true,
  pdfcreator={LaTeX via pandoc with the Eisvogel template}}

\title{Contabilidad Financiera II}
\usepackage{etoolbox}
\makeatletter
\providecommand{\subtitle}[1]{% add subtitle to \maketitle
  \apptocmd{\@title}{\par {\large #1 \par}}{}{}
}
\makeatother
\subtitle{Temario}
\author{Ismael Sallami Moreno}
\date{Mayo de 2025}


%
% for the background color of the title page
%
\usepackage{pagecolor}
\usepackage{afterpage}

%
% break urls
%
\PassOptionsToPackage{hyphens}{url}

%
% When using babel or polyglossia with biblatex, loading csquotes is recommended
% to ensure that quoted texts are typeset according to the rules of your main language.
%
\usepackage{csquotes}

%
% captions
%
\definecolor{caption-color}{HTML}{777777}
\usepackage[font={stretch=1.2}, textfont={color=caption-color}, position=top, skip=4mm, labelfont=bf, singlelinecheck=false, justification=raggedright]{caption}
\setcapindent{0em}

%
% blockquote
%
\definecolor{blockquote-border}{RGB}{221,221,221}
\definecolor{blockquote-text}{RGB}{119,119,119}
\usepackage{mdframed}
\newmdenv[rightline=false,bottomline=false,topline=false,linewidth=3pt,linecolor=blockquote-border,skipabove=\parskip]{customblockquote}
\renewenvironment{quote}{\begin{customblockquote}\list{}{\rightmargin=0em\leftmargin=0em}%
\item\relax\color{blockquote-text}\ignorespaces}{\unskip\unskip\endlist\end{customblockquote}}

%
% Source Sans Pro as the default font family
% Source Code Pro for monospace text
%
% 'default' option sets the default
% font family to Source Sans Pro, not \sfdefault.
%
\ifnum 0\ifxetex 1\fi\ifluatex 1\fi=0 % if pdftex
    \usepackage[default]{sourcesanspro}
  \usepackage{sourcecodepro}
  \else % if not pdftex
    \fi

%
% heading color
%
\definecolor{heading-color}{RGB}{40,40,40}
\addtokomafont{section}{\color{heading-color}}
% When using the classes report, scrreprt, book,
% scrbook or memoir, uncomment the following line.
%\addtokomafont{chapter}{\color{heading-color}}

%
% variables for title, author and date
%
\usepackage{titling}
\title{Contabilidad Financiera II}
\author{Ismael Sallami Moreno}
\date{Mayo de 2025}

%
% tables
%

\definecolor{table-row-color}{HTML}{F5F5F5}
\definecolor{table-rule-color}{HTML}{999999}

%\arrayrulecolor{black!40}
\arrayrulecolor{table-rule-color}     % color of \toprule, \midrule, \bottomrule
\setlength\heavyrulewidth{0.3ex}      % thickness of \toprule, \bottomrule
\renewcommand{\arraystretch}{1.3}     % spacing (padding)


%
% remove paragraph indentation
%
\setlength{\parindent}{0pt}
\setlength{\parskip}{6pt plus 2pt minus 1pt}
\setlength{\emergencystretch}{3em}  % prevent overfull lines

%
%
% Listings
%
%


%
% header and footer
%
\usepackage[headsepline,footsepline]{scrlayer-scrpage}

\newpairofpagestyles{eisvogel-header-footer}{
  \clearpairofpagestyles
  \ihead*{Contabilidad Financiera II}
  \chead*{}
  \ohead*{\small Ingeniería Informática + ADE}
  \ifoot*{Ismael Sallami Moreno}
  \cfoot*{}
  \ofoot*{\thepage}
  \addtokomafont{pageheadfoot}{\upshape}
}
\pagestyle{eisvogel-header-footer}

\deftripstyle{ChapterStyle}{}{}{}{}{\pagemark}{}
\renewcommand*{\chapterpagestyle}{ChapterStyle}


%
% Define watermark
%

\begin{document}

\begin{titlepage}
\newgeometry{left=6cm}
\newcommand{\colorRule}[3][black]{\textcolor[HTML]{#1}{\rule{#2}{#3}}}
\begin{flushleft}
\noindent
\\[-1em]
\color[HTML]{333333}
\makebox[0pt][l]{\colorRule[4682B4]{1.3\textwidth}{4pt}}
\par
\noindent

{
  \setstretch{1.4}
  \vfill
  \noindent {\huge \textbf{\textsf{Contabilidad Financiera II}}}
    \vskip 1em
  {\Large \textsf{Temario}}
    \vskip 2em
  \noindent {\Large \textsf{Ismael Sallami Moreno}}
  \vfill
}

\noindent
\includegraphics[width=35mm, left]{../../Practica/EjerciciosMD/images/logo-universidad.jpg}

\textsf{Mayo de 2025}
\end{flushleft}
\end{titlepage}
\restoregeometry
\pagenumbering{arabic}

\frontmatter
% don't generate the default title
% \maketitle


{
\hypersetup{linkcolor=}
\setcounter{tocdepth}{1}
\tableofcontents
}
\mainmatter
\hypertarget{fondos-propios}{%
\chapter{Fondos Propios}\label{fondos-propios}}

\hypertarget{elementos-de-patrimonio-neto-concepto-y-caracteruxedsticas}{%
\section{Elementos de Patrimonio Neto: Concepto y
Características}\label{elementos-de-patrimonio-neto-concepto-y-caracteruxedsticas}}

\begin{itemize}
\item
  \textbf{Perspectiva de las Normas Internacionales de Contabilidad
  (IASB):} El Marco Conceptual del IASB (IASB, 2018) no define el
  Patrimonio como tal, sino que lo delimita de forma indirecta,
  indicando que ``Patrimonio es la parte residual de los activos de la
  entidad, una vez deducidos todos sus pasivos''. A la hora de medirlo,
  tampoco establece un criterio específico, sino que, conceptualmente,
  ``es igual al total del importe en libros de todos los activos
  reconocidos menos el total de los importes en libros de todos los
  pasivos reconocidos''.
\item
  \textbf{Perspectiva del Plan General de Contabilidad (PGC):} El PGC
  también asume que el Patrimonio Neto es la parte residual al deducir
  el valor de los pasivos del valor de los activos. Sin embargo, el PGC
  añade que el Patrimonio Neto ``incluye las aportaciones realizadas, ya
  sea en el momento de su constitución o en otros posteriores, por sus
  socios o propietarios, que no tengan la consideración de pasivos, así
  como los resultados acumulados u otras variaciones que le afecten''.
\item
  \textbf{Componentes Clave destacados por el PGC:} La definición del
  PGC, influenciada por la normativa mercantil española, introduce la
  figura de socios y propietarios de esos derechos residuales sobre el
  activo. Delimita el concepto de Capital (vinculado a las aportaciones
  de socios) y el de Reservas (resultados acumulados que permanecen en
  la empresa), que conforman dos de los principales bloques del
  patrimonio, estructurados en los subgrupos 10 (Capital), 11 (Reservas)
  y 12 (Resultados).
\item
  \textbf{Distinción entre Instrumento de Patrimonio y Pasivo
  Financiero:} Para referirse al patrimonio, es necesario distinguir
  correctamente un ``instrumento de patrimonio'' de un ``pasivo
  financiero''. Se retoma la norma de registro y valoración (NRV) 9ª,
  que establece que ``un instrumento financiero es un contrato que da
  lugar a un activo financiero en una empresa y, simultáneamente, a un
  pasivo financiero o a un instrumento de patrimonio en otra empresa''.
  Un pasivo financiero supone una obligación de pago, una deuda futura
  para la empresa, mientras que un instrumento de patrimonio representa
  una participación en los activos de la empresa.
\item
  \textbf{Características de los Instrumentos de Patrimonio (NRV 9.4):}

  \begin{itemize}
  \tightlist
  \item
    Son un negocio jurídico que evidencia/refleja la participación
    residual en los activos, una vez deducidos los pasivos.
  \item
    Si son adquiridos por la propia empresa, nunca serán un activo, sino
    que minorarán los Fondos Propios.
  \item
    Los gastos derivados de su emisión son parte del Patrimonio,
    contabilizándose como reservas, salvo cuando se haya desistido de su
    emisión, que se contabilizan en la cuenta de Pérdidas y Ganancias.
  \end{itemize}
\end{itemize}

\hypertarget{capital-y-reservas}{%
\section{Capital y Reservas}\label{capital-y-reservas}}

Esta sección del capítulo se adentra en dos componentes fundamentales de
los fondos propios: el capital y las reservas, recurriendo a una
perspectiva tanto contable como jurídica.

\hypertarget{capital-concepto-y-contabilizaciuxf3n}{%
\subsection{Capital: Concepto y
Contabilización}\label{capital-concepto-y-contabilizaciuxf3n}}

El Marco Conceptual del PGC define el Capital como la financiación
procedente de las aportaciones realizadas por los propietarios, ya sea
en la constitución de la empresa o posteriormente. Desde una óptica
jurídica, la cifra de capital es una garantía para los acreedores, ya
que se espera que esté materializada en activos. La legislación
mercantil busca mantener la integridad del capital, obligando, por
ejemplo, a ``reservar'' parte del resultado.

Las cuentas que conforman el subgrupo 10 «Capital» son:

\begin{itemize}
\tightlist
\item
  \textbf{100. Capital social:} Para sociedades mercantiles (anónima,
  limitada, comanditaria por acciones). Refleja el capital formalmente
  escriturado.
\item
  \textbf{101. Fondo social:} Para entidades sin forma mercantil, como
  cooperativas.
\item
  \textbf{102. Capital:} Para empresas individuales. Al final del
  ejercicio, los resultados se integran directamente en esta cuenta. La
  cuenta (550) «Titular de la explotación» recoge movimientos entre el
  empresario y la empresa durante el ejercicio, trasladando su saldo a
  la (102) al cierre.
\item
  \textbf{103. Socios por desembolsos no exigidos:} Deudas de socios por
  la parte del capital suscrito pendiente de pago.
\item
  \textbf{104. Socios por aportaciones no dinerarias pendientes:}
  Similar a la 103, pero para aportaciones en especie.
\item
  \textbf{108. Acciones o participaciones propias en situaciones
  especiales.}
\item
  \textbf{109. Acciones o participaciones propias para reducción de
  capital.}

  \begin{itemize}
  \tightlist
  \item
    Las cuentas 108 y 109 se refieren a situaciones especiales donde la
    empresa adquiere sus propias acciones, minorando el Patrimonio Neto.
  \end{itemize}
\end{itemize}

La contabilización del \textbf{Capital Social (cuenta 100)} en
sociedades mercantiles se detalla a través del proceso de
\textbf{constitución de la empresa}, centrándose en la \textbf{fundación
simultánea} (creación en un único acto por acuerdo de los socios
fundadores). Los pasos contables son:

\begin{enumerate}
\def\labelenumi{\arabic{enumi}.}
\tightlist
\item
  \textbf{Emisión de las acciones:} Se utiliza el subgrupo 19
  «Situaciones transitorias de financiación» antes de la inscripción
  registral.

  \begin{itemize}
  \tightlist
  \item
    Asiento: (190) Acciones emitidas a (194) Capital emitido pendiente
    de inscripción. El importe incluye el nominal y, si la hay, la prima
    de emisión.
  \end{itemize}
\item
  \textbf{Suscripción de las acciones:} Implica un compromiso de
  aportación de los socios.

  \begin{itemize}
  \tightlist
  \item
    Asiento: (1034) Socios por desembolsos no exigidos, capital
    pendiente de inscripción y/o (1044) Socios por aportaciones no
    dinerarias pendientes, capital pendiente de inscripción a (190)
    Acciones emitidas.
  \end{itemize}
\item
  \textbf{Desembolso de las acciones:} La ley exige un desembolso mínimo
  (al menos el 25\% del valor nominal para S.A., y el 100\% de la prima
  de emisión si existe).

  \begin{itemize}
  \tightlist
  \item
    Asiento: (57) Tesorería a (1034) Socios por desembolsos no exigidos,
    capital pendiente de inscripción.
  \item
    Los desembolsos parciales posteriores se conocen como ``dividendo
    pasivo''. Su exigencia se contabiliza: (558) Socios por desembolsos
    exigidos a (103) Socios por desembolsos no exigidos.
  \end{itemize}
\item
  \textbf{Inscripción de la sociedad en el Registro Mercantil:} La
  empresa adquiere personalidad jurídica.

  \begin{itemize}
  \tightlist
  \item
    Asiento 1 (Capital): (194) Capital emitido pendiente de inscripción
    a (100) Capital Social y, si procede, a (110) Prima de Emisión.
  \item
    Asiento 2 (Socios): (1030) Socios por desembolsos no exigidos y/o
    (1040) Socios por aportaciones no dinerarias pendientes a (1034) y/o
    (1044) por la parte del capital suscrito no desembolsado que ahora
    corresponde a una sociedad inscrita.
  \end{itemize}
\item
  \textbf{Gastos de constitución:} Se contabilizan directamente contra
  el Patrimonio Neto, cargando una cuenta de reservas (generalmente
  (113) Reservas voluntarias) con abono a tesorería o la deuda
  correspondiente.
\end{enumerate}

El proceso es similar para las ampliaciones de capital.

\hypertarget{reservas-concepto-y-tipologuxeda}{%
\subsection{Reservas: Concepto y
Tipología}\label{reservas-concepto-y-tipologuxeda}}

Las Reservas son los beneficios obtenidos por la empresa que no han sido
distribuidos entre sus propietarios. Se clasifican según su origen:

\begin{enumerate}
\def\labelenumi{\arabic{enumi}.}
\tightlist
\item
  \textbf{De Autofinanciación (o autofinanciación de enriquecimiento):}
  Derivadas del resultado positivo del ejercicio que los propietarios
  deciden mantener en la empresa.

  \begin{itemize}
  \tightlist
  \item
    \textbf{(112) Reserva legal:} Obligatoria por ley (art. 274 RDL
    1/2010). Se destina un 10\% del beneficio del ejercicio hasta que
    esta reserva alcance el 20\% del capital social. Es, salvo casos
    especiales, indisponible.
  \item
    \textbf{(1141) Reservas estatutarias:} Establecidas en los estatutos
    de la empresa.
  \item
    \textbf{(113) Reservas voluntarias:} Constituidas libremente por la
    empresa y de libre disposición. También recogen los gastos de
    constitución o ajustes por cambios de criterio contable o errores.
  \end{itemize}
\item
  \textbf{Procedentes de Ampliaciones de Capital:}

  \begin{itemize}
  \tightlist
  \item
    \textbf{(110) Prima de emisión de acciones:} Es el sobreprecio
    pagado por las acciones sobre su valor nominal. Se justifica por la
    riqueza acumulada previamente por la empresa (reservas
    preexistentes) o buenas expectativas futuras. La normativa española
    prohíbe emisiones ``por debajo de la par''. La prima de emisión debe
    desembolsarse en su totalidad en el momento inicial.
  \item
    \textbf{(111) Otros instrumentos de patrimonio neto.}
  \end{itemize}
\item
  \textbf{Para el Mantenimiento Efectivo del Capital:} Cuentas de la
  subdivisión (114) «Reservas especiales» que buscan asegurar que la
  cifra de Capital sea representativa de una garantía frente a terceros
  (ej. reservas por acciones propias, acciones de la sociedad dominante,
  o por capital amortizado).
\item
  \textbf{Por Ajuste en el Capital:}

  \begin{itemize}
  \tightlist
  \item
    \textbf{(119) Diferencias por ajustes del capital en euros:}
    Derivada de los ajustes por redondeo en el traspaso de pesetas a
    euros.
  \end{itemize}
\item
  \textbf{Por la Compensación de Pérdidas:}

  \begin{itemize}
  \tightlist
  \item
    \textbf{(118) Aportaciones de socios o propietarios:} Aportaciones
    de socios diferentes a la suscripción de capital, habitualmente para
    compensar pérdidas y que no aparezcan en el Balance.
  \end{itemize}
\end{enumerate}

\hypertarget{asientos-contables-relacionados}{%
\subsection{Asientos Contables
Relacionados}\label{asientos-contables-relacionados}}

\begin{enumerate}
\def\labelenumi{\arabic{enumi}.}
\item
  \textbf{Emisión de acciones:}

  \begin{longtable}[]{@{}
    >{\raggedright\arraybackslash}p{(\columnwidth - 2\tabcolsep) * \real{0.5000}}
    >{\raggedright\arraybackslash}p{(\columnwidth - 2\tabcolsep) * \real{0.5000}}@{}}
  \toprule\noalign{}
  \begin{minipage}[b]{\linewidth}\raggedright
  DEBE
  \end{minipage} & \begin{minipage}[b]{\linewidth}\raggedright
  HABER
  \end{minipage} \\
  \midrule\noalign{}
  \endhead
  \bottomrule\noalign{}
  \endlastfoot
  (190) Acciones emitidas & (194) Capital emitido pendiente de
  inscripción \\
  \end{longtable}
\item
  \textbf{Suscripción de acciones:}

  \begin{longtable}[]{@{}
    >{\raggedright\arraybackslash}p{(\columnwidth - 2\tabcolsep) * \real{0.5000}}
    >{\raggedright\arraybackslash}p{(\columnwidth - 2\tabcolsep) * \real{0.5000}}@{}}
  \toprule\noalign{}
  \begin{minipage}[b]{\linewidth}\raggedright
  DEBE
  \end{minipage} & \begin{minipage}[b]{\linewidth}\raggedright
  HABER
  \end{minipage} \\
  \midrule\noalign{}
  \endhead
  \bottomrule\noalign{}
  \endlastfoot
  (1034) Socios por desembolsos no & (190) Acciones emitidas \\
  exigidos, capital pendiente de & \\
  inscripción & \\
  (1044) Socios por aportaciones no & \\
  dinerarias pendientes, capital & \\
  pendiente de inscripción & \\
  \end{longtable}
\item
  \textbf{Desembolso de acciones:}

  \begin{longtable}[]{@{}
    >{\raggedright\arraybackslash}p{(\columnwidth - 2\tabcolsep) * \real{0.5000}}
    >{\raggedright\arraybackslash}p{(\columnwidth - 2\tabcolsep) * \real{0.5000}}@{}}
  \toprule\noalign{}
  \begin{minipage}[b]{\linewidth}\raggedright
  DEBE
  \end{minipage} & \begin{minipage}[b]{\linewidth}\raggedright
  HABER
  \end{minipage} \\
  \midrule\noalign{}
  \endhead
  \bottomrule\noalign{}
  \endlastfoot
  (57) Tesorería & (1034) Socios por desembolsos no exigidos,capital
  pendiente de inscripción \\
  \end{longtable}
\item
  \textbf{Exigencia de aportaciones pendientes:}

  \begin{longtable}[]{@{}
    >{\raggedright\arraybackslash}p{(\columnwidth - 2\tabcolsep) * \real{0.5000}}
    >{\raggedright\arraybackslash}p{(\columnwidth - 2\tabcolsep) * \real{0.5000}}@{}}
  \toprule\noalign{}
  \begin{minipage}[b]{\linewidth}\raggedright
  DEBE
  \end{minipage} & \begin{minipage}[b]{\linewidth}\raggedright
  HABER
  \end{minipage} \\
  \midrule\noalign{}
  \endhead
  \bottomrule\noalign{}
  \endlastfoot
  (558) Socios por desembolsos exigidos & (103) Socios por desembolsos
  no exigidos \\
  \end{longtable}
\item
  \textbf{Inscripción en el Registro Mercantil:}

  \begin{itemize}
  \item
    \textbf{Capital:}

    \begin{longtable}[]{@{}
      >{\raggedright\arraybackslash}p{(\columnwidth - 2\tabcolsep) * \real{0.4722}}
      >{\raggedright\arraybackslash}p{(\columnwidth - 2\tabcolsep) * \real{0.5278}}@{}}
    \toprule\noalign{}
    \begin{minipage}[b]{\linewidth}\raggedright
    DEBE
    \end{minipage} & \begin{minipage}[b]{\linewidth}\raggedright
    HABER
    \end{minipage} \\
    \midrule\noalign{}
    \endhead
    \bottomrule\noalign{}
    \endlastfoot
    (194) Capital emitido pendiente & (100) Capital Social \\
    de inscripción & (110) Prima de Emisión \\
    \end{longtable}
  \item
    \textbf{Socios:}

    \begin{longtable}[]{@{}
      >{\raggedright\arraybackslash}p{(\columnwidth - 2\tabcolsep) * \real{0.4722}}
      >{\raggedright\arraybackslash}p{(\columnwidth - 2\tabcolsep) * \real{0.5278}}@{}}
    \toprule\noalign{}
    \begin{minipage}[b]{\linewidth}\raggedright
    DEBE
    \end{minipage} & \begin{minipage}[b]{\linewidth}\raggedright
    HABER
    \end{minipage} \\
    \midrule\noalign{}
    \endhead
    \bottomrule\noalign{}
    \endlastfoot
    (1030) Socios por desembolsos no & (1034) Socios por desembolsos no
    exigidos, \\
    exigidos & capital pendiente de inscripción \\
    (1040) Socios por aportaciones no & (1044) Socios por aportaciones
    no dinerarias \\
    dinerarias pendientes & pendientes, capital pendiente de
    inscripción \\
    \end{longtable}
  \end{itemize}
\end{enumerate}

\hypertarget{otras-partidas-del-patrimonio-neto}{%
\section{Otras Partidas del Patrimonio
Neto}\label{otras-partidas-del-patrimonio-neto}}

Esta sección se centra en cómo se gestionan y aplican los resultados
económicos de la empresa, que son una parte crucial del patrimonio neto.
La idea fundamental es que los incrementos (ingresos) y decrementos
(gastos) del patrimonio se deben contabilizar para conocer la evolución
periódica del mismo.

\textbf{Resultados Pendientes de Aplicación (Subgrupo 12):}

Este subgrupo recoge el resultado económico que no se imputa
directamente al patrimonio y está pendiente de una decisión sobre su
destino. Las cuentas principales son:

\begin{itemize}
\tightlist
\item
  \textbf{129. «Resultados del ejercicio»:} Esta es la cuenta más
  representativa. Al final del ejercicio, aglutina todos los ingresos
  (grupo 7) y gastos (grupo 6) para determinar si ha habido beneficio
  (saldo acreedor) o pérdida (saldo deudor). Su saldo es transitorio
  hasta que la Junta General de Accionistas (o el órgano
  correspondiente) decide qué hacer con él.
\item
  \textbf{121. «Resultados negativos de ejercicios anteriores»:} Si el
  ejercicio arroja pérdidas, estas se traspasan de la cuenta (129) a la
  (121). Esta cuenta siempre tendrá saldo deudor y minora los fondos
  propios. Debe ser compensada con beneficios futuros.

  \begin{itemize}
  \tightlist
  \item
    \emph{Asiento de traspaso de pérdidas:} (121) Resultados negativos
    de ejercicios anteriores a (129) Resultado del ejercicio.
  \end{itemize}
\item
  \textbf{120. «Remanente»:} Son beneficios no repartidos ni aplicados
  específicamente a ninguna otra cuenta tras la aprobación de las
  cuentas anuales y la distribución de resultados. Quedan a disposición
  de los administradores para su uso futuro (convertirlos en reservas o
  repartirlos).
\end{itemize}

\textbf{Aplicación del Resultado del Ejercicio (cuando hay beneficios):}

Si la cuenta (129) presenta un beneficio, la empresa debe decidir su
destino. Las opciones principales son:

\begin{enumerate}
\def\labelenumi{\arabic{enumi}.}
\tightlist
\item
  \textbf{Compensar pérdidas anteriores:} Si la cuenta (121) tiene
  saldo, se utiliza parte del beneficio para reducirlo o eliminarlo.
\item
  \textbf{Dotar Reservas:} Se traspasa parte del beneficio a cuentas de
  reservas (subgrupo 11), como:

  \begin{itemize}
  \tightlist
  \item
    Reserva Legal (obligatoria por ley).
  \item
    Reservas Estatutarias (si así lo establecen los estatutos).
  \item
    Reservas Voluntarias.
  \end{itemize}
\item
  \textbf{Repartir Dividendos:} Se reconoce una deuda con los
  accionistas.

  \begin{itemize}
  \tightlist
  \item
    La cuenta \textbf{(526) «Dividendo activo a pagar»} recoge la
    obligación de pago desde que se aprueba el dividendo hasta que se
    paga.
  \item
    Si durante el ejercicio se han adelantado dividendos ``a cuenta''
    del beneficio previsible, estos se habrán contabilizado en la cuenta
    \textbf{(557) «Dividendo activo a cuenta»} (saldo deudor). Antes de
    determinar el dividendo final a pagar, se debe saldar esta cuenta
    (557).
  \end{itemize}
\item
  \textbf{Destinar a Remanente:} La parte del beneficio que no se aplica
  a los puntos anteriores puede quedar en la cuenta (120) «Remanente».
\end{enumerate}

El asiento contable para la aplicación del resultado del ejercicio
(cuando hay beneficios) típicamente sería:

\begin{longtable}[]{@{}
  >{\raggedright\arraybackslash}p{(\columnwidth - 2\tabcolsep) * \real{0.3908}}
  >{\raggedright\arraybackslash}p{(\columnwidth - 2\tabcolsep) * \real{0.6092}}@{}}
\toprule\noalign{}
\begin{minipage}[b]{\linewidth}\raggedright
DEBE
\end{minipage} & \begin{minipage}[b]{\linewidth}\raggedright
HABER
\end{minipage} \\
\midrule\noalign{}
\endhead
\bottomrule\noalign{}
\endlastfoot
(129) Resultado del ejercicio & a (121) Rdos negativos de ejercicios
anteriores (si hay) \\
& a (11X) Reservas (Legal, Estatutaria, Voluntarias, etc.) \\
& a (557) Dividendo activo a cuenta (para saldarlo) \\
& a (526) Dividendo activo a pagar \\
& a (120) Remanente \\
\end{longtable}

Todas estas cuentas del subgrupo 12 figuran en el patrimonio neto del
Balance, con signo positivo o negativo según corresponda.

\hypertarget{ajustes-por-cambio-de-valor}{%
\section{Ajustes por Cambio de
Valor}\label{ajustes-por-cambio-de-valor}}

Esta sección explica que, aunque la mayoría de los ingresos y gastos del
ejercicio se imputan a la cuenta de Pérdidas y Ganancias, algunos deben
imputarse directamente al patrimonio neto. Esta distinción se realiza
porque ciertas variaciones del patrimonio, aunque se registran
contablemente cuando ocurren, no se consideran ``realizadas''.
Imputarlas directamente a resultados podría llevar a una
descapitalización si, por ejemplo, se repartieran dividendos basados en
beneficios no consolidados. El objetivo final es evitar la
descapitalización de la empresa.

\textbf{Proceso Contable General:}

\begin{enumerate}
\def\labelenumi{\arabic{enumi}.}
\tightlist
\item
  \textbf{Reconocimiento Inicial:}

  \begin{itemize}
  \tightlist
  \item
    Los decrementos de valor (gastos) que no están ``consolidados'' se
    contabilizan utilizando cuentas del \textbf{grupo 8} («Gastos
    imputados al patrimonio neto»).
  \item
    Los incrementos de valor (ingresos) que no están ``consolidados'' se
    contabilizan utilizando cuentas del \textbf{grupo 9} («Ingresos
    imputados al patrimonio neto»).
  \end{itemize}
\item
  \textbf{Regularización al Cierre del Ejercicio:}

  \begin{itemize}
  \tightlist
  \item
    Al final del ejercicio, los saldos de las cuentas de los grupos 8 y
    9 se agrupan y se trasladan a las cuentas correspondientes del
    \textbf{subgrupo 13} («Subvenciones, donaciones y ajustes por
    cambios de valor»). Estas cuentas del subgrupo 13 reflejarán el
    efecto neto de estas variaciones en el Balance.
  \end{itemize}
\item
  \textbf{Realización y Transferencia a Resultados:}

  \begin{itemize}
  \tightlist
  \item
    Cuando las variaciones de patrimonio previamente imputadas
    directamente al patrimonio neto se ``realizan'' (por ejemplo, a
    través de una venta o transacción), se transfieren desde el subgrupo
    13 a la cuenta de Pérdidas y Ganancias (cuenta 129 «Resultado del
    ejercicio»).
  \item
    Este traspaso implica utilizar cuentas específicas de los grupos 6 o
    7 para los gastos o ingresos realizados, y sus contrapartidas son
    cuentas de los grupos 8 o 9 destinadas a la ``transferencia'' de
    estas partidas desde el patrimonio.
  \item
    El efecto de este traspaso en el Patrimonio Neto es cualitativo
    (cambia la composición dentro del patrimonio), pero no cuantitativo
    (la cantidad total del patrimonio no varía por este traspaso en sí
    mismo, ya que una partida disminuye en el subgrupo 13 y aumenta en
    la cuenta de resultados).
  \end{itemize}
\end{enumerate}

\textbf{Ejemplo con Activos Financieros a Valor Razonable con Cambios en
Patrimonio Neto:}

\begin{itemize}
\tightlist
\item
  \textbf{Pérdida de valor no realizada:}

  \begin{itemize}
  \tightlist
  \item
    Reconocimiento del gasto: (800) Pérdidas en activos financieros a VR
    con cambios en PN a (250) Inversiones financieras l/p en
    instrumentos de patrimonio.
  \item
    Regularización al cierre: (133) Ajustes por valoración en activos
    financieros a VR con cambios en PN a (800).
  \end{itemize}
\item
  \textbf{Ganancia de valor no realizada:}

  \begin{itemize}
  \tightlist
  \item
    Reconocimiento del ingreso: (250) Inversiones financieras l/p en
    instrumentos de patrimonio a (900) Beneficios en activos financieros
    a VR con cambios en PN.
  \item
    Regularización al cierre: (900) a (133).
  \end{itemize}
\item
  \textbf{Venta del activo (realización):}

  \begin{itemize}
  \tightlist
  \item
    Si existía una pérdida acumulada en (133) (saldo deudor):

    \begin{itemize}
    \tightlist
    \item
      Traspaso a resultados: (663) Pérdidas por valoración de
      instrumentos financieros por su VR a (902) Transferencia de
      pérdidas de activos financieros a VR con cambios en PN.
    \item
      Regularización en PN: (902) a (133) Ajustes por valoración en
      activos financieros a VR con cambios en PN.
    \end{itemize}
  \item
    Si existía un beneficio acumulado en (133) (saldo acreedor):

    \begin{itemize}
    \tightlist
    \item
      Traspaso a resultados: (802) Transferencia de beneficios en
      activos financieros a VR con cambios en PN a (763) Beneficios por
      valoración de instrumentos financieros por su VR.
    \item
      Regularización en PN: (133) a (802).
    \end{itemize}
  \end{itemize}
\end{itemize}

\textbf{Presentación en Cuentas Anuales:}

\begin{itemize}
\tightlist
\item
  Estas operaciones se reflejan detalladamente en el \textbf{Estado de
  Cambios en el Patrimonio Neto (ECPN)}.

  \begin{itemize}
  \tightlist
  \item
    La primera parte del ECPN, el \textbf{Estado de Ingresos y Gastos
    Reconocidos}, muestra:

    \begin{itemize}
    \tightlist
    \item
      Primero, las variaciones del patrimonio que han ido al resultado
      del ejercicio (grupos 6 y 7).
    \item
      Luego, las variaciones imputadas directamente al patrimonio neto
      (grupos 8 y 9, agregadas en el subgrupo 13).
    \item
      Finalmente, las transferencias realizadas desde el patrimonio neto
      (subgrupo 13) hacia la cuenta de Resultados.
    \end{itemize}
  \end{itemize}
\end{itemize}

El subgrupo 13 incluye cuentas como:

\begin{itemize}
\tightlist
\item
  \textbf{130. Subvenciones oficiales de capital:} Subvenciones
  otorgadas por organismos oficiales destinadas a financiar activos no
  corrientes o inversiones a largo plazo.
\item
  \textbf{131. Donaciones y legados de capital:} Aportaciones gratuitas
  de terceros, ya sean personas físicas o jurídicas, destinadas a
  incrementar el patrimonio de la empresa.
\item
  \textbf{133. Ajustes por valoración en activos financieros a valor
  razonable con cambios en el patrimonio neto:} Variaciones en el valor
  razonable de activos financieros que se registran directamente en el
  patrimonio neto, sin pasar por la cuenta de resultados.
\item
  \textbf{134. Operaciones de cobertura:} Ajustes relacionados con
  instrumentos financieros utilizados para cubrir riesgos específicos,
  como tipos de interés, tipos de cambio o precios de materias primas,
  que afectan directamente al patrimonio neto.
\item
  \textbf{Otras:} Cualquier otra partida que, según la normativa
  contable, deba imputarse directamente al patrimonio neto y no encaje
  en las categorías anteriores.
\end{itemize}

\hypertarget{subvenciones-donaciones-y-legados}{%
\section{Subvenciones, Donaciones y
Legados}\label{subvenciones-donaciones-y-legados}}

Esta sección trata la contabilidad de las aportaciones recibidas por la
empresa que no provienen de sus socios como contraprestación por su
participación en el capital.

\textbf{Definiciones Clave:}

\begin{itemize}
\tightlist
\item
  \textbf{Subvención:} Importe (en efectivo o en especie) otorgado por
  un organismo (estatal u otro) a título gratuito, condicionado a que se
  emplee en una actividad económica específica y se cumplan ciertos
  requisitos.
\item
  \textbf{Donación:} Acto de liberalidad por el que un donante
  transfiere un activo a un beneficiario sin recibir remuneración.
\item
  \textbf{Legado:} Patrimonio entregado a un tercero mediante
  disposición testamentaria.
\item
  \textbf{Exclusión Importante:} Las aportaciones realizadas
  incondicionalmente por los socios a la empresa (que no sean capital)
  no se consideran subvenciones o donaciones a efectos contables, sino
  que se registran como Fondos Propios en la cuenta (118) «Aportaciones
  de socios o propietarios» (según la Norma de Registro y Valoración -
  NRV 18.2).
\end{itemize}

\textbf{Valoración (NRV 18.1.2):}

\begin{itemize}
\tightlist
\item
  \textbf{Monetarias:} Se valoran por el importe recibido.
\item
  \textbf{No monetarias:} Se valoran por el valor razonable del bien en
  el momento de su reconocimiento. Este criterio también aplica a las
  aportaciones de socios mencionadas en la NRV 18.2.
\end{itemize}

\textbf{Reconocimiento y Contabilización (NRV 18.1.1):}

La contabilización depende de si están sujetas a condiciones y si estas
se han cumplido:

\begin{itemize}
\tightlist
\item
  \textbf{Subvenciones Reintegrables:} Si la recepción está supeditada a
  condiciones y su cumplimiento no está garantizado, se contabilizan
  como un pasivo. Se utilizan cuentas como la (172) «Deudas a largo
  plazo transformables en subvenciones, donaciones y legados» o la (522)
  si es a corto plazo.
\item
  \textbf{Subvenciones No Reintegrables:} Son cantidades recibidas
  incondicionalmente o aquellas cuyas condiciones ya se han cumplido y
  no hay dudas razonables sobre su recepción. Generalmente, se imputan
  inicialmente al Patrimonio Neto, aunque en algunos casos específicos
  van directamente a la Cuenta de Pérdidas y Ganancias.
\end{itemize}

\textbf{Criterios de Imputación para Subvenciones No Reintegrables (NRV
18.1.3):}

La imputación a resultados depende de la finalidad de la subvención:

\begin{enumerate}
\def\labelenumi{\arabic{enumi}.}
\tightlist
\item
  \textbf{Para asegurar rentabilidad mínima o compensar déficits de
  explotación:} Se imputan como ingresos del ejercicio en que se
  conceden (directamente a la cuenta 740).
\item
  \textbf{Para financiar gastos específicos:} Se imputan como ingresos
  en el mismo ejercicio en que se devengan los gastos que financian
  (inicialmente a patrimonio neto, luego a resultados).
\item
  \textbf{Para adquirir activos (inmovilizado intangible, material,
  inversiones inmobiliarias):} Se imputan como ingresos del ejercicio en
  proporción a la amortización de dichos activos, o cuando se enajenen,
  corrijan valorativamente por deterioro o se den de baja (inicialmente
  a patrimonio neto, luego a resultados).
\item
  \textbf{Para adquirir existencias o activos financieros:} Se imputan
  como ingresos del ejercicio en que se produzca su enajenación,
  corrección valorativa por deterioro o baja en balance (inicialmente a
  patrimonio neto, luego a resultados).
\item
  \textbf{Para cancelación de deudas:} Se imputan como ingresos del
  ejercicio en que se produzca la cancelación, salvo que se otorguen
  para una financiación específica (en cuyo caso, la imputación sigue al
  elemento financiado) (inicialmente a patrimonio neto, luego a
  resultados).
\item
  \textbf{Importes monetarios sin asignación a una finalidad
  específica:} Se imputan como ingresos del ejercicio en que se
  reconozcan (directamente a la cuenta 740).
\end{enumerate}

\textbf{Proceso Contable Simplificado para Subvenciones No
Reintegrables:}

\begin{itemize}
\tightlist
\item
  \textbf{Imputación Directa a Resultados (Pérdidas y Ganancias):}

  \begin{itemize}
  \tightlist
  \item
    Ocurre cuando la subvención financia la actividad general de la
    empresa o no tiene una finalidad concreta (casos 1 y 6 de la NRV
    18.1.3).
  \item
    Se utiliza la cuenta (740) «Subvenciones, donaciones y legados a la
    explotación».
  \end{itemize}
\item
  \textbf{Imputación Inicial a Patrimonio Neto y Posterior Transferencia
  a Resultados:}

  \begin{itemize}
  \tightlist
  \item
    Ocurre para las subvenciones con una finalidad concreta (casos 2, 3,
    4, 5).
  \item
    \textbf{Paso 1 (Reconocimiento inicial):} Se contabiliza como un
    ingreso directamente en el patrimonio neto, utilizando cuentas del
    \textbf{subgrupo 94} («Ingresos por subvenciones, donaciones y
    legados»). Por ejemplo, (940) para subvenciones oficiales de
    capital, (941) para donaciones y legados de capital, o (942) para
    otras.
  \item
    \textbf{Paso 2 (Regularización al cierre):} El saldo de estas
    cuentas del subgrupo 94 se traspasa al final del ejercicio a la
    cuenta correspondiente del \textbf{subgrupo 13} (por ejemplo, (130),
    (131), (132)).
  \item
    \textbf{Paso 3 (Transferencia a Resultados):} La cantidad
    subvencionada se va trasladando a la Cuenta de Pérdidas y Ganancias
    de forma sistemática y correlacionada con los gastos que financia.
    Para ello, se carga una cuenta del \textbf{subgrupo 84}
    («Transferencias de subvenciones, donaciones y legados») y se abona
    una cuenta del \textbf{subgrupo 74} (por ejemplo, (746) o (747)). El
    saldo de la cuenta del subgrupo 84 también se imputa a la cuenta del
    subgrupo 13, de modo que esta última quedará saldada cuando la
    subvención haya sido completamente traspasada al resultado.
  \end{itemize}
\end{itemize}

\textbf{Consideraciones Adicionales:}

\begin{itemize}
\tightlist
\item
  \textbf{Deterioro de bienes financiados con subvenciones:} Si un
  activo financiado con una subvención sufre un deterioro, una parte
  proporcional de la subvención se transfiere a resultados. La NRV 18
  establece que, para simplificar, las correcciones valorativas por
  deterioro de la parte de los elementos financiada gratuitamente se
  consideran irreversibles.
\item
  \textbf{Efecto impositivo:} La imputación a Patrimonio Neto de las
  subvenciones debe realizarse neta del efecto impositivo. Esto implica
  contabilizar diferencias temporarias (utilizando el subgrupo 83 y la
  cuenta 479), un tema que se trata en otro capítulo del manual.
\end{itemize}

\hypertarget{provisiones-y-contingencias}{%
\chapter{Provisiones y
Contingencias}\label{provisiones-y-contingencias}}

\hypertarget{definiciones-y-caracteruxedsticas}{%
\section{Definiciones y
características}\label{definiciones-y-caracteruxedsticas}}

\begin{itemize}
\tightlist
\item
  \texttt{Concepto\ de\ pasivo}: obligaciones actuales surgidas como
  consecuencia de sucesos pasados, para cuya extinción la empresa espera
  desprenderse de recursos que puedan producir beneficios o rendimientos
  económicos futuros. A estos efectos se incluyen las
  \texttt{provisiones}.
\item
  Se trata del grupo \textbf{14}.
\item
  En esta parte del PGC \textbf{las provisiones} se conocen como
  obligaciones expresas o tácitas a largo plazo, claramente específicas
  en cuanto a naturaleza, pero que en la fecha de cierre del ejercicio
  son indeterminadas en cuanto a su importe exacto o a la fecha en que
  se producirán.
\item
  Se reconocen como \texttt{provisiones} los pasivos que resulten
  \texttt{indeterminados} respecto a su importe o a la fecha en la que
  se cancelará.
\end{itemize}

\textbf{Provisiones:}

\begin{itemize}
\tightlist
\item
  Son \textbf{obligaciones actuales} de una empresa, surgidas de hechos
  pasados.
\item
  Implican una \textbf{salida probable de recursos} para cancelarlas
  (desprenderse de recursos).
\item
  Existe \textbf{incertidumbre sobre su importe exacto o la fecha} de
  cancelación.
\item
  Deben poder \textbf{estimarse de forma fiable}.
\item
  Se reconocen en el balance como un pasivo.
\end{itemize}

\textbf{Contingencias (Pasivos Contingentes):}

\begin{itemize}
\tightlist
\item
  Pueden ser \textbf{obligaciones posibles}, cuya existencia se
  confirmará por eventos futuros inciertos no controlados por la
  empresa.
\item
  O pueden ser \textbf{obligaciones presentes} que no se reconocen
  porque:

  \begin{itemize}
  \tightlist
  \item
    No es probable que la empresa tenga que desprenderse de recursos
    para cancelarla.
  \item
    El importe no puede valorarse con suficiente fiabilidad.
  \end{itemize}
\item
  \textbf{No se reconocen en el balance}, pero se informan en la memoria
  si no son remotas.
\end{itemize}

\textbf{Diferencia Clave:}

\begin{itemize}
\tightlist
\item
  Una \textbf{provisión} es una obligación presente, probable y medible
  con fiabilidad (se contabiliza).
\item
  Una \textbf{contingencia} es una obligación posible, o una obligación
  presente que no es probable o no se puede medir fiablemente (se revela
  en notas, no se contabiliza en el balance).
\end{itemize}

\textbf{Activos Contingentes:}

\begin{itemize}
\tightlist
\item
  Son \textbf{derechos posibles} (no presentes) surgidos de sucesos
  pasados, cuya existencia se confirmará por eventos futuros inciertos.
\item
  \textbf{No se reconocen en el balance} porque no cumplen la definición
  de activo (no son presentes ni es seguro que generen beneficios). Se
  informan en la memoria si es probable la entrada de beneficios.
\end{itemize}

\hypertarget{criterios-de-reconocimiento-y-valoraciuxf3n-de-las-provisiones}{%
\section{Criterios de Reconocimiento y Valoración de las
Provisiones}\label{criterios-de-reconocimiento-y-valoraciuxf3n-de-las-provisiones}}

\hypertarget{reconocimiento-de-provisiones}{%
\subsection{Reconocimiento de
Provisiones}\label{reconocimiento-de-provisiones}}

Según la NIC 37, se debe reconocer una provisión cuando se cumplen las
siguientes condiciones:

\begin{itemize}
\tightlist
\item
  La entidad tiene una \textbf{obligación presente} (legal o implícita)
  como resultado de un suceso pasado.
\item
  Es \textbf{probable} que la entidad tenga que desprenderse de recursos
  que incorporen beneficios económicos para cancelar dicha obligación.
\item
  Puede hacerse una \textbf{estimación fiable} del importe de la
  obligación.
\end{itemize}

Si no se cumplen estas tres condiciones, no se debe reconocer la
provisión. Las contingencias se mencionan en la Memoria, mientras que
las provisiones, además de otras menciones, implican un asiento
contable. El Plan General de Contabilidad (PGC) establece el subgrupo 14
para el registro de las provisiones a largo plazo.

El PGC contempla diversas cuentas para provisiones, como (de la
140-147):

\begin{itemize}
\tightlist
\item
  Provisión por retribuciones a largo plazo al personal.
\item
  Provisión para impuestos.
\item
  Provisión para otras responsabilidades.
\item
  Provisión por desmantelamiento, retiro o rehabilitación del
  inmovilizado.
\item
  Provisión para actuaciones medioambientales.
\item
  Provisión para reestructuraciones.
\item
  Provisión por transacciones con pagos basados en instrumentos de
  patrimonio.
\end{itemize}

Estas cuentas tienen su equivalente a corto plazo en las divisionarias
de la cuenta (529) Provisiones a corto plazo.

\hypertarget{valoraciuxf3n-de-las-provisiones}{%
\subsection{Valoración de las
Provisiones}\label{valoraciuxf3n-de-las-provisiones}}

La NRV 15.2 establece que las provisiones se valorarán por:

\begin{itemize}
\tightlist
\item
  El \textbf{valor actual de la mejor estimación posible} del importe
  necesario para cancelar o transferir a un tercero la obligación.
\item
  Los \textbf{ajustes por actualización} se registrarán como un gasto
  financiero conforme se devenguen.
\end{itemize}

Esto implica que:

\begin{enumerate}
\def\labelenumi{\arabic{enumi}.}
\tightlist
\item
  \textbf{Imputación inicial}: Se cuantifica la provisión por el valor
  actual de la mejor estimación de la deuda.
\item
  \textbf{Cierre de cada ejercicio}:

  \begin{itemize}
  \tightlist
  \item
    \textbf{Actualización financiera}: Se ajusta el valor por el paso
    del tiempo, utilizando la cuenta (660) Gastos financieros por
    actualización de provisiones. El tipo de interés debe considerar el
    valor temporal del dinero y el riesgo específico del pasivo.
  \item
    \textbf{Ajuste de la estimación}: Se vuelve a estimar el valor
    actual de la deuda con la mejor información disponible.

    \begin{itemize}
    \tightlist
    \item
      Si la provisión debe aumentarse, se carga a la misma cuenta de
      gasto que la originó.
    \item
      Si debe disminuirse, se utiliza alguna de las divisionarias de la
      cuenta (795) Exceso de provisiones.
    \end{itemize}
  \end{itemize}
\end{enumerate}

\textbf{Puntualizaciones importantes:}

\begin{itemize}
\tightlist
\item
  \textbf{Provisión por desmantelamiento, retiro y rehabilitación
  (cuenta 143)}:

  \begin{itemize}
  \tightlist
  \item
    Está relacionada con inmovilizados materiales y su valoración
    inicial (NRV 2).
  \item
    Los cambios de valor derivados de la mejora en la estimación de esta
    provisión repercuten sobre el valor del inmovilizado
    correspondiente, incluso si ya está en funcionamiento.
  \end{itemize}
\item
  \textbf{Provisiones a corto plazo (vencimiento \(leq\) 1 año)}:

  \begin{itemize}
  \tightlist
  \item
    Si el efecto financiero no es significativo, \textbf{no será
    necesario llevar a cabo ningún tipo de descuento} (actualización
    financiera). Se contabilizarán directamente en la cuenta 529.
  \end{itemize}
\end{itemize}

\hypertarget{compensaciones-de-terceros}{%
\subsection{Compensaciones de
Terceros}\label{compensaciones-de-terceros}}

Cuando se espere recibir una compensación de un tercero al liquidar la
obligación (por ejemplo, de un seguro):

\begin{itemize}
\tightlist
\item
  \textbf{No se minorará el importe de la deuda} (provisión).
\item
  Se reconocerá un \textbf{derecho de cobro en el activo} (por un
  importe que no exceda el de la obligación) siempre que no existan
  dudas de que dicho reembolso será percibido.
\item
  \textbf{Excepción}: Si existe un vínculo legal o contractual por el
  que se haya exteriorizado parte del riesgo y la empresa no esté
  obligada a responder por esa parte, la provisión se estimará por el
  importe neto que asume la empresa.
\end{itemize}

Claro, aquí tienes el resumen de la sección ``Provisiones por
operaciones comerciales'':

\hypertarget{provisiones-por-operaciones-comerciales}{%
\section{Provisiones por Operaciones
Comerciales}\label{provisiones-por-operaciones-comerciales}}

La actividad comercial de una empresa conlleva la asunción de
obligaciones con los clientes, como garantías post-venta, reparaciones o
la aceptación de devoluciones. Estas situaciones se ajustan a la
definición de provisiones.

\textbf{Reconocimiento y Valoración:}

\begin{itemize}
\tightlist
\item
  El Plan General de Contabilidad (PGC) incluye la cuenta \textbf{(499)
  Provisiones por operaciones comerciales} dentro del subgrupo 49.
\item
  Específicamente, se utiliza la cuenta \textbf{(4999) Provisión para
  otras operaciones comerciales} para estos fines.
\item
  \textbf{Procedimiento Contable Particular:}

  \begin{enumerate}
  \def\labelenumi{\arabic{enumi}.}
  \tightlist
  \item
    \textbf{Al cierre del ejercicio:} Se dota la provisión por la
    cuantía estimada de estas obligaciones futuras. Esto se hace
    cargando la cuenta (6959) Dotación a la provisión para otras
    operaciones comerciales y abonando la (4999) Provisión para otras
    operaciones comerciales.
  \item
    \textbf{Al año siguiente:} Se anula la provisión dotada el año
    anterior por el mismo importe. Esto se hace cargando la (4999) y
    abonando la (79549) Exceso de provisión para otras operaciones
    comerciales.
  \item
    Los gastos reales por devoluciones, reparaciones, etc., se
    contabilizan en sus respectivas cuentas de gasto cuando ocurren
    durante el ejercicio.
  \end{enumerate}
\item
  Este método asegura una adecuada \textbf{correlación de ingresos y
  gastos}: el gasto estimado por estas operaciones se reconoce en el
  mismo año que la venta que las origina. Al anular la provisión al año
  siguiente, en el resultado solo queda la diferencia entre lo estimado
  y los gastos reales.
\end{itemize}

\textbf{Estimación de la Provisión:}

\begin{itemize}
\tightlist
\item
  Se debe considerar la \textbf{mejor estimación posible}.
\item
  Para valorar un conjunto importante de casos individuales (como el
  total de ventas de un ejercicio), la NIC 37 sugiere utilizar el
  \textbf{``valor esperado''}. Esto implica calcular un promedio de los
  posibles desenlaces ponderado por sus probabilidades de ocurrencia.
\end{itemize}

\hypertarget{impuesto-sobre-beneficios}{%
\chapter{Impuesto sobre beneficios}\label{impuesto-sobre-beneficios}}

\hypertarget{definiciuxf3n-y-caracteruxedsticas}{%
\section{Definición y
características}\label{definiciuxf3n-y-caracteruxedsticas}}

Son aquellos impuestos directos que se liquidan a partir de un resultado
empresarial de acuerdo a las normas fiscales. Como características,
podemos destacar:

\begin{itemize}
\tightlist
\item
  Impuesto directo, personal, hace referencia a una persona en concreto
  (jurídica), proporcional y periódico.
\item
  El hecho imponible lo constituyen la obtención de la renta.
\item
  Se aplica un gravamen único.
\item
  Básicamente, es un gasto más del ejercicio.
\end{itemize}

Por otro lado, debemos de mencionar el
\texttt{impuesto\ sobre\ sociedades}, cuya base imponible, se determina
a partir del resultado contable.

\hypertarget{el-resultado-contable-y-la-base-imponible}{%
\section{El Resultado Contable y la Base
Imponible}\label{el-resultado-contable-y-la-base-imponible}}

Este apartado detalla el proceso para determinar la base imponible del
Impuesto sobre Sociedades a partir del resultado contable de una
empresa, así como los pasos subsecuentes para calcular la cuota final
del impuesto.

\textbf{Normativa y Punto de Partida:}

\begin{itemize}
\tightlist
\item
  Los contribuyentes están obligados a presentar la declaración del
  Impuesto sobre Sociedades conforme a la Ley 27/2014, de 27 de
  noviembre, del Impuesto sobre Sociedades, y el Real Decreto 634/2015,
  de 10 de julio, que aprueba el Reglamento del Impuesto sobre
  Sociedades.
\item
  El cálculo para determinar la cuota a pagar se inicia con el
  \textbf{resultado contable} de la entidad.
\end{itemize}

\begin{longtable}[]{@{}
  >{\raggedright\arraybackslash}p{(\columnwidth - 2\tabcolsep) * \real{0.5053}}
  >{\raggedright\arraybackslash}p{(\columnwidth - 2\tabcolsep) * \real{0.4947}}@{}}
\toprule\noalign{}
\begin{minipage}[b]{\linewidth}\raggedright
\textbf{Análisis}
\end{minipage} & \begin{minipage}[b]{\linewidth}\raggedright
\end{minipage} \\
\midrule\noalign{}
\endhead
\bottomrule\noalign{}
\endlastfoot
Resultado contable antes de Impuestos & \\
(+/-) Ajustes & \\
(=) Base imponible previa & \\
(+/-) Bases imponibles negativas de ejercicios anteriores & \\
(=) Base imponible & \\
(x) Tipo de gravamen & \\
(=) Cuota íntegra & \\
(-) Deducciones y bonificaciones fiscales & \\
(=) Cuota líquida & \\
(-) Retenciones y pagos a cuenta & \\
(=) Cuota diferencial & \\
\end{longtable}

\textbf{Proceso de Cálculo de la Base Imponible y Cuota:}

\begin{enumerate}
\def\labelenumi{\arabic{enumi}.}
\tightlist
\item
  \textbf{Resultado Contable Antes de Impuestos (RCAI)}: Es el punto de
  inicio.
\item
  \textbf{Ajustes al Resultado Contable}:

  \begin{itemize}
  \tightlist
  \item
    Se realizan correcciones o ajustes al resultado contable debido a
    diferencias entre los criterios de cálculo contables y los
    permitidos por la Ley del Impuesto sobre Sociedades.
  \item
    Estos ajustes pueden ser positivos o negativos.
  \item
    Su origen puede estar en \textbf{diferencias de carácter temporal o
    permanente}.
  \end{itemize}
\item
  \textbf{Base Imponible Previa}: Resultado de aplicar los ajustes al
  RCAI.
\item
  \textbf{Compensación de Bases Imponibles Negativas}:

  \begin{itemize}
  \tightlist
  \item
    Se tienen en cuenta las bases imponibles negativas de ejercicios
    anteriores (pérdidas fiscales).
  \item
    La normativa permite su compensación en el cálculo del impuesto.
  \end{itemize}
\item
  \textbf{Base Imponible}: Resultado tras la posible compensación de
  bases imponibles negativas.
\item
  \textbf{Aplicación del Tipo de Gravamen}:

  \begin{itemize}
  \tightlist
  \item
    Al resultado obtenido (Base Imponible) se le aplica el tipo de
    gravamen que corresponda.
  \end{itemize}
\item
  \textbf{Cuota Íntegra}: Es el resultado de aplicar el tipo de gravamen
  a la base imponible.
\item
  \textbf{Deducciones y Bonificaciones}:

  \begin{itemize}
  \tightlist
  \item
    Se restan las deducciones o bonificaciones fiscales a las que la
    empresa tenga derecho.
  \item
    El objetivo de estas deducciones es fomentar la realización de
    ciertas actividades que benefician tanto a la entidad como a la
    economía general.
  \item
    Ejemplos de estas actividades incluyen: investigación y desarrollo,
    innovación tecnológica, creación de empleo, etc.
  \end{itemize}
\item
  \textbf{Cuota Líquida}: Resulta tras restar las deducciones y
  bonificaciones de la cuota íntegra.
\item
  \textbf{Retenciones y Pagos a Cuenta}:

  \begin{itemize}
  \tightlist
  \item
    Finalmente, de la cuota líquida se deducen las cantidades que hayan
    sido objeto de retención (cuando la empresa recibe ciertas rentas) y
    los pagos que la empresa haya realizado a la Administración
    Tributaria como pagos a cuenta del impuesto.
  \end{itemize}
\item
  \textbf{Cuota Diferencial}: Es el importe final a ingresar o a
  devolver por la Hacienda Pública.
\end{enumerate}

\hypertarget{el-impuesto-corriente}{%
\section{El Impuesto Corriente}\label{el-impuesto-corriente}}

Este apartado se enfoca en el concepto y registro contable del impuesto
corriente.

\textbf{Definición y Cálculo:}

\begin{itemize}
\tightlist
\item
  El \textbf{impuesto corriente} es la cantidad que una empresa debe
  pagar (o puede recuperar) por el impuesto sobre beneficios devengado
  en el ejercicio actual. Este monto se calcula aplicando las normas
  fiscales correspondientes tras realizar los ajustes pertinentes al
  resultado contable.
\item
  En esencia, el impuesto corriente como gasto se corresponde con la
  \textbf{cuota líquida} que se obtiene después de aplicar el tipo
  impositivo a la base imponible y restar las deducciones y
  bonificaciones fiscales.
\end{itemize}

\textbf{Factores que Afectan el Cálculo:}

\begin{itemize}
\tightlist
\item
  El impuesto corriente (entendido como la cuota líquida o gasto por
  impuesto) puede verse reducido por:

  \begin{itemize}
  \tightlist
  \item
    \textbf{Deducciones y ventajas fiscales} aplicables sobre la cuota
    íntegra.
  \end{itemize}
\item
  Es importante distinguir que la \textbf{compensación de pérdidas de
  ejercicios anteriores} reduce la base imponible sobre la cual se
  calcula el impuesto, afectando así la cuota íntegra y, por ende, la
  cuota líquida (impuesto corriente).
\item
  Por otro lado, hay elementos que \textbf{no modifican el importe del
  impuesto corriente (gasto)}, pero sí afectan la cantidad final a pagar
  o a devolver (cuota diferencial):

  \begin{itemize}
  \tightlist
  \item
    \textbf{Retenciones} soportadas por la empresa.
  \item
    \textbf{Pagos a cuenta} realizados durante el ejercicio.
  \end{itemize}
\end{itemize}

\textbf{Cuentas Contables Involucradas:}

El registro contable del impuesto corriente utiliza principalmente las
siguientes cuentas:

\begin{itemize}
\tightlist
\item
  \textbf{(6300) Impuesto corriente}: Refleja el gasto por el impuesto
  sobre beneficios devengado en el ejercicio. Corresponde a la cuota
  líquida.
\item
  \textbf{(4752) Hacienda Pública acreedora por impuesto de sociedades}:
  Es una cuenta de pasivo que representa la cantidad que la empresa debe
  pagar a Hacienda por este impuesto, una vez descontadas todas las
  deducciones, retenciones y pagos a cuenta.
\item
  \textbf{(473) Hacienda Pública retenciones y pagos a cuenta}: Cuenta
  de activo que acumula las cantidades retenidas a la empresa y los
  pagos fraccionados realizados a cuenta del impuesto del ejercicio. Se
  regulariza al contabilizar el impuesto.
\item
  \textbf{(4709) Hacienda Pública deudora por devoluciones de
  impuestos}: Cuenta de activo que surge cuando las retenciones y pagos
  a cuenta superan la cuota líquida, reflejando el derecho de la empresa
  a recibir una devolución.
\end{itemize}

\textbf{Ejemplificación:}

Podemos ejemplificar dos casos en los que se contabiliza el impuesto
corriente: 1. Cuando las retenciones y pagos a cuenta son inferiores a
la cuota del impuesto, resultando en una cantidad a pagar. 2. Cuando las
retenciones y pagos a cuenta son superiores a la cuota del impuesto,
resultando en una cantidad a devolver por Hacienda.

Finalmente, \textbf{si el resultado del ejercicio es de pérdida, surge
una problemática que da lugar a un impuesto diferido}.

\begin{itemize}
\tightlist
\item
  Por impuesto diferido entendemos el efecto contable que surge de las
  diferencias temporarias entre el resultado contable y la base
  imponible fiscal. Estas diferencias generan activos o pasivos por
  impuestos diferidos, que reflejan el impacto fiscal futuro derivado de
  dichas discrepancias. El impuesto diferido se registra para garantizar
  que los impuestos se imputen al ejercicio en el que realmente se
  devengan, independientemente de cuándo se paguen o recuperen.
\end{itemize}

\hypertarget{diferencias-permanentes-y-temporarias}{%
\section{Diferencias Permanentes y
Temporarias}\label{diferencias-permanentes-y-temporarias}}

Este apartado profundiza en los ajustes entre el resultado contable y la
base imponible, centrándose en las diferencias temporarias y
permanentes, y su relación con el impuesto diferido.

\hypertarget{definiciuxf3n}{%
\subsection{Definición}\label{definiciuxf3n}}

Se retoma el concepto de los ajustes que surgen por la discrepancia
entre los criterios contables y fiscales.

\begin{itemize}
\tightlist
\item
  \textbf{Diferencias Temporarias}:

  \begin{itemize}
  \tightlist
  \item
    Son aquellas que se derivan de la diferente valoración contable y
    fiscal atribuida a los activos, pasivos y determinados instrumentos
    de patrimonio propio de la empresa.
  \item
    Estas diferencias tienen incidencia en la carga fiscal futura.
  \item
    Pueden originar dos situaciones:

    \begin{itemize}
    \tightlist
    \item
      \textbf{Diferencias Imponibles}: Darán lugar a mayores cantidades
      a pagar (\textbf{en la práctica se restan}) o menores cantidades a
      devolver por impuestos en ejercicios futuros, normalmente a medida
      que se recuperen los activos o se liquiden los pasivos de los que
      se derivan.
    \item
      \textbf{Diferencias Deducibles}: Darán lugar a menores cantidades
      a pagar (\textbf{en la práctica se suman}) o a mayores cantidades
      a devolver por impuestos en ejercicios futuros, a medida que se
      recuperen los activos o se liquiden los pasivos de los que se
      derivan.
    \item
      ``Imponible'' se refiere a aquello que está sujeto a impuesto, es
      decir, la cantidad o valor sobre el cual se va a aplicar una tasa
      impositiva para calcular el impuesto a pagar. Constituye la
      materia prima del tributo. Podemos pensar que es toda renta, bien,
      servicio o transacción, susceptible de ser gravada.
    \item
      Deducible'' se refiere a aquellos gastos, costes o minoraciones
      que la ley permite restar de los ingresos brutos o de la base
      imponible para determinar la base imponible neta o la cuota
      tributaria. El objetivo de las deducciones es ajustar la capacidad
      económica real del contribuyente.
    \item
      En relación: Primero se determinan los ingresos o valores
      imponibles para obtener una base bruta. Luego, se restan los
      gastos y otros conceptos deducibles permitidos por la ley para
      llegar a una base neta o liquidable, sobre la cual se aplica el
      porcentaje de impuesto correspondiente.
    \end{itemize}
  \end{itemize}
\item
  \textbf{Diferencias Permanentes}:

  \begin{itemize}
  \tightlist
  \item
    Son aquellas diferencias entre las magnitudes contables y fiscales
    que no se identifican como temporarias.
  \item
    No se registran contablemente como impuesto diferido, aunque sí
    deben considerarse en el cálculo del impuesto corriente del
    ejercicio.
  \end{itemize}
\item
  \textbf{Ejemplos:}:

  \begin{itemize}
  \tightlist
  \item
    Un beneficio por la cesión de una patente que está parcialmente
    exento fiscalmente genera una \textbf{diferencia permanente
    negativa} (menor base imponible que resultado contable).
  \item
    Una multa de tráfico no deducible fiscalmente es una
    \textbf{diferencia permanente positiva} (mayor base imponible que
    resultado contable).
  \item
    Un deterioro de valor de un cliente contabilizado como gasto, pero
    que fiscalmente será deducible el próximo año, origina una
    \textbf{diferencia temporaria deducible}.
  \item
    La amortización acelerada de un activo fiscalmente (libertad de
    amortización) mientras contablemente se amortiza en más años, genera
    una \textbf{diferencia temporaria imponible}.
  \end{itemize}
\end{itemize}

\hypertarget{diferencias-temporarias-e-impuesto-diferido}{%
\subsection{Diferencias Temporarias e Impuesto
Diferido}\label{diferencias-temporarias-e-impuesto-diferido}}

Las diferencias temporarias son la clave para entender el impuesto
diferido.

\begin{itemize}
\tightlist
\item
  Se originan por la \textbf{diferente valoración contable y fiscal}
  (diferencia entre la base fiscal y el valor contable) de activos,
  pasivos e instrumentos de patrimonio, siempre que tengan incidencia en
  la carga fiscal futura.

  \begin{itemize}
  \tightlist
  \item
    La \textbf{base fiscal} es el importe atribuido a un elemento según
    la legislación fiscal.
  \end{itemize}
\item
  Normalmente, surgen de \textbf{diferencias temporales} en la
  imputación de ingresos y gastos entre el resultado contable y la base
  imponible, las cuales revierten en periodos siguientes.
\item
  Estas diferencias (imponibles o deducibles) se reflejan contablemente
  mediante cuentas específicas, principalmente:

  \begin{itemize}
  \tightlist
  \item
    \textbf{(6301) Impuesto diferido}
  \item
    \textbf{(4740) Activos por diferencias temporarias deducibles}
  \item
    \textbf{(479) Pasivos por diferencias temporarias imponibles}
  \end{itemize}
\item
  \textbf{Registro de Diferencias Temporarias Imponibles (originan un
  pasivo por impuesto diferido)}:

  \begin{itemize}
  \tightlist
  \item
    \textbf{Origen}: Se carga la cuenta (6301) Impuesto diferido y se
    abona la (479) Pasivos por diferencias temporarias imponibles.
    \texttt{XXX\ (6301)\ Impuesto\ diferido\ \ \ \ \ \ \ \ \ \ \ \ \ \ \ a\ (479)\ Pasivos\ por\ diferencias\ temporarias\ imponibles\ XXX}
  \item
    \textbf{Reversión}: Se carga la (479) Pasivos por diferencias
    temporarias imponibles y se abona la (6301) Impuesto diferido.
    \texttt{XXX\ (479)\ Pasivos\ por\ diferencias\ temporarias\ imponibles\ \ \ \ \ \ \ \ \ \ \ \ \ \ \ a\ (6301)\ Impuesto\ diferido\ XXX}
  \end{itemize}
\item
  \textbf{Registro de Diferencias Temporarias Deducibles (originan un
  activo por impuesto diferido)}:

  \begin{itemize}
  \tightlist
  \item
    \textbf{Origen}: Se carga la cuenta (4740) Activos por diferencias
    temporarias deducibles y se abona la (6301) Impuesto diferido.
    \texttt{XXX\ (4740)\ Activos\ por\ diferencias\ temporarias\ deducibles\ \ \ \ \ \ \ \ \ \ \ \ \ \ \ a\ (6301)\ Impuesto\ diferido\ XXX}
  \item
    \textbf{Reversión}: Se carga la (6301) Impuesto diferido y se abona
    la (4740) Activos por diferencias temporarias deducibles.
    \texttt{XXX\ (6301)\ Impuesto\ diferido\ \ \ \ \ \ \ \ \ \ \ \ \ \ \ a\ (4740)\ Activos\ por\ diferencias\ temporarias\ deducibles\ XXX}
  \end{itemize}
\end{itemize}

\hypertarget{explicaciuxf3n-extra}{%
\section{Explicación Extra}\label{explicaciuxf3n-extra}}

Ahora vamos a añadir una serie de ejemplos para que se entienda mejor en
cuanto a las casuísticas que pueden surgir en la vida real, ya que puede
ser un poco lioso, por ejemplo, el caso de que una multa de tráfico sea
una diferencia permanente positiva: Cuando analizamos el Impuesto sobre
Sociedades, uno de los conceptos que más dudas genera son las
\textbf{diferencias permanentes}. En concreto, vamos a entender por qué
una \textbf{multa de tráfico} se considera una \textbf{diferencia
permanente positiva}.

\begin{center}\rule{0.5\linewidth}{0.5pt}\end{center}

\hypertarget{la-multa-de-truxe1fico-un-gasto-contable-que-no-es-deducible-fiscalmente}{%
\subsection{La Multa de Tráfico: Un Gasto Contable que No es Deducible
Fiscalmente}\label{la-multa-de-truxe1fico-un-gasto-contable-que-no-es-deducible-fiscalmente}}

Para entenderlo, debemos diferenciar dos perspectivas:

\begin{enumerate}
\def\labelenumi{\arabic{enumi}.}
\tightlist
\item
  \textbf{Perspectiva Contable:}

  \begin{itemize}
  \tightlist
  \item
    Desde el punto de vista contable, una multa de tráfico es un
    \textbf{gasto} para la empresa.
  \item
    Se registra en la contabilidad como tal (por ejemplo, en una cuenta
    de gastos excepcionales o servicios exteriores).
  \item
    Al ser un gasto, \textbf{resta} al resultado contable de la empresa.
    Es decir, reduce el beneficio que la empresa ha obtenido según sus
    libros contables.
  \end{itemize}
\item
  \textbf{Perspectiva Fiscal:}

  \begin{itemize}
  \tightlist
  \item
    La Ley del Impuesto sobre Sociedades establece claramente que las
    \textbf{sanciones y multas (incluidas las de tráfico) no son gastos
    fiscalmente deducibles}.
  \item
    Esto significa que, a ojos de Hacienda, la multa no puede ser
    utilizada para reducir la base imponible sobre la que se calculará
    el impuesto.
  \end{itemize}
\end{enumerate}

\begin{center}\rule{0.5\linewidth}{0.5pt}\end{center}

\hypertarget{por-quuxe9-se-produce-un-ajuste-positivo}{%
\subsection{¿Por qué se produce un ``Ajuste
Positivo''?}\label{por-quuxe9-se-produce-un-ajuste-positivo}}

Aquí es donde entra en juego la necesidad de realizar un
\textbf{ajuste}. Imagina el proceso de cálculo del impuesto:

\begin{itemize}
\tightlist
\item
  Se parte del \textbf{Resultado Contable Antes de Impuestos}. Este
  resultado ya tiene la multa restada como gasto.
\item
  Sin embargo, la ley fiscal nos dice que esa multa no debe haber
  restado. Para ``deshacer'' el efecto de esa resta contable, tenemos
  que \textbf{sumar de nuevo} el importe de la multa al resultado
  contable.
\end{itemize}

\textbf{Ejemplo numérico:}

Imagina que tu empresa tiene:

\begin{itemize}
\tightlist
\item
  Un \textbf{Resultado Contable Antes de Impuestos de 50.000 €}.
\item
  Dentro de esos 50.000 €, ya se ha restado una \textbf{multa de 1.000
  €}. (Es decir, si no hubiera habido multa, el resultado contable
  habría sido 51.000 €).
\end{itemize}

Cuando calculas la \textbf{Base Imponible Fiscal}:

\begin{itemize}
\tightlist
\item
  \textbf{Partes de:} 50.000 € (Resultado Contable)
\item
  \textbf{Ajuste por multa:} Como fiscalmente esos 1.000 € no son
  deducibles, los \textbf{sumas} para revertir la resta que se hizo
  contablemente.
\item
  \textbf{Base Imponible Fiscal:} 50.000 € + 1.000 € = \textbf{51.000 €}
\end{itemize}

En este caso, la multa provoca que tu \textbf{base imponible fiscal
(51.000 €) sea mayor que tu resultado contable (50.000 €)}. Por eso se
le denomina \textbf{``positiva''}, porque incrementa la cantidad sobre
la que pagarás impuestos.

\begin{center}\rule{0.5\linewidth}{0.5pt}\end{center}

\hypertarget{por-quuxe9-es-permanente}{%
\subsection{¿Por qué es
``Permanente''?}\label{por-quuxe9-es-permanente}}

Se le llama \textbf{permanente} porque esta diferencia \textbf{nunca se
va a revertir o compensar} en el futuro. La multa nunca será un gasto
deducible fiscalmente, ni en el ejercicio actual ni en los siguientes.

Esto la distingue de las ``diferencias temporarias'', que sí se
compensan con el tiempo (por ejemplo, un gasto que es deducible ahora
contablemente pero fiscalmente lo será en el futuro, o viceversa).

\begin{center}\rule{0.5\linewidth}{0.5pt}\end{center}

\hypertarget{en-resumen}{%
\subsection{En Resumen:}\label{en-resumen}}

Una multa de tráfico es una \textbf{diferencia permanente positiva}
porque:

\begin{itemize}
\tightlist
\item
  Es un \textbf{gasto} que reduce tu resultado contable.
\item
  \textbf{Nunca será deducible fiscalmente}, lo que la convierte en
  \textbf{permanente}.
\item
  Para calcular la base imponible fiscal, debes \textbf{sumarla de
  nuevo} al resultado contable, lo que \textbf{incrementa} dicha base
  imponible y, por lo tanto, es un ajuste \textbf{positivo}.
\end{itemize}

Esto asegura que, a efectos fiscales, la empresa no se beneficie de una
reducción de su base imponible por un gasto que la ley no considera
deducible.

Creo que este ejemplo es bastante global y abarca todas las posibles
dudas que el lector pueda tener.

\begin{center}\rule{0.5\linewidth}{0.5pt}\end{center}

\hypertarget{diferencias-temporarias-por-puxe9rdidas-a-compensar}{%
\subsection{Diferencias Temporarias por Pérdidas a
Compensar}\label{diferencias-temporarias-por-puxe9rdidas-a-compensar}}

Este subapartado explica cómo se tratan las pérdidas fiscales de
ejercicios anteriores que están pendientes de compensar con beneficios
futuros.

\begin{itemize}
\tightlist
\item
  \textbf{Naturaleza}: Las pérdidas de ejercicios anteriores se
  consideran un tipo de \textbf{diferencia temporaria}. Esto se debe a
  que, cuando una empresa incurre en pérdidas fiscales, se genera un
  derecho (un crédito fiscal) que solo podrá aplicarse (revertir) cuando
  la empresa obtenga beneficios fiscales en el futuro.
\item
  \textbf{Cuenta Contable}: La cuenta principal para reflejar este
  derecho es la \textbf{(4745) Crédito por pérdidas a compensar del
  ejercicio}. Esta cuenta recoge el importe de la reducción del Impuesto
  sobre Beneficios que se espera pagar en el futuro, gracias a la
  compensación de estas bases imponibles negativas pendientes.
\item
  \textbf{Reconocimiento del Activo por Impuesto Diferido}:

  \begin{itemize}
  \tightlist
  \item
    Siguiendo el \textbf{principio de prudencia}, un activo por impuesto
    diferido derivado de estas pérdidas solo se reconocerá si es
    \textbf{probable} que la empresa genere ganancias fiscales futuras
    suficientes para poder aplicar estos créditos.
  \item
    Siempre que se cumpla la condición anterior, se reconocerá un activo
    por: el derecho a compensar pérdidas fiscales en ejercicios
    posteriores, por las diferencias temporarias deducibles, y por las
    deducciones y otras ventajas fiscales no utilizadas que queden
    pendientes de aplicar fiscalmente.
  \item
    Si existe un historial de pérdidas continuadas, se presume (salvo
    prueba en contrario) que no es probable obtener dichas ganancias
    futuras que permitan compensar las bases imponibles negativas en un
    plazo no superior al previsto por la legislación tributaria, lo que
    limitaría el reconocimiento de este activo.
  \end{itemize}
\item
  \textbf{Registro Contable}:

  \begin{itemize}
  \tightlist
  \item
    \textbf{Origen del crédito fiscal (al surgir la pérdida compensable
    y ser probable su aprovechamiento)}:
    \texttt{XXX\ (4745)\ Crédito\ por\ pérdidas\ a\ compensar\ del\ ejercicio\ \ \ \ \ \ \ \ \ \ \ \ \ \ \ a\ (6301)\ Impuesto\ diferido\ XXX}
  \item
    \textbf{Reversión del crédito fiscal (al compensar las pérdidas con
    beneficios futuros)}:
    \texttt{XXX\ (6301)\ Impuesto\ diferido\ \ \ \ \ \ \ \ \ \ \ \ \ \ \ a\ (4745)\ Crédito\ por\ pérdidas\ a\ compensar\ del\ ejercicio\ XXX}
  \end{itemize}
\end{itemize}

\hypertarget{estados-financieros}{%
\chapter{Estados Financieros}\label{estados-financieros}}

\hypertarget{los-estados-financieros-en-el-pgc}{%
\section{Los estados financieros en el
PGC}\label{los-estados-financieros-en-el-pgc}}

\hypertarget{definiciuxf3n-objetivo-y-tipologuxeda}{%
\subsection{Definición, objetivo y
tipología}\label{definiciuxf3n-objetivo-y-tipologuxeda}}

Los estados financieros constituyen una representación estructurada de
la \textbf{situación financiera} y del \textbf{rendimiento financiero}
de una entidad.

EL objetivo es \textbf{proporcionar} a los distintos usuarios
información sobre la situación financiera para que sea \emph{útil para
la toma de decisiones}.

Como venimos viendo en las asignaturas anteriores de Contabilidad, este
tema se enfoca en la información útil para la toma de decisiones.

La legislación mercantil obliga a las empresas a elaborar un conjunto de
estados financieros denominados \textbf{cuentas anuales}. Sabemos que el
\emph{Marco conceptula} posee los cinco documentos que las integran.

\hypertarget{normas-para-la-elaboraciuxf3n-de-las-cuentas-anuales-y-formatos-de-presentaciuxf3n}{%
\subsection{Normas para la elaboración de las cuentas anuales y formatos
de
presentación}\label{normas-para-la-elaboraciuxf3n-de-las-cuentas-anuales-y-formatos-de-presentaciuxf3n}}

\begin{longtable}[]{@{}
  >{\raggedright\arraybackslash}p{(\columnwidth - 2\tabcolsep) * \real{0.4359}}
  >{\raggedright\arraybackslash}p{(\columnwidth - 2\tabcolsep) * \real{0.5641}}@{}}
\toprule\noalign{}
\begin{minipage}[b]{\linewidth}\raggedright
\textbf{NECA (PGC)}
\end{minipage} & \begin{minipage}[b]{\linewidth}\raggedright
\textbf{NECA (PGC Pymes)}
\end{minipage} \\
\midrule\noalign{}
\endhead
\bottomrule\noalign{}
\endlastfoot
1. Documentos que integran las CCAA & 1. Documentos que integran las
CCAA \\
2. Formulación de las CCAA & 2. Formulación de las CCAA \\
3. Estructura de las CCAA & 3. Estructura de las CCAA \\
4. Cuentas anuales abreviadas & 4. Normas comunes al balance, la cuenta
de pérdidas y ganancias y el estado de cambios en el patrimonio neto \\
5. Normas comunes al balance, la cuenta de pérdidas y ganancias, el
estado de cambios en el patrimonio neto y el estado de flujos de
efectivo & 5. Balance \\
6. Balance & 6. Cuenta de pérdidas y ganancias \\
7. Cuentas de pérdidas y ganancias & 7. Estado de cambios en el
patrimonio neto \\
8. Estado de cambios en el patrimonio neto & 8. Memoria \\
9. Estado de flujos de efectivo & 9. Número medio de trabajadores \\
10. Memoria & 10. Empresas del grupo, multigrupo y asociadas \\
11. Cifra anual de negocios & 11. Estados financieros intermedios \\
12. Número medio de trabajadores & 12. Partes vinculadas \\
13. Empresas del grupo, multigrupo y asociadas & \\
14. Estados financieros intermedios & \\
15. Partes vinculadas & \\
\end{longtable}

\hypertarget{diferencias-entre-el-pgc-ruxe9gimen-general-y-el-pgc-pymes}{%
\subsection{Diferencias entre el PGC Régimen General y el PGC
Pymes}\label{diferencias-entre-el-pgc-ruxe9gimen-general-y-el-pgc-pymes}}

\begin{longtable}[]{@{}
  >{\raggedright\arraybackslash}p{(\columnwidth - 2\tabcolsep) * \real{0.4014}}
  >{\raggedright\arraybackslash}p{(\columnwidth - 2\tabcolsep) * \real{0.5986}}@{}}
\toprule\noalign{}
\begin{minipage}[b]{\linewidth}\raggedright
\textbf{NECA (PGC Régimen General)}
\end{minipage} & \begin{minipage}[b]{\linewidth}\raggedright
\textbf{NECA (PGC Pymes)}
\end{minipage} \\
\midrule\noalign{}
\endhead
\bottomrule\noalign{}
\endlastfoot
Balance (normal y abreviado) & Balance \\
Cuenta de pérdidas y ganancias (normal y abreviada) & Cuenta de pérdidas
y ganancias \\
Memoria (normal y abreviada) & Memoria \\
Estado de Cambios en el Patrimonio Neto & Estado de Cambios en el
Patrimonio Neto (optativo) \\
Estado de Flujos de efectivo & El Estado de Flujos de efectivo es
optativo, se seguirá, en su caso, el formato establecido en el PGC \\
\end{longtable}

Vemos que se hace distinción repecto de las Pymes. Esto se debe a que
ciertas partes de su PGC varían y son optativas, como es el caso de
\emph{formular el estado de flujos de efectivo}.

\hypertarget{documentos-que-integran-las-cuentas-anuales}{%
\subsection{Documentos que integran las cuentas
anuales}\label{documentos-que-integran-las-cuentas-anuales}}

Como hemos estudiado en asignaturas anteriores estas son: 1. Balance 2.
Cuenta de Pérdidas y Ganancias. 3. Estado de cambios en el PN. 4. Estado
de flujos de efectivo. 5. Memoria.

Cada documento que conformas las cuentas anuales se ha de redactar de
conformidad con lo previsto en la legislación mercantil.

\hypertarget{formulaciuxf3n-de-las-cuentas-anuales}{%
\subsection{Formulación de las cuentas
anuales}\label{formulaciuxf3n-de-las-cuentas-anuales}}

\begin{itemize}
\tightlist
\item
  Periodicidad, responsabilidad,fecha y firma, identificabilidad y
  valores.

  \begin{itemize}
  \tightlist
  \item
    Periodicidad: las cuentas anuales, por norma general, se han de
    elaborar con una periodicidad de doce meses, salvo en determinados
    casos.
  \item
    Responsabilidad: éstas deben de ser formuladas por el
    \emph{empresario o administradores}, quienes \textbf{responderán de
    su veracidad}.
  \item
    Fecha y firma: deben de ser firmadas por el empresario y en ellas
    debe de aparecer la fecha.

    \begin{itemize}
    \tightlist
    \item
      \textbf{Plazos para la formulación, aprobación y depósito de las
      cuentas anuales}:

      \begin{itemize}
      \tightlist
      \item
        \textbf{Formulación}: Las cuentas anuales deben ser formuladas
        por los administradores de la empresa en un plazo máximo de tres
        meses desde el cierre del ejercicio social.
      \item
        \textbf{Aprobación}: Una vez formuladas, las cuentas anuales
        deben ser aprobadas por la junta general de socios o accionistas
        en un plazo máximo de seis meses desde el cierre del ejercicio
        social.
      \item
        \textbf{Depósito}: Tras su aprobación, las cuentas anuales deben
        ser depositadas en el Registro Mercantil correspondiente en el
        plazo de un mes desde la fecha de su aprobación.
      \end{itemize}
    \end{itemize}

    Estos plazos son esenciales para garantizar el cumplimiento de las
    obligaciones legales y la transparencia en la información financiera
    de las empresas.
  \item
    Identificabilidad: todas las cuentas deben de estar identificadas de
    forma \emph{clara} aportando la información necesaria en cada una.
  \item
    Valores: Se elaborarán siguiendo sus valores expresados en euros.
  \end{itemize}
\item
  Obligación de auditar las cuentas anuales y excepción por razón de
  tamaño. Las cuentas han de ser revisadas por un \textbf{auditor de
  cuentas}.

  \begin{itemize}
  \item
    Excepción por tamaño a la obligación de auditar las cuentas anuales

    \begin{longtable}[]{@{}
      >{\raggedright\arraybackslash}p{(\columnwidth - 2\tabcolsep) * \real{0.6912}}
      >{\raggedright\arraybackslash}p{(\columnwidth - 2\tabcolsep) * \real{0.3088}}@{}}
    \toprule\noalign{}
    \begin{minipage}[b]{\linewidth}\raggedright
    \textbf{Criterios para la excepción}
    \end{minipage} & \begin{minipage}[b]{\linewidth}\raggedright
    \textbf{Límite}
    \end{minipage} \\
    \midrule\noalign{}
    \endhead
    \bottomrule\noalign{}
    \endlastfoot
    Total Activos & \(\leq\) 2.850.000 € \\
    Cifra de Negocios & \(\leq\) 5.700.000 € \\
    Número medio de trabajadores & \(\leq\) 50 \\
    \end{longtable}
  \end{itemize}
\end{itemize}

\hypertarget{estructura-de-las-cuentas-anuales-y-formatos-para-su-presentaciuxf3n}{%
\subsection{Estructura de las cuentas anuales y formatos para su
presentación}\label{estructura-de-las-cuentas-anuales-y-formatos-para-su-presentaciuxf3n}}

Se consideran dos formatos:

\begin{itemize}
\item
  Régimen General del PGC
\item
  \textbf{Formato Normal:}

  \begin{itemize}
  \tightlist
  \item
    Es el modelo general que deben utilizar las sociedades anónimas,
    limitadas, en comandita por acciones, sociedades colectivas,
    comanditarias simples y cooperativas.
  \item
    Las entidades de interés público siempre deben utilizar el formato
    normal.
  \item
    Incluye modelos para el Balance, Cuenta de Pérdidas y Ganancias,
    Memoria, Estado de Cambios en el Patrimonio Neto y Estado de Flujos
    de Efectivo.
  \end{itemize}
\item
  \textbf{Formato Abreviado:}

  \begin{itemize}
  \tightlist
  \item
    Es la obligación mínima para empresarios individuales y empresas con
    otras formas societarias no mencionadas para el formato normal.
  \item
    Una sociedad mercantil que generalmente usaría el formato normal
    puede optar por el abreviado si durante dos ejercicios consecutivos
    no supera dos de los siguientes límites:

    \begin{itemize}
    \tightlist
    \item
      \textbf{Para Balance y Memoria abreviados}:

      \begin{itemize}
      \tightlist
      \item
        Total de activo: 4.000.000 €.
      \item
        Importe neto de la cifra anual de negocio: 8.000.000 €.
      \item
        Número medio de trabajadores: 50.
      \end{itemize}
    \item
      \textbf{Para Cuenta de Pérdidas y Ganancias abreviada}:

      \begin{itemize}
      \tightlist
      \item
        Total de activo: 11.400.000 €.
      \item
        Importe neto de la cifra anual de negocio: 22.800.000 €.
      \item
        Número medio de trabajadores: 250.
      \end{itemize}
    \end{itemize}
  \item
    En el primer ejercicio desde su constitución, transformación o
    fusión, las sociedades pueden formular cuentas anuales abreviadas si
    al cierre de ese primer ejercicio no superan dos de los límites
    mencionados.
  \item
    Se pierde la posibilidad de usar formatos abreviados si durante dos
    ejercicios consecutivos se superan dos de los límites.
  \item
    Si se puede formular balance y memoria abreviados, el Estado de
    Cambios en el Patrimonio Neto y el Estado de Flujos de Efectivo no
    son obligatorios.
  \end{itemize}
\item
  PGC para PYMES
\item
  \textbf{Formato para PYMES:}

  \begin{itemize}
  \tightlist
  \item
    Las PYMES pueden presentar las cuentas anuales siguiendo el formato
    establecido en el PGC para PYMES, que es similar al modelo abreviado
    del régimen general, con ciertas diferencias en criterios de
    valoración y formulación.
  \item
    Una empresa puede acogerse al PGC para PYMES si durante dos
    ejercicios consecutivos, a fecha de cierre de cada uno, no supera
    dos de los tres límites siguientes:

    \begin{itemize}
    \tightlist
    \item
      Total de activo: 4.000.000 €.
    \item
      Importe neto de la cifra anual de negocios: 8.000.000 €.
    \item
      Número medio de trabajadores: 50.
    \end{itemize}
  \item
    Se pierde la facultad de aplicar el PGC para PYMES si durante dos
    ejercicios consecutivos se superan dos de los límites anteriores.
  \item
    El PGC para PYMES incluye un único formato para el Balance, la
    Cuenta de Pérdidas y Ganancias, la Memoria y el Estado de Cambios en
    el Patrimonio Neto (este último es optativo).
  \item
    El Estado de Flujos de Efectivo es optativo para las PYMES; si
    deciden formularlo, deben usar el formato del PGC general.
  \item
    No pueden aplicar el PGC de PYMES:

    \begin{itemize}
    \tightlist
    \item
      Entidades de interés público.
    \item
      Empresas que forman parte de un grupo obligado a formular cuentas
      anuales consolidadas.
    \item
      Empresas cuya moneda funcional sea distinta del euro.
    \item
      Entidades financieras que captan fondos del público y las que
      asuman la gestión de las anteriores.
    \end{itemize}
  \item
    Las PYMES que opten por aplicar la versión específica para PYMES
    deben mantener esta opción de forma continuada al menos durante tres
    años, salvo que superen los límites.
  \end{itemize}
\end{itemize}

\hypertarget{normas-comunes-al-balance-cuenta-de-puxe9rdidas-y-ganancias-estado-de-cambios-en-el-patrimonio-neto-y-estado-de-flujos-de-efectivo}{%
\subsection{Normas Comunes al Balance, Cuenta de Pérdidas y Ganancias,
Estado de Cambios en el Patrimonio Neto y Estado de Flujos de
Efectivo}\label{normas-comunes-al-balance-cuenta-de-puxe9rdidas-y-ganancias-estado-de-cambios-en-el-patrimonio-neto-y-estado-de-flujos-de-efectivo}}

\begin{itemize}
\tightlist
\item
  \textbf{Comparabilidad de Cifras}: En cada partida deben figurar las
  cifras del ejercicio actual (n) y las del ejercicio anterior (n-1). Si
  no son comparables (por cambios en estructura, criterio contable o
  corrección de error), se debe adaptar el ejercicio precedente y
  explicarlo en la memoria.
\item
  \textbf{Partidas sin Importe}: No se incluirán partidas sin importe en
  el ejercicio actual ni en el precedente.
\item
  \textbf{Uniformidad de Criterios}: Los criterios de contabilización no
  deben modificarse de un ejercicio a otro, salvo casos excepcionales
  que se justificarán en la memoria.
\item
  \textbf{Flexibilidad en Partidas}:

  \begin{itemize}
  \tightlist
  \item
    Se pueden añadir nuevas partidas a las previstas.
  \item
    Se puede realizar una subdivisión más detallada de las partidas
    existentes.
  \item
    Se pueden agrupar partidas (precedidas de números árabes en balance
    y estado de cambios en el patrimonio neto, o letras en la cuenta de
    pérdidas y ganancias y estado de flujos de efectivo) si su importe
    es irrelevante para la imagen fiel o si favorece la claridad.
  \end{itemize}
\item
  \textbf{Referencias Cruzadas}: Cada partida, cuando proceda, tendrá
  una referencia cruzada a la información correspondiente en la memoria.
\item
  \textbf{Empresas del Grupo y Asociadas}: Los créditos, deudas,
  ingresos y gastos con empresas del grupo y asociadas figurarán en las
  partidas correspondientes, pero separadas del resto.
\item
  \textbf{Negocios Conjuntos sin Personalidad Jurídica}: Las empresas
  que participen en negocios conjuntos (como UTEs, comunidades de
  bienes, etc.) deben presentar esta información según la norma de
  registro y valoración relativa a negocios conjuntos.
\end{itemize}

\hypertarget{balance-de-situaciuxf3n}{%
\subsection{Balance de Situación}\label{balance-de-situaciuxf3n}}

El Balance de Situación es el estado contable que refleja los recursos
económicos (Activo) y financieros (Pasivo y Patrimonio Neto) de la
empresa en un momento dado. Su objetivo es mostrar la situación
financiera de la entidad.

\begin{itemize}
\tightlist
\item
  \textbf{Estructura y Clasificación}:

  \begin{itemize}
  \tightlist
  \item
    Las partidas se clasifican en \textbf{corrientes} (vinculadas al
    ciclo normal de explotación, generalmente inferior a un año, o
    mantenidas para negociar) y \textbf{no corrientes} (el resto, con
    permanencia superior a un año). Esto aplica tanto al Activo como al
    Pasivo.
  \item
    Los activos y pasivos financieros pueden presentarse por su importe
    neto si existe el derecho exigible de compensarlos y la intención de
    liquidarlos por el neto.
  \item
    Las correcciones valorativas por deterioro y las amortizaciones
    acumuladas minoran la partida del activo correspondiente,
    presentándose los activos por su valor neto contable.
  \end{itemize}
\item
  \textbf{Partidas Específicas Destacadas}:

  \begin{itemize}
  \tightlist
  \item
    \textbf{Inversiones Inmobiliarias}: Terrenos o construcciones
    destinados a obtener rentas o plusvalías fuera del curso ordinario
    de las operaciones.
  \item
    \textbf{Capital Social y Prima de Emisión}: Deben figurar en el
    patrimonio neto si la inscripción en el Registro Mercantil es
    anterior a la formulación de las cuentas; de lo contrario, irán a
    deudas a corto plazo.
  \item
    \textbf{Instrumentos de Patrimonio Propios}: Se registran en el
    Patrimonio Neto, generalmente con signo negativo.
  \item
    \textbf{Subvenciones, Donaciones y Legados no Reintegrables}: Las
    otorgadas por terceros distintos a socios forman parte del
    patrimonio neto en una sub-agrupación específica; las otorgadas por
    socios se incluyen en fondos propios.
  \item
    \textbf{Activos no Corrientes Mantenidos para la Venta y Pasivos
    Vinculados}: Se presentan de forma separada en el activo y pasivo,
    respectivamente, y no se compensan.
  \end{itemize}
\end{itemize}

\hypertarget{cuenta-de-puxe9rdidas-y-ganancias}{%
\subsection{Cuenta de Pérdidas y
Ganancias}\label{cuenta-de-puxe9rdidas-y-ganancias}}

La Cuenta de Pérdidas y Ganancias (o Cuenta de Resultados) muestra el
total de ingresos y gastos reconocidos por la empresa durante un
ejercicio económico, con el fin de determinar el resultado del periodo
(beneficio o pérdida), excepto aquellos que se imputen directamente al
patrimonio neto.

\begin{itemize}
\tightlist
\item
  \textbf{Estructura por Resultados Parciales}:

  \begin{itemize}
  \tightlist
  \item
    \textbf{Resultado de Explotación}: Diferencia entre ingresos y
    gastos de las actividades ordinarias y no financieras.
  \item
    \textbf{Resultado Financiero}: Diferencia entre ingresos y gastos
    derivados de instrumentos financieros.
  \item
    \textbf{Resultado antes de Impuestos}: Suma de los dos anteriores.
  \item
    \textbf{Resultado del Ejercicio Procedente de Operaciones
    Continuadas}: Resultado antes de impuestos menos el impuesto sobre
    beneficios.
  \item
    \textbf{Resultado del Ejercicio Procedente de Operaciones
    Interrumpidas}: Resultado (neto de impuestos) de componentes de la
    empresa que se han enajenado o clasificado como mantenidos para la
    venta y representan una línea de negocio o área geográfica
    principal.
  \item
    \textbf{Resultado del Ejercicio}: Suma del resultado de operaciones
    continuadas e interrumpidas; esta cifra se refleja en el patrimonio
    neto.
  \end{itemize}
\item
  \textbf{Principios de Formulación Clave}:

  \begin{itemize}
  \tightlist
  \item
    Los ingresos y gastos se clasifican según su \textbf{naturaleza}.
  \item
    Ventas, prestaciones de servicios y otros ingresos de explotación se
    reflejan por su \textbf{importe neto} de devoluciones y descuentos.
  \item
    Las \textbf{subvenciones} se imputan a resultados según su finalidad
    (explotación, financiación de activos no financieros, o de carácter
    financiero).
  \item
    Los \textbf{excesos de provisiones} (reversiones) se recogen en una
    partida específica, con excepciones para las de personal y
    operaciones comerciales.
  \item
    Ingresos o gastos de \textbf{carácter excepcional y cuantía
    significativa} (ej. inundaciones, multas) se pueden agrupar en
    ``Otros resultados'' dentro del resultado de explotación.
  \end{itemize}
\end{itemize}

\hypertarget{la-memoria}{%
\subsection{La Memoria}\label{la-memoria}}

La Memoria es un documento que completa, amplía y comenta la información
contenida en los otros estados financieros que integran las cuentas
anuales. Su principal función es aportar información, en su mayoría
cualitativa, para facilitar la comprensión de la situación patrimonial,
financiera y los resultados de la empresa.

\begin{itemize}
\tightlist
\item
  \textbf{Objetivo}: Aclarar y detallar la información sintética de los
  otros estados para que las cuentas anuales reflejen la imagen fiel de
  la empresa.
\item
  \textbf{Contenido}:

  \begin{itemize}
  \tightlist
  \item
    Incluye información mínima a cumplimentar; los apartados no
    significativos pueden omitirse.
  \item
    Debe incluir cualquier otra información relevante no solicitada
    explícitamente si es necesaria para la imagen fiel, incluyendo datos
    cualitativos del ejercicio anterior si son significativos.
  \item
    Incorpora información que otras normativas exijan incluir.
  \item
    La información cuantitativa se refiere al ejercicio actual y al
    anterior (comparativo), salvo que una norma contable indique lo
    contrario.
  \item
    Un ejemplo de información específica es el período medio de pago a
    proveedores para empresas que elaboren la memoria en modelo normal.
  \end{itemize}
\end{itemize}

\hypertarget{estado-de-flujos-de-efectivo-efe}{%
\section{Estado de Flujos de Efectivo
(EFE)}\label{estado-de-flujos-de-efectivo-efe}}

El Estado de Flujos de Efectivo informa sobre el origen y la utilización
de los activos monetarios representativos de efectivo y otros activos
líquidos equivalentes, clasificando los movimientos por actividades e
indicando la variación neta de dicha magnitud en el ejercicio.

\begin{itemize}
\tightlist
\item
  \textbf{Concepto y Características}:

  \begin{itemize}
  \tightlist
  \item
    \textbf{Objetivo}: Mostrar los cobros y pagos realizados por la
    empresa para explicar la variación de la tesorería.
  \item
    \textbf{Efectivo y equivalentes}: Incluye la tesorería en caja y
    bancos a la vista, instrumentos financieros convertibles en efectivo
    con vencimiento no superior a tres meses desde su adquisición (sin
    riesgo significativo de cambio de valor y parte de la gestión normal
    de tesorería), y en algunos casos, descubiertos bancarios
    ocasionales.
  \item
    \textbf{Características}: Proporciona magnitudes objetivas (cobros y
    pagos), informa sobre las causas de las variaciones de tesorería,
    aumenta la transparencia para la toma de decisiones y es de fácil
    comprensión.
  \end{itemize}
\item
  \textbf{Clasificación de los Flujos de Efectivo}:

  \begin{itemize}
  \tightlist
  \item
    \textbf{Actividades de Explotación}: Principalmente los ocasionados
    por las actividades que constituyen la principal fuente de ingresos
    de la empresa y otras no calificables como inversión o financiación.
    Incluye pagos de intereses, cobros de intereses y dividendos, y el
    pago del impuesto de beneficios.
  \item
    \textbf{Actividades de Inversión}: Pagos por adquisición de activos
    no corrientes (intangibles, materiales, inversiones inmobiliarias,
    financieras) y otros activos no equivalentes, así como los cobros
    por su enajenación o amortización.
  \item
    \textbf{Actividades de Financiación}: Cobros de la emisión de
    títulos o recursos de terceros (préstamos) y los pagos por su
    amortización o devolución. También incluye los pagos de dividendos a
    los accionistas.
  \end{itemize}
\item
  \textbf{Metodología de Cálculo}:

  \begin{itemize}
  \tightlist
  \item
    \textbf{Flujos de Explotación}: El PGC utiliza un método mixto.

    \begin{itemize}
    \tightlist
    \item
      Se parte del ``Resultado del ejercicio antes de impuestos''.
    \item
      Se realizan ``Ajustes del resultado'' para eliminar partidas que
      no suponen movimiento de efectivo (ej. amortizaciones, deterioros,
      resultados por bajas de inmovilizado, ingresos y gastos
      financieros que se tratarán por separado) y partidas cuyos flujos
      se clasifican como de inversión o financiación.
    \item
      Se consideran los ``Cambios en el capital corriente'' (variaciones
      en existencias, deudores, acreedores, etc.).
    \item
      Finalmente, se incorporan de forma directa otros flujos como pagos
      de intereses, cobros de dividendos e intereses, y cobros/pagos por
      impuesto sobre beneficios.
    \end{itemize}
  \item
    \textbf{Flujos de Inversión y Financiación}: Se utiliza el método
    directo, presentando los cobros y pagos brutos asociados a estas
    actividades.

    \begin{itemize}
    \tightlist
    \item
      \textbf{Inversión}: Pagos por adquisiciones y cobros por
      desinversiones de activos no corrientes.
    \item
      \textbf{Financiación}: Cobros por emisión de instrumentos de
      patrimonio o pasivo financiero, y pagos por su amortización o por
      dividendos.
    \end{itemize}
  \end{itemize}
\item
  \textbf{Consideraciones Adicionales}:

  \begin{itemize}
  \tightlist
  \item
    Los flujos de activos y pasivos financieros de alta rotación (plazo
    entre adquisición y vencimiento no superior a seis meses) pueden
    presentarse netos.
  \item
    Las transacciones en moneda extranjera se convierten al tipo de
    cambio de la fecha del flujo o una media ponderada.
  \item
    Se debe informar sobre saldos significativos de efectivo no
    disponibles para ser utilizados.
  \end{itemize}
\item
  \textbf{Utilidad}:

  \begin{itemize}
  \tightlist
  \item
    Evalúa la capacidad de la empresa para generar flujos de efectivo.
  \item
    Permite analizar la política de dividendos y la capacidad para
    atender deudas.
  \item
    Ayuda a comprender la relación entre el resultado contable y los
    movimientos reales de tesorería.
  \item
    Facilita la proyección de futuros flujos de efectivo.
  \end{itemize}
\end{itemize}

\hypertarget{estado-de-cambios-en-el-patrimonio-neto-ecpn}{%
\section{Estado de Cambios en el Patrimonio Neto
(ECPN)}\label{estado-de-cambios-en-el-patrimonio-neto-ecpn}}

El Estado de Cambios en el Patrimonio Neto tiene como objetivo mostrar
las variaciones ocurridas en el patrimonio de la empresa durante el
ejercicio. Es fundamental para que los accionistas y otros interesados
puedan evaluar la evolución de la riqueza de la empresa.

Este estado se compone de dos partes principales:

\begin{enumerate}
\def\labelenumi{\arabic{enumi}.}
\tightlist
\item
  \textbf{Estado de Ingresos y Gastos Reconocidos (EIGR)}:

  \begin{itemize}
  \tightlist
  \item
    Recoge los cambios en el patrimonio neto derivados de:

    \begin{itemize}
    \tightlist
    \item
      El \textbf{resultado del ejercicio} obtenido de la Cuenta de
      Pérdidas y Ganancias.
    \item
      Los \textbf{ingresos y gastos} que, según las normas de registro y
      valoración, deben imputarse \textbf{directamente al patrimonio
      neto} (por ejemplo, ciertas valoraciones de instrumentos
      financieros, coberturas de flujos de efectivo, subvenciones de
      capital recibidas, diferencias de conversión, y el efecto
      impositivo de estas partidas).
    \item
      Las \textbf{transferencias realizadas a la Cuenta de Pérdidas y
      Ganancias} de partidas que previamente se habían reconocido en el
      patrimonio neto (por ejemplo, cuando se realiza un activo
      financiero valorado con cambios en el patrimonio neto, o la
      imputación de subvenciones).
    \end{itemize}
  \end{itemize}
\item
  \textbf{Estado Total de Cambios en el Patrimonio Neto (ETCPN)}:

  \begin{itemize}
  \tightlist
  \item
    Informa de \textbf{todos los cambios} habidos en el patrimonio neto,
    detallando:

    \begin{itemize}
    \tightlist
    \item
      El \textbf{saldo total de los ingresos y gastos reconocidos} (que
      es el resultado final del EIGR).
    \item
      Las \textbf{variaciones originadas por operaciones con los socios
      o propietarios} cuando actúen como tales (ej. ampliaciones o
      reducciones de capital, distribución de dividendos, operaciones
      con acciones propias).
    \item
      Las \textbf{restantes variaciones} que se produzcan en el
      patrimonio neto (ej. traspaso del resultado del ejercicio a
      reservas, aplicación de resultados de ejercicios anteriores).
    \item
      Los \textbf{ajustes al patrimonio neto debidos a cambios en
      criterios contables y correcciones de errores} de ejercicios
      anteriores. Estos ajustes se presentan modificando los saldos
      iniciales del patrimonio del ejercicio más antiguo presentado a
      efectos comparativos.
    \end{itemize}
  \end{itemize}
\end{enumerate}

\begin{itemize}
\tightlist
\item
  \textbf{Funciones y Utilidad}:

  \begin{itemize}
  \tightlist
  \item
    Permite a los accionistas conocer cómo ha variado su inversión y la
    política de la empresa respecto a dividendos y autofinanciación.
  \item
    Informa sobre los efectos de cambios en políticas contables o
    correcciones de errores.
  \item
    Detalla las modificaciones en el valor de activos que afectan
    directamente al patrimonio.
  \item
    Muestra las pérdidas y beneficios no realizados (por ejemplo, por
    ajustes a valor razonable de ciertos instrumentos financieros o por
    operaciones de cobertura).
  \item
    Sirve de base para que otros interesados (acreedores, empleados,
    Administración) evalúen la solvencia y garantía que ofrece el
    patrimonio de la empresa.
  \item
    Ayuda a analizar el excedente generado por la empresa, distinguiendo
    entre el realizado y el no realizado.
  \end{itemize}
\end{itemize}

\backmatter
\end{document}
