\documentclass[12pt]{article}
\usepackage[a4paper, margin=1in]{geometry}
\usepackage{amsmath, amssymb, amsfonts}
\usepackage{graphicx}
\usepackage{setspace}
\usepackage{xcolor}
\usepackage{lmodern}

% Fuente principal
\renewcommand{\rmdefault}{ptm} % Usa Times New Roman
\usepackage{titlesec}
\titleformat{\section}{\bfseries\Large\sffamily\color{blue!70!black}}{\thesection.}{1em}{}

% Configuración de listas
\usepackage{enumitem}
\setlist{nosep, leftmargin=2em}

% Portada creativa
\newcommand{\makecover}{
    \begin{titlepage}
        \centering
        {\Huge\bfseries\sffamily\color{blue!70!black} Preguntas y Respuestas}\par
        \vspace{2cm}
        \includegraphics[width=0.4\textwidth]{build/IMAGEN_FR.jpg}\par
        \vspace{2cm}
        \large \textit{Un compendio de preguntas desarrolladas paso a paso.}\par
        \vfill
        \textbf{Autor:} Ismael Sallami Moreno\\
        \textbf{Fecha:} \today
        \vspace{1cm}
    \end{titlepage}
}

\begin{document}

% Portada
\makecover

\tableofcontents
\newpage

\section{Primera Pregunta}
Describa el funcionamiento de los protocolos POP3 e IMAP, para qué son utilizados y las
diferencias de funcionamiento entre ellos.

\section{Segunda Pregunta}
Explique las diferencias que hay entre el control de congestión y el control de flujo

\section{Tercera Pregunta}
Identifique los niveles del modelo OSI y explique brevemente la funcionalidad de cada
nivel.

\section{Cuarta Pregunta}
(0,75 ptos) Describa los mensajes generados desde un equipo correctamente configurado para acceder
a Internet desde que solicita una URL en el navegador hasta que se muestra la página web completa.

\section{Quinta Pregunta}
(0,75 ptos) Suponga que A y B tienen sus correspondientes KPUB\_A/KPRIV\_A y KPUB\_B/KPRIV\_B y una
autoridad tiene sus KPUB\_AUT/KPRIV\_AUT. Explique cómo y qué primitivas se cumplirían en una
comunicación segura entre A y B usando dichas claves.

\section{Sexta Pregunta}
(0,75 ptos) Se tiene un paquete de 5KB de los que 14 bytes son de cabecera de nivel de Ethernet, 20
de nivel IP y 8 de nivel UDP. Este paquete debe pasar por una red Ethernet con una MTU de
1514Bytes. Si se precisa su fragmentación ¿Cuántos paquetes se generarían y con qué tamaños?

\section{Séptima Pregunta}
(0,75 ptos) Discuta uso y secuencia de paquetes de protocolo DHCP.

\section{Octava Pregunta}
(0,75 ptos) El ejercicio 2 del boletín de ejercicios 2 resuelto, se trata de describir whatsapp, youtube, ...

\section{Novena Pregunta}
Lo mismo que la anterior pero con vide/audio, ...


\end{document}
