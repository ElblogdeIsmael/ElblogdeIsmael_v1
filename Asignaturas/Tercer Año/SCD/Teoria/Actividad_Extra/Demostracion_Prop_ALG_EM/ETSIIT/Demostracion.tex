\documentclass[a4paper,12pt]{article}

% Paquetes básicos
\usepackage[utf8]{inputenc}
\usepackage[T1]{fontenc}
\usepackage[spanish]{babel}
\usepackage{graphicx}
\usepackage{xcolor}
\usepackage{lipsum}
\usepackage{geometry}
\geometry{top=3cm, bottom=3cm, left=2.5cm, right=2.5cm}

% Paquetes para diseño
\usepackage{titlesec}
\usepackage{fancyhdr}
\usepackage{amsmath}
\usepackage{amssymb}
\usepackage{hyperref}

% Paquetes para el entorno lstlisting
\usepackage{listings}
\usepackage{inconsolata}

% Paquete para fondo
\usepackage{background}

% Configuración de lstlisting
\lstset{
    inputencoding=utf8,          % Permite UTF-8
    extendedchars=true,          % Reconoce caracteres extendidos
    literate=                    % Configuración manual para tildes y símbolos
        {á}{{\'a}}1
        {é}{{\'e}}1
        {í}{{\'i}}1
        {ó}{{\'o}}1
        {ú}{{\'u}}1
        {ñ}{{\~n}}1
        {Á}{{\'A}}1
        {É}{{\'E}}1
        {Í}{{\'I}}1
        {Ó}{{\'O}}1
        {Ú}{{\'U}}1
        {Ñ}{{\~N}}1
        {¿}{{\textquestiondown}}1
        {¡}{{\textexclamdown}}1,
    basicstyle=\ttfamily,        % Fuente monoespaciada
    breaklines=true,             % Habilita salto de línea automático
    frame=single,                % Marco alrededor del código
    backgroundcolor=\color{gray!10}, % Fondo gris claro
    keywordstyle=\color{blue},   % Color para palabras clave
    commentstyle=\color{green},  % Color para comentarios
    stringstyle=\color{red}      % Color para strings
}
\lstdefinestyle{customcpp}{
    language=C++,                % Lenguaje de programación
    showspaces=false,            % No mostrar espacios
    showtabs=false,              % No mostrar tabulaciones
    tabsize=4,                   % Tamaño de tabulación
    showstringspaces=false,      % No mostrar espacios en strings
    numbers=left,                % Números de línea a la izquierda
    numberstyle=\tiny\color{gray}, % Estilo de los números de línea
    numbersep=5pt,               % Separación de los números de línea
    stepnumber=1,                % Mostrar número en cada línea
    basicstyle=\ttfamily\footnotesize, % Estilo básico del código
    keywordstyle=\bfseries\color{blue}, % Estilo de las palabras clave
    commentstyle=\itshape\color{green!50!black}, % Estilo de los comentarios
    stringstyle=\color{red},     % Estilo de los strings
    identifierstyle=\color{black}, % Estilo de los identificadores
    % procnamekeys={def,class},    % Palabras clave para nombres de funciones
    morekeywords={constexpr,nullptr,size_t}, % Más palabras clave
    emph={int,char,double,float,unsigned}, % Palabras a enfatizar
    emphstyle=\color{magenta},   % Estilo de las palabras enfatizadas
    backgroundcolor=\color{gray!10}, % Color de fondo
    frame=shadowbox,             % Marco con sombra
    rulesepcolor=\color{gray},   % Color de la línea de separación
    breakatwhitespace=false,     % No cortar en espacios en blanco
    breaklines=true,             % Cortar líneas largas
    captionpos=b,                % Posición del título (abajo)
    escapeinside={(*@}{@*)},     % Delimitadores para escapar a LaTeX
    morecomment=[l][\color{magenta}]{\#}, % Comentarios de una línea
    morecomment=[s][\color{orange}]{/*}{*/}, % Comentarios multilínea
    morestring=[b]",             % Strings entre comillas dobles
    morestring=[b]'              % Strings entre comillas simples
}

% Configuración de título
\titleformat{\section}{\normalfont\Large\bfseries}{\thesection}{1em}{}

% Información del documento
\title{
    \vspace{-2cm}
    \includegraphics[width=0.2\textwidth]{images/etsiit.png} \\ % Cambia el logo si es necesario
    \LARGE \textbf{Demostración de Propiedades de Algoritmo de Exclusión Mutua} \\[0.5cm]
    \large \textbf{Sistemas Concurrentes y Distribuidos} \\[0.5cm]}
\author{
    \textbf{Ismael Sallami Moreno} \\
    \textit{Ingenería Informática + ADE} \\
    \textit{Universidad de Granada (UGR)} \\
}
\date{\today}

% Configuración del fondo
\backgroundsetup{
    scale=1,
    color=black,
    opacity=0.2,
    angle=0,
    position=current page.south,
    vshift=0pt,
    hshift=0pt,
    %contents={\includegraphics[width=\paperwidth,height=\paperheight,keepaspectratio]{images/granada.jpg}}
}

% Inicio del documento
\begin{document}

% Portada
\maketitle
\thispagestyle{empty}

% \begin{center}
%     \includegraphics[width=\textwidth,height=0.4\textheight,keepaspectratio]{images/algEM.jpg} \\ % Añade tu imagen de fondo
%     \vfill
% \end{center}

\newpage

% Índice (opcional)
\tableofcontents
\newpage

\section{Enunciado}

Supongamos un conjunto de $n$ procesos distribuidos y conectados en anillo. El algoritmo de exclusión mutua se basa en las siguientes reglas:

\begin{itemize}
    \item Tenemos los procesos $0, \ldots, n-1$. Cada proceso $i$ tiene una variable $s[i]$, inicializada a 0, que puede tomar los valores $0$ o $1$.
    \item Para que un proceso $i$ pueda entrar en la sección crítica (SC), deben cumplirse las siguientes condiciones:
    \begin{enumerate}
        \item Si $i > 0$: $s[i] \neq s[i-1]$.
        \item Si $i = 0$: $s[0] = s[n-1]$.
    \end{enumerate}
    \item Al salir de la sección crítica, el proceso $i$ ejecuta:
    \begin{enumerate}
        \item Si $i > 0$: $s[i] := s[i-1]$, siempre que antes de la asignación $s[i] \neq s[i-1]$.
        \item Si $i = 0$: $s[0] := (s[0] + 1) \mod 2$.
    \end{enumerate}
\end{itemize}

Se pide:

\begin{enumerate}
    \item Demostrar que el algoritmo permite únicamente a un proceso acceder a la sección crítica en cualquier configuración.
    \item Demostrar que cada proceso $p[i]$ conseguirá entrar en la sección crítica infinitamente a menudo.
\end{enumerate}

\section{Demostración}

% \subsection{Parte I: Exclusión mutua}

% \textbf{Condiciones de entrada a la SC:}
% \begin{itemize}
%     \item Para $i > 0$: $s[i] \neq s[i-1]$.
%     \item Para $i = 0$: $s[0] = s[n-1]$.
% \end{itemize}

% \textbf{Condiciones al salir de la SC:}
% \begin{itemize}
%     \item Para $i > 0$: $s[i] := s[i-1]$, si $s[i] \neq s[i-1]$.
%     \item Para $i = 0$: $s[0] := (s[0] + 1) \mod 2$.
% \end{itemize}

% \textbf{Demostración:}

% \begin{enumerate}
%     \item La entrada a la SC para un proceso $i$ depende únicamente de las variables $s[i]$ y $s[i-1]$:
%     \begin{itemize}
%         \item Si $i > 0$, la condición $s[i] \neq s[i-1]$ asegura que no puede haber dos procesos consecutivos en la SC, ya que los valores de $s$ no pueden cumplir las condiciones simultáneamente.
%         \item Si $i = 0$, la condición $s[0] = s[n-1]$ asegura que el proceso 0 puede entrar solo si su estado está sincronizado con el del último proceso, evitando conflictos.
%     \end{itemize}
%     \item Al salir de la SC:
%     \begin{itemize}
%         \item Para $i > 0$, la operación $s[i] := s[i-1]$ iguala el estado del proceso $i$ al de su predecesor, modificando las condiciones para permitir que otro proceso entre en la SC.
%         \item Para $i = 0$, $s[0] := (s[0] + 1) \mod 2$ altera cíclicamente el estado del proceso inicial, asegurando que las condiciones del anillo cambien continuamente.
%     \end{itemize}
% \end{enumerate}

% Por lo tanto, el algoritmo garantiza \textbf{exclusión mutua}, ya que las condiciones de entrada aseguran que solo un proceso puede estar en la SC en cualquier momento.

\subsection{Parte I: Demostrar que el algoritmo garantiza exclusión mutua}

La exclusión mutua significa que, en cualquier configuración del sistema, como máximo un proceso puede estar en su sección crítica (SC) en un momento dado. A continuación, se analiza cómo las reglas del algoritmo garantizan esta propiedad.

\subsubsection{Condiciones para entrar en la sección crítica}

\begin{itemize}
    \item Para $i > 0$: $s[i] \neq s[i-1]$.  
    Esto implica que un proceso $i$ solo puede entrar a la SC si su variable $s[i]$ es distinta de la de su vecino inmediato anterior $s[i-1]$. Por lo tanto:
    \begin{itemize}
        \item Si un proceso $i$ está en la SC, esto obliga a que $s[i] \neq s[i-1]$.
        \item Su vecino inmediato anterior $i-1$, que también depende de $s[i-1]$, no podrá entrar a la SC al mismo tiempo porque para $i-1$ entrar se requeriría $s[i-1] \neq s[i-2]$, lo que sería inconsistente con la condición de $i$.
    \end{itemize}

    \item Para $i = 0$: $s[0] = s[n-1]$.  
    El proceso $0$ solo puede entrar si su variable $s[0]$ coincide con la del último proceso $s[n-1]$. Esto asegura la sincronización entre el principio y el final del anillo. Si $0$ está en la SC, ningún otro proceso puede cumplir las condiciones necesarias, porque las relaciones $s[i] \neq s[i-1]$ o $s[0] = s[n-1]$ no se mantendrían simultáneamente.
\end{itemize}

Por lo tanto, estas condiciones aseguran que, como máximo, un proceso puede satisfacer las reglas de entrada a la SC en cualquier momento.

\subsubsection{Condiciones al salir de la sección crítica}

Cuando un proceso $i$ termina de ejecutar su sección crítica, realiza las siguientes actualizaciones:

\begin{itemize}
    \item Para $i > 0$: $s[i] := s[i-1]$, solo si $s[i] \neq s[i-1]$.  
    Al asignar $s[i]$ el valor de $s[i-1]$, el proceso $i$ ``cede el paso'' a su vecino. Esto significa que, después de que $i$ sale de la SC:
    \begin{itemize}
        \item Su estado $s[i]$ se iguala al de $s[i-1]$, deshabilitando inmediatamente su propia entrada a la SC.
        \item Esto permite que otros procesos cumplan las condiciones necesarias para acceder a la SC.
    \end{itemize}

    \item Para $i = 0$: $s[0] := (s[0] + 1) \mod 2$.  
    Al incrementar cíclicamente $s[0]$, el proceso $0$ cambia el estado global del anillo, lo que obliga a que las condiciones de entrada se actualicen para otros procesos.
\end{itemize}

Estas actualizaciones aseguran que:
\begin{enumerate}
    \item Un proceso no permanezca en la SC indefinidamente.
    \item Los estados del sistema se actualicen de forma consistente para permitir que otros procesos entren en la SC.
\end{enumerate}

\subsubsection{Exclusión mutua en el anillo}

Las condiciones de entrada y salida interactúan de la siguiente manera para garantizar exclusión mutua:

\begin{enumerate}
    \item \textbf{Vecinos no pueden entrar simultáneamente:}  
    La condición $s[i] \neq s[i-1]$ asegura que dos procesos consecutivos en el anillo no pueden entrar a la SC al mismo tiempo.  
    Por ejemplo, si el proceso $i$ entra en la SC, entonces $s[i] \neq s[i-1]$. Esto implica que el vecino $i-1$ no puede cumplir la condición para entrar en la SC al mismo tiempo.

    \item \textbf{Inicio y fin del anillo sincronizados:}  
    Para $i = 0$, la condición $s[0] = s[n-1]$ asegura que el proceso $0$ solo puede entrar en la SC cuando está sincronizado con el estado del último proceso $n-1$. Esto evita conflictos en los extremos del anillo.

    \item \textbf{Imposibilidad de más de un acceso:}  
    Supongamos que dos procesos distintos, $i$ y $j$, intentan entrar en la SC simultáneamente.
    \begin{itemize}
        \item Si $i > 0$ y $j > 0$, ambos procesos requerirían que $s[i] \neq s[i-1]$ y $s[j] \neq s[j-1]$, lo que es imposible porque sus estados están interdependientes y los valores $s[k]$ forman un único conjunto de valores en el anillo.
        \item Si $i = 0$, entonces $s[0] = s[n-1]$ debe cumplirse, pero esto es inconsistente con las condiciones $s[1] \neq s[0]$ necesarias para que otros procesos entren simultáneamente.
    \end{itemize}
\end{enumerate}


\subsubsection{El diseño del algoritmo asegura que:}

\begin{itemize}
    \item Cada proceso depende exclusivamente de su propio estado $s[i]$ y el de su vecino inmediato ($s[i-1]$ o $s[n-1]$ para $i = 0$).
    \item Las condiciones de entrada y salida están sincronizadas para evitar conflictos.
    \item En cualquier configuración, solo un proceso puede cumplir las condiciones de entrada, lo que garantiza \textbf{exclusión mutua}.
\end{itemize}

Por lo que queda demostrado que el algoritmo asegura la Exclusión Mutua.



% \subsection{Parte II: Progreso y justicia}

% \textbf{Progreso:} En cada iteración del protocolo, un proceso $i$ que cumple las condiciones puede entrar en la SC. Al salir de la SC, el proceso actual actualiza $s[i]$ para que otro proceso tenga la oportunidad de cumplir las condiciones.

% \textbf{Justicia:} La configuración del anillo y las condiciones locales aseguran que cada proceso eventualmente podrá entrar en la SC:
% \begin{itemize}
%     \item Si un proceso $i$ no puede entrar en la SC porque $s[i] = s[i-1]$, entonces su vecino $i-1$ eventualmente entrará en la SC y actualizará su estado, lo que permitirá a $i$ cumplir las condiciones.
%     \item Debido a la actualización cíclica de $s[0]$, las condiciones se modifican periódicamente, asegurando que todos los procesos tengan una oportunidad de entrar en la SC.
% \end{itemize}

% \textbf{Demostración:}
% \begin{enumerate}
%     \item Dado que las condiciones de entrada dependen únicamente de los estados locales $s[i]$ y $s[i-1]$, cada proceso está garantizado a tener la oportunidad de entrar en la SC en un número finito de pasos.
%     \item El protocolo avanza en un anillo, y la actualización de estados asegura que ningún proceso quede bloqueado indefinidamente.
% \end{enumerate}

% Por lo tanto, el algoritmo asegura que cada proceso \textbf{entra en la SC infinitamente a menudo}.

% \section{Conclusión}

% El algoritmo cumple con las propiedades fundamentales de exclusión mutua, progreso y justicia:
% \begin{enumerate}
%     \item Garantiza que solo un proceso puede estar en la SC a la vez.
%     \item Asegura que todos los procesos tengan oportunidades iguales y periódicas de entrar en la SC.
% \end{enumerate}

\subsection{Parte II: Demostrar que cada proceso entra a la sección crítica infinitamente a menudo}

El objetivo es demostrar que el algoritmo garantiza que ningún proceso queda bloqueado indefinidamente y que cada proceso puede entrar en su sección crítica (SC) un número infinito de veces.

\subsubsection{Progreso del sistema}

El progreso significa que, en cualquier configuración válida, siempre habrá al menos un proceso que cumpla las condiciones necesarias para entrar en la SC. Esto se garantiza por las siguientes razones:

\begin{enumerate}
    \item \textbf{Dependencia local de las condiciones de entrada:}  
    Cada proceso $i$ puede decidir entrar en la SC únicamente basándose en el valor de $s[i]$ y $s[i-1]$ (o $s[n-1]$ si $i = 0$). Por lo tanto:
    \begin{itemize}
        \item La decisión de un proceso $i$ no depende de los valores de procesos distantes, lo que permite que las decisiones se tomen de forma local y eficiente, es decir, depende de sus vecinos inmediatos.
        \item Esto implica que el estado global del anillo no bloquea el progreso de ningún proceso en particular.
    \end{itemize}

    \item \textbf{Actualización consistente de los valores de $s[i]$:}  
    Cuando un proceso $i$ sale de la SC, actualiza su variable $s[i]$ de acuerdo con el protocolo:
    \begin{itemize}
        \item Si $i > 0$: $s[i] := s[i-1]$, lo que sincroniza el estado del proceso con su vecino anterior y permite que $i+1$ cumpla las condiciones para entrar.
        \item Si $i = 0$: $s[0] := (s[0] + 1) \mod 2$, lo que altera el estado global del anillo, asegurando que las condiciones para otros procesos cambien eventualmente.
    \end{itemize}

    Estas actualizaciones garantizan que los valores de $s[i]$ evolucionen de manera predecible, permitiendo que los procesos se turnen para entrar en la SC.

    \item \textbf{Anillo cerrado y cíclico:}  
    El diseño del anillo asegura que todos los procesos están conectados en una estructura cíclica.  
    Si un proceso $i$ no puede entrar en la SC debido a su vecino, ese vecino eventualmente actualizará su estado y permitirá que $i$ entre.  
    Por ejemplo, si $s[i] = s[i-1]$, el vecino $i-1$ debe entrar en la SC antes de que $s[i-1]$ cambie, lo que desbloqueará $i$.
\end{enumerate}

En conjunto, estas propiedades garantizan que siempre haya al menos un proceso con las condiciones necesarias para entrar en la SC, asegurando progreso continuo.

\subsubsection{Acceso infinito a la sección crítica}

Ningún proceso puede quedar bloqueado indefinidamente, y cada proceso $i$ podrá entrar en la SC un número infinito de veces.

Esto podemos garantizarlo en base a las siguientes razones:

\begin{enumerate}
    \item \textbf{Evolución de las condiciones de entrada:}  
    Las actualizaciones de $s[i]$ al salir de la SC aseguran que las condiciones necesarias para que otros procesos entren se modifiquen periódicamente.  
    Por ejemplo:
    \begin{itemize}
        \item Cuando un proceso $i$ actualiza $s[i] := s[i-1]$, su vecino $i+1$ tiene una mayor probabilidad de cumplir $s[i+1] \neq s[i]$ en el futuro.
        \item De manera similar, cuando $i = 0$, la actualización $s[0] := (s[0] + 1) \mod 2$ modifica cíclicamente las condiciones del anillo.
    \end{itemize}
    Estas modificaciones aseguran que las condiciones necesarias para entrar en la SC no permanezcan estáticas, permitiendo que todos los procesos tengan una oportunidad justa de acceder.

    \item \textbf{Estructura cíclica del anillo:}  
    En un sistema de $n$ procesos conectados en anillo:
    \begin{itemize}
        \item Si el proceso $i$ no puede entrar en la SC porque $s[i] = s[i-1]$, su vecino inmediato $i-1$ deberá entrar en la SC y actualizar $s[i-1]$ antes de que $i$ pueda intentarlo de nuevo.
        \item Este comportamiento garantiza que todos los procesos sean atendidos eventualmente, ya que no hay dependencia circular que pueda bloquear un proceso de manera indefinida.
    \end{itemize}

    \item \textbf{Periodicidad garantizada:}  
    Debido a que $s[0]$ se actualiza cíclicamente como $(s[0] + 1) \mod 2$, las configuraciones del anillo eventualmente regresan a un estado similar al inicial.  
    En cada ciclo completo, todos los procesos tienen la oportunidad de cumplir las condiciones necesarias para entrar en la SC.  
    Este comportamiento asegura que cada proceso tendrá acceso a la SC un número infinito de veces.
\end{enumerate}

\subsubsection{Análisis combinando Progreso y Acceso Infinito a la Sección Crítica}

\begin{enumerate}
    \item \textbf{Progreso garantiza que el sistema no se detiene:}  
    Siempre hay al menos un proceso que puede entrar en la SC. Cuando un proceso sale, actualiza las condiciones para permitir que otros procesos avancen.

    \item \textbf{Justicia garantiza que nadie queda excluido:}  
    Gracias a la estructura cíclica y las actualizaciones periódicas de $s[i]$, todos los procesos eventualmente tendrán una oportunidad de cumplir las condiciones para entrar en la SC.

    \item \textbf{Interacción entre progreso y justicia:}  
    Si un proceso $i$ no puede entrar porque su vecino $i-1$ no ha actualizado $s[i-1]$, ese vecino eventualmente tendrá la oportunidad de entrar y actualizar $s[i-1]$. Esto evita bloqueos indefinidos y asegura que todos los procesos accedan a la SC un número infinito de veces.
\end{enumerate}

Por lo tanto, se cumplen las propiedades de \textbf{progreso} y \textbf{acceso infinito a la sección crítica}, asegurando que ningún proceso quede bloqueado indefinidamente y que todos los procesos puedan acceder a la SC un número infinito de veces, por lo que esta propiedad queda demostrada.

\subsection{Relación con los axiomas de la programación concurrente}

El algoritmo cumple con los cinco axiomas fundamentales de la programación concurrente. A continuación, se describe cómo se satisfacen:

\subsubsection{Atomicidad e Intercalación de Instrucciones}

Este axioma establece que cada instrucción del programa concurrente se ejecuta de forma atómica y el resultado depende del orden en que las instrucciones se intercalan. En el contexto del algoritmo:

\begin{itemize}
    \item \textbf{Atomicidad:}  
    Las operaciones de lectura y escritura sobre las variables \( s[i] \) son atómicas. Por ejemplo, al verificar la condición \( s[i] \neq s[i-1] \) o al actualizar \( s[i] := s[i-1] \), se garantiza que la operación se completa antes de que otro proceso acceda a las mismas variables.
    
    \item \textbf{Intercalación:}  
    Los procesos avanzan en su ejecución intercalando las instrucciones, asegurando que sólo un proceso a la vez pueda acceder a la sección crítica. Esto se deriva de las condiciones de entrada (\( s[i] \neq s[i-1] \) y \( s[0] = s[n-1] \)) y de las actualizaciones consistentes al salir de la sección crítica.
\end{itemize}

\subsubsection{Consistencia de datos después del acceso concurrente}

Este axioma requiere que el estado del sistema sea consistente después de cualquier acceso concurrente. En el algoritmo:

\begin{itemize}
    \item La actualización de las variables \( s[i] \) tras salir de la sección crítica garantiza que el sistema mantenga un estado consistente:
    \begin{itemize}
        \item Para \( i > 0 \): \( s[i] := s[i-1] \) sincroniza el estado del proceso con su vecino anterior.
        \item Para \( i = 0 \): \( s[0] := (s[0] + 1) \mod 2 \) asegura que el estado global del anillo avance de manera predecible.
    \end{itemize}
    \item Estas reglas evitan inconsistencias, como permitir que dos procesos cumplan simultáneamente las condiciones para entrar en la sección crítica.
\end{itemize}

\subsubsection{Irrepetibilidad de las secuencias de instrucciones}

Este axioma implica que una misma secuencia de instrucciones puede producir diferentes resultados en ejecuciones distintas debido a la concurrencia. En el algoritmo:

\begin{itemize}
    \item Las decisiones para entrar en la sección crítica dependen del estado actual de las variables \( s[i] \), que varía dinámicamente durante la ejecución del sistema.
    \item Por ejemplo, el proceso \( i \) puede cumplir \( s[i] \neq s[i-1] \) en una configuración específica, pero no en otra, dependiendo de cómo los procesos vecinos actualicen sus valores \( s[j] \).
\end{itemize}

Esta irrepetibilidad asegura que el comportamiento del sistema no sea rígido, adaptándose a las condiciones cambiantes del entorno concurrente.

\subsubsection{Independencia de la velocidad de proceso}

Este axioma establece que el sistema debe funcionar correctamente independientemente de las velocidades relativas de los procesos. En el algoritmo:

\begin{itemize}
    \item La independencia de la velocidad se garantiza porque las condiciones de entrada a la sección crítica dependen únicamente de los estados locales (\( s[i] \) y \( s[i-1] \)) y no de la velocidad con la que otros procesos se ejecutan.
    \item Incluso si un proceso avanza más rápido que los demás, no puede bloquear indefinidamente a otros procesos, ya que:
    \begin{itemize}
        \item El diseño en anillo asegura que cada proceso eventualmente actualice su estado.
        \item Las actualizaciones periódicas de \( s[0] \) mediante \( (s[0] + 1) \mod 2 \) permiten que las condiciones cambien para todos los procesos.
    \end{itemize}
\end{itemize}

\subsubsection{Hipótesis de progreso finito}

Este axioma implica que cada proceso debe avanzar y alcanzar su objetivo (por ejemplo, entrar en la sección crítica) en un tiempo finito. En el algoritmo:

\begin{itemize}
    \item \textbf{Progreso finito} se garantiza porque:
    \begin{itemize}
        \item Las actualizaciones de \( s[i] \) al salir de la sección crítica permiten que los procesos desbloqueen a sus vecinos.
        \item La estructura cíclica del anillo asegura que las condiciones necesarias para entrar en la sección crítica evolucionen periódicamente, evitando bloqueos indefinidos.
    \end{itemize}
    \item Cada proceso tiene garantizado el acceso infinito a la sección crítica gracias a la combinación de las propiedades de progreso (ningún proceso bloquea indefinidamente a otro) y justicia (todos los procesos tienen oportunidades equitativas de acceder a la sección crítica).
\end{itemize}


Podemos concluir con que el algoritmo cumple con los cinco axiomas de la programación concurrente.

\subsection{Propiedades de vivacidad y seguridad}

El algoritmo cumple con las dos propiedades fundamentales de los sistemas concurrentes: \textbf{vivacidad} y \textbf{seguridad}.

\subsubsection{Vivacidad}

La vivacidad asegura que el sistema no se bloquea indefinidamente y que cada proceso tiene garantizado el progreso. En este caso:

\begin{itemize}
    \item Se garantiza que siempre hay al menos un proceso que puede entrar en la sección crítica debido a las condiciones dinámicas de entrada:
    \begin{itemize}
        \item Para \( i > 0 \): \( s[i] \neq s[i-1] \) asegura que un proceso solo depende de su vecino inmediato.
        \item Para \( i = 0 \): \( s[0] = s[n-1] \) garantiza que el anillo evoluciona cíclicamente.
    \end{itemize}
    \item Las actualizaciones de \( s[i] \) y la estructura en anillo aseguran que cada proceso tenga la oportunidad de entrar en la sección crítica después de que otros procesos completen la suya.
    \item Por lo tanto, no hay bloqueos ni inanición, cumpliéndose la propiedad de vivacidad.
\end{itemize}

\subsubsection{Seguridad}

La seguridad garantiza que nunca se violan las reglas del sistema, como permitir que más de un proceso acceda simultáneamente a la sección crítica. En este caso:

\begin{itemize}
    \item Las condiciones de entrada a la sección crítica (\( s[i] \neq s[i-1] \) para \( i > 0 \), y \( s[0] = s[n-1] \) para \( i = 0 \)) aseguran que solo un proceso pueda entrar en la sección crítica a la vez:
    \begin{itemize}
        \item Si dos procesos intentaran entrar simultáneamente, las condiciones de entrada impedirían que ambos lo hagan, ya que al menos uno de ellos no cumpliría \( s[i] \neq s[i-1] \).
    \end{itemize}
    \item Las actualizaciones consistentes de \( s[i] \) tras salir de la sección crítica evitan que otros procesos accedan simultáneamente, manteniendo la exclusión mutua.
    \item Por lo tanto, se cumple la propiedad de seguridad, garantizando que el acceso a la sección crítica sea exclusivo.
\end{itemize}

Podemos concluir que el algoritmo cumple con las propiedades de vivacidad y de seguridad.

\section{Bibliografía}

\begin{thebibliography}{9}

\bibitem{Diapositivas de clase}
Diapositivas de la asignatura \textit{Sistemas Concurrentes y Distribuidos}, en la Universidad de Granada.


\bibitem{lamport1978}
L. Lamport, 
\textit{Time, Clocks, and the Ordering of Events in a Distributed System}, 
Communications of the ACM, vol. 21, no. 7, pp. 558-565, 1978.

\bibitem{concurrencyaxioms}
M. Ben-Ari, 
\textit{Principles of Concurrent and Distributed Programming}, 
2nd Edition, Addison-Wesley, 2006.

\end{thebibliography}

\section{Anotaciones}

Cabe destacar que la extensión de esta práctica se debe a la intención de utilizar la mayor cantidad posible de conceptos del temario, así como de proporcionar una demostración y explicación detallada de las propiedades de un algoritmo de exclusión mutua.

\end{document}
