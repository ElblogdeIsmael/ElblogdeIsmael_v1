\documentclass[a4paper,12pt]{book}

% Paquetes necesarios
\usepackage[utf8]{inputenc}   % Codificación de caracteres
\usepackage[spanish]{babel}   % Idioma español
\usepackage[T1]{fontenc}      % Codificación de fuentes
\usepackage{amsmath, amssymb} % Símbolos matemáticos
\usepackage{graphicx}         % Inclusión de gráficos
\usepackage{cite}             % Gestión de citas
\usepackage{hyperref}         % Enlaces y referencias
\usepackage{geometry}         % Configuración de márgenes
\usepackage{fancyhdr}         % Encabezados y pies de página
\usepackage{titlesec}         % Formato de títulos
\usepackage{booktabs}         % Tablas profesionales
\usepackage{caption}          % Personalización de leyendas
\usepackage{enumitem}         % Personalización de listas
\usepackage{float}
\usepackage{tcolorbox}

% Configuración de márgenes
\geometry{left=3cm, right=3cm, top=2.5cm, bottom=2.5cm}

% Configuración de encabezados y pies de página
\setlength{\headheight}{14.49998pt}
\pagestyle{fancy}
\fancyhf{}
\fancyhead[L]{Universidad de Granada}
% \fancyhead[C]{Escuela Técnica Superior de Ingenierías Informática}
\fancyhead[R]{Grado en Ingeniería Informática + ADE}
\fancyfoot[L]{Ismael Sallami Moreno}
\fancyfoot[C]{\thepage}
\fancyfoot[R]{\today}

% Formato de títulos
\titleformat{\section}{\large\bfseries}{\thesection.}{0.5em}{}
\titleformat{\subsection}{\normalsize\bfseries}{\thesubsection.}{0.5em}{}

% Datos del documento
\title{\textbf{Temario Inteligencia Artificial}}
\author{
    Ismael Sallami Moreno \\
    \texttt{ism350zsallami@correo.ugr.es}
}
\date{
    \vspace{1cm}
    \begin{tabular}{rl}
        \textbf{Asignatura:} & Inteligencia Artificial \\
        \textbf{Tema:} & Teoría \\
        \textbf{Fecha:} & \today
    \end{tabular}
}

\begin{document}

% Portada
\maketitle
% \begin{center}
%     \includegraphics[width=0.2\textwidth]{images/logo_ugr.png}
% \end{center}
\newpage

% Resumen
% \begin{abstract}
% \noindent
% \textbf{Resumen:} Este documento presenta los apuntes y/o resúmenes del tema que se expone que se han realizado a lo largo del curso de la asignatura de Inteligencia Artificial. En este caso, se trata de las prácticas de la asignatura.
% \end{abstract}
% \bigskip



% Tabla de contenidos
\tableofcontents
\newpage

\chapter{Tema 1}

\section{Relación de Ejercicios T1}

\begin{enumerate}
    \item ¿Cuáles son las principales diferencias entre un sistema de procesamiento de archivos y un sistema de bases de datos?
    \\\\
    Por un lado, en un sistema de procesammiento de archivos los datos se almacenan en archivos y para trabajar con ellos se utilizan programas específicos. Esto hace que sean desordenados, con repeticiones causando problemas de organización. Además, cabe destacar que la seguridad es limitada. 

    Por otro lado, en un sistema de bases de datos, los datos se estructuran en tablas relacionadas entre sí. Para gestionar y manipular estos datos, se utiliza un lenguaje de consulta estructurado conocido como SQL, que facilita su manejo de manera eficiente y segura. Además, las bases de datos ofrecen mejores mecanismos de seguridad y son altamente escalables, permitiendo gestionar grandes volúmenes de datos y múltiples usuarios simultáneamente sin inconvenientes.


    \item Describe las características más importantes en un sistema de base de datos y también las propiedades más deseables. Explica a tu juicio cuál es la propiedad más importante.
    
    Características y propiedades más importantes de un SGBD:

    \begin{enumerate}[label=\alph*)]
        \item Estructura y organización: los datos estan estructurados de manera eficiente, lo que permite un acceso rápido y fácil a la información.
        \item Integridad: los datos deben ser precisos y consistentes, evitando errores y duplicaciones.
        \item Seguridad: el acceso a los datos debe estar controlado y protegido contra accesos no autorizados.
        \item Escalabilidad: el sistema debe ser capaz de manejar un aumento en la cantidad de datos y usuarios sin perder rendimiento.
        \item Confiabilidad: el sistema debe ser capaz de recuperarse de fallos y mantener la disponibilidad de los datos.
    \end{enumerate}

    Bajo mi punto de vista, la propiedad más importante es la seguridad ya que de esta manera se evitan accesos a la información no autorizados y otros problemas, como la pérdida de datos o la corrupción de la base de datos. La seguridad es fundamental para proteger la información sensible y garantizar la confianza de los usuarios en el sistema.

    \item Hemos conocido las ventajas de utilizar un sistema de bases de datos, ¿podrías comentar también algunos inconvenientes?
    
    \begin{itemize}
        \item \textbf{Costo:} La implementación y configuración de un sistema de bases de datos puede ser costosa, especialmente en proyectos grandes o en empresas con necesidades complejas. Además, el mantenimiento continuo del sistema también implica costos adicionales.
        \item \textbf{Complejidad:} Es necesario contar con personal capacitado con conocimientos técnicos especializados para gestionar y utilizar la base de datos de manera eficiente. Asimismo, la organización y estructuración de grandes volúmenes de datos o relaciones complejas entre entidades puede ser un desafío.
        \item \textbf{Seguridad:} Las bases de datos pueden ser vulnerables a ataques de hackers interesados en acceder a información sensible. Por ello, es fundamental implementar medidas de seguridad sólidas y contar con un plan de recuperación ante desastres, lo que puede aumentar la complejidad de la gestión.
    \end{itemize}

    \item Explica la diferencia entre independencia física e independencia lógica.
    
    La independencia física permite cambiar cómo se almacenan los datos sin afectar su uso lógico, es decir, que si se cambia el hardware las aplicaciones no se ven afectadas, mientras que la independencia lógica permite modificar la organización y presentación de los datos sin alterar su almacenamiento físico. En esencia, la primera se centra en el almacenamiento y la segunda en la estructura lógica de cara al usuario.

    \item Definir brevemente los siguientes conceptos:
    \begin{itemize}
        \item Base de datos: Conjunto de datos relacionados y organizados de manera estructurada, que se almacenan y gestionan mediante un sistema de gestión de bases de datos (SGBD).
        \item DBMS (DataBase Management System): Programas para describir las estructuras y gestionar la información de BD.
        \item DBA (DataBase Administrator): Persona responsable de la administración y gestión de una base de datos, asegurando su rendimiento, seguridad y disponibilidad.
        \item Acceso concurrente: Capacidad de múltiples usuarios para acceder y manipular datos en una base de datos al mismo tiempo, garantizando la integridad y consistencia de la información.
        \item Vista de usuario: Representación personalizada de los datos en una base de datos, que permite a los usuarios interactuar con la información de manera específica y adaptada a sus necesidades, sin necesidad de conocer la estructura interna de la base de datos.
    \end{itemize}
    \item Explicar brevemente los conceptos de: Integridad, fiabilidad y seguridad en una base de datos.
        \begin{itemize}
            \item Ordenarlos por importancia, explicando los criterios utilizados para elaborar el orden.
            
            \begin{enumerate}
                \item Seguridad: Es fundamental proteger la información sensible y garantizar que solo los usuarios autorizados tengan acceso a los datos. Sin seguridad, la integridad y fiabilidad de los datos pueden verse comprometidas.
                \item Integridad: La precisión y consistencia de los datos son esenciales para garantizar que la información sea útil y confiable. Si los datos no son íntegros, las decisiones basadas en ellos pueden ser erróneas.
                \item Fiabilidad: Aunque la fiabilidad es importante, si los datos son seguros e íntegros, la fiabilidad se convierte en un aspecto secundario. La fiabilidad se basa en la seguridad y la integridad de los datos, por lo que su importancia es menor en comparación con los otros dos aspectos.
            \end{enumerate}

            \item ¿En qué etapa de la vida de una base de datos se deben tener en cuenta unos y otros?
            
            \begin{itemize}
                \item Seguridad: Desde el inicio del diseño de la base de datos, es fundamental establecer políticas y medidas de seguridad para proteger la información.
                \item Integridad: Durante la fase de diseño y modelado de la base de datos, se deben definir reglas y restricciones para garantizar la integridad de los datos.
                \item Fiabilidad: A lo largo de toda la vida útil de la base de datos, es importante mantener un sistema confiable mediante copias de seguridad y recuperación ante desastres.
            \end{itemize}

            \item ¿Cómo se mantienen en una base de datos? 
            \begin{itemize}
                \item Seguridad: Se implementan controles de acceso, autenticación y autorización, así como cifrado de datos y auditorías de seguridad.
                \item Integridad: Se utilizan restricciones de integridad, como claves primarias, foráneas y reglas de validación, para asegurar la calidad de los datos.
                \item Fiabilidad: Se realizan copias de seguridad periódicas, pruebas de recuperación y mantenimiento preventivo del sistema para garantizar su disponibilidad y rendimiento.
            \end{itemize}
        \end{itemize}
\end{enumerate}


\chapter{Tema 2}

Uno de los ejercicios de examen será identificar los componentes de una imagen de una máquina (PC, Pórtatil).

Cada uno de los portátiles tiene un formato propio.

Se proporciona información sobre los componentes de la placa base, en especial, en los manuales de Prado.

En cuanto al montaje de los componentes de la placa base, debemos de tener cuidado:
\begin{itemize}
    \item Que esté apagado.
    \item Debemos de usar guantes que no conducen la electricidad, aunque en los actuales, suelen estar protegidos.
    \item No tocar nada metálico con la placa base.
    \item Un componente solo se instala de una manera, no debemos de forzarlo.
\end{itemize}

\subsection{Fuente de Alimentación}

Suele ser la parte que más se avería. Se puede comprobar con un polímetro. La fuente de alimentación posee numerosos puertos para cargar cada una de las partes de la máquina que requieren enegía.

Hay varios tipos de fuentes de alimentación.

Otra de las partes es el módulo regular de voltaje, esta pensado para evitar que el ordenador se pueda quemar, entre otras palabras.

En cuanto a los procesadores, hay dos tipos:
\begin{itemize}
    \item PGA.
    \item LGA.  
\end{itemize}


\begin{tcolorbox}[colback=yellow!5!white,colframe=yellow!75!black]
    \textbf{Pregunta de examen:} ¿Un conector de tipo PGA tiene pines alargados (patillas) que se enganchan al procesador? \textbf{Verdadero o Falso} Es falso.
    
\end{tcolorbox}

\begin{figure}[H]
    \centering
    \includegraphics[width=0.8\textwidth]{images/Tema2/evol.png}
    \caption{Evolución histórica de los microprocesadores.}
    \label{fig:1}
\end{figure}

\textbf{Preguntas de examen:}
\begin{itemize}
    \item ¿Que pasó en 2005 relacionado con la frecuencia de los procesadores?. Responder en base a la imagen de arriba. Que deciden poner más cores,\dots
    \item ¿Que diferencia a un procesador de sobremesa a uno de servidores? Número de cores.
    \item Diferencias enter un microprocesador de PC a los de servidores (Transparencia número 15).
    \item ¿Que es un canal de RAM? Conecta la memoria con el procesador, de manera que se puede leer más rápido.
\end{itemize}

Un servidor tiene instrucciones distintas a las de ordenadores de sobremesa, debido a que estan destinados por ejemplo a virtualización de máquinas virtuales.

Los procesadores AMD para servidores, son multichips, es decir, tienen varios procesadores en un mismo chip (no todos).

Otra pregunta:
¿La primera arquitectura de 64 bits fue en 2004? Falso, ya había en los años 90.

Otra de las partes de un ordenador que podemos destacar es el \textit{IBM POWER}, que significa Performance Optimization With Enhanced RISC.

Luego podemos encontrar el disipador de calor, que se encarga de refrigerar el procesador.

Tenemos ranuras de memoria DRAM (Dynamic RAM), que es donde se conecta los módulos de memoria principal. Dentro del procesador hay memoria RAM, pero es estática. La memoria RAM es volátil, es decir, que si se apaga el ordenador, se pierde la información.

\begin{figure}
    \centering
    \includegraphics[width=0.8\textwidth]{images/Tema2/piramide.png}
    \caption{Jerarquía de memoria.}
    \label{fig:2}
\end{figure}
Las cintas de la imagen, se usan para copias de seguridad, ya que son muy lentas.

En cuanto a la evolución histórica de la memoria RAM, podemos destacar que primero estaba la DRAM, luego surge la PM, luego la SDRAM,\dots

Al principio tenían lo que conocemos como 'patitas', luego pasan a conectores, y luego a conectores que se encuentran por delante y por detrás.

Tenemos diversos tipos de DMIM (Dual Data Rate Inline Memory Module), que son módulos de memoria RAM:
\begin{itemize}
    \item Para servidores.
    \begin{itemize}
        \item EU-DIMM. U-DDIM con corrección de errores. Por cada línea\footnote{Como línea nos referimos a las líneas de datos, de manera gráfica ver la diapositiva 23, líneas con números en ECC RAM.} de datos se pasan 8 bits de corrección de errores y otros 8 bits de datos.
        \item R-DIMM: Registered DIMM. Hay un registro que almacena
        las señales de control (operación a realizar, líneas de
        dirección…). Mayor latencia que EU-DIMM pero permiten
        módulos de mayor tamaño. Tienen ECC.
        \item LR-DIMM: Load Reduced DIMM. Hay un buffer que
        almacena tanto las señales de control como los datos a
        leer/escribir. Mayor latencia que R-DIMM, pero son las que
        permiten los módulos con mayor tamaño. Tienen ECC
    \end{itemize}
    \item Para PC y portátiles.
    \begin{itemize}
        \item DIMM: Unbuffered (ó Unregistered) DIMM.
        \item SO-DIMM. Small Outline DIMM. Tamaño más reducido para
        equipos portátiles (tienen menos contactos).
    \end{itemize}
\end{itemize}

Cuando se aumenta el número de pines es más eficiente. En cuanto al ancho de banda debemps de tener en cuenta que va aumentando también. En la diapositiva 21, no debemos de aprendernos los números, pero si debemos de saber ocmo crecen y demás.

Un procesador accede a los módulos de memoria DRAM a  través de los \textit{canales de memoria}. Debemos de tener en cuenta que cuando las encajamos no debemos de hacer mucha fuerza. Algunos módulos tienes disipadores de calor, chips,... Hay un espacio entre las memorias para que se refrigere. En cada instante de lectura no se puede leer de manera simultánea, sino que se lee de manera secuencial. Podemos distinguir entre \textit{bancos}(leemos solo de uno) y entre \textit{canales}(leemos de todos los canales a la vez), cuando son canales podemos leer al final solo de un módulo de memoria (Diapositiva 23).

El concepto de rango se refiere al hecho de tener bancos de memoria. En cuanto a la notación de $1R\times8$ se refiere a que es un módulo de memoria con un rango, y que tiene 8 chips de memoria, por lo que lee de 8 chips a la vez. (A nivel de módulo, banco es a nivel de zócalo).

Se puede dar el caso que sea Double Sided, pero 1 rank, por lo que aunque tenga rangos por delante y por detrás, solo puede leer de uno a la vez.











\newpage
% Referencias
\begin{thebibliography}{99}
\bibitem{Referencia1}
Autor(es), \emph{Título del artículo}, Nombre de la Revista, volumen(número), páginas, año.

\bibitem{Referencia2}
Autor(es), \emph{Título del libro}, Editorial, año.

\bibitem{Referencia3}
Autor(es), \emph{Título del documento}, Nombre de la Conferencia, páginas, año.
\end{thebibliography}

\end{document}
