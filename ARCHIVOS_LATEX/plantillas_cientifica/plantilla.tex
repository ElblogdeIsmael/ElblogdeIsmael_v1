\documentclass[a4paper,12pt]{article}

% Paquetes necesarios
\usepackage[utf8]{inputenc}   % Codificación de caracteres
\usepackage[spanish]{babel}   % Idioma español
\usepackage{amsmath, amssymb} % Símbolos matemáticos
\usepackage{graphicx}         % Inclusión de gráficos
\usepackage{cite}             % Gestión de citas
\usepackage{hyperref}         % Enlaces y referencias
\usepackage{geometry}         % Configuración de márgenes
\usepackage{fancyhdr}         % Encabezados y pies de página
\usepackage{titlesec}         % Formato de títulos
\usepackage{booktabs}         % Tablas profesionales
\usepackage{caption}          % Personalización de leyendas
\usepackage{enumitem}         % Personalización de listas

% Configuración de márgenes
\geometry{left=3cm, right=3cm, top=2.5cm, bottom=2.5cm}

% Configuración de encabezados y pies de página
\setlength{\headheight}{14.49998pt}
\pagestyle{fancy}
\fancyhf{}
%\fancyhead[L]{Universidad de Granada}
\fancyhead[L]{\nouppercase{\chaptername \thechapter. \leftmark}}
\fancyhead[R]{Grado en Ingeniería Informática + ADE}
\fancyfoot[L]{\rule[0pt]{\textwidth}{0.2pt}\\Ismael Sallami Moreno}
\fancyfoot[C]{\rule[0pt]{\textwidth}{0.2pt}\\\thepage}
\fancyfoot[R]{\rule[0pt]{\textwidth}{0.2pt}\\\today}

\renewcommand{\sectionmark}[1]{\markboth{#1}{}} % Configura \leftmark para que solo muestre la sección


% Formato de títulos
\titleformat{\section}{\large\bfseries}{\thesection.}{0.5em}{}
\titleformat{\subsection}{\normalsize\bfseries}{\thesubsection.}{0.5em}{}

% Datos del documento
\title{\textbf{Práctica 0 de Fundamentos de Ingeniería del Software}}
\author{
    Nombre del Autor \\
    \texttt{correo@ejemplo.com}
}
\date{
    \vspace{1cm}
    \begin{tabular}{rl}
        \textbf{Asignatura:} & Nombre de la Asignatura \\
        \textbf{Tema:} & Nombre del Tema \\
        \textbf{Fecha:} & \today
    \end{tabular}
}

\begin{document}

% Portada
\maketitle
\begin{center}
    \includegraphics[width=0.5\textwidth]{images/logo-ugr.png}
\end{center}
\newpage

% Resumen
\begin{abstract}
\noindent
\textbf{Resumen:} Este documento presenta una plantilla en LaTeX para la elaboración de trabajos académicos en el ámbito tecnológico. Se incluyen secciones típicas como introducción, metodología, resultados y conclusiones, así como el formato para las referencias bibliográficas.
\end{abstract}
\bigskip

% Palabras clave
\noindent
\textbf{Palabras clave:} plantilla, LaTeX, trabajos académicos, tecnología
\newpage

% Tabla de contenidos
\tableofcontents
\newpage

% Introducción
\section{Introducción}
La introducción debe proporcionar una visión general del tema, los objetivos del trabajo y la relevancia del estudio. Aquí se contextualiza el problema y se establece la hipótesis o las preguntas de investigación.

% Metodología
\section{Metodología}
En esta sección se describe detalladamente el enfoque, los métodos y las técnicas utilizadas para llevar a cabo la investigación o el desarrollo del proyecto. Es importante que la metodología sea replicable por otros investigadores.

% Resultados
\section{Resultados}
Se presentan los hallazgos obtenidos durante la investigación. Pueden incluirse tablas, figuras y descripciones que faciliten la comprensión de los datos. Asegúrate de que todas las figuras y tablas estén correctamente numeradas y referenciadas en el texto.

% Discusión
\section{Discusión}
En la discusión se interpretan los resultados, se comparan con estudios previos y se analizan las implicaciones de los hallazgos. También se pueden abordar las limitaciones del estudio y posibles líneas de investigación futura.

% Conclusiones
\section{Conclusiones}
Las conclusiones resumen los aspectos más relevantes del trabajo, responden a las preguntas de investigación y destacan la contribución del estudio al campo tecnológico.

% Agradecimientos
\section{Agradecimientos}
Sección opcional donde se agradece a personas o instituciones que hayan contribuido significativamente al desarrollo del trabajo.

% Referencias
\begin{thebibliography}{99}
\bibitem{Referencia1}
Autor(es), \emph{Título del artículo}, Nombre de la Revista, volumen(número), páginas, año.

\bibitem{Referencia2}
Autor(es), \emph{Título del libro}, Editorial, año.

\bibitem{Referencia3}
Autor(es), \emph{Título del documento}, Nombre de la Conferencia, páginas, año.
\end{thebibliography}

\end{document}
