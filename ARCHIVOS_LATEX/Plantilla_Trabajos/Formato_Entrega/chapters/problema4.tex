\chapter{Problema 4}

\section{Descripción del problema}
El problema consiste en encontrar el camino de menor coste al ascender la "Pixel Mountain". La montaña se representa como una matriz de costes, donde cada elemento $(i, j)$ de la matriz indica el coste de posicionarse en ese punto. El objetivo es encontrar el camino desde la base (fila 0) hasta la cumbre (última fila) con el menor coste posible. Se permiten movimientos hacia arriba, arriba a la izquierda, y arriba a la derecha. Formalmente, queremos minimizar el coste total al llegar a cualquier posición en la cumbre $(f-1, j)$.

\section{Solución: Diseño de componentes y del algoritmo}

\subsection{Resolución del problema por etapas}
Para resolver este problema, utilizamos una técnica de programación dinámica. La idea es construir una tabla $T$ donde cada entrada $T(i, j)$ represente el coste mínimo para alcanzar la posición $(i, j)$ de la montaña. Consideremos la siguiente matriz de costes como ejemplo:

\[
\text{Montaña} = \begin{bmatrix}
4 & 7 & 8 & 6 & 4 \\
6 & 7 & 3 & 9 & 2 \\
3 & 8 & 1 & 2 & 4 \\
7 & 1 & 7 & 3 & 7 \\
2 & 9 & 8 & 9 & 3 \\
\end{bmatrix}
\]

La matriz $T$ se llenará de la siguiente manera:

\[
T(i, j) = \text{coste}(i, j) + \min \begin{cases} 
T(i-1, j) \\
T(i-1, j-1) & \text{si } j > 0 \\
T(i-1, j+1) & \text{si } j < \text{número de columnas} - 1 
\end{cases}
\]

Comenzamos inicializando la primera fila de $T$ con los valores de la primera fila de la matriz de costes, ya que esas son las posiciones de partida.

\[
T(0, j) = \text{Montaña}(0, j)
\]

Para las demás filas, aplicamos la ecuación recurrente para llenar la tabla:

\[
T = \begin{bmatrix}
4 & 7 & 8 & 6 & 4 \\
10 & 11 & 7 & 15 & 6 \\
13 & 15 & 8 & 9 & 10 \\
20 & 9 & 15 & 12 & 16 \\
11 & 18 & 17 & 21 & 15 \\
\end{bmatrix}
\]

Finalmente, el coste mínimo para alcanzar la cumbre de la montaña es el valor mínimo en la última fila de $T$:

\[
V = \min \{ T(f-1, j) \mid 0 \leq j < \text{número de columnas} \} = 11
\]

Para recuperar el camino de menor coste, partimos desde la posición en la última fila de $T$ que contiene el valor mínimo y rastreamos hacia atrás utilizando las posiciones anteriores de donde provino el coste mínimo. Esto nos permite reconstruir el camino óptimo.
\subsection{Ecuación recurrente}
La ecuación recurrente para calcular el coste mínimo para cada posición se define como:
\[
T(i, j) = \text{coste}(i, j) + \min \begin{cases} 
T(i-1, j) \\
T(i-1, j-1) & \text{si } j > 0 \\
T(i-1, j+1) & \text{si } j < \text{número de columnas} - 1 
\end{cases}
\]

Aquí, $\text{coste}(i, j)$ es el coste asociado a la posición $(i, j)$. La función $\min$ selecciona el coste mínimo entre las posiciones desde las que se puede llegar a $(i, j)$, es decir, desde $(i-1, j)$, $(i-1, j-1)$ si $j > 0$, y $(i-1, j+1)$ si $j < \text{número de columnas} - 1$.

\subsection{Valor objetivo}
El valor objetivo es el coste mínimo para alcanzar cualquier posición en la última fila (cumbre) de la matriz. Formalmente, esto se expresa como:
\[
V = \min \{ T(f-1, j) \mid 0 \leq j < \text{número de columnas} \}
\]
donde $f$ es el número de filas de la matriz.

\subsection{Verificación del cumplimiento del P.O.B}

El Principio de Optimalidad de Bellman (P.O.B.) establece que una solución óptima de un problema de optimización se puede obtener a partir de las soluciones óptimas de sus subproblemas. En nuestro contexto, esto significa que el coste mínimo para llegar a la posición $(i, j)$ en la montaña debe ser calculado usando las soluciones óptimas de las posiciones anteriores $(i-1, j)$, $(i-1, j-1)$ y $(i-1, j+1)$. 

Para verificar que nuestra ecuación recurrente cumple con el P.O.B., consideremos que estamos en la posición $(i, j)$:

\[
T(i, j) = \text{coste}(i, j) + \min \begin{cases} 
T(i-1, j) \\
T(i-1, j-1) & \text{si } j > 0 \\
T(i-1, j+1) & \text{si } j < \text{número de columnas} - 1 
\end{cases}
\]

Cada subproblema $T(i-1, j)$, $T(i-1, j-1)$ y $T(i-1, j+1)$ representa el coste mínimo para alcanzar las posiciones inmediatas anteriores desde donde se puede llegar a $(i, j)$. Al tomar el mínimo de estos subproblemas y sumarle el coste de la posición actual $\text{coste}(i, j)$, garantizamos que estamos construyendo la solución óptima de forma inductiva, basada en soluciones óptimas de subproblemas más pequeños. Por lo tanto, nuestra solución cumple con el Principio de Optimalidad de Bellman.


\subsection{Diseño de la memoria}
Con este diseño, construimos el algoritmo de Programación Dinámica como sigue:


\subsection{Diseño del algoritmo de cálculo de coste óptimo}
\begin{lstlisting}
Algoritmo CalcularCosteMinimo(Montana)
    f = numero de filas en Montana
    c = numero de columnas en Montana
    T = matriz de tamaño f x c, inicializada con infinito

    Para j desde 0 hasta c-1 hacer
        T(0, j) = Montana(0, j)

    Para i desde 1 hasta f-1 hacer
        Para j desde 0 hasta c-1 hacer
            T(i, j) = Montana(i, j) + T(i-1, j)
            Si j > 0 entonces
                T(i, j) = min(T(i, j), Montana(i, j) + T(i-1, j-1))
            Fin Si
            Si j < c-1 entonces
                T(i, j) = min(T(i, j), Montana(i, j) + T(i-1, j+1))
            Fin Si
        Fin Para
    Fin Para

    V =  min(T(f-1))
    
    devolver T, V
Fin Algoritmo
\end{lstlisting}

En este pseudocódigo:
\begin{itemize}
    \item Inicializamos la primera fila de la matriz $T$ con los valores de coste de la primera fila de la montaña.
    \item Llenamos la matriz $T$ de manera iterativa usando la ecuación recurrente.
    \item Finalmente, encontramos el valor objetivo $V$, que es el mínimo valor en la última fila de $T$.
\end{itemize}

\subsection{Diseño del algoritmo de recuperación de la solución}
Para recuperar la solución (el camino de menor coste), seguimos el rastro desde la cumbre hasta la base utilizando la tabla $T$:

\newpage
\begin{lstlisting}
Algoritmo RecuperarSolucion(T, Montana)
    f = numero de filas en Montana
    c = numero de columnas en Montana
    Solucion = lista vacia

    j = Indice de la columna con el valor minimo en T(f-1, :)

    Para i desde f-1 hasta 0 hacer
        anadir (i, j) a Solucion
        Si i > 0 entonces
            Si j > 0 y T(i-1, j-1) == T(i, j) - Montana(i, j) entonces
                j = j - 1
            Sino Si j < c-1 y T(i-1, j+1) == T(i, j) - Montana(i, j) entonces
                j = j + 1
            Fin Si
        Fin Si
    Fin Para

    invertir Solucion
    devolver Solucion
Fin Algoritmo
\end{lstlisting}

En este pseudocódigo:
\begin{itemize}
    \item Iniciamos en la última fila de $T$ y encontramos la columna $j$ con el valor mínimo.
    \item Rastreamos el camino desde la cumbre hasta la base, verificando de dónde provino el mínimo coste en cada paso.
    \item Almacenamos cada paso en la lista `Solucion`.
    \item Invertimos la lista `Solucion` para obtener el camino desde la base hasta la cumbre.
\end{itemize}